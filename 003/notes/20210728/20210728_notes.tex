\documentclass{article}

\usepackage{graphicx}
\usepackage[margin=1in]{geometry}
\usepackage{afterpage}
\usepackage{natbib}

\title{Research notes}
\date{July 28, 2021}
\author{Osamu Miyawaki}

\begin{document}
\maketitle

Project title: Investigating Arctic climate change in response to increasing CO$_2$ as a regime transition from Radiative Advective to Radiative Convective Advective Equilibrium

\section*{Introduction}
\begin{itemize}
    \item The Arctic is expected to undergo large and rapid changes associated with sea ice melt in response to anthropogenic increases in CO$_2$ \citep{dai2019, feldl2020}.
    \begin{itemize}
        \item Largest surface warming is projected in the Arctic \citep[Arctic Amplification, e.g.][]{manabe1975, held1993a}, especially during wintertime
        \item Large fractional increase in precipitation \citep{bintanja2014,siler2018}
    \end{itemize}
    \item The near-surface lapse rate feedback associated with the weakening of the surface inversion is thought to play a dominant role on surface amplified warming \citep{bintanja2011, pithan2014} along with the ice albedo feedback, both of which are thought to be driven by sea ice melt \citep{feldl2020}.
    \item However, \cite{previdi2020} show by investigating the transient response of the abrupt4$\times$CO2 run that amplified surface warming precedes significant sea ice melt, emphasizing the importance of different timescales in Arctic climate change in response to increased CO$_2$. The transient response to projected future changes in CO$_2$ has not yet been investigated.
    \item The transient response to projected future increases in CO$_2$ is important because 1) it tells us about the timing of when the projected changes will occur and 2) the Arctic atmospheric response is likely state-dependent (e.g., the warming rate before, during, and after sea ice melt).
    \item Here, we will
    \begin{enumerate}
        \item diagnose the the transient response of the Arctic energy budget to increasing CO$_2$ across a hierarchy of climate models using the moist static energy budget and $R_1$,
        \item show that the response can be categorized into three phases: 1) onset of the RAE/RCAE regime transition, 2) sea ice melt following the regime transition, and 3) perennially ice free stable state.
        \item Phase 1 is dominated by changes in radiative cooling, and phase 2 is associated with a decrease in MSE flux convergence into the Arctic.
        \item We show that the increase in radiative cooling associated with phase 1 is predominantly driven by clear-sky longwave cooling. This suggests that the RAE model \citep{cronin2016} may be able to predict its own demise (using the criterion that the surface inversion vanishes) purely through increasing the optical thickness (i.e., isolating the radiative component of $\Delta R_1$),
        \item quantify the decrease in MSE flux convergence into contributions from the mean meridional circulation, transient eddies, and stationary eddies,
        \item show that the decrease in MSE flux convergence is dominated by a decrease in transient eddy heat transport,
        \item further decompose this into DSE and latent energy transport as we expect there to be compensation between the two \citep{feldl2017},
        \item test whether the change in MSE flux convergence in the Arctic is consistent with a diffusive closure using a moist-diffusive EBM
        % \item test the hypothesis that melting sea ice is the key mechanism that controls the decrease in MSE flux convergence in the Arctic by using idealized climate models (aquaplanet and possibly a moist-diffusive EBM with sea ice) 
    \end{enumerate}
\end{itemize}

\section{Transient response of the Arctic energy budget}
\begin{itemize}
    \item CMIP5 multimodel mean of the extended RCP8.5 run (8 models) shows a wintertime regime transition over the Arctic (poleward of ~75$^\circ$N, Fig.~\ref{fig:r1-mon-lat-ext}). Interestingly, the region between 60--70$^\circ$N (dominated by land) remains in RAE.
    \item In MPI-ESM-LR, the regime transition is associated with the emergence of nonzero wintertime convective precipitation (Fig.~\ref{fig:r1-prc}) and the disappearance of the surface inversion (Fig.~\ref{fig:r1-gadev}). The near surface lapse rate stabilizes around year 125, suggesting that the lower troposphere lapse rate feedback may be weakening after this time. The connection between $R_1$, the lapse rate structure, and precipitation suggests that understanding the time evolution of $R_1$ may help us better understand the temperature and hydrologic cycle response.
    \item \textbf{TO DO}: make Fig.~\ref{fig:r1-prc} and \ref{fig:r1-gadev} for the CMIP5 multimodel mean.
\end{itemize}

\section{$R_1$ and MSE budget decomposition}
\begin{itemize}
    \item The high latitude (poleward of 80$^\circ$N) $R_1$ evolution closely follows the radiative component until the regime transition that occurs around year 2100 (Fig.~\ref{fig:decomp-ext}). The associated decrease in $R_1$ is consistent with stronger radiative cooling (Fig.~\ref{fig:flux-dev-ext}).
    \item In MPI-ESM-LR, the stronger radiative cooling is predominantly associated with clear-sky longwave cooling, suggesting that enhanced optical thickness from increased CO$_2$ and water vapor dominants the $R_a$ response (Fig.~\ref{fig:ra-lwcs}).
    \item \textbf{TO DO:} make Fig.~\ref{fig:ra-lwcs} for the CMIP5 multimodel mean.
\end{itemize}

\begin{figure}
    \centering
    \includegraphics[width=\textwidth]{{/project2/tas1/miyawaki/projects/003/plot/rcp85/mmm/200601-230012/mon_lat/r1_mon_lat.djfmean}.pdf}
    \caption{The wintertime (DJF) time evolution of $R_1$ for the CMIP5 multimodel mean of the extended RCP8.5 runs.}
    \label{fig:r1-mon-lat-ext}
\end{figure}

\begin{figure}
    \centering
    \includegraphics[width=\textwidth]{{/project2/tas1/miyawaki/projects/003/plot/rcp85/mmm/200601-230012/mon_hl/r1_mon_hl.80.90.djfmean.decomp}.pdf}
    \caption{The wintertime (DJF) time evolution of $R_1$ decomposed into the dynamic (red) and radiative (gray) components for the CMIP5 multimodel mean of the extended RCP8.5 runs.}
    \label{fig:decomp-ext}
\end{figure}

\begin{figure}
    \centering
    \includegraphics[width=\textwidth]{{/project2/tas1/miyawaki/projects/003/plot/rcp85/MPI-ESM-LR/200601-230012/mon_hl/r1_mon_hl.80.90.djfmean.prc}.pdf}
    \caption{The wintertime (DJF) time evolution of $R_1$ (black, left axis) and convective precipitation (blue, right axis) in the extended RCP8.5 run of MPI-ESM-LR.}
    \label{fig:r1-prc}
\end{figure}

\begin{figure}
    \centering
    \includegraphics[width=\textwidth]{{/project2/tas1/miyawaki/projects/003/plot/rcp85/MPI-ESM-LR/200601-230012/mon_hl/r1_mon_hl.80.90.djfmean.ga_dev.1.0.9}.pdf}
    \caption{The wintertime (DJF) time evolution of $R_1$ (black, left axis) and near surface lapse rate deviation from a moist adiabat (blue, right axis) in the extended RCP8.5 run of MPI-ESM-LR.}
    \label{fig:r1-gadev}
\end{figure}

\begin{figure}
    \centering
    \includegraphics[width=\textwidth]{{/project2/tas1/miyawaki/projects/003/plot/rcp85/mmm/200601-230012/mon_hl/flux_dev_mon_hl.80.90.djfmean}.pdf}
    \caption{The wintertime (DJF) energy flux changes relative to 2006 for the CMIP5 multimodel mean of the extended RCP8.5 runs.}
    \label{fig:flux-dev-ext}
\end{figure}

\begin{figure}
    \centering
    \includegraphics[width=\textwidth]{{/project2/tas1/miyawaki/projects/003/plot/rcp85/MPI-ESM-LR/200601-230012/mon_hl/rad_lwcs_dev_mon_hl.80.90.djfmean}.pdf}
    \caption{The wintertime (DJF) radiative cooling (gray) and the clear-sky longwave radiative cooling component (green) for the extended RCP8.5 run of MPI-ESM-LR.}
    \label{fig:ra-lwcs}
\end{figure}

\section{\textbf{TO DO:} Predict the transient evolution of the near surface inversion strength in the RAE model as a function of optical thickness}
\begin{itemize}
    \item To be able to compare the RAE prediction with the CMIP5 results, I need to find a relationship between CO$_2$ concentration and total optical thickness.
\end{itemize}

\section{IN PROGRESS: MSE flux divergence decomposition}
\begin{itemize}
    \item To decompose MSE flux divergence into contributions from the MMC, SE, and TE, I follow \citep{donohoe2020a} to compute the MSE transports using monthly frequency data. 
    \item This method involves computing the total atmospheric energy transport as the residual $$F_a = 2\pi a^2\int_{-\pi/2}^{\phi}\cos\phi^\prime (R_a + \mathrm{LH + SH} - \langle\partial_t m\rangle) \mathrm{d}\phi^\prime \, ,$$ mass-conserving MMC transport $[\overline{v}][\overline{m}]$ \citep[following][]{marshall2014}, and SE transport $[\overline{v}^*\overline{m}^*]$ using the commonly available monthly frequency data. Transient eddy transport is then inferred as the residual $$[\overline{v^{*\prime} m^{*\prime}}] = F_a - [\overline{v}][\overline{m}] - [\overline{v^*}\overline{m^*}]$$
    \item The results I computed for MPI-ESM-LR are generally comparable to those computed for CESM in \cite{donohoe2020a}, except for the noisy feature in the Southern Hemisphere high latitudes (Fig~\ref{fig:vm}). My current suspicion is that my method for calculating surface MSE (I currently evaluate the 3D MSE field at surface pressure by interpolating) is inaccurate. I will try instead computing surface MSE using 2 m temperature, humidity, and surface geopotential.
    \item To compute the MSE flux divergence, I take the derivative of the energy transported through a latitudinal band ($1/2\pi a^2 cos\phi \partial_\phi(\cdot)$). I perform the derivative using a central finite difference.
    \item The aforementioned noise issue in the SH high latitudes is compounded after taking the derivative (Fig.~\ref{fig:div}). Possibly because of this issue, there is a significant residual (dashed line), where the three components do not exactly add up to the total MSE flux divergence.
\end{itemize}

\begin{figure}
    \centering
    \includegraphics[width=\textwidth]{{/project2/tas1/miyawaki/projects/003/plot/rcp85/MPI-ESM-LR/200601-230012/lat/vm.ymonmean-30}.pdf}
    \caption{The annual mean total energy transport (black) decomposed into contributions from the meridional overturning circulation (MOC, green), stationary eddies (blue), and transient eddies (red) averaged over the last 30 years of the extended RCP8.5 run of MPI-ESM-LR.}
    \label{fig:vm}
\end{figure}

\begin{figure}
    \centering
    \includegraphics[width=\textwidth]{{/project2/tas1/miyawaki/projects/003/plot/rcp85/MPI-ESM-LR/200601-230012/lat/div.ymonmean-30}.pdf}
    \caption{Similar to Fig.~\ref{fig:vm} but for the energy flux divergences.}
    \label{fig:div}
\end{figure}

\section{TO DO: using a moist-diffusive EBM and keeping the diffusivity constant, predict the change in MSE flux convergence using the meridional MSE gradient}

\bibliographystyle{apalike}
\bibliography{../../../002/draft/references.bib}

\end{document}
