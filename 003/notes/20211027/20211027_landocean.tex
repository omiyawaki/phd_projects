\documentclass{article}

\usepackage{graphicx}
\usepackage[margin=1in]{geometry}
\usepackage{afterpage}
\usepackage{natbib}
\usepackage{amsmath,amssymb,amsfonts}

\title{Decomposing $R_1$ into land and ocean contributions}
\date{October 27, 2021}
\author{Osamu Miyawaki}

\begin{document}
\maketitle

\section*{Derivation 3}
Take the zonal mean of the MSE budget over the ocean and land separately:
\begin{align}\label{eq:mse-lo}
    [\partial_t m + \nabla\cdot (\mathbf{v}m)]_L &= [R_a]_L + [\mathrm{LH+SH}]_L \\
    [\partial_t m + \nabla\cdot (\mathbf{v}m)]_O &= [R_a]_O + [\mathrm{LH+SH}]_O 
\end{align}
where
\begin{align}\label{eq:loavg}
    [\cdot]_L &= \frac{1}{2\pi}\int_0^{2\pi} \! f\,(\cdot) \, \mathrm{d}\lambda \\
    [\cdot]_O &= \frac{1}{2\pi}\int_0^{2\pi} \! (1-f)\,(\cdot) \, \mathrm{d}\lambda 
\end{align}
and $f$ is the area fraction of land. Then, $R_1$ over land and ocean is defined as
\begin{align}\label{eq:r1-lo}
    \underbrace{\frac{[\partial_t m + \nabla\cdot (\mathbf{v}m)]_L}{[R_a]_L}}_{R_{1,L}^{**}} &= 1 + \underbrace{\frac{[\mathrm{LH+SH}]_L}{[R_a]_L}}_{R_{1,L}^{**}} \\
    \underbrace{\frac{[\partial_t m + \nabla\cdot (\mathbf{v}m)]_O}{[R_a]_O}}_{R_{1,O}^{**}} &= 1 + \underbrace{\frac{[\mathrm{LH+SH}]_O}{[R_a]_O}}_{R_{1,O}^{**}}
\end{align}

To make $R_{1,L}^{**}$ comparable to the previous two definitions of $R_1$, I scale $R_{1,L}^{**}$ by the land fraction $f$. This decomposition of $R_1$ leads to the same qualitative result as before: the summertime midlatitude RCE regime transition is a signal that occurs only over land (Fig.~\ref{fig:r1ss-mon-mid}).

\begin{figure}
    \centering
    \includegraphics[width=\textwidth]{{/project2/tas1/miyawaki/projects/002/figures/rea/1980_2005/1.00/r1_lo/r1ss_mon_mid}.png}
    \caption{The seasonality of the Northern midlatitude (40$^\circ$--60$^\circ$N) $R_1$ (solid black), $fR_{1,L}^{**}$ (green), $(1-f)R_{1,O}^{**}$ (blue), and the residual (dash-dot black) for the reanalysis mean.}
    \label{fig:r1ss-mon-mid}
\end{figure}

The $R_1$ seasonality over land as defined by $R_{1,L}$ and $R_{1,L}^**$ qualitatively agrees with the $R_1$ seasonality in the 3 m AQUA simulation scaled by the land fraction $f$ (Fig.~\ref{fig:r1-landcomp}). The scaled $R_1$ seasonality in AQUA exhibits a 1 month phase lag and wintertime $R_1$ is smaller than that in the reanalysis. The smaller wintertime $R_1$ in AQUA is consistent with the lack of zonal heat transport from the ocean to land during wintertime.

\begin{figure}
    \centering
    \includegraphics[width=\textwidth]{{/project2/tas1/miyawaki/projects/002/figures/rea/1980_2005/1.00/r1_lo/r1_landcomp_mon_mid}.png}
    \caption{A comparison of the seasonality of the Northern midlatitude (40$^\circ$--60$^\circ$N) $R_{1,L}$ (solid green), $R_{1,L}^*$ (dotted green), $fR_{1,L}^{**}$ (dashed green), and $fR_1$ for 3 m AQUA (solid black).}
    \label{fig:r1-landcomp}
\end{figure}

The $R_1$ seasonality over ocean cannot be captured by any AQUA simulation regardless of the definition of $R_1$ (Fig.~\ref{fig:r1-oceancomp}). The scaled $R_1$ seasonality in the 15 m AQUA simulation approximately corresponds the amplitude of $R_1$ over the Northern midlatitude ocean but exhibits the opposite phase. Again, this is likely due to the lack of zonal heat transport in AQUA. Computing the seasonality of zonal heat transport between the land and ocean domains in the reanalyses would be the logical next step.

\begin{figure}
    \centering
    \includegraphics[width=\textwidth]{{/project2/tas1/miyawaki/projects/002/figures/rea/1980_2005/1.00/r1_lo/r1_oceancomp_mon_mid}.png}
    \caption{A comparison of the seasonality of the Northern midlatitude (40$^\circ$--60$^\circ$N) $R_{1,O}$ (solid green), $R_{1,O}^*$ (dotted green), $(1-f)R_{1,O}^{**}$ (dashed green), and $(1-f)R_1$ for 15 m AQUA (solid black).}
    \label{fig:r1-oceancomp}
\end{figure}

% \section*{Derivation 2}
% Alternatively, $R_1$ may be defined at every grid point by first nondimensionalizing equation~(\ref{eq:mse}) by $R_a$ and then taking the zonal mean:
% \begin{equation}\label{eq:r1s}
%     \underbrace{\left[\frac{\partial_t m + \nabla\cdot (\mathbf{v}m)}{R_a}\right]}_{R_1^*} = 1 + \underbrace{\left[\frac{\mathrm{LH+SH}}{R_a}\right]}_{R_2^*}
% \end{equation}
% $R_1^*$ differs from $R_1$ in that the former includes the contribution from the covariance of the deviation of advection and radiative cooling from the zonal mean:
% \begin{equation}
%     R_1^* = \left[\frac{\partial_t m + \nabla\cdot(\mathbf{v}m)}{R_a}\right] = \frac{[\partial_t m + \nabla\cdot(\mathbf{v}m)]}{[R_a]} + \left[\frac{(\partial_t m + \nabla\cdot(\mathbf{v}m))^\prime}{(R_a)^\prime}\right] = R_1 + \left[\frac{(\partial_t m + \nabla\cdot(\mathbf{v}m))^\prime}{(R_a)^\prime}\right]
% \end{equation}
% The decomposition into land and ocean domains may be performed similarly for $R_1^*$:
% \begin{equation}\label{eq:r1slo}
%     \underbrace{\left[\frac{\partial_t m + \nabla\cdot (\mathbf{v}m)}{R_a}\right]_L}_{R_{1,L}^*} + \underbrace{\left[\frac{\partial_t m + \nabla\cdot (\mathbf{v}m)}{R_a}\right]_O}_{R_{1,O}^*} = 1 + \underbrace{\left[\frac{\mathrm{LH+SH}}{R_a}\right]_L}_{R_{2,L}^*} + \underbrace{\left[\frac{\mathrm{LH+SH}}{R_a}\right]_O}_{R_{2,O}^*} \, ,
% \end{equation}

% The land-ocean decomposition of the alternative definition $R_1^*$ leads to the same qualitative result that the summertime RCE regime transition in the Northern midlatitudes occurs only over land. However, the zonal covariance of advection and radiative cooling leads to a quantitative difference between $R_1$ and $R_1^*$, primarily over land. In particular, the covariance term leads to a larger seasonality in $R_1^*$ due to a smaller summertime $R_1^*$ over land (compare green lines in Fig.~\ref{fig:r1ss-mon-mid} and \ref{fig:r1-mon-mid}).

% \begin{figure}
%     \centering
%     \includegraphics[width=\textwidth]{{/project2/tas1/miyawaki/projects/002/figures/rea/1980_2005/1.00/r1_lo/r1ss_mon_mid}.png}
%     \caption{The seasonality of the Northern midlatitude (40$^\circ$--60$^\circ$N) $R_1^*$ (solid black), $R_{1,L}^*$ (green), $R_{1,O}^*$ (blue), and the residual (dash-dot black) for the reanalysis mean.}
%     \label{fig:r1ss-mon-mid}
% \end{figure}

% The zonal covariance between advection and radiative cooling appears to be associated with topography. For example, the zonal deviation in MSE flux divergence is locally larger in the presence of high topography (e.g., see Tibetan Plateau and the Rocky Mountains in Fig.~\ref{fig:dev-div-lat-lon}), where radiative cooling is weaker (i.e., a smaller negative quantity, see Fig.~\ref{fig:dev-ra-lat-lon}).

% \begin{figure}
%     \centering
%     \includegraphics[width=\textwidth]{{/project2/tas1/miyawaki/projects/002/figures/rea/1980_2005/1.00/flux/mse_old/lo/jja/dev_div_lat_lon}.png}
%     \caption{The longitudinal and latitudinal structure of summertime (JJA) deviation of atmospheric heat storage plus advection from the zonal mean for the reanalysis mean.}
%     \label{fig:dev-div-lat-lon}
% \end{figure}

% \begin{figure}
%     \centering
%     \includegraphics[width=\textwidth]{{/project2/tas1/miyawaki/projects/002/figures/rea/1980_2005/1.00/flux/mse_old/lo/jja/dev_ra_lat_lon}.png}
%     \caption{The longitudinal and latitudinal structure of summertime (JJA) deviation of radiative cooling from the zonal mean for the reanalysis mean.}
%     \label{fig:dev-ra-lat-lon}
% \end{figure}

% \begin{figure}
%     \centering
%     \includegraphics[width=\textwidth]{{/project2/tas1/miyawaki/projects/002/figures/rea/1980_2005/1.00/flux/mse_old/lo/jja/dev_stf_lat_lon}.png}
%     \caption{The longitudinal and latitudinal structure of summertime (JJA) deviation of surface turbulent fluxes from the zonal mean for the reanalysis mean.}
%     \label{fig:dev-stf-lat-lon}
% \end{figure}

% \bibliographystyle{apalike}
% \bibliography{./references.bib}

\end{document}
