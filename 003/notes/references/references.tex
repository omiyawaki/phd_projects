\documentclass{article}

\usepackage{graphicx}
\usepackage[margin=1in]{geometry}
\usepackage{afterpage}
\usepackage{natbib}

\title{Reference notes}
\author{Osamu Miyawaki}

\begin{document}
\maketitle

\begin{itemize}
\item \textbf{\cite{manabe1975}}: pioneering GCM simulation that show Arctic Amplification and more generally the latitude-height warming structure (see Fig.~4b).
\item \textbf{\cite{hansen1984}}: pioneering climate feedback decomposition study using a 1D RCE model based on one GCM vertical profiles that show lapse rate feeback makes an important contribution to the global climate sensitivity (see Fig.~6)
\item \textbf{\cite{colman2003}}: multimodel intercomparison showing the contribution of the lapse rate feedback on global climate sensitivity.
\item \textbf{\cite{bony2006}}: review paper trying to understand the intermodel spread of climate feedback contributions from a mechanistic point of view.
\item \textbf{\cite{bintanja2012}}: shows the importance of base-state inversions on the magnitude of high latitude surface warming. They demonstrate this by varying the strength of boundary layer vertical mixing in the EC-Earth GCM.
\item \textbf{\cite{feldl2013}}: quantifies the importance of nonlinear feedbacks and nonlocal (energy transport) effects on local (as opposed to global) climate sensitivity. Shows that moisture transport into the high latitudes contributes to polar amplification.
\item \textbf{\cite{pithan2014}}: 
\item \textbf{\cite{feldl2020}}: 
\end{itemize}

\bibliographystyle{apalike}
\bibliography{../../../002/draft/references.bib}

\end{document}

