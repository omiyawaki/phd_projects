\documentclass{article}

\usepackage{graphicx}
\usepackage[margin=1in]{geometry}
\usepackage{afterpage}
\usepackage{natbib}

\title{Reference notes}
\author{Osamu Miyawaki}

\begin{document}
\maketitle

\begin{itemize}
\item \textbf{\cite{manabe1975}}: pioneering GCM simulation that show Arctic Amplification and more generally the latitude-height warming structure (see Fig.~4b).
\item \textbf{\cite{hansen1984}}: pioneering climate feedback decomposition study using a 1D RCE model based on one GCM vertical profiles that show lapse rate feeback makes an important contribution to the global climate sensitivity (see Fig.~6)
\item \textbf{\cite{schneider1997}}: SSTs in the equatorial Eastern Pacific Ocean has a global influence. Constraining the SSTs in the cold tongue region in a slab ocean GCM with a doubling of CO2 doubling leads to significantly less global warming.
\item \textbf{\cite{colman2003}}: multimodel intercomparison showing the contribution of the lapse rate feedback on global climate sensitivity.
\item \textbf{\cite{alexeev2005}}: splits forcing into tropical ($<30^\circ$) and extratropical ($>30^\circ$) and show that tropical forcing leads to a uniform response due to atmospheric heat transport and the enhanced downwelling longwave effect from warming and enhanced moisture. Arctic amplification then sets in from extratropical forcing that remains concentrated in the high latitudes.
\item \textbf{\cite{caballero2005}}: atmospheric heat transport saturates at high global mean temperatures and weak meridional temperature gradients because of high static stability in a warm climate and poleward migration of storm tracks.
\item \textbf{\cite{bony2006}}: review paper trying to understand the intermodel spread of climate feedback contributions from a mechanistic point of view.
\item \textbf{\cite{francis2006}}:
\item \textbf{\cite{graversen2006}}: Arctic surface air temperature increases after a period of strong poleward heat transport at $60^\circ$N with a 5 day lag. While atmospheric heat transport likely contributes to AA, it is secondary to local feedbacks. ERA-40 trends show that atmospheric heat transport is strengthening during 1979--2001. 
\item \textbf{\cite{sorteberg2008}}:
\item \textbf{\cite{lu2009}}:
\item \textbf{\cite{graversen2009}}:
\item \textbf{\cite{deser2010}}:
\item \textbf{\cite{screen2010}}:
\item \textbf{\cite{screen2010a}}:
\item \textbf{\cite{drijfhout2012}}: 
\item \textbf{\cite{bintanja2012}}: shows the importance of base-state inversions on the magnitude of high latitude surface warming. They demonstrate this by varying the strength of boundary layer vertical mixing in the EC-Earth GCM.
\item \textbf{\cite{feldl2013}}: quantifies the importance of nonlinear feedbacks and nonlocal (energy transport) effects on local (as opposed to global) climate sensitivity. Shows that moisture transport into the high latitudes contributes to polar amplification.
\item \textbf{\cite{pithan2014}}: 
\item \textbf{\cite{baggett2015}}: 
\item \textbf{\cite{carmack2015}}: 
\item \textbf{\cite{park2015}}: 
\item \textbf{\cite{park2015a}}: 
\item \textbf{\cite{baggett2016}}: 
\item \textbf{\cite{blackport2016}}: 
\item \textbf{\cite{laine2016}}: 
\item \textbf{\cite{overland2016}}: 
\item \textbf{\cite{overland2016a}}: 
\item \textbf{\cite{kim2016}}:
\item \textbf{\cite{woods2016}}:
\item \textbf{\cite{baggett2017}}: 
\item \textbf{\cite{blackport2017}}: 
\item \textbf{\cite{kim2017}}:
\item \textbf{\cite{hegyi2017}}:
\item \textbf{\cite{koyama2017}}:
\item \textbf{\cite{lee2017}}:
\item \textbf{\cite{mccusker2017}}: 
\item \textbf{\cite{blackport2018}}: 
\item \textbf{\cite{boeke2018}}: uses the surface energy budget to demonstrate that the strength of local feedbacks (surface albedo in summer, surface turbulent fluxes in fall/winter, and the clear sky longwave feedback) are correlated with intermodel variability in AA. Remote feedbacks are anticorrelated with AA. Describes the physical picture of the seasonal feedback loop that is employed by \citep{hankel2021}.
\item \textbf{\cite{hay2018}}:
\item \textbf{\cite{hegyi2018}}:
\item \textbf{\cite{screen2018}}:
\item \textbf{\cite{taylor2018}}:
\item \textbf{\cite{zappa2018}}:
\item \textbf{\cite{blackport2019}}: 
\item \textbf{\cite{dai2019}}: sea ice is a necessary condition for Arctic Amplification (AA). AA only occurs over regions of significant sea ice loss. Warming is amplified after open ocean is exposed due to enhanced LW, SH, and LH heat release during winter when the ocean is warmer than the overlying atmosphere. AA slows after sea ice melts.
\item \textbf{\cite{overland2019}}: 
\item \textbf{\cite{screen2019}}: 
\item \textbf{\cite{blackport2020}}: 
\item \textbf{\cite{feldl2020}}: 
\item \textbf{\cite{previdi2020}}: AA occurs within the first 3 months of the abrupt4xCO2 experiment when sea ice loss is negligible (1--2\%); thus, although sea ice melt strongly contributes to AA, it is not necessary. This rapid AA response is largely due to the lapse rate feedback.
\item \textbf{\cite{blackport2021}}: 
\item \textbf{\cite{hankel2021}}: intermodel variability in abruptness of winter sea ice melt is correlated with the strength of the springtime surface albedo feedback, which warms the summer Arctic Ocean, inhibits winter sea ice growth, which is associated with enhanced fall/winter surface turbulent fluxes and clear sky longwave feedback. Winter convection is active in all but one model (IPSL) but it is not correlated with the abruptness of winter sea ice loss, suggesting that a convective cloud feedback \citep[e.g.,][]{abbot2008} is not important for the rate of wintertime sea ice melt.
\end{itemize}

\bibliographystyle{apalike}
\bibliography{../../../002/draft/references.bib}

\end{document}

