\documentclass{article}

\usepackage{graphicx}
\usepackage[margin=1in]{geometry}
\usepackage{afterpage}
\usepackage{natbib}

\title{Research notes}
\date{July 14, 2021}
\author{Osamu Miyawaki}

\begin{document}
\maketitle

Project title: The transient response of the Arctic energy budget and the zonal-mean circulation

\section*{Introduction}
\begin{itemize}
    \item The Arctic is expected to undergo large and rapid changes associated with sea ice melt in response to anthropogenic increases in CO$_2$ \citep{dai2019, feldl2020}.
    \begin{itemize}
        \item Largest surface warming is projected in the Arctic \citep[Arctic Amplification, e.g.][]{manabe1975, held1993}, especially during wintertime
        \item Large fractional increase in precipitation \citep{bintanja2014,siler2018}
    \end{itemize}
    \item Previous studies have only investigated the equilibrium response of the Arctic atmosphere to projected sea ice loss \citep{deser2010} or observed trends \citep{screen2013}.
    \item Understanding the transient response to projected future increases in CO$_2$ is important because 1) it tells us about the timing of when the projected changes will occur and 2) the Arctic atmospheric response is likely state-dependent (e.g., before, during, and after sea ice melt).
    \item Here, we will
    \begin{enumerate}
        \item diagnose the the transient response of the Arctic energy budget to increasing CO$_2$ across a hierarchy of climate models using the moist static energy budget and $R_1$,
        \item test whether the RAE model \citep{cronin2016} can predict its own demise (surface inversion vanishes) purely through increasing the optical thickness (i.e., isolating the radiative component of $\Delta R_1$),
        \item quantify the decrease in MSE flux convergence into contributions from the mean meridional circulation, transient eddies, and stationary eddies,
        \item test whether the change in MSE flux convergence in the Arctic is consistent with a diffusive closure,
        \item test the hypothesis that melting sea ice is the key mechanism that controls the decrease in MSE flux convergence in the Arctic by using idealized climate models (aquaplanet and possibly a moist-diffusive EBM with sea ice) 
    \end{enumerate}
\end{itemize}

\section{Transient response of the Arctic energy budget}
\begin{itemize}
    \item CMIP5 multimodel mean of the extended RCP8.5 run (8 models) shows a wintertime regime transition over the Arctic (Fig.~\ref{fig:r1-mon-lat-ext}).
    \item The high latitude (poleward of 80$^\circ$N) $R_1$ evolution closely follows the radiative component until the regime transition that occurs around year 2100 (Fig.~\ref{fig:decomp-ext}).
    \item We can use the RAE model to test whether the temperature response can be mostly explained by changing the optical thickness while keeping other parameters fixed.
\end{itemize}

\begin{figure}
    \centering
    \includegraphics[width=\textwidth]{{/project2/tas1/miyawaki/projects/003/plot/rcp85/mmm/200601-230012/mon_lat/r1_mon_lat.djfmean}.pdf}
    \caption{The wintertime (DJF) time evolution of $R_1$ for the CMIP5 multimodel mean of the extended RCP8.5 runs.}
    \label{fig:r1-mon-lat-ext}
\end{figure}

\begin{figure}
    \centering
    \includegraphics[width=\textwidth]{{/project2/tas1/miyawaki/projects/003/plot/rcp85/mmm/200601-230012/mon_hl/r1_mon_hl.80.90.djfmean.decomp}.pdf}
    \caption{The wintertime (DJF) time evolution of $R_1$ decomposed into the dynamic (red) and radiative (gray) components for the CMIP5 multimodel mean of the extended RCP8.5 runs.}
    \label{fig:decomp-ext}
\end{figure}

\begin{figure}
    \centering
    \includegraphics[width=\textwidth]{{/project2/tas1/miyawaki/projects/003/plot/rcp85/mmm/200601-230012/mon_hl/flux_dev_mon_hl.80.90.djfmean}.pdf}
    \caption{The wintertime (DJF) energy flux changes relative to 2006 for the CMIP5 multimodel mean of the extended RCP8.5 runs.}
    \label{fig:flux-dev-ext}
\end{figure}

\begin{figure}
    \centering
    \includegraphics[width=\textwidth]{{/project2/tas1/miyawaki/projects/003/plot/rcp85/mmm/200601-210012/mon_lat/r1_mon_lat.djfmean}.pdf}
    \caption{Same as Fig.~\ref{fig:r1-mon-lat-ext} but for the CMIP5 multimodel mean of the standard RCP8.5 runs (37 models, ends at 2100).}
    \label{fig:r1-mon-lat-std}
\end{figure}

\begin{figure}
    \centering
    \includegraphics[width=\textwidth]{{/project2/tas1/miyawaki/projects/003/plot/rcp85/mmm/200601-210012/mon_hl/r1_mon_hl.80.90.djfmean.decomp}.pdf}
    \caption{Same as Fig.~\ref{fig:decomp-ext} but for the CMIP5 multimodel mean of the standard RCP8.5 runs.}
    \label{fig:decomp-std}
\end{figure}

\begin{figure}
    \centering
    \includegraphics[width=\textwidth]{{/project2/tas1/miyawaki/projects/003/plot/rcp85/mmm/200601-210012/mon_hl/flux_dev_mon_hl.80.90.djfmean}.pdf}
    \caption{Same as Fig.~\ref{fig:decomp-dev-ext} but for the CMIP5 multimodel mean of the standard RCP8.5 runs.}
    \label{fig:flux-dev-std}
\end{figure}


\bibliographystyle{apalike}
\bibliography{../../../002/draft/references.bib}

\end{document}
