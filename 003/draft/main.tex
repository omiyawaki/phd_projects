%%%%%%%%%%%%%%%%%%%%%%%%%%%%%%%%%%%%%%%%%%%%%%%%%%%%%%%%%%%%%%%%%%%%%%%%%%%%
% AGUJournalTemplate.tex: this template file is for articles formatted with LaTeX
%
% This file includes commands and instructions
% given in the order necessary to produce a final output that will
% satisfy AGU requirements, including customized APA reference formatting.
%
% You may copy this file and give it your
% article name, and enter your text.
%
%
% Step 1: Set the \documentclass
%
%

%% To submit your paper:
\documentclass[draft]{agujournal2019}
\usepackage{mlmodern}
\usepackage{url} %this package should fix any errors with URLs in refs.
\usepackage{lineno}
\usepackage{amsmath,amssymb,amsfonts}
\usepackage[inline]{trackchanges} %for better track changes. finalnew option will compile document with changes incorporated.
\usepackage{soul}
\linenumbers
%%%%%%%
% As of 2018 we recommend use of the TrackChanges package to mark revisions.
% The trackchanges package adds five new LaTeX commands:
%
%  \note[editor]{The note}
%  \annote[editor]{Text to annotate}{The note}
%  \add[editor]{Text to add}
%  \remove[editor]{Text to remove}
%  \change[editor]{Text to remove}{Text to add}
%
% complete documentation is here: http://trackchanges.sourceforge.net/
%%%%%%%

\draftfalse

%% Enter journal name below.
%% Choose from this list of Journals:
%
% JGR: Atmospheres
% JGR: Biogeosciences
% JGR: Earth Surface
% JGR: Oceans
% JGR: Planets
% JGR: Solid Earth
% JGR: Space Physics
% Global Biogeochemical Cycles
% Geophysical Research Letters
% Paleoceanography and Paleoclimatology
% Radio Science
% Reviews of Geophysics
% Tectonics
% Space Weather
% Water Resources Research
% Geochemistry, Geophysics, Geosystems
% Journal of Advances in Modeling Earth Systems (JAMES)
% Earth's Future
% Earth and Space Science
% Geohealth
%
% ie, \journalname{Water Resources Research}

\journalname{Geophysical Research Letters}


\begin{document}

%% ------------------------------------------------------------------------ %%
%  Title
%
% (A title should be specific, informative, and brief. Use
% abbreviations only if they are defined in the abstract. Titles that
% start with general keywords then specific terms are optimized in
% searches)
%
%% ------------------------------------------------------------------------ %%

% Example: \title{This is a test title}

\title{DRAFT\\The transient emergence of a new wintertime Arctic energy balance regime}

%% ------------------------------------------------------------------------ %%
%
%  AUTHORS AND AFFILIATIONS
%
%% ------------------------------------------------------------------------ %%

% Authors are individuals who have significantly contributed to the
% research and preparation of the article. Group authors are allowed, if
% each author in the group is separately identified in an appendix.)

% List authors by first name or initial followed by last name and
% separated by commas. Use \affil{} to number affiliations, and
% \thanks{} for author notes.
% Additional author notes should be indicated with \thanks{} (for
% example, for current addresses).

% Example: \authors{A. B. Author\affil{1}\thanks{Current address, Antartica}, B. C. Author\affil{2,3}, and D. E.
% Author\affil{3,4}\thanks{Also funded by Monsanto.}}

\authors{O. Miyawaki\affil{1}, T. A. Shaw\affil{1}, M. F. Jansen\affil{1}}


\affiliation{1}{Department of the Geophysical Sciences, The University of Chicago}
% \affiliation{2}{Second Affiliation}
% \affiliation{3}{Third Affiliation}
% \affiliation{4}{Fourth Affiliation}

%(repeat as many times as is necessary)

%% Corresponding Author:
% Corresponding author mailing address and e-mail address:

% (include name and email addresses of the corresponding author.  More
% than one corresponding author is allowed in this LaTeX file and for
% publication; but only one corresponding author is allowed in our
% editorial system.)

% Example: \correspondingauthor{First and Last Name}{email@address.edu}

\correspondingauthor{Osamu Miyawaki}{miyawaki@uchicago.edu}

%% Keypoints, final entry on title page.

%  List up to three key points (at least one is required)
%  Key Points summarize the main points and conclusions of the article
%  Each must be 100 characters or less with no special characters or punctuation and must be complete sentences

% Example:
% \begin{keypoints}
% \item	List up to three key points (at least one is required)
% \item	Key Points summarize the main points and conclusions of the article
% \item	Each must be 100 characters or less with no special characters or punctuation and must be complete sentences
% \end{keypoints}

\begin{keypoints}
\item The Arctic is projected to undergo a wintertime energy balance regime transition that is associated with sea ice loss, vanishing surface temperature inversion, and emergence of convective activity.
\item The change in the Arctic energy balance is dominated by enhanced atmospheric radiative cooling prior to the regime transition and a decrease in MSE flux convergence thereafter.
\item The decrease in MSE flux convergence in the Arctic is consistent with decreased sensible heat transported by transient eddies. 
\end{keypoints}

%% ------------------------------------------------------------------------ %%
%
%  ABSTRACT and PLAIN LANGUAGE SUMMARY
%
% A good Abstract will begin with a short description of the problem
% being addressed, briefly describe the new data or analyses, then
% briefly states the main conclusion(s) and how they are supported and
% uncertainties.

% The Plain Language Summary should be written for a broad audience,
% including journalists and the science-interested public, that will not have 
% a background in your field.
%
% A Plain Language Summary is required in GRL, JGR: Planets, JGR: Biogeosciences,
% JGR: Oceans, G-Cubed, Reviews of Geophysics, and JAMES.
% see http://sharingscience.agu.org/creating-plain-language-summary/)
%
%% ------------------------------------------------------------------------ %%

%% \begin{abstract} starts the second page

\begin{abstract}
The energy balance in the Arctic atmosphere today involves atmospheric radiative cooling and advective heating, so-called Radiative Advective Equilibrium (RAE). Here we show that the transient response of Arctic climate change under the RCP8.5 scenario involves the emergence of a new energy balance regime characterized by radiative cooling, convective heating, and advective heating, so-called Radiative Convective Advective Equilibrium (RCAE). The timing of the RAE to RCAE regime transition coincides with wintertime Arctic sea ice loss, the disappearance of a surface inversion, and the emergence of convective precipitation. A transient decomposition of the moist static energy balance shows the regime transition is associated with the following mechanisms: 1) enhanced radiative cooling from 2050 to 2150 and 2) decreased advective heating from 2150 onward. We configure experiments in a single column model with interactive thermodynamic sea ice to quantify the importance of increased CO2 and water vapor for the enhanced radiative cooling response. We decompose the decreased advective heating response and show it is dominated by decreased dry static energy transport by transient eddies consistent with the reduced meridional temperature gradient. The results show that the transient climate change response of the Arctic involves multiple mechanisms, suggesting that historical trends likely do not reveal the full picture of the long-term response.
\end{abstract}

\section*{Plain Language Summary}
[ enter your Plain Language Summary here or delete this section]


%% ------------------------------------------------------------------------ %%
%
%  TEXT
%
%% ------------------------------------------------------------------------ %%

%%% Suggested section heads:
% \section{Introduction}
%
% The main text should start with an introduction. Except for short
% manuscripts (such as comments and replies), the text should be divided
% into sections, each with its own heading.

% Headings should be sentence fragments and do not begin with a
% lowercase letter or number. Examples of good headings are:

% \section{Materials and Methods}
% Here is text on Materials and Methods.
%
% \subsection{A descriptive heading about methods}
% More about Methods.
%
% \section{Data} (Or section title might be a descriptive heading about data)
%
% \section{Results} (Or section title might be a descriptive heading about the
% results)
%
% \section{Conclusions}


%%%%%%%%%%%%%%%%%%%%%%%%%%%%%%%%%%%%%%%%%%%%%%%%%%%%%%%%%
\section{Introduction}
\label{sec:int}
%%%%%%%%%%%%%%%%%%%%%%%%%%%%%%%%%%%%%%%%%%%%%%%%%%%%%%%%%
The modern Arctic climate in wintertime is characterized by sea ice cover, a strong surface temperature inversion, and the absence of convective activity \cite<e.g.,>[]{peixoto1992}. The wintertime Arctic is projected to undergo significant changes in response to anthropogenic forcing. The robust response of the wintertime Arctic is characterized by sea ice loss \cite{dai2019,hankel2021}, amplified surface warming \cite{manabe1975,bintanja2011}, and an enhanced hydrologic cycle \cite{bengtsson2011,bintanja2014,pithan2021}. Due to the large magnitude of the Arctic response and its associated transformation of the surface boundary condition (sea ice to bare ocean) and precipitation type (large-scale precipitation and snow to convective precipitation and rain), Arctic climate change has been described as an emergence of a new climate regime \cite{landrum2020}.

While the emergence of a new Arctic climate regime has been investigated for the response of sea ice loss, surface warming, and precipitation phase, we lack a complete understanding of the transient emergence of wintertime convection in the Arctic atmosphere. Previous studies have found that convection occurs across the hierarchy of model complexity in an ice-free wintertime Arctic \cite{huber1999, abbot2008, abbot2008a, abbot2009, arnold2014, hankel2021}. Quantifying and understanding the transient emergence of convection is important because its greenhouse effect has been hypothesized to be important for melting wintertime sea ice \cite{abbot2009} and maintaining warm surface temperatures in the ice-free Arctic during polar night \cite{abbot2009a,abbot2009b}.

Here, we propose the energy balance framework is useful for quantifying and understanding the mechanisms that influence the transient emergence of wintertime convection in the Arctic. In the energy balance framework, the modern Arctic is characterized by a state of idealized energy balance where net atmospheric radiative cooling is predominantly balanced by advective heating \cite{nakamura1988}, or Radiative Advective Equilibrium \cite<RAE,>[]{cronin2016}. In the quasi-equilibrium response to anthropogenic forcing, the wintertime Arctic is projected to transition to the so-called Radiative Convective Advective Equilibrium (RCAE) regime \cite{miyawaki2022} consistent with the emergence of convection. Thus the research questions that guide this work are:
\begin{enumerate}
    \item What is the transient response of the wintertime Arctic energy balance to anthropogenic forcing? How is the RAE to RCAE regime transition linked to sea ice, temperature, and convective activity?
    \item What mechanisms control the transient response of this regime transition?
\end{enumerate}

%%%%%%%%%%%%%%%%%%%%%%%%%%%%%%%%%%%%%%%%%%%%%%%%%%%%%%%%%
\section{Methods}
\label{sec:met}
\subsection{CMIP5 data}
We quantify the transient response of wintertime (DJF) Arctic climate change using the extended RCP8.5 runs of Coupled Model Intercomparison Project Phase 5 \cite<CMIP5,>[]{meinshausen2011,taylor2012}. We focus on the multimodel mean response of 7 models (Supplementary Table~1) following \citeA{hankel2021}. We quantify relative changes [$\Delta(\cdot)$] as the difference between the RCP8.5 run and the 1975--2005 climatology of the historical run [$\overline{(\cdot)}$].

\subsection{Energy balance regimes}
We quantify energy balance regimes using the nondimensional number $R_1$ \cite{miyawaki2022}. $R_1$ quantifies the ratio of advective heating to radiative cooling:
\begin{equation}
    R_1 = \frac{\langle \partial_t [m] + \partial_y [vm] \rangle}{\langle [R_{a}] \rangle}
\end{equation}
where $m=c_pT + gz + Lq$ is moist static energy, $v$ is meridional wind, $R_a$ is net atmospheric radiative cooling, LH is surface latent heat flux, SH is surface sensible heat flux, $[\cdot]$ is the zonal mean, and $\langle \cdot \rangle$ is the mass-weighted vertical integral. Following \cite{miyawaki2022}, $R_1$ is computed using monthly frequency LH, SH, and $R_a$. RAE is defined as where $R_1 \ge 0.9$, which corresponds to a state of atmospheric energy balance where advective heating is the predominant heat source that balances radiative cooling. We focus on $R_1$ in the Arctic, which we define as the area-weighted average of $R_1$ from $80^\circ$ to $90^\circ$N. We choose $80^\circ$N as the lower bound of the Arctic domain as it corresponds to the equatorward extent of the zonal mean RAE regime in the modern climate \cite<see Fig.~3a in>[]{miyawaki2022}.

\subsection{RRTMG}
We quantify the direct and indirect CO$_2$ effects on the transient radiative cooling response using Rapid Radiative Transfer Model for General Circulation Models \cite<RRTMG,>[]{mlawer1997,price2014}. Specifically, we use RRTMG included in CLIMLAB \cite{rose2018}. RRTMG is configured with zero insolation consistent with polar night in the wintertime Arctic. Ozone and well-mixed radiatively active gases aside from CO$_2$ are prescribed according to the Aquaplanet Experiment (APE) protocol \cite{blackburn2013}. We focus on the clear-sky radiative cooling response in RRTMG.

Clear-sky radiative cooling in RRTMG ($RR_a$) is then computed as a function of three variables: CO$_2$ concentration ($pCO_2$), vertical temperature profile ($T(p)$), and vertical vapor pressure profile ($e(p)$). The total clear-sky radiative cooling response in RRTMG (All) is then computed as
\begin{equation}
    \mathrm{All} = RR_a(pCO_2, T, e) - RR_a(\overline{pCO_2}, \overline{T}, \overline{e})\, ,
\end{equation}
where $pCO_2$, $T$, and $e$ correspond to the CMIP5 mean quantities from the RCP8.5 run. To quantify the direct effect of CO$_2$ (CO2), we vary $pCO_2$ following the RCP8.5 protocol while holding $T$ and $e$ fixed at the historical climatology.
\begin{equation}
    \mathrm{CO2} = RR_a(pCO_2, \overline{T}, \overline{e}) - RR_a(\overline{pCO_2}, \overline{T}, \overline{e})\, .
\end{equation}
The indirect effect of CO$_2$ consists of changes to radiative cooling associated with warming and moistening of the atmosphere. We quantify the warming contribution (T) by varying temperature and vapor pressure expected from the Clausius-Clapeyron relation (i.e. holding relative humidity, $RH$, fixed):
\begin{equation}
    \mathrm{T} = RR_a(\overline{pCO_2}, T, \overline{RH}e^{\ast}(T)) - RR_a(\overline{pCO_2}, \overline{T}, \overline{e})\, ,
\end{equation}
where $e^\ast$ is saturation vapor pressure. Finally we quantify the moistening contribution (RH) associated with vapor pressure changes that depart from Clausius-Clapeyron as follows:
\begin{equation}
    \mathrm{RH} = RR_a(\overline{pCO_2}, \overline{T}, RHe^{\ast}(\overline{T})) - RR_a(\overline{pCO_2}, \overline{T}, \overline{e})\, .
\end{equation}

\subsection{Aquaplanet experiments}
We run the ECHAM6 aquaplanet (AQUA) with and without thermodynamic sea ice to test the importance of sea ice loss on the transient response of Arctic energy balance regimes \cite{stevens2013, shaw2020, shaw2022}. Thermodynamic sea ice in AQUA is based on a zero-layer Semtner model \cite{semtner1976} and grid cells are either completely ice free or ice covered \cite{giorgetta2013,salameh2018}. AQUA with a 40 m mixed layer depth configured with ice (hereafter AQUAice) was previously shown to capture the observed wintertime Arctic sea ice thickness, energy balance regime, and inversion strength \cite{miyawaki2022}. Here, we initialize AQUAice from its control climate equilibrium ($pCO_2=348$ ppmv) and ramp up the CO$_2$ concentration following the RCP8.5 protocol starting from the year 1987 ($pCO_2=348.6$ ppmv).

To test the role of sea ice loss on the Arctic transient response, we configure AQUA with 40 m mixed layer and Q-flux but no sea ice (hereafter AQUAqflux). The imposed Q-flux is seasonally varying but does not vary year-by-year (Supplementary Fig.~1). The Q-flux is computed to reproduce the wintertime Arctic surface energy budget in AQUAice control climate as follows:

\textbf{OLD EQUATION} (doesn't account for albedo difference):
\begin{equation}
    Q_{\mathrm{flux}} = F_{SFC} - \frac{C_w}{C} \left[ F_{SFC} + \rho_iL_f\frac{\partial h}{\partial t}\right] \, ,
\end{equation}

\textbf{NEW EQUATION} (accounts for albedo difference):
\begin{equation}
    Q_{\mathrm{flux}} = \left[1-\alpha_{s,o}-\frac{C_W}{C}(1-\alpha_s)\right]SW^{\downarrow}_{SFC} + \left(1-\frac{C_w}{C}\right) \left( LW^{\text{net}}_{SFC} + LH + SH \right) - \frac{C_w}{C}\rho_iL_f\frac{\partial h}{\partial t}\, ,
\end{equation}
where $F_{SFC}$ is the net surface heat flux in AQUAice, $C_w$ is the surface heat capacity of the 40 m mixed layer ocean, $C$ is the effective surface heat capacity in AQUAice, $rho_i=905$ kg m$^{-3}$ is the density of ice, $L_f=3.337\times10^{5}$ J kg$^{-1}$ is the latent heat of fusion, and $h$ is sea ice thickness in AQUAice. $C$ is determined as follows:
\[
    C = \left\{\begin{array}{ll}
            C_w & \text{if sic}<1\text{ or }\partial_t(h)<0\\ 
            C_i & \text{else}
    \end{array}\right.
\]
where sic is the sea ice concentration. Thus the effective surface heat capacity in AQUAice is assumed to correspond to that of a 40 m mixed layer ocean in regions of bare ocean or melting ice and otherwise takes on the value $C_i$. We choose $C_i$ such that $C_w/C_i = 25$ to closely capture the wintertime Arctic climate in AQUAice. $C_w/C_i=25$ corresponds to an ice heat capacity that is equivalent to a 1.6 m deep mixed layer ocean.

%%%%%%%%%%%%%%%%%%%%%%%%%%%%%%%%%%%%%%%%%%%%%%%%%%%%%%%%%

%%%%%%%%%%%%%%%%%%%%%%%%%%%%%%%%%%%%%%%%%%%%%%%%%%%%%%%%%
\section{Results}
\label{sec:res}
\subsection{The transient energy balance response of the wintertime Arctic}
\label{sec:r1}
%%%%%%%%%%%%%%%%%%%%%%%%%%%%%%%%%%%%%%%%%%%%%%%%%%%%%%%%%
The modern Arctic atmosphere is in the RAE regime ($R_1=1.04$) and undergoes a regime transition to RCAE in the late 21st century (black line crosses to white region in Fig.~\ref{fig:r1}a). $R_1$ continues to decrease until 2200 and stabilizes in the RCAE regime corresponding to $R_1=0.7$.

The timing of the energy balance regime transition coincides closely with the disappearance of the surface temperature inversion as measured by the lapse rate deviation from a moist adiabat (blue line, Fig.~\ref{fig:r1}a). The modern Arctic is characterized by the existence of a strong inversion (near surface lapse rate deviation exceeds 100\%). The near surface lapse rate weakens until 2200 and stabilizes to the moist adiabatic lapse rate. Importantly, the timing of the inversion response is closely captured by $R_1$ as evidenced by the coincident timing of the RAE to RCAE regime transition and the disappearance of the inversion (where the deviation equals 100\%).

The $R_1$ response also coincides with the emergence of Arctic convection as measured by the convective precipitation fraction (blue line, Fig.~\ref{fig:r1}b). The modern Arctic is characterized by the absence of convection (convective precipitation fraction is 0). Convective precipitation increases until 2200 and stabilizes to 35\%, consistent with $R_1=0.7$ indicating that 30\% of radiative cooling is balanced by convective heating. The timing of the inversion response is also closely captured by $R_1$. Thus, the transient response of energy balance regimes is useful for understanding the disappearing inversion and the emerging convection in the Arctic.

\begin{figure}
    \centering
    \includegraphics[width=\textwidth]{{/project2/tas1/miyawaki/projects/003/plotmerge/fig_1/fig_1}.pdf}
    \caption{The response of wintertime (DJF) $R_1$ (a,b, black, left axis), near-surface lapse rate deviation from a moist adiabat (a, blue, right axis), and convective precipitation fraction (b, blue, right axis) for the CMIP5 multimodel mean of the extended RCP8.5 run. The shading indicates the multimodel spread (25th and 75th percentiles).}
    \label{fig:r1}
\end{figure}

%%%%%%%%%%%%%%%%%%%%%%%%%%%%%%%%%%%%%%%%%%%%%%%%%%%%%%%%%
\subsection{The radiative and advective phases of the regime transition}
%%%%%%%%%%%%%%%%%%%%%%%%%%%%%%%%%%%%%%%%%%%%%%%%%%%%%%%%%
\label{sec:dc}

To diagnose the possible physical mechanisms that control the $R_1$ response (Fig.~\ref{fig:r1}d), we decompose $\Delta R_1(t) = R_1(t) - \overline{R_1}$ into radiative and advective components following \citeA{miyawaki2022}:

\begin{equation} \label{eq:dc}
    \Delta R_1 = \overline{R_1}\left( \underbrace{ \frac{\Delta(\partial_t m + \partial_y (vm))}{\overline{\partial_t m + \partial_y (vm)}} }_{\mathrm{advective}} \; \underbrace{ - \frac{\Delta R_a }{\overline{R_a}} }_{\mathrm{radiative}} \right) + \mathrm{Residual} \, .
\end{equation}

The advective component [first term in Equation~(\ref{eq:dc})] quantifies the importance of the advective heating response and the radiative component [second term in Equation~(\ref{eq:dc})] quantifies the importance of the radiative cooling response. The residual quantifies the contribution of higher order terms.

The decomposition shows that there are two stages to the transient response of $R_1$: 1) the radiative phase prior to 2100 and 2) the advective phase after 2100 (Fig.~\ref{fig:r1-decomp}a). During the radiative phase, $\Delta{R_1}$ is predominantly associated with enhanced radiative cooling and the contribution from changes in advective heating is small. During the advective phase, $\Delta{R_1}$ is predominantly associated with reduced advective heating. The two phases are closely associated with the time prior to and after the RAE to RCAE regime transition (compare Fig.~\ref{fig:r1-decomp}a with Fig.~\ref{fig:r1}a). 

% The role of sea ice on the $R_1$ response can be understood through their connection with the surface turbulent fluxes. The presence of sea ice modulates the magnitude of the surface turbulent fluxes by insulating the atmosphere from the warmer ocean temperature below. In the Arctic, where $R_1\approx1$, $\Delta R_1\approx \Delta(\mathrm{LH+SH})/\overline{R_a}$ \cite<see Equations~5 and 6 in>[]{miyawaki2022}. Thus, the $R_1$ response can be understood to first order by the surface turbulent flux response (Fig.~\ref{fig:r1-decomp}c), which is consistent with the transient response of sea ice melt.  

\begin{figure}
    \centering
    \includegraphics[width=\textwidth]{{/project2/tas1/miyawaki/projects/003/plotmerge/fig_2/fig_2}.pdf}
    \caption{The wintertime (DJF) transient response (relative to the 1975--2005 historical mean) of (a) $R_1$ (solid black) decomposed into the advective (red) and radiative (gray) components and the residual (dash-dot black) for the CMIP5 multimodel mean of the extended RCP8.5 runs. (b) The smoothed (80-year rolling mean) time rate of change of each term in (a). (c) $\Delta R_1$ (black) compared with the approximation based solely on surface turbulent flux response (blue).}
    \label{fig:r1-decomp}
\end{figure}

%%%%%%%%%%%%%%%%%%%%%%%%%%%%%%%%%%%%%%%%%%%%%%%%%%%%%%%%%
\subsection{Quantifying the direct and indirect CO$_2$ effects on the radiative cooling response}
%%%%%%%%%%%%%%%%%%%%%%%%%%%%%%%%%%%%%%%%%%%%%%%%%%%%%%%%%
\label{sec:rap}

\begin{figure}
    \centering
    \includegraphics[width=\textwidth]{{/project2/tas1/miyawaki/projects/003/plotmerge/fig_3/fig_3}.pdf}
    \caption{(a) Wintertime (DJF) radiative cooling response (gray) decomposed into shortwave (cyan) and longwave clear- (dashed red) and cloudy-sky (dotted red) fluxes for the CMIP5 multimodel mean and the RRTMG clear-sky response (black). (b) The RRTMG clear-sky radiative cooling response is further decomposed into changes associated with the direct CO$_2$ effect (green) and indirect effects (i.e. warming and moistening following Clausius Clapeyron, dashed blue).}
    \label{fig:ra-lwcs}
\end{figure}

%%%%%%%%%%%%%%%%%%%%%%%%%%%%%%%%%%%%%%%%%%%%%%%%%%%%%%%%%
\subsection{Testing the importance of sea ice melt on the advective heating response}
%%%%%%%%%%%%%%%%%%%%%%%%%%%%%%%%%%%%%%%%%%%%%%%%%%%%%%%%%
\label{sec:adp}
% \begin{itemize}
%     \item As a first step toward understanding the mechanism for the advectively-driven change in $\Delta R_1$, we decompose the change in MSE flux divergence into stationary and transient components:
%     \begin{equation}
%         \langle \partial_y(vm) \rangle = \langle \partial_y(\overline{v}\,\overline{m}) \rangle + \langle \partial_y(\overline{v^\prime m^\prime}) \rangle
%     \end{equation}
%     \item Is the result consistent with the existing literature that show the change in meridional energy transport in the extratropics is predominantly due to transient eddies \cite{feldl2021}?
%     \item While there are studies that show the change in energy transport, the change in MSE flux divergence has not (to my knowledge) been decomposed in this way.
%         \item To do: Can a moist-diffusive EBM capture the advectively-driven change in $\Delta R_1$?
% \end{itemize}

\begin{figure}
    \centering
    \includegraphics[width=\textwidth]{{/project2/tas1/miyawaki/projects/003/plotmerge/fig_4/fig_4}.pdf}
    \caption{(a) The wintertime (DJF) MSE flux convergence (maroon) decomposed into contributions due to stationary (blue) and transient (red) transport for the CMIP5 multimodel mean of the extended RCP8.5 runs. (b) Change in MSE flux convergence due to stationary and transient transport are further decomposed into change in DSE (dashed) and latent energy (dotted) flux convergence.}
    \label{fig:dyn-decomp}
\end{figure}


%%%%%%%%%%%%%%%%%%%%%%%%%%%%%%%%%%%%%%%%%%%%%%%%%%%%%%%%%
\section{Summary and Discussion}
%%%%%%%%%%%%%%%%%%%%%%%%%%%%%%%%%%%%%%%%%%%%%%%%%%%%%%%%%
\label{sec:end}

%Text here ===>>>


%%

%  Numbered lines in equations:
%  To add line numbers to lines in equations,
%  \begin{linenomath*}
%  \begin{equation}
%  \end{equation}
%  \end{linenomath*}



%% Enter Figures and Tables near as possible to where they are first mentioned:
%
% DO NOT USE \psfrag or \subfigure commands.
%
% Figure captions go below the figure.
% Table titles go above tables;  other caption information
%  should be placed in last line of the table, using
% \multicolumn2l{$^a$ This is a table note.}
%
%----------------
% EXAMPLE FIGURES
%
% \begin{figure}
% \includegraphics{example.png}
% \caption{caption}
% \end{figure}
%
% Giving latex a width will help it to scale the figure properly. A simple trick is to use \textwidth. Try this if large figures run off the side of the page.
% \begin{figure}
% \noindent\includegraphics[width=\textwidth]{anothersample.png}
%\caption{caption}
%\label{pngfiguresample}
%\end{figure}
%
%
% If you get an error about an unknown bounding box, try specifying the width and height of the figure with the natwidth and natheight options. This is common when trying to add a PDF figure without pdflatex.
% \begin{figure}
% \noindent\includegraphics[natwidth=800px,natheight=600px]{samplefigure.pdf}
%\caption{caption}
%\label{pdffiguresample}
%\end{figure}
%
%
% PDFLatex does not seem to be able to process EPS figures. You may want to try the epstopdf package.
%

%
% ---------------
% EXAMPLE TABLE
%
% \begin{table}
% \caption{Time of the Transition Between Phase 1 and Phase 2$^{a}$}
% \centering
% \begin{tabular}{l c}
% \hline
%  Run  & Time (min)  \\
% \hline
%   $l1$  & 260   \\
%   $l2$  & 300   \\
%   $l3$  & 340   \\
%   $h1$  & 270   \\
%   $h2$  & 250   \\
%   $h3$  & 380   \\
%   $r1$  & 370   \\
%   $r2$  & 390   \\
% \hline
% \multicolumn{2}{l}{$^{a}$Footnote text here.}
% \end{tabular}
% \end{table}

%% SIDEWAYS FIGURE and TABLE
% AGU prefers the use of {sidewaystable} over {landscapetable} as it causes fewer problems.
%
% \begin{sidewaysfigure}
% \includegraphics[width=20pc]{figsamp}
% \caption{caption here}
% \label{newfig}
% \end{sidewaysfigure}
%
%  \begin{sidewaystable}
%  \caption{Caption here}
% \label{tab:signif_gap_clos}
%  \begin{tabular}{ccc}
% one&two&three\\
% four&five&six
%  \end{tabular}
%  \end{sidewaystable}

%% If using numbered lines, please surround equations with \begin{linenomath*}...\end{linenomath*}
%\begin{linenomath*}
%\begin{equation}
%y|{f} \sim g(m, \sigma),
%\end{equation}
%\end{linenomath*}

%%% End of body of article

%%%%%%%%%%%%%%%%%%%%%%%%%%%%%%%%
%% Optional Appendix goes here
%
% The \appendix command resets counters and redefines section heads
%
% After typing \appendix
%
%\section{Here Is Appendix Title}
% will show
% A: Here Is Appendix Title
%
%\appendix
%\section{Here is a sample appendix}

%%%%%%%%%%%%%%%%%%%%%%%%%%%%%%%%%%%%%%%%%%%%%%%%%%%%%%%%%%%%%%%%
%
% Optional Glossary, Notation or Acronym section goes here:
%
%%%%%%%%%%%%%%
% Glossary is only allowed in Reviews of Geophysics
%  \begin{glossary}
%  \term{Term}
%   Term Definition here
%  \term{Term}
%   Term Definition here
%  \term{Term}
%   Term Definition here
%  \end{glossary}

%
%%%%%%%%%%%%%%
% Acronyms
%   \begin{acronyms}
%   \acro{Acronym}
%   Definition here
%   \acro{EMOS}
%   Ensemble model output statistics
%   \acro{ECMWF}
%   Centre for Medium-Range Weather Forecasts
%   \end{acronyms}

%
%%%%%%%%%%%%%%
% Notation
%   \begin{notation}
%   \notation{$a+b$} Notation Definition here
%   \notation{$e=mc^2$}
%   Equation in German-born physicist Albert Einstein's theory of special
%  relativity that showed that the increased relativistic mass ($m$) of a
%  body comes from the energy of motion of the body—that is, its kinetic
%  energy ($E$)—divided by the speed of light squared ($c^2$).
%   \end{notation}




%%%%%%%%%%%%%%%%%%%%%%%%%%%%%%%%%%%%%%%%%%%%%%%%%%%%%%%%%%%%%%%%
%
%  ACKNOWLEDGMENTS
%
% The acknowledgments must list:
%
% >>>>	A statement that indicates to the reader where the data
% 	supporting the conclusions can be obtained (for example, in the
% 	references, tables, supporting information, and other databases).
%
% 	All funding sources related to this work from all authors
%
% 	Any real or perceived financial conflicts of interests for any
%	author
%
% 	Other affiliations for any author that may be perceived as
% 	having a conflict of interest with respect to the results of this
% 	paper.
%
%
% It is also the appropriate place to thank colleagues and other contributors.
% AGU does not normally allow dedications.


\acknowledgments
Enter acknowledgments, including your data availability statement, here.


%% ------------------------------------------------------------------------ %%
%% References and Citations

%%%%%%%%%%%%%%%%%%%%%%%%%%%%%%%%%%%%%%%%%%%%%%%
%
% \bibliography{<name of your .bib file>} don't specify the file extension
%
% don't specify bibliographystyle
%%%%%%%%%%%%%%%%%%%%%%%%%%%%%%%%%%%%%%%%%%%%%%%

% \bibliography{/project2/tas1/miyawaki/projects/002/draft/references.bib}
\bibliography{/project2/tas1/miyawaki/projects/003/draft/references}

%Reference citation instructions and examples:
%
% Please use ONLY \cite and \citeA for reference citations.
% \cite for parenthetical references
% ...as shown in recent studies (Simpson et al., 2019)
% \citeA for in-text citations
% ...Simpson et al. (2019) have shown...
%
%
%...as shown by \citeA{jskilby}.
%...as shown by \citeA{lewin76}, \citeA{carson86}, \citeA{bartoldy02}, and \citeA{rinaldi03}.
%...has been shown \citeAjskilbye}.
%...has been shown \citeAlewin76,carson86,bartoldy02,rinaldi03}.
%... \cite <i.e.>[]{lewin76,carson86,bartoldy02,rinaldi03}.
%...has been shown by \cite <e.g.,>[and others]{lewin76}.
%
% apacite uses < > for prenotes and [ ] for postnotes
% DO NOT use other cite commands (e.g., \citet, \cite, \citeyear, \nocite, \citealp, etc.).
%



\end{document}



More Information and Advice:

%% ------------------------------------------------------------------------ %%
%
%  SECTION HEADS
%
%% ------------------------------------------------------------------------ %%

% Capitalize the first letter of each word (except for
% prepositions, conjunctions, and articles that are
% three or fewer letters).

% AGU follows standard outline style; therefore, there cannot be a section 1 without
% a section 2, or a section 2.3.1 without a section 2.3.2.
% Please make sure your section numbers are balanced.
% ---------------
% Level 1 head
%
% Use the \section{} command to identify level 1 heads;
% type the appropriate head wording between the curly
% brackets, as shown below.
%
%An example:
%\section{Level 1 Head: Introduction}
%
% ---------------
% Level 2 head
%
% Use the \subsection{} command to identify level 2 heads.
%An example:
%\subsection{Level 2 Head}
%
% ---------------
% Level 3 head
%
% Use the \subsubsection{} command to identify level 3 heads
%An example:
%\subsubsection{Level 3 Head}
%
%---------------
% Level 4 head
%
% Use the \subsubsubsection{} command to identify level 3 heads
% An example:
%\subsubsubsection{Level 4 Head} An example.
%
%% ------------------------------------------------------------------------ %%
%
%  IN-TEXT LISTS
%
%% ------------------------------------------------------------------------ %%
%
% Do not use bulleted lists; enumerated lists are okay.
% \begin{enumerate}
% \item
% \item
% \item
% \end{enumerate}
%
%% ------------------------------------------------------------------------ %%
%
%  EQUATIONS
%
%% ------------------------------------------------------------------------ %%

% Single-line equations are centered.
% Equation arrays will appear left-aligned.

Math coded inside display math mode \[ ...\]
 will not be numbered, e.g.,:
 \[ x^2=y^2 + z^2\]

 Math coded inside \begin{equation} and \end{equation} will
 be automatically numbered, e.g.,:
 \begin{equation}
 x^2=y^2 + z^2
 \end{equation}


% To create multiline equations, use the
% \begin{eqnarray} and \end{eqnarray} environment
% as demonstrated below.
\begin{eqnarray}
  x_{1} & = & (x - x_{0}) \cos \Theta \nonumber \\
        && + (y - y_{0}) \sin \Theta  \nonumber \\
  y_{1} & = & -(x - x_{0}) \sin \Theta \nonumber \\
        && + (y - y_{0}) \cos \Theta.
\end{eqnarray}

%If you don't want an equation number, use the star form:
%\begin{eqnarray*}...\end{eqnarray*}

% Break each line at a sign of operation
% (+, -, etc.) if possible, with the sign of operation
% on the new line.

% Indent second and subsequent lines to align with
% the first character following the equal sign on the
% first line.

% Use an \hspace{} command to insert horizontal space
% into your equation if necessary. Place an appropriate
% unit of measure between the curly braces, e.g.
% \hspace{1in}; you may have to experiment to achieve
% the correct amount of space.


%% ------------------------------------------------------------------------ %%
%
%  EQUATION NUMBERING: COUNTER
%
%% ------------------------------------------------------------------------ %%

% You may change equation numbering by resetting
% the equation counter or by explicitly numbering
% an equation.

% To explicitly number an equation, type \eqnum{}
% (with the desired number between the brackets)
% after the \begin{equation} or \begin{eqnarray}
% command.  The \eqnum{} command will affect only
% the equation it appears with; LaTeX will number
% any equations appearing later in the manuscript
% according to the equation counter.
%

% If you have a multiline equation that needs only
% one equation number, use a \nonumber command in
% front of the double backslashes (\\) as shown in
% the multiline equation above.

% If you are using line numbers, remember to surround
% equations with \begin{linenomath*}...\end{linenomath*}

%  To add line numbers to lines in equations:
%  \begin{linenomath*}
%  \begin{equation}
%  \end{equation}
%  \end{linenomath*}



