%%%%%%%%%%%%%%%%%%%%%%%%%%%%%%%%%%%%%%%%%%%%%%%%%%%%%%%%%%%%%%%%%%%%%%%%%%%%

%% To submit your paper:
\documentclass[draft]{agujournal2019}
\usepackage{mlmodern}
\usepackage{url} %this package should fix any errors with URLs in refs.
\usepackage{lineno}
\usepackage{amsmath,amssymb,amsfonts}
\usepackage[inline]{trackchanges} %for better track changes. finalnew option will compile document with changes incorporated.
\usepackage{soul}
\linenumbers

\draftfalse

\journalname{Environmental Research: Climate}


\begin{document}

\title{The emergence of a new wintertime Arctic energy balance regime}

\authors{O. Miyawaki\affil{1}, T. A. Shaw\affil{2}, M. F. Jansen\affil{2}}


\affiliation{1}{Climate and Global Dynamics Laboratory, National Center for Atmospheric Research}
\affiliation{2}{Department of the Geophysical Sciences, The University of Chicago}

\correspondingauthor{Osamu Miyawaki}{miyawaki@ucar.edu}

\begin{keypoints}
\item Wintertime Arctic sea-ice loss, vanishing inversion, and emergence of convective precipitation is tightly coupled to the emergence of a new Arctic energy balance regime (RAE to RCAE).
\item The time-dependent energy balance response is dominated by enhanced atmospheric radiative cooling prior to the regime transition and reduced advective heating thereafter.
\item Sea-ice loss is a necessary condition for capturing both radiative and advective heating responses.
\end{keypoints}

\begin{abstract}
The modern Arctic climate during wintertime is characterized by sea-ice cover, a strong surface temperature inversion, and the absence of convection. Correspondingly, the energy balance in the Arctic atmosphere today is dominated by atmospheric radiative cooling and advective heating, so-called Radiative Advective Equilibrium (RAE). Climate change in the Arctic involves sea-ice loss, vanishing of the surface inversion, and emergence of convective precipitation. Here we show climate change in the Arctic involves the emergence of a new energy balance regime characterized by radiative cooling, convective heating, and advective heating, so-called Radiative Convective Advective Equilibrium (RCAE). A time-dependent decomposition of the atmospheric energy balance shows the regime transition is associated with enhanced radiative cooling followed by decreased advective heating. The radiative cooling response consists of a lasting and robust clear-sky greenhouse effect and a transient cloud contribution that varies across models. Mechanism-denial experiments in an aquaplanet with and without interactive sea ice highlights the essential role of sea-ice loss on the radiative cooling and advective heating responses. The results show that climate change in the Arctic involves temporally evolving mechanisms, suggesting that an emergent constraint based on historical data or trends may not constrain the long-term response.
\end{abstract}

\section*{Plain Language Summary}
The Arctic is projected to undergo significant changes in response to anthropogenic emissions. Long-term changes (up to year 2200) during wintertime include complete sea-ice loss, surface amplified warming that eliminates the temperature inversion, and the emergence of convective precipitation that are absent in the modern Arctic climate. These changes can be understood in terms of a shift in the types of energy fluxes (radiation, convection, and advection) that maintain energy balance of the atmosphere. Arctic climate change is driven by the enhanced greenhouse effect of water vapor and clouds prior to the regime transition followed by a decrease in the energy transported poleward by large-scale atmospheric motions. The timing of this transition corresponds to when Arctic sea ice is projected to melt. Experiments in an idealized climate model with and without sea-ice loss demonstrate the critical importance of Arctic sea ice on the time-dependent response of Arctic energy balance, warming, and precipitation type to anthropogenic emissions.

%%%%%%%%%%%%%%%%%%%%%%%%%%%%%%%%%%%%%%%%%%%%%%%%%%%%%%%%%
\section{Introduction}
\label{sec:int}
%%%%%%%%%%%%%%%%%%%%%%%%%%%%%%%%%%%%%%%%%%%%%%%%%%%%%%%%%
The modern Arctic climate in wintertime is characterized by sea-ice cover, a strong surface temperature inversion, and the absence of convective activity \cite<e.g.,>[]{hartmann2016}. The modern Arctic is also characterized by a state of energy balance where net atmospheric radiative cooling is predominantly balanced by advective heating, so called Radiative Advective Equilibrium \cite<RAE,>[]{nakamura1988a,cronin2016,miyawaki2022}.

The wintertime Arctic is projected to undergo significant changes in response to anthropogenic forcing by the end of the century. Climate models project sea-ice loss \cite{dai2019,hankel2021}, amplified surface warming \cite{manabe1975,bintanja2011,vallis2015}, enhanced hydrological cycle \cite{bengtsson2011,bintanja2014,pithan2021}, vanishing surface inversion \cite{bintanja2011,ruman2022}, and emergence of convection \cite{huber1999,arnold2014,hankel2021}. Due to the large magnitude of its response, Arctic climate change has been described as an emergence of a new climate regime \cite{landrum2020}. \citeA{miyawaki2022} showed the end-of-century Arctic climate exhibits a regime transition from RAE to a regime where radiative cooling is balanced by both advective heating and convective heating, so called Radiative Convective Advective Equilibrium (RCAE).

Two mechanisms have been proposed in the literature to control the Arctic climate change response. The first mechanism is sea-ice loss, which controls both the sea-ice albedo and lapse rate feedback. These positive feedbacks amplify surface warming and induce further sea-ice loss \cite{pithan2014,feldl2020}. Sea-ice loss is closely coupled to energy balance regimes as surface latent and sensible heating are enhanced over an exposed open ocean with warmer temperatures \cite{taylor2018,feldl2020,shaw2022}. The associated Arctic amplification and the decrease in the meridional gradient of moist static energy (MSE) is consistent with the projected decrease in advective heating into the Arctic \cite{armour2019,feldl2021,shaw2022,cardinale2023}. The second mechanism is related to radiative changes associated with increased CO$_2$ and water vapor. Atmospheric radiative cooling increases over the Arctic under climate change \cite{bintanja2011} and has been hypothesized to energetically constrain Arctic precipitation \cite{pithan2021}. It is important to compare and understand these mechanisms because they can potentially be used to constrain the climate change response using an emergent constraints approach \cite{klein2015}.

Diagnosing energy balance regimes is a new framework that can quantify the relative importance of different mechanisms (e.g., radiative cooling and advective heating responses) for both equilibrium and transient climate change. The regimes are defined using the metric $R_1$, which is the ratio of vertically-integrated advective heating and atmospheric energy storage to radiative cooling. Energy balance regimes were previously shown to be useful for understanding the seasonal and latitudinal structure of tropospheric lapse rates as well as their equilibrium warming response \cite{miyawaki2022}.

Here, we investigate the mechanisms underlying the time-dependent climate regime transition during the wintertime Arctic in CMIP6 models. We first demonstrate the usefulness of the energy balance framework as a way to quantify and understand Arctic climate change. We then use the framework to quantify the role of changes in atmospheric radiative cooling and advective heating. We use idealized models to further understand how sea ice influences the radiative cooling and advective heating responses. Lastly, we summarize and discuss our results.

%%%%%%%%%%%%%%%%%%%%%%%%%%%%%%%%%%%%%%%%%%%%%%%%%%%%%%%%%
\section{Methods}
\label{sec:met}
\subsection{CMIP6 data}
We quantify the time-dependent response of wintertime (DJF) Arctic climate change using the extended SSP585 runs of Coupled Model Intercomparison Project Phase 6 \cite<CMIP6,>[]{oneill2016,meinshausen2020}. We focus on the multimodel mean response of 10 models (Table~S1). We quantify relative changes (denoted by $\Delta(\cdot)$) as the difference between the SSP585 run and the 1984--2014 climatology of the historical run (denoted by $\overline{(\cdot)}$).

\subsection{Energy balance regimes}
We quantify energy balance regimes using the ratio of advective heating to radiative cooling defined as $R_1$ \cite{miyawaki2022}:
\begin{equation}\label{eq:defr1}
    R_1 = \frac{\langle \partial_t [m] + \partial_y [vm] \rangle}{[R_{a}]} = 1 + \frac{[LH]+[SH]}{[R_{a}]}
\end{equation}
where $m=c_pT + gz + Lq$ is moist static energy, $v$ is meridional wind, $R_a$ is net atmospheric radiative cooling, LH is surface latent heat flux, SH is surface sensible heat flux, $[\cdot]$ is the zonal mean, and $\langle \cdot \rangle$ is the mass-weighted vertical integral. Following \citeA{miyawaki2022}, $R_1$ is computed using monthly frequency LH, SH, and $R_a$. RAE is defined as where $R_1 \ge 0.9$, which corresponds to a state of atmospheric energy balance where advective heating balances radiative cooling. RCAE is defined for $0.1<R_1<0.9$. We focus on $R_1$ in the Arctic, which we define as the area-weighted average of $R_1$ from $80^\circ$ to $90^\circ$N. We choose $80^\circ$N as the lower bound of the Arctic domain as it corresponds to the equatorward extent of the zonal-mean RAE regime in the modern climate \cite<see Fig.~3a in>[]{miyawaki2022}. Using an alternative definition of $R_1$ that excludes the storage term leads to the same qualitative results (compare solid and dashed black lines Fig.~S1).

\subsection{Decomposing the radiative cooling response using an offline radiative transfer model}
We quantify the mechanisms controlling the time-dependent clear-sky radiative cooling response using the Rapid Radiative Transfer Model for General Circulation Models \cite<RRTMG,>[]{mlawer1997,price2014}. Specifically, we use RRTMG included in the Climlab Python package \cite{rose2018}. RRTMG is configured with zero insolation consistent with polar night in the wintertime Arctic. Ozone and well-mixed radiatively active gases aside from CO$_2$ are prescribed according to the Aquaplanet Experiment protocol \cite{blackburn2013}. %We focus on the clear-sky radiative cooling response in RRTMG because the sign of its response is robust across all models (see Supporting Information) and dominates the full radiative cooling response in the multimodel mean (Fig.~\ref{fig:ra-lwcs}).

Clear-sky radiative cooling in RRTMG is computed as a function of three variables:
\begin{equation}
    R_a = R_a(CO_2, T, q)\, ,
\end{equation}
where $CO_2(t)$ is CO$_2$ concentration, $T(t,p)$ is the vertical temperature profile averaged from $80^\circ$ to $90^\circ$N, and $q(t,p)=RHq^\ast$ is the vertical specific humidity profile averaged from $80^\circ$ to $90^\circ$N, where $RH$ is relative humidity, $q^\ast(T)$ is saturation specific humidity, $t$ is time (in yearly DJF-mean increments) and $p$ is pressure.

We follow \citeA{henry2021} and decompose the total radiative cooling response in RRTMG into contributions from 1) the direct CO$_2$ effect, 2) warming effect, and 3) relative humidity effect as follows:
\begin{equation} \label{eq:dra-di}
    \Delta R_{a} = \underbrace{\Delta R_{a}(\Delta CO_2, 0,0)}_{\text{Direct CO}_2\text{ effect}} + \underbrace{\Delta R_{a}(0,\Delta T,\,\overline{RH}\Delta q^\ast)}_{\text{Warming effect}} + \underbrace{\Delta R_a(0,0,\Delta RH \overline{q^\ast})}_{\text{Relative humidity effect}} + \mathrm{ Residual} \, .
\end{equation}

We decompose the water vapor contribution to radiative cooling into temperature-dependent and relative humidity-dependent changes following the convention used to compute the water vapor feedback \cite{held2012,ingram2013,jeevanjee2021}.

To quantify the direct effect of increased CO$_2$, we vary CO$_2$ following the SSP585 scenario while holding $T$ and $q$ fixed at the historical climatology:
\begin{equation} \label{eq:dra-d}
    \Delta R_\mathrm{a}(\Delta CO_2,0,0) = R_\mathrm{a}(CO_2, \overline{T}, \overline{q}) - R_\mathrm{a}(\overline{CO_2}, \overline{T}, \overline{q})\, ,
\end{equation}
where $CO_2$, $T$, and $q$ correspond to the yearly CMIP6 DJF-mean quantities from the SSP585 run.

We quantify the warming effect at fixed relative humidity as
\begin{equation} \label{eq:dra-t}
    \Delta R_\mathrm{a}(0,\Delta T,\,\overline{RH}\Delta q) = R_\mathrm{a}(\overline{CO_2}, T, \overline{RH}q^{\ast}) - R_\mathrm{a}(\overline{CO_2}, \overline{T}, \overline{q})\, .
\end{equation}

Finally, we quantify the effect of changes in relative humidity as follows:
\begin{equation} \label{eq:dra-rh}
    \Delta R_\mathrm{a}(0,0,\Delta RH \overline{q^\ast}) = R_\mathrm{a}(\overline{CO_2}, \overline{T}, RH \overline{q^{\ast}}) - R_\mathrm{a}(\overline{CO_2}, \overline{T}, \overline{q})\, .
\end{equation}

The discrepancy between the sum of equation~(\ref{eq:dra-d})--(\ref{eq:dra-rh}) and the total RRTMG clear-sky radiative cooling response is the residual, which is small (Fig.~\ref{fig:ra-lwcs}b).

\subsection{Aquaplanet experiments}
\label{subsec:ai-an}
We configure the ECHAM6 aquaplanet \cite<AQUA,>[]{stevens2013a} with and without thermodynamic sea ice to test the importance of sea-ice loss for climate change in the Arctic following previous work \cite{shaw2020, shaw2022}. The zero-layer Semtner thermodynamic sea-ice model \cite{semtner1976} is used (hereafter AQUAice). Grid cells are either completely ice free or ice covered \cite{giorgetta2013,salameh2018}. AQUAice with a 40 m mixed layer depth was previously shown to capture the observed wintertime Arctic sea-ice thickness, energy balance regime, and inversion strength \cite{miyawaki2022}. Here, we initialize AQUAice from its control climate equilibrium (CO$_2=348$ ppmv) and prescribe the CO$_2$ concentration following the historical forcing up to 2014 and the SSP585 forcing thereafter.

To test the role of sea-ice loss on the time-dependent response of the Arctic to increased CO$_2$, we configure AQUA with no sea ice (AQUAnoice) and a 40 m mixed layer. We impose a Q flux with a seasonal cycle that repeats every year to mimic the thermodynamic effect of sea ice in AQUAice (see the Supplementary Information for the derivation of the Q flux). AQUA with an imposed Q flux reproduces the climatology of AQUAice in both the annual mean and seasonal cycle (compare blue and purple lines in Fig.~S8 and S9). Whereas the sea ice melts in AQUAice in response to anthropogenic forcing, the thermodynamic effect of sea ice remains fixed in AQUAnoice because the same seasonal cycle of Q flux repeats every year.

%%%%%%%%%%%%%%%%%%%%%%%%%%%%%%%%%%%%%%%%%%%%%%%%%%%%%%%%%

%%%%%%%%%%%%%%%%%%%%%%%%%%%%%%%%%%%%%%%%%%%%%%%%%%%%%%%%%
\section{Results}
\label{sec:res}
\subsection{The evolving energy balance of the wintertime Arctic}
\label{sec:r1}
%%%%%%%%%%%%%%%%%%%%%%%%%%%%%%%%%%%%%%%%%%%%%%%%%%%%%%%%%
The wintertime Arctic atmosphere in the modern climate is in the RAE regime ($R_1=1.05\,\pm\,0.01$, spread is quantified as the 5--95\% confidence interval across the CMIP6 models) and undergoes a regime transition to RCAE (black line cross from blue to white region in Fig.~\ref{fig:r1}a). The Arctic energy balance equilibrates in the RCAE regime by the end of the next century ($R_1=0.70\,\pm\,0.03$). The energy balance response closely follows sea ice, which decreases from 100\% sea-ice cover in the modern climate to 0\% in the future (purple line in Fig.~\ref{fig:r1}). 

The timing of the energy balance response also coincides closely with the disappearance of the surface temperature inversion as measured by the near-surface lapse rate deviation from a moist adiabat (blue line, Fig.~\ref{fig:r1}a). The modern Arctic is characterized by the existence of a strong inversion (near-surface lapse rate deviation from a moist adiabat exceeds 100\%). The near-surface lapse rate weakens in response to forcing and equilibrates around the moist adiabatic lapse rate (i.e., 0\% deviation).

The energy balance response also coincides with the emergence of Arctic convection as measured by the convective precipitation fraction (blue line, Fig.~\ref{fig:r1}b). The modern Arctic is characterized by the absence of convection (convective precipitation fraction is 0). Convective precipitation fraction increases in response to forcing and equilibrates around $0.25\,\pm\,0.03$. Thus, the time-dependent response of energy balance regimes is useful for understanding the timing of the disappearing inversion and emerging convection in the Arctic. The consistency across the response of energy balance, near-surface lapse rate, convective precipitation fraction, and sea-ice loss suggests sea-ice loss is a key driver of climate change in the Arctic.

\begin{figure}
    \centering
    % \includegraphics[width=\textwidth]{{/project2/tas1/miyawaki/projects/003/plotmerge/cmip5/fig_1/fig_1}.pdf}
    \includegraphics[width=\textwidth]{{/project2/tas1/miyawaki/projects/003/plotmerge/cmip6/fig_1/fig_1}.pdf}
    \caption{The wintertime (DJF) energy balance regime quantified using $R_1$ (black, left axis, see equation~(\ref{eq:defr1})), sea-ice fraction (purple, right axis), near-surface lapse rate deviation from a moist adiabat (a, blue, right axis), and convective precipitation fraction (b, blue, right axis) for the CMIP6 multimodel mean of the extended SSP585 run. Blue and white regions indicate RAE and RCAE, respectively. The shading indicates the 5--95\% confidence interval based on intermodel spread.}
    \label{fig:r1}
\end{figure}

%%%%%%%%%%%%%%%%%%%%%%%%%%%%%%%%%%%%%%%%%%%%%%%%%%%%%%%%%
\subsection{The radiative and advective phases of the regime transition}
%%%%%%%%%%%%%%%%%%%%%%%%%%%%%%%%%%%%%%%%%%%%%%%%%%%%%%%%%
\label{sec:dc}

To diagnose the physical mechanisms that control the time-dependent energy balance response to the SSP585 scenario, we decompose $\Delta R_1(t) = R_1(t) - \overline{R_1}$ into radiative and advective responses following \citeA{miyawaki2022}:

\begin{equation} \label{eq:dc}
    \Delta R_1 = \overline{R_1}\left( \underbrace{ \frac{\Delta(\partial_t m + \partial_y (vm))}{\overline{\partial_t m + \partial_y (vm)}} }_{\mathrm{advective}} \; \underbrace{ - \frac{\Delta R_a }{\overline{R_a}} }_{\mathrm{radiative}} \right) + \mathrm{Residual} \, .
\end{equation}

The advective response (first term in equation~(\ref{eq:dc})) quantifies the importance of advective heating. The radiative response (second term in equation~(\ref{eq:dc})) quantifies the importance of radiative cooling. The residual quantifies the contribution of higher order terms.

The decomposition shows that there are two stages to the energy balance response in the Arctic: 1) the radiative phase when enhanced radiative cooling dominates (black line follows change in gray line in Fig.~\ref{fig:r1-decomp}) and 2) the advective phase when reduced advective heating dominates (black line follows change in maroon line in Fig.~\ref{fig:r1-decomp}). The timing that separates the two phases is similar to when the RAE to RCAE regime transition occurs (i.e., the maroon line begins to change when black line crosses from blue into white region in Fig.~\ref{fig:r1-decomp}). The contribution of higher-order terms are small (dash-dot line in Fig.~\ref{fig:r1-decomp}).

\begin{figure}
    \centering
    % \includegraphics[width=\textwidth]{{/project2/tas1/miyawaki/projects/003/plotmerge/cmip5/fig_2/fig_2}.pdf}
    \includegraphics[width=\textwidth]{{/project2/tas1/miyawaki/projects/003/plotmerge/cmip6/fig_2/fig_2}.pdf}
    \caption{The wintertime (DJF) response (relative to the 1984--2014 historical mean) of energy balance regimes (solid black) decomposed into the advective (red) and radiative (gray) components (see equation~(\ref{eq:dc})) and the residual (dash-dot black) for the CMIP6 multimodel mean of the extended SSP585 runs. Blue and white regions indicate RAE and RCAE, respectively. The shading indicates the 5--95\% confidence interval based on intermodel spread.}
    \label{fig:r1-decomp}
\end{figure}

%%%%%%%%%%%%%%%%%%%%%%%%%%%%%%%%%%%%%%%%%%%%%%%%%%%%%%%%%
\subsection{Decomposing the radiative cooling response}
%%%%%%%%%%%%%%%%%%%%%%%%%%%%%%%%%%%%%%%%%%%%%%%%%%%%%%%%%
\label{sec:rap}
The wintertime radiative cooling response in the Arctic (gray line in Fig.~\ref{fig:ra-lwcs}) is entirely associated with the longwave component as there is zero shortwave absorption during polar night (cyan line in Fig.~\ref{fig:ra-lwcs}a). Enhanced atmospheric longwave cooling is consistent with an enhanced greenhouse effect over the Arctic. Several mechanisms have been proposed to control the enhanced longwave cooling response to increased CO$_2$, including the direct effect of CO$_2$ (anthropogenic forcing), lapse rate feedback, water vapor feedback, and increased cloud optical thickness \cite<e.g.,>[]{curry1995,curry1996,vavrus2004,abbot2009a,taylor2013,pithan2014,cronin2015a,henry2021}. However, past studies focused on the top of atmosphere energy balance and surface warming. Here, we quantify the mechanisms controlling the time-dependent response of atmospheric radiative cooling because of its link to energy balance regimes.

Clear-sky and cloudy-sky processes contribute equally to the enhanced longwave cooling response prior to the regime transition (red and purple lines in Fig.~\ref{fig:ra-lwcs}a). The longwave cloud radiative effect begins to decrease thereafter and returns back to the same strength as in the modern climate. The response of the cloud radiative effect varies significantly across individual models (purple lines in Fig.~S5). In contrast, clear-sky radiative cooling robustly increases across models (red line in Fig.~\ref{fig:ra-lwcs}a and S5). As the clear-sky response is more robust, we focus on understanding it using RRTMG.

RRTMG reproduces the multimodel mean clear-sky longwave cooling response reasonably well (compare the black and red lines in Fig.~\ref{fig:ra-lwcs}a). The small discrepancy arises from a small subset of models where RRTMG overpredicts the GCM response (Fig.~S5). RRTMG shows the enhanced radiative cooling is dominated by warming and the associated increase in water vapor at fixed relative humidity (equation~(\ref{eq:dra-t}), compare orange and black lines in Fig.~\ref{fig:ra-lwcs}b). It is primarily the greenhouse effect of water vapor that enhances radiative cooling (blue line in Fig.~S10a). Warming in the absence of moistening reduces radiative cooling (orange line in Fig.~S10a) because the lapse rate effect is stronger than the Planck effect (compare dotted and dashed lines in Fig.~S10b). The direct effect of CO$_2$ (equation~(\ref{eq:dra-d}), green line in Fig.~\ref{fig:ra-lwcs}b) and relative humidity changes (equation~(\ref{eq:dra-rh}), blue line in Fig.~\ref{fig:ra-lwcs}b) contribute to less than 2~W~m$^{-2}$ of the radiative cooling response.

\begin{figure}
    \centering
    % \includegraphics[width=\textwidth]{{/project2/tas1/miyawaki/projects/003/plotmerge/cmip5/fig_3/fig_3}.pdf}
    \includegraphics[width=\textwidth]{{/project2/tas1/miyawaki/projects/003/plotmerge/cmip6/fig_3/fig_3}.pdf}
    \caption{(a) Wintertime (DJF) radiative cooling response (gray) decomposed into shortwave (cyan) and longwave clear- (red) and cloudy-sky (purple) fluxes for the CMIP6 multimodel mean and the RRTMG clear-sky response (black). (b) The RRTMG clear-sky radiative cooling response is further decomposed (see equation~(\ref{eq:dra-di})) into changes associated with the direct CO$_2$ effect (green line, see equation~(\ref{eq:dra-d})), the warming effect including the associated moistening assuming fixed relative humidity (orange line, see equation~(\ref{eq:dra-t})), and the drying effect from a decrease in relative humidity (blue line, see equation~(\ref{eq:dra-rh})). Shading denotes the 5--95\% confidence interval based on intermodel spread.}
    \label{fig:ra-lwcs}
\end{figure}

%%%%%%%%%%%%%%%%%%%%%%%%%%%%%%%%%%%%%%%%%%%%%%%%%%%%%%%%%
\subsection{Testing the importance of sea-ice loss on the regime transition}
%%%%%%%%%%%%%%%%%%%%%%%%%%%%%%%%%%%%%%%%%%%%%%%%%%%%%%%%%
\label{sec:adp}
Sea ice has been proposed in the literature to be a key mechanism controlling both the radiative \cite<e.g.,>[]{screen2010,dai2019} and advective responses \cite<e.g.,>[]{feldl2020,shaw2022} to forcing. Here, we test the hypothesis that sea-ice loss is a necessary criteria for the time-dependent radiative and advective responses by performing mechanism-denial experiments in an aquaplanet with and without sea-ice loss (see Section~\ref{subsec:ai-an}). Specifically, we quantify the response to SSP585 forcing in AQUAice (with sea-ice loss) and AQUAnoice (without sea-ice loss).

In AQUAice, the wintertime climatology in the Arctic is in the RAE regime ($R_1=1.08$) and $R_1$ decreases in response to anthropogenic forcing. AQUAice broadly captures the two-phased response of $R_1$. $R_1$ initially follows the radiative contribution (gray line, Fig.~\ref{fig:aqua}a) then the advective contribution thereafter (marooon line, Fig.~\ref{fig:aqua}a) consistent with the response in CMIP6. Unlike in the CMIP ensemble, the radiative contribution again dominates after about year 2100. As in the CMIP models, the Arctic equilibrates in the RCAE regime. The RAE to RCAE regime transition in AQUAice also coincides with the vanishing of the surface inversion and emergence of convective precipitation (Fig.~S11). The regime transition occurs earlier than the CMIP multimodel mean, consistent with an earlier onset of sea-ice loss (compare Fig.~S11 and \ref{fig:r1}).

In AQUAnoice, the wintertime climatology in the Arctic is comparable to that in AQUAice but there is no robust change in the Arctic energy balance in response to the SSP585 scenario (black line in Fig.~\ref{fig:aqua}b). The Arctic energy balance remains in the RAE regime, the surface inversion persists (Fig.~S12a), and convective precipitation remains absent (Fig.~S12b). The lack of an $R_1$ response is a result of negligible long-term change in both radiative and advective contributions (gray and maroon lines in Fig.~\ref{fig:aqua}b). The small radiative and advective responses in AQUAnoice combined with the fact that there is no Arctic amplification of surface warming (Arctic surface warming is 4.6 K in AQUAnoice compared to 33.0 K in AQUAice; compare Fig.~S13a and b) suggest sea-ice loss plays a fundamental role in controlling the response of the Arctic energy balance to anthropogenic forcing. Sea-ice loss strongly influences the magnitude of Arctic surface warming which controls both the radiative cooling response (via the enhanced greenhouse effect from moistening associated with warming, see Fig.~S14) and the advective heating response (via a decrease in meridional MSE gradient).

\begin{figure}
    \centering
    \includegraphics[width=\textwidth]{{/project2/tas1/miyawaki/projects/003/plotmerge/fig_4/fig_4}.pdf}
    \caption{Same as Fig.~\ref{fig:r1-decomp} but for (a) AQUAice and (b) AQUAnoice.}
    \label{fig:aqua}
\end{figure}

%%%%%%%%%%%%%%%%%%%%%%%%%%%%%%%%%%%%%%%%%%%%%%%%%%%%%%%%%
\section{Summary and Discussion}
%%%%%%%%%%%%%%%%%%%%%%%%%%%%%%%%%%%%%%%%%%%%%%%%%%%%%%%%%
\label{sec:end}
\subsection{Summary}
The wintertime Arctic (poleward of 80$^\circ$N) in the modern climate is characterized by a strong near-surface inversion, the absence of convection, and complete sea-ice cover. The Arctic end-of-century response to anthropogenic forcing involves the vanishing of the inversion, emergence of convection, and vanishing sea ice. Here we show these changes are coincident with a shift in the wintertime Arctic energy balance regime from a balance between radiative cooling and advective heating (RAE) in the modern Arctic to a regime where radiative cooling is balanced by convective and advective heating (RCAE) in the future Arctic.

Here, we investigated the time-dependent response of the wintertime Arctic climate to anthropogenic forcing using the energy balance regime framework. In this framework, Arctic energy balance regimes are quantified by the ratio of radiative cooling to advective heating plus atmospheric storage ($R_1$). We show the evolution of $R_1$ in response to the extended SSP585 emissions scenario is quantitatively linked to the vanishing of the surface temperature inversion, emergence of convective precipitation, and vanishing sea ice.

We used the energy balance framework to quantitatively compare the importance, in terms of magnitude and timing, of two previously proposed mechanisms (enhanced radiative cooling and reduced advective heating) on the time-dependent response to anthropogenic forcing. We linearly decomposed the energy balance response into contributions from changes in radiative cooling and advective heating. The decomposition showed that the regime transition is characterized by two phases. In the first phase, the $R_1$ response is dominated by enhanced radiative cooling, with clear-sky and cloud radiative effects contributing roughly equally to the radiative cooling response in the multimodel mean. Offline radiative transfer calculations showed clear-sky radiative cooling follows from increased water vapor associated with warming. In the second phase, the $R_1$ response is dominated by reduced advective heating into the Arctic. The two-phased transition suggests that different mechanisms are important at different times, highlighting the importance of investigating the time-dependent response of wintertime Arctic climate change.

We tested the hypothesis that sea-ice loss is a necessary condition for the regime transition using an aquaplanet configured with and without sea-ice loss. The aquaplanet with sea-ice loss exhibits an Arctic regime transition including a time dependent response of radiative cooling and advective heating. The aquaplanet without sea-ice loss exhibits no change in $R_1$ and the wintertime Arctic remains in RAE. The absence of significant energy flux changes in the case without sea-ice loss is consistent with the key role that sea ice plays in the surface warming response.

\subsection{Discussion}
The results demonstrate that the response of the Arctic to anthropogenic forcing involves time-dependent mechanisms controlled by sea-ice loss. This implies that historical records and near-term projections of wintertime Arctic climate change do not reveal the full picture of the long-term response. Furthermore, the results highlight the importance of using time-dependent responses to test mechanisms \cite{shaw2019} and suggest emergent constraints based on the historical Arctic climate cannot fully constrain the longer term response.

Quantifying the emergence of the new Arctic regime clarifies when assumptions applicable for the modern Arctic regime will break down. For example, we expect the temperature response predicted by the RAE model \cite{payne2015,cronin2016} to be valid for the Arctic prior to the regime transition but fail thereafter when convective heating becomes important. Similarly, the energy-constrained Arctic precipitation response hypothesis \cite{pithan2021} is only expected to hold in the absence of surface turbulent fluxes. Indeed, the precipitation response deviates from the radiative cooling response after the Arctic energy balance regime transition (Fig.~S15).

The mechanism-denial experiments support previous studies that show sea-ice loss plays an essential role in Arctic climate change \cite{screen2010,boeke2018,dai2019,shaw2022}. However previous studies have also shown that Arctic Amplification occurs in the absence of sea-ice loss \cite{alexeev2005,kim2018,merlis2018, previdi2020}. During wintertime, Arctic Amplification can occur in the absence of sea-ice loss because of the cloud response \cite{kim2018}, the lapse rate feedback \cite{previdi2020}, and an increase in poleward latent energy transport associated with moist air intrusions \cite{woods2013,woods2016,pithan2018} and the nonlinear temperature depedence of the Clausius-Clapeyron relation \cite{manabe1980,hwang2011,shaw2016c,graversen2016,yoshimori2017,merlis2018,feldl2021}. While the results do not preclude the importance of the above processes on the Arctic warming response (latent energy transport increases but is dominated by a decrease in dry static energy transport), we do not find Arctic Amplification in the absence of sea-ice loss here.

While the mechanism-denial experiment demonstrates that sea-ice loss is essential for both radiative and advective responses, additional experiments are necessary to understand the two-phased nature of the response. A fruitful avenue for future work may be to quantify the effect of sea-ice loss into 1) reduced sea-ice coverage (i.e., surface albedo changes from sea-ice covered to open ocean) and 2) reduced sea-ice thickness (i.e., surface remains covered in sea ice) through their distinct influences on the effective surface heat capacity \cite{hahn2022} and surface turbulent heat fluxes \cite{taylor2022}. The advective response emerges concurrently with the sea-ice fraction response, so one plausible hypothesis is that the advective response is connected to ice retreat (a change in surface type to open ocean) whereas the radiative response is connected to ice thinning (no change in surface type).

The Q-flux method introduced here quantifies sea-ice loss by taking the difference of an aquaplanet configured with and without interactive sea ice. This has the advantage over methods used to control sea-ice loss such as the ghost flux \cite<e.g.,>[]{alexeev2005,deser2015} and nudging methods \cite<e.g.,>[]{mccusker2017, sun2018}, where a time varying surface forcing introduces spurious warming that overestimates the true warming contribution of sea-ice loss in response to radiative forcing \cite{england2022}. As imposing a Q flux is a simple and ubiquitous feature in climate models, the method introduced here may be of interest to the broader polar climate change community seeking to configure mechanism-denial experiments to isolate the effect of sea-ice loss on climate change.

\acknowledgments
We acknowledge support from the National Science Foundation (AGS-2033467). We thank Tim Cronin for helpful discussions. We acknowledge the University of Chicago Research Computing Center for providing the computational resources used to carry out this work.

%% ------------------------------------------------------------------------ %%
%% References and Citations

% \bibliography{/project2/tas1/miyawaki/projects/002/draft/references.bib}
\bibliography{/project2/tas1/miyawaki/projects/003/draft/references}

\end{document}



