%%%%%%%%%%%%%%%%%%%%%%%%%%%%%%%%%%%%%%%%%%%%%%%%%%%%%%%%%%%%%%%%%%%%%%%%%%%%
% AGUJournalTemplate.tex: this template file is for articles formatted with LaTeX
%
% This file includes commands and instructions
% given in the order necessary to produce a final output that will
% satisfy AGU requirements, including customized APA reference formatting.
%
% You may copy this file and give it your
% article name, and enter your text.
%
%
% Step 1: Set the \documentclass
%
%

%% To submit your paper:
\documentclass[draft]{agujournal2019}
\usepackage{url} %this package should fix any errors with URLs in refs.
\usepackage{lineno}
\usepackage[inline]{trackchanges} %for better track changes. finalnew option will compile document with changes incorporated.
\usepackage{soul}
\linenumbers
%%%%%%%
% As of 2018 we recommend use of the TrackChanges package to mark revisions.
% The trackchanges package adds five new LaTeX commands:
%
%  \note[editor]{The note}
%  \annote[editor]{Text to annotate}{The note}
%  \add[editor]{Text to add}
%  \remove[editor]{Text to remove}
%  \change[editor]{Text to remove}{Text to add}
%
% complete documentation is here: http://trackchanges.sourceforge.net/
%%%%%%%

\draftfalse

%% Enter journal name below.
%% Choose from this list of Journals:
%
% JGR: Atmospheres
% JGR: Biogeosciences
% JGR: Earth Surface
% JGR: Oceans
% JGR: Planets
% JGR: Solid Earth
% JGR: Space Physics
% Global Biogeochemical Cycles
% Geophysical Research Letters
% Paleoceanography and Paleoclimatology
% Radio Science
% Reviews of Geophysics
% Tectonics
% Space Weather
% Water Resources Research
% Geochemistry, Geophysics, Geosystems
% Journal of Advances in Modeling Earth Systems (JAMES)
% Earth's Future
% Earth and Space Science
% Geohealth
%
% ie, \journalname{Water Resources Research}

\journalname{Geophysical Research Letters}


\begin{document}

\title{Supplementary material for\\The transient emergence of a new wintertime Arctic energy balance regime}


\authors{O. Miyawaki\affil{1}, T. A. Shaw\affil{1}, M. F. Jansen\affil{1}}


\affiliation{1}{Department of the Geophysical Sciences, The University of Chicago}

\correspondingauthor{Osamu Miyawaki}{miyawaki@uchicago.edu}

\renewcommand{\thefigure}{S\arabic{figure}}
\renewcommand{\thetable}{S\arabic{table}}

\newpage

\begin{table}
    \centering
    \caption{List of the 7 CMIP5 and 5 CMIP6 models that are used for the multimodel mean of the extended RCP8.5 and SSP585 runs, respectively. Following \citeA{hankel2021}, GISS-E2-H and GISS-E2-R are omitted as outliers from the CMIP5 mean (deemed outliers because wintertime sea ice does not melt during the extended RCP8.5 run).}
    \label{tab:models}
    \begin{tabular}{l l}
        \hline
        CMIP5 & CMIP6 \\
        \hline
        bcc-csm1-1 & ACCESS-CM2\\
        CCSM4 & ACCESS-ESM1-5\\
        CNRM-CM5 & CanESM5\\
        CSIRO-Mk3-6-0 & IPSL-CM6A-LR\\
        HadGEM2-ES & MRI-ESM2-0\\
        IPSL-CM5A-LR \\
        MPI-ESM-LR\\
        \hline
    \end{tabular}
\end{table}

\begin{figure}
    \centering
    \includegraphics[width=\textwidth]{{/project2/tas1/miyawaki/projects/003/echam/qflux/qflux}.pdf}
    \caption{The seasonally varying Q-flux (positive values correspond to heat flux divergence, a cooling influence) that is prescribed for the AQUAqflux run. The zero contour is indicated in black.}
\end{figure}

\begin{figure}
    \centering
    \includegraphics[width=\textwidth]{{/project2/tas1/miyawaki/projects/003/plotmerge/qflux/clima}.pdf}
    \caption{(a) Surface temperature tendency, (b) surface net longwave radiative flux, (c) surface latent heat flux, and (d) surface sensible heat flux for 40 m AQUA with ice (blue), with q flux but no ice (red), and no q flux and no ice (black).}
\end{figure}

\begin{figure}
    \centering
    \includegraphics[width=\textwidth]{{/project2/tas1/miyawaki/projects/003/plotmerge/cmip6/fig_2/fig_2}.pdf}
    \caption{Same as Fig.~2 but for the CMIP6 mean of the extended SSP585 run.}
\end{figure}

\begin{figure}
    \centering
    \includegraphics[width=\textwidth]{{/project2/tas1/miyawaki/projects/003/plotmerge/cmip5/fig_2_all/fig_2_all}.pdf}
    \caption{Same as Fig.~2 (a) but for individual CMIP5 models (b--h).}
\end{figure}

\begin{figure}
    \centering
    \includegraphics[width=\textwidth]{{/project2/tas1/miyawaki/projects/003/plotmerge/cmip6/fig_2_all/fig_2_all}.pdf}
    \caption{Same as Fig.~S3 (a) but for individual CMIP6 models (b--f).}
\end{figure}

\begin{figure}
    \centering
    \includegraphics[width=\textwidth]{{/project2/tas1/miyawaki/projects/003/plotmerge/cmip5/fig_3a_all/fig_3a_all}.pdf}
    \caption{Same as Fig.~3a (a) but for individual CMIP5 models (b--h).}
\end{figure}

\begin{figure}
    \centering
    \includegraphics[width=\textwidth]{{/project2/tas1/miyawaki/projects/003/plotmerge/cmip5/fig_3b_all/fig_3b_all}.pdf}
    \caption{Same as Fig.~3b (a) but for individual CMIP5 models (b--h).}
\end{figure}


%
% ---------------
% EXAMPLE TABLE
%
% \begin{table}
% \caption{Time of the Transition Between Phase 1 and Phase 2$^{a}$}
% \centering
% \begin{tabular}{l c}
% \hline
%  Run  & Time (min)  \\
% \hline
%   $l1$  & 260   \\
%   $l2$  & 300   \\
%   $l3$  & 340   \\
%   $h1$  & 270   \\
%   $h2$  & 250   \\
%   $h3$  & 380   \\
%   $r1$  & 370   \\
%   $r2$  & 390   \\
% \hline
% \multicolumn{2}{l}{$^{a}$Footnote text here.}
% \end{tabular}
% \end{table}

\clearpage

\bibliography{/project2/tas1/miyawaki/projects/003/draft/references.bib}

\end{document}
