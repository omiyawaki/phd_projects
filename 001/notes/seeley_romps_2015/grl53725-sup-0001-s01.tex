%%%%%%%%%%%%%%%%%%%%%%%%%%%%%%%%%%%%%%%%%%%%%%%%%%%%%%%%%%%%%%%%%%%%%%%%%%%%
% AGUtmpl.tex: this template file is for articles formatted with LaTeX2e,
% Modified November 2013
%
% This template includes commands and instructions
% given in the order necessary to produce a final output that will
% satisfy AGU requirements.
%
% PLEASE DO NOT USE YOUR OWN MACROS
% DO NOT USE \newcommand, \renewcommand, or \def.
%
% FOR FIGURES, DO NOT USE \psfrag
%
%%%%%%%%%%%%%%%%%%%%%%%%%%%%%%%%%%%%%%%%%%%%%%%%%%%%%%%%%%%%%%%%%%%%%%%%%%%%
%
% All questions should be e-mailed to latex@agu.org.
%
%%%%%%%%%%%%%%%%%%%%%%%%%%%%%%%%%%%%%%%%%%%%%%%%%%%%%%%%%%%%%%%%%%%%%%%%%%%%
%
% Step 1: Set the \documentclass
%
% There are two options for article format: two column (default)
% and draft.
%
% PLEASE USE THE DRAFT OPTION TO SUBMIT YOUR PAPERS.
% The draft option produces double spaced output.
%
% Choose the journal abbreviation for the journal you are
% submitting to:

% jgrga JOURNAL OF GEOPHYSICAL RESEARCH
% gbc   GLOBAL BIOCHEMICAL CYCLES
% grl   GEOPHYSICAL RESEARCH LETTERS
% pal   PALEOCEANOGRAPHY
% ras   RADIO SCIENCE
% rog   REVIEWS OF GEOPHYSICS
% tec   TECTONICS
% wrr   WATER RESOURCES RESEARCH
% gc    GEOCHEMISTRY, GEOPHYSICS, GEOSYSTEMS
% sw    SPACE WEATHER
% ms    JAMES
% ef    EARTH'S FUTURE
%
%
%
% (If you are submitting to a journal other than jgrga,
% substitute the initials of the journal for "jgrga" below.)

\documentclass[draft,grl]{agutexSI}
\usepackage{color,soul}
\newcommand{\hilight}[1]{%
  \colorbox{yellow}{$\displaystyle#1$}}


%%%%%%%%%%%%%%%%%%%%%%%%%%%%%%%%%%%%%%%%%%%%%%%%%%%%%%%%%%%%%%%%%%%%%%%%%
%
%  SUPPORTING INFORMATION TEMPLATE
%
%% ------------------------------------------------------------------------ %%
%
%
%Please use this template when formatting and submitting your Supporting Information.

%This template serves as both a “table of contents” for the supporting information for your article and as a summary of files.
%
%
%OVERVIEW
%
%Please note that all supporting information will be peer reviewed with your manuscript.
%In general, the purpose of the supporting information is to enable authors to provide and archive auxiliary information such as data %tables, method information, figures, video, or computer software, in digital formats so that other scientists can use it.
%The key criteria are that the data:
% 1. supplement the main scientific conclusions of the paper but are not essential to the conclusions (with the exception of
%    including %data so the experiment can be reproducible);
% 2. are likely to be usable or used by other scientists working in the field;
% 3. are described with sufficient precision that other scientists can understand them, and
% 4. are not exe files.
%
%USING THIS TEMPLATE
%
%All Supporting text and figures should be included in this document. Insert supporting information content into each appropriate section of the template. %Figures and tables should appear above each caption.  To add additional captions, simply copy and paste each sample caption as needed.

%Tables may be included, but can also be uploaded separately, especially if they are larger than 1 page, or if necessary for retaining table formatting. Data sets, large tables, movie files, and audio files should be uploaded separately, following AGU naming conventions. Include their captions in this document and list the file name with the caption. You will be prompted to upload these files on the Upload Files tab during the submission process, using file type “Supporting Information (SI)”

%IMPORTANT NOTE ON FIGURES AND TABLES
% Placeholders for figures and tables appear after the \end{article} command, after references.
% DO NOT USE \psfrag or \subfigure commands.
%
%  Uncomment the following command to include .eps files
  \usepackage{graphicx}
%
%  Uncomment the following command to allow illustrations to print
%   when using Draft:
  \setkeys{Gin}{draft=false}
%
% Substitute one of the following for [dvips] above
% if you are using a different driver program and want to
% proof your illustrations on your machine:
%
% [xdvi], [dvipdf], [dvipsone], [dviwindo], [emtex], [dviwin],
% [pctexps],  [pctexwin],  [pctexhp],  [pctex32], [truetex], [tcidvi],
% [oztex], [textures]
%
%
%% ------------------------------------------------------------------------ %%
%
%  ENTER PREAMBLE
%
%% ------------------------------------------------------------------------ %%

% Author names in capital letters:
\authorrunninghead{SEELEY AND ROMPS}

% Shorter version of title entered in capital letters:
\titlerunninghead{WHY DOES CAPE INCREASE WITH WARMING?}

%Corresponding author mailing address and e-mail address:
\authoraddr{Corresponding author: J. T. Seeley,
Department of Earth and Planetary Science, University of
California, Berkeley, 449 McCone Hall, Berkeley, CA 94703, USA.
(jseeley@berkeley.edu)}

\begin{document}

%% ------------------------------------------------------------------------ %%
%
%  TITLE
%
%% ------------------------------------------------------------------------ %%

%\includegraphics{agu_pubart-white_reduced.eps}


\title{Supporting Information for ``Why does tropical convective available potential energy (CAPE) increase with warming?"}
%
% e.g., \title{Supporting Information for "Terrestrial ring current:
% Origin, formation, and decay $\alpha\beta\Gamma\Delta$"}
%
%DOI: 10.1002/%insert paper number here%

%% ------------------------------------------------------------------------ %%
%
%  AUTHORS AND AFFILIATIONS
%
%% ------------------------------------------------------------------------ %%


%Use \author{\altaffilmark{}} and \altaffiltext{}

% \altaffilmark will produce footnote;
% matching \altaffiltext will appear at bottom of page.

\authors{Jacob T. Seeley\altaffilmark{1,2}
and David M. Romps\altaffilmark{1,2}}

\altaffiltext{1}{Department of Earth and Planetary Science,
University of California, Berkeley, California, USA.}

\altaffiltext{2}{Climate and Ecosystem Sciences Division, Lawrence Berkeley National Laboratory,
Berkeley, California, USA.}


%% ------------------------------------------------------------------------ %%
%
%  BEGIN ARTICLE
%
%% ------------------------------------------------------------------------ %%

% The body of the article must start with a \begin{article} command
%
% \end{article} must follow the references section, before the figures
%  and tables.

\begin{article}

%% ------------------------------------------------------------------------ %%
%
%  TEXT
%
%% ------------------------------------------------------------------------ %%



\noindent\textbf{Contents of this file}
%%%Remove or add items as needed%%%
\begin{enumerate}
\item Text S1 to S3
\item Figure S1
%if Tables are larger than 1 page, upload as separate excel file
\end{enumerate}
%\noindent\textbf{Additional Supporting Information (Files uploaded separately)}
%\begin{enumerate}
%\item Captions for Datasets S1 to Sx
%\item Captions for large Tables S1 to Sx (if larger than 1 page, upload as separate excel file)
%\item Captions for Movies S1 to Sx
%\item Captions for Audio S1 to Sx
%\end{enumerate}

%\noindent\textbf{Introduction}
%Type or paste your text here. The introduction gives a brief overview of the supporting information. You should include information %about as many of the following as possible (when appropriate):
% 1. a general overview of the kind of data files;
% 2. information about when and how the data were collected or created;
% 3. a general description of processing steps used;
% 4. any known imperfections or anomalies in the data.

\clearpage

%Delete all unused file types below. Copy/paste for multiples of each file type as needed.
\noindent\textbf{Text S1: Simple zero-buoyancy model}
%Type or paste text here. This should be additional explanatory text, such as: extended descriptions of results, full details of models, extended lists of acknowledgements etc.  It should not be additional discussion, analysis, interpretation or critique. It should not be an additional scientific experiment or paper.
%
%Repeat for any additional Supporting Text

The simple zero-buoyancy model is based on a simplified thermodynamics in which there is no ice phase. In addition, the effect of water on the density and heat capacity of air is neglected. Accordingly, MSE is defined here as $h = c_p T + L q_v + gz$. The plume equations describing the vertical profiles of saturated MSE $h^*$, total water ($q_t = q_v + q_c$, where $q_c$ is the non-precipitating condensed water), and pressure are: 
\begin{eqnarray}
&\partial_z h^* = -\epsilon \left(h^* - h\right)\ ,& \label{eq:h_budg}\\
&\partial_z q_t = -\epsilon \left(q_t - q_{v}\right)\ ,&\label{eq:q_budg} \\ 
&\partial_z \log p = -g/(R_a T)\ . & \label{eq:p_eq}
\end{eqnarray}
\noindent The MSE of the environment is given by $h$, $q_v$ is the specific humidity of the environment, and $R_a$ is the dry-air gas constant. The plume equations are integrated vertically by first specifying the temperature, total water, and pressure at plume base. If the plume is initially unsaturated, as it typically is when initialized with values taken from the near-surface level of a CRM simulation, equations \ref{eq:h_budg}--\ref{eq:p_eq} are advanced with the entrainment rate $\epsilon$ set to 0 to generate a dry adiabat until saturation occurs. At and above the level of plume saturation, the specific humidity and the MSE of the environment are calculated using the supplied RH profile and the known temperature of the plume/environment. The supplied entrainment profile is then used to calculate the plume's $q_t$ and $h^*$ at the next vertical step with a simple forward-difference method. A root-solver is used to calculate the temperature that is consistent with the known value of $h^*$ at the next vertical step. Any water in excess of $q_v^*$ is dumped into the $q_c$ category. As it is written, equation \ref{eq:q_budg} assumes no fallout of condensed water, but our plume model includes a parameter, $\gamma$, that determines what fraction of liquid water precipitates out at each step. If $1\ge \gamma > 0$, any new $q_c$ that forms when stepping vertically is reduced by the factor ($1-\gamma$). This ``precipitation" is removed at constant pressure and temperature. Since we neglect the effect of water on the density and pressure of air in this simple model, $q_c$ only functions as a reserve of liquid water that can re-evaporate to maintain saturation after entrainment reduces the specific humidity of the plume. Iterating this procedure generates a plume/environment temperature profile. The buoyancy of an undilute parcel is then computed by lifting a parcel that conserves its (simple) MSE, computing its temperature as a function of height, and comparing to the plume/environment temperature profile. Note that the zero-buoyancy plume is always colder than the undilute parcel---the zero-buoyancy model does not predict an LNB. To get a value of CAPE from the buoyancy profile predicted by the zero-buoyancy model, one must supply an upper bound for the CAPE integration. The plume models are implemented in Python; code is available from the first author upon request.

\noindent\textbf{Text S2: Complex zero-buoyancy model}

Like the simple version, the complex zero-buoyancy model operates on the principle of neutrality between an entraining plume and its environment, but takes full account of the ice phase and the effects of water on the density and heat capacity of air. The total water mass fraction in this case is $q_t = q_v + q_l + q_s$, where $q_l$ and $q_s$ are the mass fractions of liquid and solid water, respectively. The MSE is given by 
\begin{equation}\label{eq:MSE_full}
h = c_{pm}(T-T_0) + q_v(E_{0v} + R_vT_0) - q_sE_{0s} + gz,
\end{equation}
\noindent where $c_{pm}$ is the constant-pressure specific heat capacity of moist air, $T_0=273.16$ K is the triple-point temperature, $E_{0v}$ is the difference in specific internal energy between water vapor and liquid at the triple-point temperature, $R_v$ is the gas constant for water vapor, and $E_{0s}$ is the difference in specific internal energy between water liquid and solid at the triple-point temperature. The moist-air heat capacity, $c_{pm}$, is given as a mass-fraction-weighted linear combination of the constant-pressure heat capacities of dry air (subscript $a$) and the three water phases (subscripts $v$, $l$, and $s$): $c_{pm} = q_a c_{pa} + q_v c_{pv} + q_l c_{pl} + q_s c_{ps}$.

The complex version of the zero-buoyancy model integrates the same plume equations as the simple version, but the pressure equation is replaced with
\begin{equation}
\partial_z \log p = \frac{-g}{R_m T_e}\ ,
\end{equation}
\noindent where $R_m = q_{v}R_v + (1 - q_{v})R_d$ is the gas constant for moist environmental air and $T_e$ is the environment temperature. The vertical integration of the plume equations is carried out as for the simple model, but since we include the effects of water phases on the density of air in this case, the neutrality of the entraining plume is enforced as a constraint on density rather than temperature. A rootsolving algorithm uses the known plume density and the supplied environmental relative humidity to calculate the environmental temperature that is consistent with a neutrally buoyant plume. The complex version of the zero-buoyancy model also includes the full effects of the ice phase; $q_v^*$ is defined with respect to liquid at temperatures warmer than the triple-point temperature (273.16 K), with respect to ice at temperatures below 240 K, and as a linear combination of the two at temperatures in between. This corresponds to a non-isothermal mixed phase regime between the triple-point temperature and the temperature of homogeneous freezing; the partitioning of condensates in the plume transitions linearly in temperature from all-liquid to all-ice between these two temperatures. (For a more complete description of the moist thermodynamics used in this model, including explicit equations for $q_v^*$, see the appendix of \cite{Romps2015}). To calculate undilute parcel buoyancy, a near-surface parcel is lifted assuming conservation of MSE - CAPE, and this parcel's density is then compared to the plume/environment density profile produced by the zero-buoyancy model. The plume models are implemented in Python; code is available from the first author upon request. 

\noindent\textbf{Text S3: Method for nudging RH}

There are many ways one could ``nudge" relative humidity in a numerical model, so here we will be explicit about how this forcing was implemented in our simulations. Our relative humidity nudging was performed by nudging the local $q_v$ according to
\begin{equation}
F_{q_v} = \rho \frac{\textrm{RH}^\dagger(z) q^*_v - q_v}{\tau},
\end{equation}
\noindent where RH$^\dagger(z)$ is the target RH profile and the nudging timescale is $\tau$. Note that $F_{q_v}$ has units of density per time, and represents an artificial convergence of pure water vapor into the Eulerian finite volumes in the numerical model (we use  ``convergence" as shorthand for convergence or divergence). That is, $F_{q_v}$ appears as a source in the governing equation for water vapor as follows:
\begin{equation}
\partial_t\left(q_v \rho\right) = -\vec{\nabla} \cdot \left(q_v \rho \vec{u}\right) + e - \vec{\nabla} \cdot \vec{d_v} + F_{q_v},
\end{equation}
\noindent where the other source terms are the resolved-flow moisture convergence $-\vec{\nabla} \cdot \left(q_v \rho \vec{u}\right)$, the evaporation $e$, and the convergence of diffusive vapor fluxes $-\vec{\nabla} \cdot \vec{d_v}$, which is nonzero only in the near-surface level in our model.

However, since we are interested in the effect of environmental relative humidity on the temperature profile of a convecting atmosphere, we need to adjust RH in such a way that the forcing itself has negligible effects on temperature. The convergence of water vapor does work on the gas in a finite volume, and therefore has an effect on temperature. To counteract this, we also specify a countervailing convergence of an equal and opposite number of moles of dry air per volume per time:
\begin{equation}
F_{q_a}(z) = -\frac{R_v}{R_a}F_{q_v}.
\end{equation}
\noindent The end result of this combination of forcings is effectively a mole-for-mole swap of dry air and water vapor. The corresponding effect on the model's finite-volume energy budget was accounted for by keeping track of the enthalpies of the exchanged gases. We apply this relative humidity nudging in every model level below 15 km, but not in the stratosphere.

To minimize the possibility of convective preconditioning (i.e., the probability that a developing cloud will grow through the moist detritus of a prior convective event), we should like to use a short $\tau$ in the free troposphere. However, if we nudge RH locally on too short a timescale near cloud base, we will never give clouds a chance to be born. Therefore, we specify an altitude-dependent $\tau$ given by 
\begin{equation}\label{eq:tau_rh}
\tau(z) = \cases{
      1 \textrm{ minute} & $z < 200$ \textrm{ m} \cr
      1 \textrm{ day} & $200 \textrm{m} \leq z \leq 600 \textrm{ m}$ \cr
      \frac{10^7}{z^{1.35}} \textrm{ seconds} & $600 \textrm{ m} < z$ }.
\end{equation}
\noindent This profile of $\tau$ has the desirable properties of maintaining constant humidity in the subcloud-layer, giving clouds a chance to become saturated in the neighborhood of cloud base ({\raise.17ex\hbox{$\scriptstyle\mathtt{\sim}$}}500 m), and quickly adjusting the RH of air outside of clouds to the target value in the free troposphere.


%%Enter Data Set, Movie, and Audio captions here
%%EXAMPLE CAPTIONS

%\noindent\textbf{Data Set S1.} %Type or paste caption here.
%upload your dataset(s) to AGU's journal submission site and select "Supporting Information (SI)" as the file type. Following naming %convention: ds01.

%Repeat for any additional Supporting data sets

%\noindent\textbf{Movie S1.} %Type or paste caption here.
%upload your movie(s) to AGU's journal submission site and select, "Supporting Information %(SI)" as the file type. Following naming convention: ms01.

%Repeat any additional Supporting movies

%\noindent\textbf{Audio S1.} %Type or paste caption here.
%upload your audio file(s) to AGU's journal submission site and select "Supporting Information %(SI)" as the file type. Following naming convention: auds01.

%Repeat for any additional Supporting audio files

%%% End of body of article:
%%%%%%%%%%%%%%%%%%%%%%%%%%%%%%%%%%%%%%%%%%%%%%%%%%%%%%%%%%%%%%%%
%
% Optional Notation section goes here
%
% Notation -- End each entry with a period.
% \begin{notation}
% Term & definition.\\
% Second term & second definition.\\
% \end{notation}
%%%%%%%%%%%%%%%%%%%%%%%%%%%%%%%%%%%%%%%%%%%%%%%%%%%%%%%%%%%%%%%%


%% ------------------------------------------------------------------------ %%
%%  REFERENCE LIST AND TEXT CITATIONS
%
% Either type in your references using
 \begin{thebibliography}{}
\bibitem[{\textit{Romps}(2015)}]{Romps2015}
Romps, D.~M. (2015), {MSE minus CAPE is the true conserved variable for an
  adiabatically lifted parcel}, \textit{Journal of the Atmospheric Sciences},
  pp. 0--13.
% Text
 \end{thebibliography}
%
% Or,
%
% If you use BiBTeX for your references, please use the agufull08.bst file (available at % ftp://ftp.agu.org/journals/latex/journals/Manuscript-Preparation/) to produce your .bbl
% file and copy the contents into your paper here.
%
% Follow these steps:
% 1. Run LaTeX on your LaTeX file.
%
% 2. Make sure the bibliography style appears as \bibliographystyle{agufull08}. Run BiBTeX on your LaTeX
% file.
%
% 3. Open the new .bbl file containing the reference list and
%   copy all the contents into your LaTeX file here.
%
% 4. Comment out the old \bibliographystyle and \bibliography commands.
%
% 5. Run LaTeX on your new file before submitting.
%
% AGU does not want a .bib or a .bbl file. Please copy in the contents of your .bbl file here.

%\begin{thebibliography}{}

%\providecommand{\natexlab}[1]{#1}
%\expandafter\ifx\csname urlstyle\endcsname\relax
%  \providecommand{\doi}[1]{doi:\discretionary{}{}{}#1}\else
%  \providecommand{\doi}{doi:\discretionary{}{}{}\begingroup
%  \urlstyle{rm}\Url}\fi
%
%\bibitem[{\textit{Atkinson and Sloan}(1991)}]{AtkinsonSloan}
%Atkinson, K., and I.~Sloan (1991), The numerical solution of first-kind
%  logarithmic-kernel integral equations on smooth open arcs, \textit{Math.
%  Comp.}, \textit{56}(193), 119--139.
%
%\bibitem[{\textit{Colton and Kress}(1983)}]{ColtonKress1}
%Colton, D., and R.~Kress (1983), \textit{Integral Equation Methods in
%  Scattering Theory}, John Wiley, New York.
%
%\bibitem[{\textit{Hsiao et~al.}(1991)\textit{Hsiao, Stephan, and
%  Wendland}}]{StephanHsiao}
%Hsiao, G.~C., E.~P. Stephan, and W.~L. Wendland (1991), On the {D}irichlet
%  problem in elasticity for a domain exterior to an arc, \textit{J. Comput.
%  Appl. Math.}, \textit{34}(1), 1--19.
%
%\bibitem[{\textit{Lu and Ando}(2012)}]{LuAndo}
%Lu, P., and M.~Ando (2012), Difference of scattering geometrical optics
%  components and line integrals of currents in modified edge representation,
%  \textit{Radio Sci.}, \textit{47},  RS3007, \doi{10.1029/2011RS004899}.

%\end{thebibliography}

%Reference citation examples:

%...as shown by \textit{Kilby} [2008].
%...as shown by {\textit  {Lewin}} [1976], {\textit  {Carson}} [1986], {\textit  {Bartholdy and Billi}} [2002], and {\textit  {Rinaldi}} [2003].
%...has been shown [\textit{Kilby et al.}, 2008].
%...has been shown [{\textit  {Lewin}}, 1976; {\textit  {Carson}}, 1986; {\textit  {Bartholdy and Billi}}, 2002; {\textit  {Rinaldi}}, 2003].
%...has been shown [e.g., {\textit  {Lewin}}, 1976; {\textit  {Carson}}, 1986; {\textit  {Bartholdy and Billi}}, 2002; {\textit  {Rinaldi}}, 2003].

%...as shown by \citet{jskilby}.
%...as shown by \citet{lewin76}, \citet{carson86}, \citet{bartoldy02}, and \citet{rinaldi03}.
%...has been shown \citep{jskilbye}.
%...has been shown \citep{lewin76,carson86,bartoldy02,rinaldi03}.
%...has been shown \citep [e.g.,][]{lewin76,carson86,bartoldy02,rinaldi03}.
%
% Please use ONLY \citet and \citep for reference citations.
% DO NOT use other cite commands (e.g., \cite, \citeyear, \nocite, \citealp, etc.).

%% ------------------------------------------------------------------------ %%
%
%  END ARTICLE
%
%% ------------------------------------------------------------------------ %%
\end{article}
\clearpage

% Delete all unused file types below. Copy/paste for multiples of each file type as needed.

% enter figures and tables here:
%
% EXAMPLE FIGURE
% ---------------
% \begin{figure}
%\setfigurenum{S1} %%Change number for each figure
% \noindent\includegraphics[width=20pc]{samplefigure.eps}
%\caption{Caption text here}
 %\label{figure_label}
 %\end{figure}
 
\begin{figure}
\setfigurenum{S1}
\begin{center}
\includegraphics[scale=.6]{./figs/w_RH_sst.png}
\caption{Profiles of mean vertical velocity in cloud updrafts in the CRM simulations from (a) the SST-warming experiment, and (b) the RH-varying experiment. Colors correspond to the set of SSTs and target free-tropospheric RH values as in Figures 2a and 4a of the main text. Note that (a) and (b) have different horizontal and vertical scales.} 
\label{fig:w_RH}
\end{center}
\end{figure}

% ---------------
% EXAMPLE TABLE
%
%\begin{table}
%\settablenum{S1} %%Change number for each table
%\caption{Time of the Transition Between Phase 1 and Phase 2\tablenotemark{a}}
%\centering
%\begin{tabular}{l c}
%\hline
% Run  & Time (min)  \\
%\hline
%  $l1$  & 260   \\
%  $l2$  & 300   \\
%  $l3$  & 340   \\
%  $h1$  & 270   \\
%  $h2$  & 250   \\
%  $h3$  & 380   \\
%  $r1$  & 370   \\
%  $r2$  & 390   \\
%\hline
%\end{tabular}
%\tablenotetext{a}{Footnote text here.}
%\end{table}
% ---------------
%
% EXAMPLE LARGE TABLE (UPLOADED SEPARATELY)
%\begin{table}
%\settablenum{S1} %%Change number for each table
%\caption{Time of the Transition Between Phase 1 and Phase 2\tablenotemark{a}}
%\end{table}


\end{document}

%%%%%%%%%%%%%%%%%%%%%%%%%%%%%%%%%%%%%%%%%%%%%%%%%%%%%%%%%%%%%%%

More Information and Advice:

%% ------------------------------------------------------------------------ %%
%
%  SECTION HEADS
%
%% ------------------------------------------------------------------------ %%

% Capitalize the first letter of each word (except for
% prepositions, conjunctions, and articles that are
% three or fewer letters).

% AGU follows standard outline style; therefore, there cannot be a section 1 without
% a section 2, or a section 2.3.1 without a section 2.3.2.
% Please make sure your section numbers are balanced.
% ---------------
% Level 1 head
%
% Use the \section{} command to identify level 1 heads;
% type the appropriate head wording between the curly
% brackets, as shown below.
%
%An example:
%\section{Level 1 Head: Introduction}
%
% ---------------
% Level 2 head
%
% Use the \subsection{} command to identify level 2 heads.
%An example:
%\subsection{Level 2 Head}
%
% ---------------
% Level 3 head
%
% Use the \subsubsection{} command to identify level 3 heads
%An example:
%\subsubsection{Level 3 Head}
%
%---------------
% Level 4 head
%
% Use the \subsubsubsection{} command to identify level 3 heads
% An example:
%\subsubsubsection{Level 4 Head} An example.
%
%% ------------------------------------------------------------------------ %%
%
%  IN-TEXT LISTS
%
%% ------------------------------------------------------------------------ %%
%
% Do not use bulleted lists; enumerated lists are okay.
% \begin{enumerate}
% \item
% \item
% \item
% \end{enumerate}
%
%% ------------------------------------------------------------------------ %%
%
%  EQUATIONS
%
%% ------------------------------------------------------------------------ %%

% Single-line equations are centered.
% Equation arrays will appear left-aligned.

Math coded inside display math mode \[ ...\]
 will not be numbered, e.g.,:
 \[ x^2=y^2 + z^2\]

 Math coded inside \begin{equation} and \end{equation} will
 be automatically numbered, e.g.,:
 \begin{equation}
 x^2=y^2 + z^2
 \end{equation}

% IF YOU HAVE MULTI-LINE EQUATIONS, PLEASE
% BREAK THE EQUATIONS INTO TWO OR MORE LINES
% OF SINGLE COLUMN WIDTH (20 pc, 8.3 cm)
% using double backslashes (\\).

% To create multiline equations, use the
% \begin{eqnarray} and \end{eqnarray} environment
% as demonstrated below.
\begin{eqnarray}
  x_{1} & = & (x - x_{0}) \cos \Theta \nonumber \\
        && + (y - y_{0}) \sin \Theta  \nonumber \\
  y_{1} & = & -(x - x_{0}) \sin \Theta \nonumber \\
        && + (y - y_{0}) \cos \Theta.
\end{eqnarray}

%If you don't want an equation number, use the star form:
%\begin{eqnarray*}...\end{eqnarray*}

% Break each line at a sign of operation
% (+, -, etc.) if possible, with the sign of operation
% on the new line.

% Indent second and subsequent lines to align with
% the first character following the equal sign on the
% first line.

% Use an \hspace{} command to insert horizontal space
% into your equation if necessary. Place an appropriate
% unit of measure between the curly braces, e.g.
% \hspace{1in}; you may have to experiment to achieve
% the correct amount of space.


%% ------------------------------------------------------------------------ %%
%
%  EQUATION NUMBERING: COUNTER
%
%% ------------------------------------------------------------------------ %%

% You may change equation numbering by resetting
% the equation counter or by explicitly numbering
% an equation.

% To explicitly number an equation, type \eqnum{}
% (with the desired number between the brackets)
% after the \begin{equation} or \begin{eqnarray}
% command.  The \eqnum{} command will affect only
% the equation it appears with; LaTeX will number
% any equations appearing later in the manuscript
% according to the equation counter.
%

% If you have a multiline equation that needs only
% one equation number, use a \nonumber command in
% front of the double backslashes (\\) as shown in
% the multiline equation above.

%% ------------------------------------------------------------------------ %%
%
%  SIDEWAYS FIGURE AND TABLE EXAMPLES
%
%% ------------------------------------------------------------------------ %%
%
% For tables and figures, add \usepackage{rotating} to the paper and add the rotating.sty file to the folder.
% AGU prefers the use of {sidewaystable} over {landscapetable} as it causes fewer problems.
%
% \begin{sidewaysfigure}
% \includegraphics[width=20pc]{samplefigure.eps}
% \caption{caption here}
% \label{label_here}
% \end{sidewaysfigure}
%
%
%
% \begin{sidewaystable}
% \caption{}
% \begin{tabular}
% Table layout here.
% \end{tabular}
% \end{sidewaystable}
%
%

