% Created 2020-07-08 Wed 16:02
% Intended LaTeX compiler: pdflatex
\documentclass[11pt]{article}
\usepackage[utf8]{inputenc}
\usepackage[T1]{fontenc}
\usepackage{graphicx}
\usepackage{grffile}
\usepackage{longtable}
\usepackage{wrapfig}
\usepackage{rotating}
\usepackage[normalem]{ulem}
\usepackage{amsmath}
\usepackage{textcomp}
\usepackage{amssymb}
\usepackage{capt-of}
\usepackage{hyperref}
\usepackage[margin=1in]{geometry} \usepackage[parfill]{parskip}
\date{}
\title{}
\hypersetup{
 pdfauthor={Osamu Miyawaki},
 pdftitle={},
 pdfkeywords={},
 pdfsubject={},
 pdfcreator={Emacs 26.3 (Org mode 9.4)}, 
 pdflang={English}}
\begin{document}


\section*{Reviewer 1}
\label{sec:org70fe662}
Framing the large scale circulation: I did not understand what was meant when you wrote that the ``large-scale circulation'' contributed to the over-prediction, even after reading lines 65--78 (which I had interpreted as a useful summary of interesting/relevant results by others). I initially thought you were referring to the CO\(_2\) effect on the circulation change. But the ``large-scale circulation'' simply means that a moist adiabatic response should really only be expected in regions of deep convection (and not in regions of large-scale descent). It wasn't until Section 3.2 that I understood what as meant and my reaction to Figure 2 was that ``no one expects a moist adiabatic scaling relative to the eastern Pacific surface temperature response.'' Perhaps making the meaning of ``large-scale circulation'' more clear with something like this (perhaps the topic sentence(s) or concluding sentence(s) 65--78): ``The tropical atmospheric circulation is zonally asymmetric with regions of climatological ascent and descent. Since the tropospheric temperature is only expected to be coupled to the surface temperature in regions of deep atmospheric convection, a moist adiabatic temperature scaling does not necessarily hold in areas of large scale descent or in the tropical average. As such, the presence of the tropical atmospheric circulation may influence the scaling between tropical tropospheric and surface temperature change. In contrast, aquaplanets do not include a tropical, zonally symmetric circulation, and may adhere more closely to moist adiabatic warming throughout the tropics.''

This manuscript includes a tremendous amount of information. At times, the many complementary (and subtly different) comparisons can dilute the take home messages. I think it would be extremely useful to try to incorporate the findings into a summary figure that could illustrate the estimated contribution that each process makes to the moist adiabatic warming over-estimation. For example, it is very helpful that you state the contribution to the over-prediction from the circulation and CO\(_2\) in the abstract. I don't think you actually state how you arrive at these estimates in the main text. I've constructed a schematic of the type of figure I was thinking of (you could include other simulations, but I think the key is to allow the reader to see the contribution each process makes to the over-prediction). This would essentially combine the information of Figure 1 and 3.

A similar comment regarding the presentation in the text. The average over-prediction values are listed in Sections 3.1--3.3 (rather than the estimate change attributed to each physical process). It would be extremely helpful to 1) re-order the presentation of experiments into a more linear narrative, 2) emphasize a subset of the experiments (and use the others of further evidence of your claims), and 3) focus more on the impact of each process (rather than the mean over-prediction for each experiment). For example, you could:
\begin{itemize}
\item start by stating in Section 3.1 that the over-prediction is \(23.7\%\) in abrupt4\(\times\)CO\(_2\) and the magnitude is similar in amipFuture+4\(\times\)CO\(_2\) and amip4K+4\(\times\)CO\(_2\) runs (\(20.1\) and \(21.1\%\)).
\item then in Section 3.2, you could state that when you look in regions of deep convection, the abrupt4\(\times\)CO\(_2\) over-prediction drops by \(7.1\%\) to \(16.6\%\). This is supported by the results from the amipFuture+4\(\times\)CO\(_2\) and amip4K simulations where the overprediction drops by \(7.1\) and \(3.8\%\). Last, you could point out that the over-prediction in aqua4K+4\(\times\)CO\(_2\) is similar to the amipFuture+4\(\times\)CO\(_2\) and amip4K+4\(\times\)CO\(_2\) simulations (as you do in lines 292--294).
\item in Section 3.3, you could shift your focus to the amipFuture+4\(\times\)CO\(_2\)/amipFuture simulations and state that when you remove the direct effect of CO\(_2\) the overprediction falls by \(3.7\%\) to \(9.3\%\). As support, you could note that AMIP and aqua simulations also show declines in the over-prediction when the effects of CO\(_2\) changes are removed.
\item in section 3.4 you could state that entrainment also contributes to the over-prediction (and perhaps state an estimate from the GFDL model experiments).
\item this narrative would mean that in each section, you are removing one piece of complexity (first circulation, then CO\(_2\), then entrainment) and emphasizing a subset of experiments (e.g., abrupt+amip, and using the others to bolster your claims, e.g., aqua). I think this would make it a bit easier to follow an dunderstand what the many experiments are telling us.
\end{itemize}

\textbf{TODO: discuss and respond.}

Title: You do a bit more than ``quantify.'' You could consider noting that you identify key processes contributing to the over-estimation.

\textbf{TODO: discuss and respond.}

Key Point 2 / Line 19: Consider ``after accounting for the presence of a large-scale climatological circulation and the direct effect of CO\(_2\) on circulation changes''.

\textbf{Revised text following the reviewer's suggestion. TODO: refer to edited line number.}

Line 15: Consider making it clear that warming is amplified relative to the surface (or lower troposphere) by inserting ``surface.'' Here you say that the amplification is in response to CO\(_2\) increases, but in other places you are careful to say that it is due to surface warming (and not the direct effect of CO\(_2\)). Maybe you could address this subtlety with ``greenhouse warming'' in place of ``increased CO\(_2\)''?

\textbf{Revised text following the reviewer's suggestion. TODO: refer to edited line number.}

Line 21: It wasn't immediately clear how to interpret these numbers, because the range of overprediction is large across CMIP5. So this would account for \(\approx40\%\) or \(\approx75\%\) of the over-prediction, depending on the CMIP5 model considered.

\textbf{TODO: discuss and respond.}

Plain Language Summary: There are some places where this reads a bit jargon-y and emphasizes detailed, field specific results rather than broader take-home messages. Consider simplifying this a bit and/or focusing on the implications for the research. For example, you could center the discussion around rising plumes (which leads to thermodynamic heating as water vapor condenses into clouds and rain droplets). This would allow you to substitute out some specialty-specific language such as ``mixing of dry environmental air into moist ascent'' into something like ``dry air mixes into the rising plume, which dilutes the water vapor content and reduces warming from condensation.'' I view this as an optional editorial comment since plain-language summaries are new enough where there isn't a standard on which audience these should be geared to.

\textbf{TODO: discuss and respond.}

Line 16, 30, 47: Consider in one or more places clarifying that this is increased ``atmospheric'' CO\(_2\).

\textbf{Revised text following the reviewer's suggestion. TODO: refer to edited line number.}

Line 21: I wasn't initially sure how to interpret this. It would be helpful to say that these values are the multimodel average contributions to the over-prediction.

\textbf{Revised text following the reviewer's suggestion. TODO: refer to edited line number.}

Line 50: One paper that is useful to demonstrating this in models is \cite{santer_amplification_2005}, because it shows that amplification occurs relative to surface warming irrespective of timescale.

\textbf{Added reference following the reviewer's suggestion. TODO: refer to edited line number.}

Line 48--51: Use a comma in place of the first and (``and'' is used twice in this sentence). Consider inserting ``\ldots{}and \textbf{high-resolution} cloud-resolving models (CRMs)\ldots{}''

\textbf{Revised text following the reviewer's suggestion. TODO: refer to edited line number.}

Line 53: Here or at the discussion of the \cite{andrews_dependence_2018} paper. I think a useful point to add that the stability also affects the cloud response (e.g., \cite{zhou_impact_2016}).

\textbf{Revised text and added reference following the reviewer's suggestion. TODO: refer to edited line number.}

Line 60--62: This is a useful example, but will be sensitive to the assumptions (e.g., boundary layer relative humidity and the surface temperature). Consider appending your assumptions, e.g., (``\ldots{}predicts warming aloft of 10 K (for a typical tropical surface temperature of XXX K and a relative humidity of YY\%.)'')

\textbf{Revised text following the reviewer's suggestion. Note that previously, we computed the predicted warming of 10 K aloft by inputing the CMIP5 multi-model mean response of surface temperature and relative humidity. In the revised text, the predicted warming of 9 K corresponds to the simpler assumption of 4 K warming starting at a surface temperature of 298 K and a fixed relative humidity of \(80\%\).  TODO: refer to edited line number.}

Line 68: Consider referencing \cite{sobel_enso_2002}, which I believe motivated some of the work that you cite.

\textbf{TODO: read paper and respond.}

Line 75--76: Consider inserting ``\textbf{largely} confined'' since this isn't evident from their Figure 5 (though they do say ``largely confined'' in their text).

\textbf{Revised text following the reviewer's suggestion. TODO: refer to edited line number.}

Line 79 onwards: I was unclear about what you meant by the ``direct effect of CO\(_2\).'' Is this the ``fast response''? I don't think this was adequately defined. You might re-frame this, while simultaneously defining what you mean by the direct effect: ``Changes in carbon dioxide result in changes in precipitation and the atmospheric circulation in the absence of surface temperature change \cite{bony_robust_2013}. This so-called direct effect of CO\(_2\) on atmospheric temperature change is nearly uniform in height\ldots{}''

\textbf{Revised text following the reviewer's suggestion. TODO: refer to edited line number.}

Line 79--84: Should this be described as a circulation/precipitation response? The aquaplanet model shows a similar profile of response (but has no zonally asymmetric circulation in the deep tropics). Could this be alternatively described as the atmosphere coming into balance (with the radiative effects of CO\(_2\)) with a different atmospheric temperature profile?

\textbf{TODO: discuss and respond}

Line 90--91: Consider ``unvarying'' instead of ``climatological''

\textbf{Revised text following the reviewer's suggestion. TODO: refer to edited line number.}

Line 98: Consider inserting ``\ldots{}in response to \textbf{greenhouse gas-induced surface} warming'' [I realize you don't force all experiments with CO\(_2\) changes, but you classify the AMIP style experiments as characterizing the ``indirect effect of CO\(_2\) change'' so I think this may still apply]

\textbf{Revised text following the reviewer's suggestion. TODO: refer to edited line number.}

Line 107: Define CMIP5

\textbf{Defined the acronym CMIP5. TODO: refer to edited line number.}

Line 107--109: Consider clarifying these are experiments, with language like ``\ldots{}in 29 models using the abrupt\(4\times\)CO\(_2\) and piControl experiments, respectively.''

\textbf{Revised text following the reviewer's suggestion. TODO: refer to edited line number.}

Line 113--114: It isn't clear what ``indirect effect'' of CO\(_2\) increase is and this is the first time you use the term. Perhaps you could explain the meaning in the introduction (I assume it is the surface warming response to CO\(_2\)).

\textbf{The indirect effect indeed refers to the surface warming effect. We revised the introduction (insert line number here) to clarify this terminology. TODO: refer to edited line number.}

Line 115: I was confused by amipF. Is this officially ``amipFuture''?

\textbf{We use amipF as an abbreviation for amipFuture. As this is not an official acronym, we clarified this in the text (insert line number here). TODO: refer to edited line number.}

Line 115--119: Similar to the comment at line 107, perhaps you could simply add ``experiment,'' e.g., ``(amip4K \textbf{experiment})''

\textbf{Revised text following the reviewer's suggestion. TODO: refer to edited line number.}

Line 123: This is a little unclear. Is the qObs information important to note here? If so, maybe just add a couple sentences to explain this. I assume the SSTs were derived from an aquaplanet with a mixed layer ocean with some prescribed heat flux (qObs)?

\textbf{TODO: discuss and respond.}

Line 135--136: This is a useful point to make and a good way to motivate the values you chose. Consider including other works here or perhaps in the introduction (\cite{jang_simulation_2013}, \cite{ham_what_2013}, \cite{kim_ninosouthern_2011}).

\textbf{Revised text and added references following the reviewer's suggestion. TODO: refer to edited line number.}

Line 142--143: Above, you say that this parameter only comes into play for plumes rising above 500 hPa. If most of the variations in the entrainment occur above 500 hPa when varying \(\alpha\), consider just averaging above 500 hPa.

\textbf{As $\alpha$ only sets the minimum entrainment rate, there are significant variations in entrainment below 500 hPa as well (TODO: attach figure). In addition, it is also important to consider the entrainment profile below 500 hPa for the zero-buoyancy bulk-plume models, where entrainment rates are varied throughout the entire troposphere.}

Line 155--158: Consider breaking this into two sentences: ``We compare the tropical tropospheric temperature response to surface warming in aquaplanet models and zero-buoyancy bulk-plume models subject to varying entrainment rates. We consider bulk-plume models from \ldots{}''

\textbf{Revised text following the reviewer's suggestion. TODO: refer to edited line number.}

Line 166: Consider replacing ``to be'' with ``which is'' or ``so that the R16 model is''

\textbf{Revised text following the reviewer's suggestion. TODO: refer to edited line number.}

Line 168: Replace ``the literature.'' with ``each model's respective publication.'' (assuming this is the case)

\textbf{Revised text following the reviewer's suggestion. TODO: refer to edited line number.}

Line 170: I assume this is tropical? 20 N--S? Over land and ocean? Or was this done at each grid cell? Suggest specifying a bit more here.

\textbf{We calculate the moist adiabat at each grid cell, then take the tropical average. We clarified this in the text (insert line number here). TODO: refer to edited line number.}

Eq. 2: This isn't exactly what is on the AMS website, but I trust that it is equivalent.

\textbf{The two equations are mathematically equivalent as} \(\Gamma_d=\frac{g}{c_{pd}}\) \textbf{in height coordinates. We choose to write the equation this way as it is also valid in pressure coordinates, where} \(\Gamma_d=\frac{R_dT}{pc_{pd}}\).

SI Table 2, 4, 5: should this be ``indistinguishable from zero''?

\textbf{Corrected captions for SI Tables 2, 4, and 5.}

Line 185--186: It would be very useful to look at this and include a statement that this assumption does not matter. I think \cite{flannaghan_tropical_2014} include the effects of freezing in their appendix.

\textbf{show effect of freezing and respond.}

Line 199--200: I think this means that you use each model's vertical velocity field to derive ascent regions (not a multimodel average)?

\textbf{Correct, the ascent regions are derived separately for each model. We do this to account for differences in ascent regions across models. However, we also compute the multi-model mean ascent region for illustrative purposes in Figure 2.}

Line 207: It seems like it would be useful to point this out in Section 3.3: isn't this table essentially showing that CO\(_2\) is important? If you reference this in Section 3.3, it might be worthwhile to analyze the overprediction in regions of ascent (since you will have already discussed the large-scale circulation component).

\textbf{discuss and respond.}

Line 209 / Figure 1: I would encourage you to start with the amipFuture\(+4\times\)CO\(_2\), amip4K\(+4\times\)CO\(_2\), and aqua4K\(+4\times\)CO\(_2\) results, which are more comparable to abrupt\(4\times\)CO\(_2\). This would help with the flow as you get to Section 3.3. See major comments.

\textbf{discuss with major comments and respond.}

Line 222: Consider saying ``smaller'' in regions of deep convection (since you go on to show that other factors matter and there is still substantive over-prediction).

\textbf{Revised text following the reviewer's suggestion. TODO: refer to edited line number.}

Line 225--227: One point that isn't made in this paper is that the tropical upper tropospheric warming should be relatively uniform and so some of these pattern effects (e.g., imperfect scaling in the eastern Pacific) is due to the remote influence of the tropical western Pacific over the eastern Pacific surface.

Line 240--241: Please state the experiment you are using. I assume these are the amip4K\(+4\times\)CO\(_2\) experiments, but it's also possible you are quantifying this as the difference between the (amip\(4\times\))CO\(_2\) minus amip4K simulations. Where do you use the amip\(4\times\)CO\(_2\) experiment (mentioned in line 118)?

\textbf{revisit after discussing major comments and respond.}

Line 246: Consider replacing ``non-zero'' with ``the''

Figure 4: Consider plotting the moist adiabat for reference.

\textbf{revise plot.}

Line 269: Do you know why this scales with the logarithm of entrainment?

\textbf{discuss and respond.}

Line 289: It would be worthwhile to quantify the average contribution of the circulation by taking the difference of the over-prediction in the tropical average and the ascent region. Actually, I see you do this in the abstract, so it should also be quantified in the text/conclusion.

\textbf{revisit after discussing major comments and respond.}

Line 295: Similar comment. You could quantify this with (amip4K\(+4\times\)CO\(_2\) minus amip4K, amipF\(+4\times\)CO\(_2\) minus amipF, and aqua4K\(+4\times\)CO\(_2\) minus aqua4K).

\textbf{revisit after discussing major comments and respond.}

Line 300: Similar comment. Could you look at the y-intercept in Figure 4c and 4d versus the overprediction from the standard entrainment value to estimate the impact of entrainment? Or perhaps you can turn entrainment to zero in the bulk-plume models?

\textbf{compute the y-intercept, discuss, and respond.}

Line 317--320: Why was it not included here? Is it complicated to use or have a lot of free parameters that would complicate the story?

\textbf{discuss and respond.}

Figure S5 and similar box and whisker plots: Is the range of the blue lines the \(5\)--\(95\%\) CI and the red box +/- one standard deviation? The caption seems reversed.

\textbf{The caption labels are correct. The \(5\)--\(95\%\) confidence interval (CI) of the mean is related to the standard deviation \(\sigma\), mean \(\mu\), and sample size \(n\) as} \(\mu\pm1.96\frac{\sigma}{\sqrt{n}}\). \textbf{In Figure S5, the sample sizes are close to 9, so the CI is approximately} \(\mu\pm\frac{2}{3}\sigma\). \textbf{Thus, it makes sense that the CI of the mean is smaller than the standard deviation.}

\section*{Reviewer 2}
\label{sec:orgf0edd62}

\bibliographystyle{apalike}
\bibliography{../../../../../../mnt/c/Users/omiyawaki/Sync/papers/references}
\end{document}
