% Created 2020-09-11 Fri 12:11
% Intended LaTeX compiler: pdflatex
\documentclass[11pt]{article}
\usepackage[utf8]{inputenc}
\usepackage[T1]{fontenc}
\usepackage{graphicx}
\usepackage{grffile}
\usepackage{longtable}
\usepackage{wrapfig}
\usepackage{rotating}
\usepackage[normalem]{ulem}
\usepackage{amsmath}
\usepackage{textcomp}
\usepackage{amssymb}
\usepackage{capt-of}
\usepackage{hyperref}
\usepackage[margin=1in]{geometry} \usepackage[parfill]{parskip}
\date{}
\title{}
\hypersetup{
 pdfauthor={Osamu Miyawaki, Zhihong Tan, Tiffany Shaw, Malte Jansen},
 pdftitle={},
 pdfkeywords={},
 pdfsubject={},
 pdfcreator={Emacs 26.3 (Org mode 9.4)}, 
 pdflang={English}}
\begin{document}


\section*{Reviewer 1}
\label{sec:org6394099}
I am grateful to the authors for their effort in improving the manuscript and for addressing my comments. The reorganization of Figure 1 is very helpful (Figure 4 is a bit duplicative, but I think it helps drive the main points home). I believe this is the first author's first paper. If so, congratulations on a very solid and comprehensive first paper.

One remaining comment on the presentation: In reading this version, I found section 2.1 to be quite compact and it may be difficult to follow for people not already familiar with these experiments. It would be worthwhile to add brief descriptions of the experiments before discussing how you evaluate the various contributions to the overprediction.

\textbf{We thank the reviewer for the additional comments. We revised the manuscript to address the concerns they raised. Our responses to their comments are presented in bold. The line numbers referenced in our response correspond to those in the unannotated version of our revised manuscript.}

Key Point 1: You had written ``Moist adiabatic adjustment'' in the first draft, which seems more accurate, but I am assuming you ran out of characters. Consider shortening another part of this (e.g., ``in climate models'' instead of ``across the model hierarchy'') so that you could use ``moist adiabatic adjustment.''

\textbf{Revised text following the reviewer's suggestion (see line 8).}

Key Point 3: I think this should be ``show that''

\textbf{Revised text following the reviewer's suggestion (see line 12).}

Line 17 (and in other places, e.g., line 99 and 108): You haven't said what the ``climate model hierarchy'' is --- consider saying something simpler like ``in climate models.'' In other places you say things like ``\emph{the} model hierarchy,'' but the reader doesn't know what this means exactly. Consider defining the model hierarchy and using ``a'' instead of ``the'' in places.

\textbf{We now specifically refer to the climate model hierarchy as the CMIP5 model hierarchy (lines 17, 103, 107).}

Line 23: It might be useful to indicate in the abstract that decreasing entrainment reduces the overprediction (here you discuss that a decreased Tokioka parameter reduces overprediction).

\textbf{Revised text following the reviewer's suggestion (see line 23--24).}

Line 32: I think you can delete ``Understanding''.

\textbf{Revised text following the reviewer's suggestion (see line 32).}

Line 54: Is there a problem with this reference? I think it has the author's first initial in the citation.

\textbf{This reference format follows the AGU style guideline}\footnote{https://www.agu.org/Publish-with-AGU/Publish/Author-Resources/Grammar-Style-Guide\#referenceformat}\textbf{, which states that the first initial of the author should be included if two different first authors share the same last name.}

Line 63/177: I think this should be ``\textbf{surface} relative humidity''.

\textbf{We revised line 63 following the reviewer's suggestion. As the SO13 model takes the tropospheric relative humidity profile (not surface relative humidity) as an input, we keep our phrasing as it is in line 186.}

Line 64: I don't think ``and humidity'' makes sense here (maybe and ``humidity change''?)

\textbf{Revised text following the reviewer's suggestion (see line 64--65).}

Line 73--76: I'm not sure it will be clear what you mean by ``can be accounted for.'' Consider something like ``We therefore expect that tropospheric warming would be best approximated by a moist adiabat in regions of deep convection, but may not replicate the scaling between tropical average surface and tropospheric warming. Previous work has shown that changes in tropical tropospheric temperature are closely connected to changes in precipitation-weighted surface temperature, which is a proxy for the surface temperature change in regions of moist convection (\ldots{}).''

\textbf{We revised lines 73--77 to clarify what the referenced papers show.}

Line 77--81: It might confuse people not familiar with the ``direct effect'' that it has no effect on global surface temperature and the definition of these terms could easily be missed by readers. Consider reminding the reader about these definitions. ``The tropospheric temperature increases in response to grennhouse warming of the Earth's surface (indirect effect). As noted earlier, increasing concentrations of atmospheric carbon dioxide also have a ''direct effect`` on the tropospheric radiative budget, large-scale circulation, and precipitation. This so-called direct effect of CO\(_2\) can also influence the magnitude of tropical tropospheric warming.''

\textbf{We simplified the first sentence in this paragraph to make it more clear that the direct effect measures the tropospheric temperature response to changes in CO\(_2\) with fixed SSTs (see lines 78--80). We prefer to not reiterate the definition of the indirect effect of CO\(_2\) here as we only focus on the direct effect in this paragraph.}

Line 83--84: On reading this section, I realize that I now have the same question as Reviewer 2 from the last review: If the direct effect leads to enhanced tropospheric warming, why does CO\(_2\) increase the overprediction? You answered this question in the review response. It might be worth alluding to the RH effect in this paragraph.

\textbf{We now discuss the expectation that the moist adiabat predicts no warming aloft in the absence of surface warming (see lines 80--82). We add that the moist adiabat may still overpredict the temperature response due to changes in the near-surface air temperature and relative humidity induced by the direct effect of CO\(_2\) (see lines 82--84)}.

Line 93: This was also shown in \cite{po-chedley2019} (Figure S2) for CMIP5 historical simulations (as noted in the previous draft). \cite{santer2005} also show this.

\textbf{We previously removed the} \cite{po-chedley2019} \textbf{reference as entrainment was not the main focus of their paper. However, we agree with the reviewer that Fig. S2 in} \cite{po-chedley2019} \textbf{demonstrate that the temperature response predicted by the moist adiabat overpredicts that of the CMIP5 responses and that the zero-buoyancy bulk-plume model is in better agreement with the GCM temperature responses than a moist adiabat. We now include the} \cite{po-chedley2019} \textbf{reference in the introduction (see lines 95--100) and the discussion (see lines 333--336).}

Line 98: Should this be the moist adiabatic prediction in response to \emph{surface temperature and humidity changes} (rather than CO\(_2\) changes) compared to climate models forced by changes in atmospheric CO\(_2\) concentration?

\textbf{We prefer to keep the original text as the changes in surface temperature and humidity we investigate here are those induced by increases in CO\(_2\).}

Line 110--119: I am familiar with these experiments and initially found this difficult to read (see major comments above). Consider briefly describing the experiments themselves and then how you quantify the different terms. A table could be useful for this section (rather than showing the difference between experiments in-line).

\textbf{We added descriptions for each experiment and formatted the expression for quantifying the total response as equations to improve readability (see lines 111--133).}

Line 111--113: Consider breaking this into 2--3 sentences (the full sentence is trying to explain the total effect, but it is a little confusing).

\textbf{Revised text following the reviewer's suggestion (see line 120--126).}

Line 115--116: Consider alluding to the purpose of these abbreviations by including ``patterned'' and ``uniform'' in the description.

\textbf{Revised text following the reviewer's suggestion (see lines 126 and 128).}

Line 120--121: This is a strong statement; consider using language like ``is expected to apply\ldots{}''

\textbf{We removed this sentence in the revised text (see line 134).}

Line 126: From the preceding lines, I thought the criteria was defined using the 75th percentile of vertical velocity? You could reverse the order to make this clear, e.g., We use the criteria of \(-35\) hPa/d\ldots{} This value corresponds to the 75th percentile in the multimodel average\ldots{}

\textbf{Revised text following the reviewer's suggestion (see line 134--138).}

Line 135: Consider making this ``tropical tropospheric'' instead of ``tropical''?

\textbf{Revised text following the reviewer's suggestion (see line 146--147).}

Line 161: Should this be ``meridional'' instead of ``zonal''?

\textbf{While the original phrase was written as intended (zonally symmetric meaning zonally averaged), we simplified the text by refering to the prescribed SST profile in GFDLaqua as the same profile used in the CMIP5 AQUA simulations (see lines 164--165).}

Eq. 3: Consider denoting that Delta T is a function of pressure (p) instead of a subscript.

\textbf{Revised text following the reviewer's suggestion (see equation~(9) and lines 233--235).}

SI Table 5: This was in the middle of the figures rather than with all the other tables in the beginning.

\textbf{We now place Supplementary Table S5 before the Supplementary Figures.}

Line 255: Would ``driven by'' be more accurate than ``amplified by''? My understanding is this is the root cause of the overprediction from the CO\(_2\) effect -- it would be helpful to be more clear about this point.

\textbf{We revised lines 277--280 to clarify the role of the relative humidity change on overprediction. Even when the change in surface relative humidity is ignored, the moist adiabat overpredicts the temperature response at 300 hPa due to the small but significant 2 m air temperature warming associated with the direct effect of CO\(_2\) (see Supplementary Fig. S6). The increase in surface relative humidity further increases the overprediction aloft, which is the reason we choose to describe the role of increasing surface relative humidity as amplifying the overprediction.}

Line 259--260: Consider removing ``parameterized'' here.

\textbf{We removed this sentence in the revised text (see line 282).}

Line 304: Should you mention ``relative humidity'' as well? It seems like this is the key for the CO\(_2\) effect.

\textbf{Revised text following the reviewer's suggestion (see line 316).}

\clearpage
\section*{Reviewer 2}
\label{sec:orgd73312e}
The authors have addressed some of my concerns. I have following questions about the revision.

\textbf{We thank the reviewer for their additional comments. The reviewer's comments below prompted us to review our comparison of the entrainment rate in the RAS scheme to those in the bulk plume and spectral plume models. This resulted in a number of important changes as outlined below. It is now clear that the bulk plume fractional entrainment rate $\epsilon(z)$ can be diagnosed from the RAS scheme and directly compared to the bulk plume models. The spectral plume fractional entrainment rate profile $\epsilon[z_d]$ used in ZX19 cannot be directly compared with the entrainment rate output from the RAS scheme because the entrainment rate of a plume in RAS decreases with height. Thus, we diagnose $\epsilon[z_d]$ using the saturation MSE profile in GFDL as discussed below. Moreover, the spectral plume entrainment rate cannot directly be compared to the bulk plume entrainment rate. Given the two distinct types of entrainment rates we present the comparison with the bulk plume and spectral plume models separately (see new Fig. 3c--f). Our responses to the reviewer's comments are presented in bold. The line numbers referenced in our response correspond to those in the unannotated version of our revised manuscript.}

Line 142: How is the convective entrainment quantified at each level? \emph{We use the convective entrainment output directly by the RAS scheme in units of 1/m on the standard output pressure levels for the GFDL model. We rephrased this sentence for clarity (see lines 154--156).}

Line 155: ----- It makes more sense to understand the GCM results using the spectral-plume model, as in \cite{zhou2019}. We added the \cite{zhou2019}, hereafter ZX19, model to the revised manuscript with the same parameters used in their paper.

The RAS convective scheme assumes a spectrum of convective clouds. Each subgroup is characterized by a single entrainment rate. It is not clear to me what the entrainment rate outputted at each vertical level refers to? Does it mean the average entrainment rate of the convective clouds that pass this level or the entrainment rate of the clouds that detrain at this level? The physical meaning of this output and why its vertical mean can be used as the bulk entrainment in simple plume models need to be clarified.

I am also curious why not compute the ``bulk'' entrainment rate (used in simple bulk-plume models) directly from the average entrainment rate of the ensemble convection (probably prescribed in the model source code).

\textbf{We revised our methodology to properly diagnose the entrainment rates in GFDL. We now diagnose the bulk plume entrainment rate in GFDL using the bulk plume continuity equation:}
\begin{equation}
\epsilon = \frac{1}{M}\left(\frac{\partial M}{\partial z}+d\right),
\end{equation}
\textbf{where \(z\) is height in m, \(M\) is the convective mass flux in kg m}\(^{-2}\) \textbf{s}\(^{-1}\) \textbf{and \(d\) is the detrainment mass flux per unit height in kg m}\(^{-3}\) \textbf{s}\(^{-1}\). \textbf{\(M\) and \(d\) are output from the RAS scheme. We average the bulk plume entrainment rate over pressure between 850--300 hPa to compute the mean strength of entrainment in the free troposphere.}

For ZX19, it closely follows the concept of RAS, assuming a spectrum of convective clouds with distinct entrainment rate. For each subgroup, the entrainment rate determines the level where the convective cloud detrains. The relation between the entrainment rate and the detrainment height of the plume is written as (see Fig. 8 in \cite{arakawa1974} for initial motivation of such approximation)

\begin{equation}
\epsilon = \epsilon_0 f(h/H)
\end{equation}

where h is the height and H is the tropopause height. f(h/H) decreases with h/H from 1 to 0 (plumes with smaller entrainment rate detrains at higher levels). With f(0)=1, \emph{the entrainment parameter \(\epsilon_0\) in ZX19 refers to the entrainment rate of the cloud that detrains immediately at the lowest model level instead of the mean(\(\epsilon\)).}

\textbf{While the ZX19 model and the RAS scheme} \cite{moorthi1992} \textbf{are conceptually similar in that they both assume a spectrum of entraining plumes, there are differences that complicate a direct comparison of the two models. For example, the ZX19 model assumes that each plume has a constant entrainment rate \(\epsilon[z_d]\), whereas the RAS scheme assumes that the entrainment rate of a plume decreases with height such that the updraft mass flux increases linearly, rather than exponentionally as in} \cite{arakawa1974} \textbf{\!\!. Thus, there is no output from the RAS scheme that is directly comparable to \(\epsilon[z_d]\) in ZX19. Given this limitation, we choose to diagnose the spectral plume entrainment rate in GFDL such that the following criteria is satisfied:}
\begin{equation}
\overline{h}^*(z_d) = h_{\epsilon[z_d]}(z_d),
\end{equation}
\textbf{where} \(\overline{h}^*\) \textbf{is the saturation moist static energy (MSE) in GFDL and} \(h_{\epsilon[z_d]}\) \textbf{is the MSE of a plume with entrainment rate \(\epsilon[z_d]\).}

The paper shows that ZX19 works less well than the bulk-plume model (Fig. 3c; Fig. S6c). This comes as a surprise to me because the ZX19 model is more close to reality by design and works well (better than bulk-plume models) to explain the observed over-prediction. Furthermore, since the ZX19 model closely follows the concept of the RAS scheme, it should reasonably reproduce its behavior.

\textbf{We identified a mistake in our previous implementation of the ZX19 model where we used the ensemble entrainment rate in the lapse rate equation where the single-plume entrainment rate should have been used. We corrected this error and verified that the temperature deviation from a moist adiabat predicted by the ZX19 model most closely follows the climatology of GFDLrce as expected for \(z_t=14.61\) km, \(\epsilon_0=0.33\) km}\(^{-1}\)\textbf{, and \(k=1.00\) (see Fig. 1d below).}

It is not clear from the paper how the paper change the \(\epsilon_0\) parameter when modifying the Tokioka parameter or if the plume model has been correctly implemented to capture the GCM profile. Such validation can be done by plotting the vertical profiles of the temperature deviation (from moist adiabat estimated in these simple models against those in the GCMs over the regions that are dominated by the entrainment effect).

\textbf{We tested the sensitivity of the ZX19 results to varying \(\epsilon[z_d]\) by varying either \(z_t\), \(\epsilon_0\), or \(k\) independently, and found that varying \(\epsilon[z_d]\) by varying \(k\) most closely follows the GFDLrce results (see Fig. 2 below). Thus, we now present the ZX19 results by varying \(k\) instead of \(\epsilon_0\) (see Fig. 3e--f in manuscript).}

\textbf{In summary, we now present the results of the zero-buoyancy bulk-plume models separately from the spectral plume models to reflect the two distinct types of entrainment rates used in these models (see Fig. 3c--f). We discuss the revised procedure in Section 2.2.1. As a result of these revisions, the ZX19 model closely follows that of GFDL. We revised the text to reflect this improved fit (see lines 295--297).}

\begin{figure}[htbp]
\centering
\includegraphics[width=.9\linewidth]{./deviation.png}
\caption{\label{fig:org081ec3f}Temperature deviation from a moist adiabat in GFDLrce for a prescribed SST of 300 K (black dashed) and 304 K (red dashed). The corresponding predictions of the temperature deviations are shown for a) the SO13 zero-buoyancy bulk-plume model (solid) for \(\hat{\epsilon}=0.7\) and \(\mathrm{RH}=85\%\), b) the R14 zero-buoyancy bulk-plume model for \(\epsilon=0.3\) km\(^{-1}\) and \(\alpha=0.8\), c) the R16 zero-buoyancy bulk-plume model for \(a=0.25\) and \(\mathrm{PE}=1\), and d) the ZX19 spectral-plume model for \(\mathrm{RH}=65\%\), \(z_t=14.61\) km, \(\epsilon_0=0.33\) km\(^{-1}\), and \(k=1.00\).}
\end{figure}

\begin{figure}[htbp]
\centering
\includegraphics[width=.9\linewidth]{./figs1.png}
\caption{\label{fig:orgf00720e}The relationship between spectral entrainment rate \(\epsilon[z_d]\) and overprediction obtained by the ZX19 model are shown as dash-dot lines compared to the a) GFDLrce and b) GFDLaqua results where \(\epsilon[z_d]\) in ZX19 is varied by varying \(\epsilon_0\) while holding \(z_t\) and \(k\) fixed. c) and d) are the same except \(\epsilon[z_d]\) is varied by varying \(z_t\) while holding \(\epsilon_0\) and \(k\) fixed. For e) and f), \(\epsilon[z_d]\) is varied by varying \(k\) while holding \(z_t\) and \(\epsilon_0\) fixed.}
\end{figure}

\bibliographystyle{apalike}
\bibliography{../../../../../../mnt/c/Users/omiyawaki/Sync/papers/references}
\end{document}
