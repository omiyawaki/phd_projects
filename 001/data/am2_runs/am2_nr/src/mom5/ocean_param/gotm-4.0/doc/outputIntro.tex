%
%$Id: outputIntro.tex,v 20.0 2013/12/14 00:13:20 fms Exp $
%

\section{Saving the results \label{sec:output} }

GOTM provides an easily extendible interface for storing calculated results.
The main specifications are given via the {\tt output} namelist in
{\tt gotmrun.inp}. The most important member in this namelist is
the integer {\tt out\_fmt}. Changing this variable will select the output format
--- presently ASCII and NetCDF are supported.

In GOTM output is triggered by {\tt do\_output()}
called inside the main integration loop (see \sect{sec:gotm}).
Completely separated from the core of GOTM, a format specific
subroutine is called to do the actual output.  We strongly recommend
to use the NetCDF format --- mainly because it is well established and
save --- but also because a large number of graphical programmes can
read NetCDF. Another reason is the powerful package `nco' which
provides some nice programs for manipulating NetCDF files. Information
about how to install and use NetCDF and nco can be found at 
\begin{itemize}
 \item {\tt http://www.unidata.ucar.edu/packages/netcdf} and 
 \item {\tt http://nco.sourceforge.net}.
\end{itemize}


