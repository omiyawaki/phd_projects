%
%$Id: airseaIntro.tex,v 20.0 2013/12/14 00:13:05 fms Exp $
%

\section{Air--Sea interaction \label{sec:airseaIntro}}

\subsection{Introduction}

This module provides the surface forcing for GOTM. For all dynamic
equations, surface boundary conditions need to be specified.  For the
momentum equations described in \sect{sec:uequation} and
\sect{sec:vequation}, these are the surface momentum fluxes $\tau_x^s$ and
$\tau_y^s$ in N\,m$^{-2}$. For the temperature
equation described in \sect{sec:temperature}, it is the total surface heat flux,
\begin{equation}
  Q_{tot}=Q_E+Q_H+Q_B
\end{equation}
in W\,m$^{-2}$ that has to be determined\footnote{Note, that $Q_{tot}$
has to be divided by the mean density and the specific heat capacity
to be used as a boundary condition in \eq{TEq}, since this equation is
formulated in terms of the temperature, and the the internal
energy}. The total surface heat flux $Q_{tot}$ is calculated as the
sum of the latent heat flux $Q_E$, the sensible heat flux $Q_H$, and
the long wave back radiation $Q_B$.  In contrast to the total surface
heat flux $Q_{tot}$, the net short wave radiation at the surface,
$I_0$, is not used as a boundary condition but as a source of heat, as
calculated by means of equation \eq{Iz}, see \cite{PaulsonSimpson77}.
For the salinity equation described in \sect{sec:salinity}, the fresh
water fluxes at the surface are given by the difference 
of the evaporation and the 
precipitation, $p_e$, given in m\,s$^{-1}$, see also the surface boundary
condition for salinity, \eq{S_sbc}. 
The way how boundary conditions for the
transport equations of turbulent quantities are derived, is discussed
in \sect{sec:turbulenceIntro}.

There are basically two ways of calculating the surface heat and
momentum fluxes implemented into GOTM. They are either prescribed (as
constant values or to be read in from files) or calculated on the
basis of standard meteorological data which have to be read in from
files. The necessary parameters are 
the wind velocity vector at 10 m height in m\,s$^{-1}$,
the sea surface temperature (SST in Celsius), 
air temperature in Celsius), air humidity (either as relative humidity
in \%, as wet bulb temperature or as dew point
temperature in Celsius) and air pressure (in hectopascal), each at 2 m height
above the sea surface, and
the wind velocity vector at 10 m height in m\,s$^{-1}$. 
Instead of the observed SST,
also the SST from the model may be used.  For the calculation of these
fluxes, the bulk formulae of \cite{Kondo75} or \cite{Fairalletal96}
are used. 
