%
%$Id: turbulenceIntro.tex,v 20.0 2013/12/14 00:13:21 fms Exp $
%

\section{The turbulence model \label{sec:turbulenceIntro}}

To close the differential equations for momentum, heat, and salt,
parameterisations of the turbulent fluxes of momentum, $\mean{u' w'}$, heat
$\mean{w'\theta'}$, and salinity $\mean{w's'}$ are required. Since
turbulence `feels' the effects of temperature and salinity
fluctuations essentially through buoyancy fluctuations, only the
buoyancy flux, $\mean{w'b'}$, is discussed in the following. The
assumptions under which one can infer the fluxes of heat and salinity
from the buoyancy flux are addressed below.

\subsection{Introduction}

There are different types and levels of closure models available in
GOTM to compute the vertical turbulent fluxes. Simple models rely on
the idea that theses fluxes can be computed as the product of a
positive turbulent diffusivity and a mean flow gradient. Contributions
to the fluxes that are not `down-gradient', are summarized in
so-called counter-gradient terms. Using these assumptions, the fluxes
of momentum and buoyancy can be expressed as
\begin{equation}
  \label{fluxes}
  \mean{u' w'}      = - \nu_t  \partder{u}{z}  + \tilde{\Gamma}_U \comma
  \mean{v' w'}      = - \nu_t  \partder{v}{z}  + \tilde{\Gamma}_V \comma
  \mean{w' b'}      = - \nu^B_t \partder{B}{z} + \tilde{\Gamma}_B
\comma
\end{equation}
where $\tilde{\Gamma}_{(U,V,B)}$ denote the counter-gradient fluxes.
They can be important under very strong stratification and in the case
of convection. Note, that the current version of GOTM identifies the
diffusivities of heat and salt with $\nu^B_t$ (see
\sect{sec:temperature} and \sect{sec:salinity}).  

Using an analogy to the kinetic theory of gases, the vertical
turbulent diffusivities, $\nu_t$ and $\nu^B_t$, are often assumed to be the
product of a typical velocity scale of turbulence, $q$, times a
typical length scale, $l$, see \cite{TennekesLumley72}. The velocity
scale $q$ can e.g.\ be identified with the average value of the
turbulent fluctuations expressed by the turbulent kinetic energy, $k=
q^2 /2$.  Then, the diffusivities of momentum and heat can be written
as
\begin{equation}
  \label{nu}
  \nu_t = c_\mu k^\frac{1}{2} l  \comma   \nu^B_t = c_\mu' k^\frac{1}{2} l  
  \comma
\end{equation}
where the dimensionless quantities $c_\mu$ and ${c_\mu}'$ are usually
referred to as the `stability functions'.  Depending on the level of
turbulent closure, these stability functions can be either constants, 
empirical functions, or functions of some non-dimensional flow parameters
resulting from a higher-order turbulence model. The same applies to
the counter-gradient fluxes $\tilde{\Gamma}_{(U,V,B)}$ defined in \eq{fluxes}.

There are different possibilities in GOTM to compute the scales $q$ (or
$k$) and $l$ appearing in \eq{nu}. According to the level of
complexity, they are ordered in GOTM in the following fashion.

\begin{enumerate}
\item Both, $k$ and $l$ are computed from algebraic relations. The
 algebraic equation for $k$ is based on a simplified form of the
 transport equation of the turbulent kinetic energy. The
 equation for the length-scale may result from different approaches. The
 most simple models assume an empirically motivated, prescribed
 vertical distribution of the length-scale. This level of closure
 corresponds to the `level 2' model of \cite{MellorYamada82}, but also
 to more recent approaches, see \cite{Chengetal2002}. Algebraic models
 are an over-simplification in numerous situations.

\item At the next level, $k$ is computed from the differential
 transport equation for the turbulent kinetic energy (`energy models'). As
 before, the length-scale is computed from an empirically or theoretically
 based relation. Models of this type are quite popular in geophysical
 modelling. A description is given in \sect{sec:algebraiclength}.

\item In the so-called two-equation models, both, $k$ and $l$, are computed from
 differential transport equations. As before, $k$ follows from the
 transport equation of the turbulent kinetic energy. Now, however, also
 the length-scale is determined from a differential transport
 equation. This equation is usually not directly formulated for the
 length-scale, but for a related, length-scale determining
 variable. Presently, there are different possibilities for the
 length-scale determining variables implemented in GOTM, such as the
 rate of dissipation, $\epsilon$, or the product $kl$. They are
 discussed in \sect{sec:updateLength}.

 The main advantage of the two-equation models is their greater
 generality. There are, for example, a number of fundamental flows
 which cannot be reproduced with an algebraically prescribed
 length-scale. Examples are the temporal decay of homogeneous
 turbulence, the behaviour of turbulence in stratified
 homogeneous shear flows, and the spatial decay of shear-free
 turbulence from a planar source. A discussion of these points is
 given in \sect{sec:generate} and \sect{sec:analyse}. Also see
 \cite{Umlaufetal2003} and \cite{UmlaufBurchard2003}.
\end{enumerate}

In addition to the hierarchy of turbulence models in terms of their
methods used to compute the turbulent kinetic energy and the
length-scale, GOTM also supports an ordering scheme according to the
extent to which transport equations for the turbulent fluxes are solved.
\begin{enumerate}
\item At the lowest level of this scheme, it is postulated that
 $c_\mu=c_\mu^0$ and $c'_\mu=c'^0_\mu$ are constant. Because these
 models implicitly assume an isotropic tensor relation between the velocity
 gradient and the tensor of the Reynolds-stresses, they usually fail
 in situations of strong anisotropy, most notably in stably
 stratified, curved or shallow flows. In unstratified flows with
 balanced aspect ratios (which seldom occur in nature), however, they
 may compute reasonable results. Models of this type are referred to
 as the `standard' models in the following.
\item Some problems associated with standard versions of the models can
 be ameliorated by making $c_\mu$ and $c'_\mu$ empirical functions of
 one or several significant non-dimensional flow parameters. At this
 level, the simplest approach would be to formulate empirical
 relations suggested from observations in the field or in the
 laboratory. An example of such a relation is the model of
 \cite{SchumannGerz95} which has been implemented in GOTM (see
 \sect{sec:sg}).
\item Another, more consistent, approach results from the solution of 
 simplified forms of the transport equations for the Reynolds-stresses
 and the turbulent heat fluxes in addition to the transport equations
 for $k$ and the length-scale determining variable.  Surprisingly, it
 turns out that under some assumptions, and after tedious algebra, the
 turbulent fluxes computed by these models can be expressed by
 \eq{nu}. The important difference is, however, that the existence of
 vertical eddy diffusivities is not a postulate, but a consequence
 of the model. The stability functions $c_\mu$ and $c'_\mu$ can be
 shown to become functions of some non-dimensional numbers like

 \begin{equation} 
   \label{alphaMN} 
    \alpha_M = \frac{k^2}{\epsilon^2} M^2 \comma 
    \alpha_N = \frac{k^2}{\epsilon^2} N^2 \comma
    \alpha_b = \frac{kk_b}{\epsilon^2}    \comma
 \end{equation}

 with the shear-frequency, $M$, and the buoyancy frequency, $N$,
 computed as described in \sect{sec:uequation} and
 \sect{sec:stratification}, respectively. $k$ and $k_b$ are the
 turbulent kinetic energy and the buoyancy variance, respectively and
 $\epsilon$ denotes the rate of dissipation.

 The most well-known  models of this type have been
 implemented into GOTM. An up-to-date account of their derivation can
 be found in \cite{Canutoetal2001a}. Their evaluation for the oceanic
 mixed layer has been extensively discussed by
 \cite{BurchardBolding2001}.
\item Even more complete models include further differential equations for the
 buoyancy variance and for some or all of the turbulent fluxes. These
 models cannot be reduced to the form \eq{nu}. The derivation of models
 of the type discussed in the latter two points are reviewed in 
 \sect{sec:secondOrder}
\end{enumerate}

Evidently, this short introduction cannot serve as an
introductory text on one-point turbulence modelling. It serves merely
as a place to define the most important quantities and relations used
in this manual. Readers not familiar with this subject will certainly
feel the need for a more in-depth discussion. An excellent introduction
to turbulence is still the book of \cite{TennekesLumley72}. A modern
and detailed approach to one and two-equation models for unstratified
flows is given in the book of \cite{Wilcox98}, and the effects of
stratification are discussed e.g.\ by \cite{Rodi87} and by
\cite{Burchard2002a}.



\subsection{Second-order models\label{sec:secondOrder}}

Since one-point second-order models are an essential part of GOTM,
this section is devoted to a detailed discussion of the derivation and
the properties of these models. Second-order models result from the
full or approximate solution of the transport equations for the
turbulent fluxes like $\mean{u'u'}$, $\mean{u'w'}$, $\mean{w'b'}$,
etc. Model equations for the turbulent momentum fluxes follow directly
from the Navier-Stokes equations. The derivation of these equations
for stratified and rotating fluids is discussed e.g.\ in
\cite{Sander98a}.

 Considering the one-point correlations for the velocity fluctuations
 $u'_i$, the momentum fluxes can be expressed as
\begin{equation}
  \label{momentumFlux}
  \meandot{u'_i u'_j} - {\cal D}_{ij} = 
  P_{ij} + G_{ij} + F_{ij} + \Phi_{ij} - \epsilon_{ij}
  \comma 
\end{equation}
where ${\cal D}_{ij}$ is the sum of the viscous and turbulent
transport terms and $\dot{\mean{\cdots}}$ denots the material derivative
of the ensemble average. The shear-production, $P_{ij}$, and the
buoyancy production, $G_{ij}$, on the right hand side are defined as
\begin{equation}
  \label{PijandGij}
  P_{ij} = - \mean{u'_i u'_m} \partder{U_j}{x_m} - \mean{u'_j u'_m} \partder{U_i}{x_m} \comma
  G_{ij} =  \delta_{i3} \mean{u'_j b'}  +  \delta_{j3} \mean{u'_i b'} \comma
\end{equation}
where $b'$ is the fluctuating part of the buoyancy, defined
analogously to the mean buoyancy, $B$, in \eq{DefBuoyancy}. The tensor
of the dissipation rate is assumed to be isotropic, leading to
$\epsilon_{ij} = 2/3 \epsilon \delta_{ij}$. $\Phi_{ij}$ denotes the
pressure redistribution terms discussed below. The influence of the
Coriolis-acceleration can be summarized in the tensor $F_{ij}$ which
is, however, neglected in the current version of GOTM.

The contraction of \eq{momentumFlux} yields the equation for the
turbulent kinetic energy, \eq{tkeA}, with production terms
defined by 
\begin{equation}
  \label{PG}
   P = \dfrac{1}{2} P_{ii} \comma G = \dfrac{1}{2} G_{ii}
  \point
\end{equation}

Similar to \eq{momentumFlux}, the transport equation for the turbulent 
buoyancy flux is given by
\begin{equation}
  \label{buoyancyFlux}
  \meandot{u'_i b'} - {\cal D}^b_{i} =  
  - \mean{u'_i u'_m} \partder{B}{x_m} - \mean{u'_m b'} \partder{U_i}{x_m} 
  + F^b_i + 2 \delta_{i3} k_b + \Phi^b_i - \epsilon^b_{i} 
  \comma
\end{equation}
where $ {\cal D}^b_{i}$ denotes the viscous and turbulent transport
terms, see \cite{Sander98a}. For the dissipation, one has
$\epsilon^b_{i}=0$, following from isotropy. The redistribution terms
$\Phi^b_i$ are discussed below. As in \eq{momentumFlux}, the Coriolis term
$F^b_i$ is neglected in the current version of GOTM.

Note that $k_b$ is half the buoyancy variance and relates to the
turbulent potential energy, $E_p$, according to
\begin{equation}
  \label{defkb}
    k_b = \mean{b'^2}/2 = E_p N^2 
   \comma	
\end{equation}
where the square of the buoyancy frequency, $N^2$, is defined in
\eq{DefBuoyancyFrequency}. 

The crucial point in \eq{momentumFlux} is the model for the
pressure-strain correlation. The most popular models in engineering
trace back to suggestions by \cite{Launderetal75} and
\cite{GibsonLaunder76}. With the modifications suggested of
\cite{Spezialeetal91}, this model can be written as
\begin{equation}
  \label{Phi}
    \Phi_{ij} = 
    - c_1 \tau^{-1}_u k \; b_{ij} 
    + c_2 k S_{ij} + c_3 k \Sigma_{ij} + c_4 k Z_{ij} 
    + c_5 k N_{ij} +  c_6 \Gamma_{ij}
   \comma
\end{equation}
usually extended by the last term to account for the effects of
buoyancy, see \cite{GibsonLaunder76}, \cite{GibsonLaunder78}. The
model \eq{Phi} is expressed here in terms of the dimensionless tensor
of the stress anisotropies,
\begin{equation}
  \label{bij}
   b_{ij} = \dfrac{\mean{u'_i u'_j}}{2k} - \dfrac{1}{3} \delta_{ij}
  \comma
\end{equation}
and two traceless and symmetric tensors,
\begin{equation}
  \label{SigmaZ}
  \Sigma_{ij} = S_{im} b_{mj} + S_{jm} b_{mi} - \frac{2}{3} S_{mn} b_{mn} \delta_{ij} \comma
       Z_{ij} = W_{im} b_{mj} + W_{jm} b_{mi}
   \comma
\end{equation}
which depend on the symmetric and the anti-symmetric parts of the velocity gradient,
\begin{equation}
  \label{SandW}
  S_{ij} = \frac{1}{2} \left( L_{ij} + L_{ji} \right) \comma
  W_{ij} = \frac{1}{2} \left( L_{ij} - L_{ji} \right) 
  \quad \text{with} \quad
  L_{ij} =  \partder{U_i}{x_j}
  \point
\end{equation}
Buoyancy enters via the symmetric and traceless tensor
\begin{equation}
  \label{Gammaij}
  \Gamma_{ij} = - \left( G_{ij} - \frac{2}{3} G \delta_{ij} \right) 
  \comma
\end{equation}
with $G_{ij}$ as defined in \eq{PijandGij}. In view of the derivation
of Explicit Algebraic Models (EASMs), the models implemented in GOTM
neglect the term $N_{ij}$ on the right hand side of \eq{Phi}, which is
non-linear in $b_{ij}$, see \cite{Spezialeetal91}. $c_1$--$c_6$ are
model constants. In geophysical applications, in contrast to
engineering, virually all authors used $c^*_1=0$ in \eq{Phi}. In GOTM,
the return-to-isotropy time scale $\tau_u$ is identified with the 
dynamic dissipation time scale
\begin{equation}
  \label{DefTau}
   \tau = \dfrac{k}{\epsilon}
   \comma
\end{equation}
which is a reasonable model assumption in many applications
(\cite{Canutoetal2001a}, \cite{Jinetal2003}).


For Explicit Algebraic Heat Flux Models, a quite general model for the
pressure buoyancy-gradient correlation appearing in \eq{buoyancyFlux}
can be written as
\begin{equation}
  \label{Phib}
  \begin{array}{rcl}
    \Phi^b_i  &=& - c_{b 1} \tau^{-1}_b \; \mean{u'_i b'} 
    + c_{b 2} S_{ij} \mean{u'_j b'} + c_{b 3} W_{ij} \mean{u'_j b'} \\[3mm]
    & & + c_{b 4} \mean{u'_i u'_j} \partder{B}{x_j} - 2 c_{b 5} k_b \delta_{i3}
  \comma
  \end{array}
\end{equation}
where $\tau_b=\tau$ is adopted for the return-to-isotropy time scale.

The models \eq{Phi} and \eq{Phib} correspond to some recent models
used in theoretical and engineering studies (\cite{Soetal2003a},
\cite{Jinetal2003}), and generalize all \emph{explicit} models so far
adopted by the geophysical community (see \cite{Burchard2002a},
\cite{BurchardBolding2001}). With all model assumptions inserted,
\eq{momentumFlux} and \eq{buoyancyFlux} constitute a closed system of 9 coupled
differential equations, provided the dissipation time scale $\tau$ and 
the buoyancy variance $k_b$ are known. Models for the latter two quantities
and simplifying assumptions reducing the differential equations to
algebraic expressions are discussed in the following subsection.


\subsection{Algebraic Models\label{sec:ASM}}


The key assumptions in deriving algebraic models have been formulated by 
\cite{Rodi76} and \cite{GibsonLaunder76}. These authors suggested
to simplify the right hand sides of \eq{momentumFlux} and \eq{buoyancyFlux}
according to
\begin{equation}
  \label{Rodi}
    \meandot{u'_i u'_j}  - {\cal D}_{ij} = \dfrac{\mean{u'_i u'_j}}{k} 
    \left( \dot{k} - {\cal D}_k \right) \comma
    \meandot{u'_i b'} - {\cal D}^b_{i}   = \dfrac{ \mean{u'_i b'} }{2} 
    \left( \dfrac{\dot{k} - {\cal D}_k }{k} + \dfrac{\dot{k}_b - {\cal D}_b  }{k_b}  \right) 
   \comma
\end{equation}
which are reasonable approximations in many situations. Moreover, \eq{Rodi}
can be shown to hold exactly in stably stratified, homogeneous shear flows,
when the flow approaches the so-called weak-equilibrium limit, see
\cite{Shihetal2000}. Using \eq{Rodi} and the pressure-strain model
\eq{Phi}, it can be shown after some algebra that the transport equations
 for the momentum flux \eq{momentumFlux} reduces to
\begin{equation}
  \label{bASM} {\cal N} b_{ij} = - a_1 \overline{S}_{ij} - a_2
  \overline{\Sigma}_{ij} - a_3 \overline{Z}_{ij} - a_4
  \overline{N}_{ij} - a_5 \overline{\Gamma}_{ij}
\end{equation}
in dimensionless form. The $a_i$ relate to the coefficients used in
\eq{Phi} according to $a_1 = 2/3 - c_2/2$, $a_2=1 - c_3/2$, 
$a_3=1 - c_4/2$, $a_4=c_5/2$, and $a_5=1/2 - c_6/2$.  
The dimensionless, traceless and symmetric tensors appearing on the right hand side of
\eq{bASM} are defined as
\begin{equation}
  \label{Sbar}
  \overline{S}_{ij}      = \dfrac{k}{\epsilon}      S_{ij}      \comma
  \overline{\Sigma}_{ij} = \dfrac{k}{\epsilon} \Sigma_{ij}      \comma
  \overline{Z}_{ij}      = \dfrac{k}{\epsilon}      Z_{ij}      \point
\end{equation}
Additionally,
\begin{equation}
  \label{Gij}
  \overline{\Gamma}_{ij} = \Gamma_{ij} / \epsilon = 
  \begin{pmatrix}
    - \frac{2}{3} \gamma_3 &           0             &      \gamma_1           \\
            0              & - \frac{2}{3} \gamma_3  &      \gamma_2           \\
       \gamma_1            &     \gamma_2            &  \frac{4}{3} \gamma_3  \\
  \end{pmatrix}
  \comma
  \gamma_i = - \dfrac{\mean{u'_i b'}}{\epsilon}
\end{equation}
has been introduced in \eq{bASM} for convenience. Here, the $\gamma_i$
correspond to the \emph{mixing efficiencies} in each coordinate
direction, respectively.  Note, that the vertical component,
\begin{equation}
  \label{gamma}
   \gamma_3 = \gamma  = - \dfrac{\mean{w'b'}}{\epsilon} 
  = - \dfrac{G}{\epsilon} = \dfrac{R_f}{1-R_f} \comma R_f = - \dfrac{G}{P}
  \comma
\end{equation}
can be identified with the classical mixing efficiency used in many
studies of stratified fluids.


Most authors proceed know in deriving, with the help of \eq{Rodi}, a
dimensionless equation for the normalised turbulent buoyancy flux,
$\zeta_i = \mean{u'_i b'} / \sqrt{(k k_b)}$, see \cite{Soetal2002a},
\cite{Jinetal2003}. It can be shown, however, that the resulting
algebraic equations alternatively can be expressed, without further
assumptions, in the form of equations for the mixing efficiencies,
\begin{equation}
  \label{gammaASM}
  {\cal N}_b \gamma_{i} = 
  - a_{b 1} \overline{S}_{ij} \gamma_j 
  - a_{b 2} \overline{W}_{ij} \gamma_j 
  + a_{b 3} b_{ij} \overline{N}_{j}  + \dfrac{1}{3} a_{b 3} \overline{N}_{i} 
  - a_{b 4} \overline{T} \delta_{i3}
  \point 
\end{equation}
Since efficiencies $\gamma_i$ are the primary variables appearing on
the right hand side of \eq{bASM} through the presence of the tensor
$\overline{\Gamma}_{ij}$ defined in \eq{Gij}, and since they are
variables with a clear physical interpretation, we prefer
\eq{gammaASM} to a mathematicall equivalent equation for the
normalised buoyancy flux, $\zeta_i$.

The new dimensionless quantities entering the problem via \eq{gammaASM} are 
\begin{equation}
  \label{Nbar}
  \overline{N}_i =  \dfrac{k^2}{\epsilon^2} \partder{B}{x_i} \comma
  \overline{T}   =  \dfrac{k k_b}{\epsilon^2}
  \point
\end{equation}
Note that the vertical component of $\overline{N}_i$ can be identified with
the square of the buoyancy frequency, $N^2$, made dimensionless with the dynamic 
dissipation time scale $\tau=k/\epsilon$.

\eq{bASM} and \eq{gammaASM} are linear in $b_{ij}$ and $\gamma_i$, with a non-linear
coupling introduced by the terms
\begin{equation}
  \label{NandNb}
  \begin{array}{rcl}
    {\cal N} &=&   \dfrac{P+G}{\epsilon} + \dfrac{c_1}{2} - 1                \\[5mm] 
    {\cal N}_b &=& \dfrac{1}{2} \left( \dfrac{P+G}{\epsilon}   - 1  \right) 
    + c_{b 1} + \dfrac{1}{2r}   \left( \dfrac{P_b}{\epsilon_b} -1   \right)
   \point
  \end{array}
\end{equation}
The production-to-dissipation ratios appearing in these expression are
exclusively related to known quantities and thus introduce no new
independent variables. However, the time scale ratio,
\begin{equation}
 \label{DefR}
  r = \dfrac{k_b}{\epsilon_b} \dfrac{\epsilon}{k}
\end{equation}
needs to be described. 

\eq{bASM} and \eq{gammaASM} are a system of 9 coupled algebraic 
equations for the anisotropies $b_{ij}$ and the mixing efficiencies
$\gamma_i$, depending solely on the non-dimensional tensors $\overline{S}_{ij}$,
$\overline{W}_{ij}$, the vector $\overline{N}_{i}$, and the scalar 
$\overline{T}$. This system is linear, if ${\cal N}$ and ${\cal N}_b$ are 
treated as knowns and if the nonlinear term $N_{ij}$ in
\eq{bASM} is neglected, $a_4=0$. No closed solution of the complete
system in three dimensions has been reported so far in the
literature. Nevertheless, separate solutions in three dimensions for
\eq{bASM} and \eq{gammaASM}, respectively, have been reported (see \cite{Jinetal2003}
and the references therein).

In geophysical applications, the system \eq{bASM} and \eq{gammaASM}
can be considerably simplified by assuming that the fluid is
horizontally homogeneous (boundary layer approximation), and closed
solutions can be obtained (see \cite{Chengetal2002}). The procedure to
obtain such solutions is discussed in the following subsection.


\subsection{Explicit models for vertical shear and stratification\label{sec:EASM}}

In the following, we restrict ourselves to flows with vertical shear
and stratification, and assume that mean quantities are horizontally
homogeneous.  Under these conditions,
\eq{Nbar} yiels $\overline{N}_1=\overline{N}_2=0$ and
\begin{equation}
  \label{NbarVertical}
   \overline{N}_3 =  \dfrac{k^2}{\epsilon^2} \partder{B}{z}
                  =  \dfrac{k^2}{\epsilon^2} N^2
  \point
\end{equation}
The velocity gradient simplifies to
\begin{equation}
  \label{LijVertical}
  L_{ij} = 
  \begin{pmatrix}
            0             &           0             &        S_U          \\
            0             &           0             &        S_V          \\
            0             &           0             &        0            \\
  \end{pmatrix}
  \comma
\end{equation}
where $S_U=\partial U / \partial z$ and $S_V=\partial V / \partial z$ 
are the vertical shear in $U$ and $V$, respectively. 


Under these conditions, and using the conventions
\begin{equation}
 \label{overlineSN}
   \overline{S}_U   = \dfrac{k}{\epsilon} S_U \comma
   \overline{S}_V   = \dfrac{k}{\epsilon} S_V \comma
   \overline{N}^2   =  \overline{N_3}         \comma
\end{equation}
\eq{bASM} reduces to
\begin{equation}
 \label{bijVertical}
 \begin{array}{rcl}
   {\cal N} b_{11} 
   &=&
   - \left( \dfrac{a_2}{3} + a_3 \right) b_{13} \overline{S}_U 
   +        \dfrac{2}{3} a_2             b_{23} \overline{S}_V 
   +        \dfrac{2}{3} a_5             \gamma_3                          
   \comma \\[5mm]
   {\cal N} b_{22} 
   &=&
   - \left( \dfrac{a_2}{3} + a_3 \right) b_{23} \overline{S}_V 
   +        \dfrac{2}{3} a_2             b_{13} \overline{S}_U 
   +        \dfrac{2}{3} a_5             \gamma_3    
   \comma \\[5mm]
   {\cal N} b_{33} 
   &=&
   - \left( \dfrac{a_2}{3} - a_3 \right) b_{13} \overline{S}_U 
   - \left( \dfrac{a_2}{3} - a_3 \right) b_{23} \overline{S}_V 
   -        \dfrac{4}{3} a_5             \gamma_3                          
   \comma \\[5mm]
   {\cal N} b_{12} 
   &=&
   -        \dfrac{a_2 + a_3}{2}         b_{13} \overline{S}_V 
   -        \dfrac{a_2 + a_3}{2}         b_{23} \overline{S}_U 
   \comma \\[5mm]
   {\cal N} b_{13} 
   &=&
   -       \dfrac{a_2 - a_3}{2}          b_{11} \overline{S}_U 
   -       \dfrac{a_2 + a_3}{2}          b_{33} \overline{S}_U 
   -       \dfrac{a_2 - a_3}{2}          b_{12} \overline{S}_V 
   -       \dfrac{1}{2} a_1                     \overline{S}_U                                 
   -                    a_5              \gamma_1
   \comma \\[5mm]                             
   {\cal N} b_{23} 
   &=&
   -       \dfrac{a_2 - a_3}{2}          b_{22} \overline{S}_V 
   -       \dfrac{a_2 + a_3}{2}          b_{33} \overline{S}_V 
   -       \dfrac{a_2 - a_3}{2}          b_{12} \overline{S}_U 
   -       \dfrac{1}{2} a_1                     \overline{S}_V                                
   -                    a_5              \gamma_2
   \point
 \end{array}
\end{equation}
Similarly, for the mixing efficiencies, \eq{gammaASM} yields 
\begin{equation}
 \label{giVertical}
 \begin{array}{rcl}
  {\cal N}_b \gamma_1
  &=&
  - \dfrac{a_{b1} + a_{b2}}{2}           \gamma_3 \overline{S}_U
  +      a_{b3}                           b_{13}  \overline{N}^2                  
  \comma \\[5mm] 
  {\cal N}_b \gamma_2
  &=&
  - \dfrac{a_{b1} + a_{b2}}{2}           \gamma_3 \overline{S}_V
  +      a_{b3}                           b_{23}  \overline{N}^2                  
  \comma \\[5mm] 
  {\cal N}_b \gamma_3
  &=&
  - \dfrac{a_{b1} - a_{b2}}{2}           \gamma_1 \overline{S}_U
  - \dfrac{a_{b1} - a_{b2}}{2}           \gamma_2 \overline{S}_V
  +         a_{b3}                         b_{33} \overline{N}^2                              
  + \dfrac{a_{b3}}{3}                             \overline{N}^2                              
  -         a_{b4}                                \overline{T}                          
 \point
 \end{array}
\end{equation}

In geophysical applications, a reasonable assumption is
$P_b=\epsilon_b$ to elimmate the dependence of \eq{giVertical} on
$\overline{T}$. From \eq{kbeq}, using
\eq{gamma} and \eq{DefR}, it follows that $\overline{T}$ can be
expressed in the form
\begin{equation}
 \label{Tequilibrium}
   \overline{T} = r \gamma_3 \overline{N}^2
 \point
\end{equation}
With the help of \eq{Tequilibrium}, the last of \eq{giVertical} can be
re-written as
\begin{equation}
 \label{giVerticalEq}
  {\cal N}_b \gamma_3
  =
  - \dfrac{a_{b1} - a_{b2}}{2}             \gamma_1 \overline{S}_U
  - \dfrac{a_{b1} - a_{b2}}{2}             \gamma_2 \overline{S}_V
  +        a_{b3}                            b_{33} \overline{N}^2                              
  +     \dfrac{a_{b3}}{3}                           \overline{N}^2                              
  -        a_{b5}                          \gamma_3 \overline{N}^2               
 \point
\end{equation}
Note that the new parameter $a_{b5} = r a_{b4}$ depends on the time
scale ratio, $r$, and is, in general, not constant. Nevertheless,
constant $r=c_b$ is frequently assumed (see below).

In the general case, \eq{bijVertical} and \eq{giVertical} can be
inverted directly to yield a solution of the form
\begin{equation}
\label{b13}
  b_{13}   = - \dfrac{1}{2} \hat{c}_\mu \overline{S}_U \comma
  b_{23}   = - \dfrac{1}{2} \hat{c}_\mu \overline{S}_V \comma
 \gamma_3  = \hat{c}_\mu' \overline{N}^2 - \Gamma
 \comma
\end{equation}
from which, by insertion into \eq{bijVertical} and \eq{giVertical},
all other quantities can be determined. Since ${\cal N}$ and ${\cal
N}_b$ defined in \eq{NandNb} have been treated as known, the solution
is not yet completely explicit. In the numerical scheme of GOTM, they
are updated from their values at past time steps. By identifying 
\begin{equation}
  \label{nuke}
  \nu_t           = \hat{c}_\mu   \dfrac{k^2}{\epsilon} \comma
  \nu'_t          = \hat{c}_\mu'  \dfrac{k^2}{\epsilon} \comma
  \tilde{\Gamma}  = \epsilon \Gamma
  \comma
\end{equation}
\eq{b13} corresponds in form exactly to \eq{fluxes}. Note that, adopting the 
equilibrium assumption \eq{Tequilibrium}, the dependence on $\Gamma$
drops in \eq{b13}. From \eq{nu} and \eq{nuke}, and using the definition
of the dissipation rate \eq{epsilon}, it is clear that 
\begin{equation}
  \label{cmuConversion}
  \hat{c}_\mu = (c^0_\mu)^3  c_\mu \comma
  \hat{c}_\mu' = (c^0_\mu)^3 c'_\mu \comma
  \point
\end{equation}


The structure of the dimensionless parameter functions apearing in
\eq{b13} is given by
\begin{equation}
\label{cm}
  \hat{c}_\mu  = \dfrac{N_n}{D} \comma
  \hat{c}_\mu' = \dfrac{N_b}{D} \comma  
  \Gamma = \dfrac{N_\Gamma}{D}
  \comma
\end{equation}
where the numerators and the denominator are polynomials of the square
of the shear number, $\alpha_M = \overline{S}^2 = \overline{S}^2_U
+\overline{S}^2_V$, the square of the buoyancy number, $\alpha_N =
\overline{N}^2 $, the mixed scalar, $\alpha_B = \overline{T}$, and the
functions ${\cal N}$ and ${\cal N}_b$. The latter two functions depend
on the production-to-dissipation ratios for $k$ and $k_b$, which for
vertical shear and stratification can be written as
\begin{equation}
  \label{PeVertical}
 \begin{array}{rcl}
  \dfrac{P}{\epsilon}     &=& - 2 b_{13} \overline{S}_U - 2 b_{23} \overline{S}_V
                          = \hat{c}_\mu \overline{S}^2 
   \comma \\[5mm]
  \dfrac{G}{\epsilon}     &=&  - \gamma_3  = - \hat{c}_\mu' \overline{N}^2 + \Gamma 
   \comma \\[5mm]
  \dfrac{P_b}{\epsilon_b} &=&  -\dfrac{G}{\epsilon} \dfrac{\epsilon}{\epsilon_b} N^2 
                           =   - r \dfrac{G}{\epsilon} \dfrac{\overline{N}^2}{\overline{T}}
  \end{array}
  \point
\end{equation}
Once $k$ and $k_b$ (and their dissipation ratios, $\epsilon$ and $\epsilon_b$)
are known, also the time scale ratio $r$ defined in \eq{DefR} can be computed, 
and the problem can be solved. Different possibilities to derive
these quantities are discussed in the following.


\subsubsection{Equilibrium states}

Some authors use simplifying assumptions to derive more compact forms
of the expressions for the solution in \eq{b13}. In the following, a
few examples, which are special cases of the general solution
discussed here, are reviewed.

In deriving their version of the general solution \eq{b13},
\cite{Canutoetal2001a} e.g.\ assumed $P_b=\epsilon_b$ and constant
$r$. Under these conditions, because of \eq{Tequilibrium}, the
dependence on $\overline{T}$ dissapears, and the counter-gradient term
$\Gamma_B$ in \eq{b13} drops. It was further assumed that
$P+G=\epsilon$ in \eq{NandNb} only, leading to ${\cal N} = (c_1 + c_1^*)/2$
and ${\cal N}_b=c_{b1}$. These particularly simple expressions
linearize the system, and a fully explicit solution can be obtained, provided
$k$ and $\epsilon$ are known. \cite{BurchardBolding2001}
adopted the solution of \cite{Canutoetal2001a} and complemented it by
$k$ and $\epsilon$ computed from dynamical equations (`$k$-$\epsilon$
model'). 

In contrast, \cite{Canutoetal2001a} and \cite{Chengetal2002} decided
for a further simplification.  They solved \eq{b13} with $k$ and
$\epsilon$ from algebraic expressions. In their case, $k$ followed from the
approximation $P+G=\epsilon$ of \eq{tkeA} (see
\sect{sec:tkealgebraic}), and $\epsilon$ from a prescribed length-scale.

Using \eq{b13}, \eq{cm}, and \eq{PeVertical}, it is easy to show that
the assumption $(P+G)/\epsilon$ leads to
\begin{equation}
  \label{FE}
     N_n \overline{S}^2 - N_b \overline{N}^2 - D = 0   
   \comma
\end{equation}
which is polynomial equation in $\overline{S}$ and
$\overline{N}$. This expression can be used to replace one of the
latter two variables by the other. An interesting consequence is the
fact that all non-dimensional turbulent quantities can be expressed in
terms of the Richardson number $Ri = \overline{N}^2 / \overline{S}^2$
only. Replacing $\overline{N}^2$ by $\overline{S}^2 Ri$ in
\eq{FE}, a quadratic equation for $\alpha_M = \overline{S}^2$ in terms for $Ri$
can be established (see e.g. \cite{Chengetal2002}. Using the
definitions given in \sect{sec:cmueC}, this equation can be written as
\begin{equation}
 \label{SSEqRi}
    \alpha_M^2 \left( -d_5 + n_2 - \left( d_3 - n_1 + n_{b2}\right) Ri 
                                     - \left( d_4 + n_{b1}      \right) Ri^2  
                   \right)
  + \alpha_M   \left( -d_2 + n_0 - \left( d_1 + n_{b0}      \right) Ri
                   \right)
  - d_0 = 0	
  \point
\end{equation}
The solution for $\alpha_M$ can, via \eq{FE}, be used to expressed also
$\overline{N}^2$ in terms of $Ri$. This implies that also the stability functions
and hence the complete solution of the problem only depends on $Ri$. 

Investigating the solution of the quadratic equation \eq{SSEqRi}, it
can be seen that $\alpha_M$ becomes infinite if the factor in front of
$\alpha_M^2$ vanishes. This is the case for a certain value of the
Richardson number, $Ri=Ri_c$, following from 
\begin{equation}
 \label{Ric}
    -d_5 + n_2 - \left( d_3 - n_1 + n_{b2}\right) Ri_c 
               - \left( d_4 + n_{b1}      \right) Ri_c^2  = 0
  \point
\end{equation}
Solutions of this equation for some popular models are given in
\tab{tab:Ric}. For $Ri=Ri_c$, equilibrium models predict complete extinction of
turbulence. For non-equilibrium models solving dynamical equations
like \eq{tkeA}, however, $Ri_c$ has no direct signifcance, because
turbulence may be sustainned by turbulent transport and/or the rate
term.
\begin{table}[!h]
 \begin{center}
  \begin{tabular}{ccccc}                                                   
    GL78    &  KC94   &  CHCD01A  & CHCD01B  &  CCH02   \\ \hline
    $0.47$  &  $0.24$ &  $0.85$   & $1.02$   &  $0.96$      
   \end{tabular}
  \end{center}
 \caption{\label{tab:Ric}Critical Richardson number for some models}
\end{table}

\subsubsection{Stability of explicit models}
A physically reasonable condition for an explicit second order model
expressed the fact that increasing (non-dimensional) shear
$\overline{S}$ should lead to increasing vertical shear-anisotropies
of turbulence, $b_{13}$ and $b_{23}$. It has been shown by
\cite{BurchardDeleersnijder2001} that a violation of this condition
may lead to numerical instabilities of the models.

Mathematically, the shear-condition is expressed by
\begin{equation}
  \label{shearCondition}
  \partder{(b_{13}^2 + b_{23}^2)^\frac{1}{2}}{\overline{S}} 
   = \dfrac{1}{2} \partder{\tilde{c}_\mu \overline{S}}{\overline{S}}\ge 0
  \comma
\end{equation}
where \eq{b13} has been used. Using the equilibrium form of the stability
function described in \sect{sec:cmueC}, this condition leads to a
cubic equation in $\alpha_M=\overline{S}^2$. A simpler condition can
be obtained, when this equation is solved after terms multiplied by
$d_5$ and $n_2$, which usually are very small, are neglected.

The resulting approximate condition is
\begin{equation}
   \label{simpleShearCondition}
   \alpha_M \le \dfrac{d_0 n_0 + (d_0 n_1 + d_1 n_0) \alpha_N 
                          + (d_1 n_1 + d_4 n_0           ) \alpha_N^2
                          +  d_4 n_1                       \alpha_N^3}{
                             d_2 n_0 
                          + (d_2 n_1 + d_3 n_0           ) \alpha_N
                          +  d_3 n_1                       \alpha_N^2}
  \point
\end{equation}
\cite{BurchardDeleersnijder2001} showed that using \eq{simpleShearCondition}
the most well-known models yield numerically stable results. However,
for some models like those of \cite{MellorYamada82} and
\cite{KanthaClayson94}, the limiter \eq{simpleShearCondition} is
almost always `active',  and hence replaces the actual turbulence model in a
questionable way.


\subsection{Parameter conversion for other models\label{sec:parameterConversion}}
Virtually all pressure-redistribution models used in engineering and
geophysical applications can be considered as special cases of \eq{Phi}
and \eq{Phib}. However, most authors adopted a different notation and 
different parameter values. In this section, paramater conversions for the 
most well-known models are discussed. 
\subsubsection{The model of \cite{GibsonLaunder78}}
The pressure-strain model of this important class of engineering models 
has been originally suggested by \cite{Launderetal75}. It can be written
as
\begin{equation}
  \label{PhiGL78}
    \Phi_{ij} = - 2 \tilde{c}_1 \epsilon b_{ij} 
    - \tilde{c}_2 k S_{ij} 
    - \tilde{c}_3 \left( P_{ij} - \dfrac{2}{3} P \delta_{ij} \right)
    - \tilde{c}_4 \left( D_{ij} - \dfrac{2}{3} P \delta_{ij} \right)
    + \tilde{c}_6 \Gamma_{ij}
   \comma
\end{equation}
where that last term has been added by \cite{GibsonLaunder78} to 
account for the effects of gravity in stratified fluids. This term is
identical to the last term in \eq{Phi}.
The new production-of-anisotropy tensor $D_{ij}$ is defined as
\begin{equation}
  \label{Dij}
  D_{ij} = - \mean{u'_i u'_m} \partder{U_m}{x_j} - \mean{u'_j u'_m} \partder{U_m}{x_i}
  \point
\end{equation}
Using the tensor relations
\begin{equation}
  \label{PijTensor}
  \begin{array}{rcl}
   P_{ij} = - 2 k \Sigma_{ij} - 2 k Z_{ij} + \dfrac{2}{3} P \delta_{ij} 
            - \dfrac{4}{3} k S_{ij} \comma   \\[4mm]
   D_{ij} = - 2 k \Sigma_{ij} + 2 k Z_{ij} + \dfrac{2}{3} P \delta_{ij} 
            - \dfrac{4}{3} k S_{ij} \comma  
  \end{array}
\end{equation}
\eq{PhiGL78} can be re-written in the form
\begin{equation}
  \label{PhiGL78new}
    \Phi_{ij} = - 2 \tilde{c}_1 \epsilon b_{ij} 
    + \left( \dfrac{4}{3} ( \tilde{c}_3  + \tilde{c}_4 ) - \tilde{c}_2 \right)   k S_{ij} 
    + 2 \left( \tilde{c}_3  + \tilde{c}_4 \right)   k \Sigma_{ij} 
    + 2 \left( \tilde{c}_3  - \tilde{c}_4 \right)   k Z_{ij} 
    + \tilde{c}_6 \Gamma_{ij}
   \point
\end{equation}
Comparting with \eq{Phi}, the following relations can be estabilished: $c_1 = 2 \tilde{c}_1$, 
$c_2 = 4/3 ( \tilde{c}_3 + \tilde{c}_4 )- \tilde{c}_2$, 
$c_3 = 2 ( \tilde{c}_3 + \tilde{c}_4 )$, $c_3 = 2 ( \tilde{c}_3 - \tilde{c}_4 )$, 
$c_5 = 0$, and $c_6 = \tilde{c}_6$.

\cite{GibsonLaunder78} use a slightly different notation for the pressure-scambling 
model \eq{Phib}. Their model is somewhat simplified form of the model of 
\cite{Jinetal2003}, which can be written as
\begin{equation}
  \label{PhibGL78}
  \begin{array}{rcl}
    \Phi^b_i  &=& - \tilde{c}_{b 1} \dfrac{\epsilon}{k} \; \mean{u'_i b'} 
    + \tilde{c}_{b 2} L_{ij}  \mean{u'_j b'} + \tilde{c}_{b 3} L_{ji} \mean{u'_j b'} \\[3mm]
    & & + \tilde{c}_{b 4} \mean{u'_i u'_j} \partder{B}{x_j} 
    - 2 \tilde{c}_{b 5} k_b \delta_{i3}
  \point
  \end{array}
\end{equation}
Using the decomposition of the velocity gradient in its symmetric and anti-
symmetric part, \eq{SandW}, the following parameter relation are evident:
$c_{b1} = \tilde{c}_{b1}$, $c_{b2} = \tilde{c}_{b2} + \tilde{c}_{b3}$,
$c_{b3} = \tilde{c}_{b2} - \tilde{c}_{b3}$, $c_{b4} = \tilde{c}_{b4}$, 
$c_{b5} = \tilde{c}_{b5}$. 


Parameter values for this model are compiled in \tab{tab:CHCD}. `GLNEW' denotes 
the revised parameter set for the pressure-strain model given in \cite{Wilcox98} and 
for the pressure-buoyancy gradient model in \cite{Zhaoetal2001}.
\begin{table}
 \begin{tabular}{c|ccccccccccc}                                                   
        & $\tilde{c}_1$      & $\tilde{c}_2$      &  $\tilde{c}_3$      &  $\tilde{c}_4$  & $\tilde{c}_6$  &   
          $ \tilde{c}_{b1}$  & $ \tilde{c}_{b2}$  &  $ \tilde{c}_{b3}$  &
          $ \tilde{c}_{b4}$  & $ \tilde{c}_{b5}$  &  $r$                                               \\ \hline  
GL78    & $1.8$              & $0$                &  $0.6$              &  $0$            &  $0.5$         & 
          $3$                & $0.33$             &  $0$                &  
          $0$                & $0.33$             &  $0.8$                                              \\ 
GLNEW   & $1.8$              & $0$                &  $0.78$             &  $0.2545$       &  $0.3$         & 
          $3.28$             & $0.4$              &  $0$                &  
          $0$                & $0.4$              &  $0.8$                                              \\ 
 \end{tabular}
 \caption{\label{tab:LRR}Some parameter sets for the model of \cite{GibsonLaunder78}}
\end{table}




\subsubsection{The model of \cite{Canutoetal2001a}}
\cite{Canutoetal2001a} and \cite{Chengetal2002} use a model 
that is virtually identical to the traditional model of
\cite{Launderetal75} and \cite{GibsonLaunder78}. The values 
of their model parameters and their notation, however, are somewhat
different.

Looking for conversion relations, it should be noted that the
anisotropy tensor $b^\text{CCHD}_{ij}$ used by \cite{Canutoetal2001a}
is twice the tensor defined in \eq{bij}, $b^\text{CCHD}_{ij} = 2 k
b_{ij}$. Also the dissipative time scale $\tau^\text{CCHD}$ of
\cite{Canutoetal2001a} is twice the time scal defined in \eq{DefTau},
$\tau^\text{CCHD} = 2 \tau$. If one further notes that the turbulent
heat flux $h_i=\mean{u'_i \theta'}$ is related to the buoyancy flux
according to $\mean{u'_i b'} = \alpha g h_i$, relations between the
model parameters can be found.

With these relations, equation (10a) of \cite{Canutoetal2001a} can be 
re-written as
\begin{equation}
  \label{bASMCHCD2001} 
    b_{ij} = - \lambda_1  \overline{S}_{ij} 
            - 2 \lambda_2 \overline{\Sigma}_{ij} 
            - 2 \lambda_3 \overline{Z}_{ij}
            -   \lambda_4 \overline{\Gamma}_{ij}
   \point
\end{equation}
The return-to-isotropy part of the pressure-strain model
of \cite{Canutoetal2001a} reads
\begin{equation}
  \label{PhiCHCD2001}
   \Phi_{ij} = - \dfrac{2}{\lambda} \epsilon b_{ij}
   \comma
\end{equation}
from which, by comparing with \eq{Phi}, it follows that $c_1 =
2/\lambda$ and $c^*_1 = 0$, and hence from \eq{NandNb} ${\cal N} =
1/\lambda$. Thus, adopting the relations $a_1 = \lambda_1/\lambda$,
$a_2 = 2 \lambda_2/\lambda$, $a_3 = 2 \lambda_3/\lambda$, $a_4 = 0$,
and $a_5 = \lambda_4/\lambda$, \eq{bASMCHCD2001} corresponds exactly
to \eq{bASM},

Similarly, equation (10a) of \cite{Chengetal2002} can be re-expressed 
in the form
\begin{equation}
  \label{gammaASMCHCD2001}
  \dfrac{\lambda_5}{2}\gamma_{i} = 
  - \lambda_6  \overline{S}_{ij} \gamma_j 
  - \lambda_7  \overline{W}_{ij} \gamma_j 
  +     2        b_{ij} \overline{N}_{j}  
  + \dfrac{2}{3} \overline{N}_{i} 
  - \lambda_0  \overline{T} \delta_{i3}
  \point 
\end{equation}
The somewhat simpler model of \cite{Canutoetal2001a} adopts the
equilibrium assumption
\eq{Tequilibrium}, and replaces the last term in \eq{gammaASMCHCD2001} by
$-\lambda_0 r \gamma_3 \overline{N}^2 \delta_{i3}$ and, assuming
constant $r$, identifies $\lambda_0 r = \lambda_8$. The time scale
ratio $r$ is computed in equation (20a) of \cite{Canutoetal2001a}.

The return-to-isotropy part of this model (see equation (6c) of
\cite{Chengetal2002}) reads
\begin{equation}
  \label{PhibCHCD2001}
   \Phi^b_i = - \dfrac{\lambda_5}{2} \dfrac{\epsilon}{k} \mean{u'_i b'}
   \comma
\end{equation}
from which follows, by comparison with \eq{Phib} and \eq{NandNb}, that 
${\cal N}_b = c_{b1} = \lambda_5/2$. Comparing \eq{gammaASMCHCD2001} with
\eq{gammaASM} one finds, by inspection, the relations $a_{b1} = \lambda_6$, 
$a_{b2} = \lambda_7$, $a_{b3} = 2$, $a_{b4} = 2 \lambda_0$, 
and $a_{b5} = 2 \lambda_8$. Some parameter sets for this model are compiled in \tab{tab:CHCD}.
\begin{table}[!h]
 \begin{tabular}{c|cccccccccc}                                                   
        & $\lambda$    & $\lambda_0$  &  $\lambda_1$  &  $\lambda_2$  & $\lambda_3$  &   
          $\lambda_4$  & $\lambda_5$  &  $\lambda_6$  &  $\lambda_7$  & $\lambda_8$     \\ \hline  
CHCD01A & $0.4$        & $2/3$        &  $0.107$      &  $0.0032$     &  $0.0864$    & 
          $0.12$       & $11.9$       &  $0.4$        &  $0$          & $0.48$          \\ 
CHCD01B & $0.4$        & $2/3$        &  $0.127$      &  $0.00336$    & $0.0906$     & 
          $0.101$      & $11.2$       &  $0.4$        &  $0$          & $0.318$         \\ 
CCH02   & $0.4$        & $2/3$        &  $0.107$      &  $0.0032$     & $0.0864$     & 
          $0.1$        & $11.04$      &  $0.786$      &  $0.643$      & $0.547$         \\ 
 \end{tabular}
 \caption{\label{tab:CHCD}Some parameter sets for the model of \cite{Canutoetal2001a}}
\end{table}





\subsubsection{The model of \cite{MellorYamada82}}
The pressure-strain model of \cite{MellorYamada82} is expressed in
terms of $q^2=2k$ and the dissipation length scale $l=q^3/(B_1
\epsilon)$, where $B_1$ is a model constant.  The time scale ratio 
in this model is set to $r=c_b=B_1/B_2$. Using these expression,
their model can be re-written as
\begin{equation}
  \label{PhiMY82}
   \Phi_{ij} = - \dfrac{B_1}{3 A_1} \epsilon b_{ij} + 4 C_1 k S_{ij}
   \comma
\end{equation}
which, by comparison with \eq{Phi}, yields $c_1 = B_1 / (3 A_1)$ and
$c_2 = 4 C_1$. All other parameters are zero.

Similarly, the pressure-scrambling model of \cite{MellorYamada82} (using 
the extensions suggested by \cite{KanthaClayson94} and \cite{Kantha2003}) 
reads
\begin{equation}
  \label{PhibMY82}
   \Phi^b_i = - \dfrac{B_1}{6 A_2} \dfrac{\epsilon}{k} \mean{u'_i b'} 
    + C_2 ( S_{ij} + W_{ij} )  \mean{u'_j b'} - 2 C_3 k_b \delta_{i3}
    \comma
\end{equation}
which can be compared to \eq{Phib} to obtain $c_{b1} = B_1 / (6 A_2)$ and
$c_{b2} = C_2$, $c_{b3} = C_2$, $c_{b5} = C_3$. 
All other parameters are zero.

Several parameter sets suggested for this model are compiled in \tab{tab:MY}
\begin{table}[!h]
 \begin{tabular}{c|ccccccc}                                                   
      &  $A_1$   &  $A_2$  &  $B_1$  & $B_2$   &   $C_1$  &  $C_2$  &  $C_3$  \\ \hline  
MY82  & $0.92$   & $0.74$  & $16.55$ & $10.1$  &  $0.08$  & $0$     & $0$     \\ 
KC94  & $0.92$   & $0.74$  & $16.55$ & $10.1$  &  $0.08$  & $0.7$   & $0.2$   \\ 
K03   & $0.58$   & $0.62$  & $16.55$ & $11.6$  &  $0.038$ & $0.7$   & $0.2$     
 \end{tabular}
 \caption{\label{tab:MY}Some parameter sets for the model of \cite{MellorYamada82}}
\end{table}


\subsection{Numerics}

The numerical approximation of the turbulence equations is in principle
carried out as explained in section \ref{SectionNumericsMean}. 
One basic difference is however due to the fact that turbulent
quantities are generally non-negative such that
it is necessary that the discretised forms of the physical
equations retain the principle of non-negativity. 
A typical model problem would be the following:

\begin{equation}
 \label{eq:burchard11}
 \partder{X}{t} =P-QX, \quad P,Q > 0
\end{equation}
with $X$ denoting any non-negative quantity, $P$ a non-negative
source term, $QX$ a non-negative linear sink term, and $t$ denoting
time. $P$ and $Q$ may depend on $X$ and $t$. It can easily be shown
that with \eq{eq:burchard11}, $X$ remains non-negative for any
non-negative initial value $X_0$ and limited $Q$.  For the $q^2
l$-equation and the $\epsilon$-equation (described in
\sect{sec:lengthscaleeq} and \sect{sec:dissipationeq}), 
$Q$ would be proportional to $q/l$ and $\epsilon
/k�$, repsectively.

A straight-forward, explicit discretisation in time of \eq{eq:burchard11}
can be written as
\begin{equation}
  \label{eq:burchard12}
  \frac{X^{n+1}-X^n}{\Delta t}=P^n-Q^nX^n
\end{equation}
with the superscripts denoting the old ($n$) and the new ($n+1$) time
level and $\Delta t$ denoting the time step.  In this case, the
numerical solution on the new time level would be
\begin{equation}
  \label{eq:burchard13}
  X^{n+1}_i= X^n_i(1-\Delta tQ^n_i)+\Delta t P_i^n
  \comma
\end{equation}
which is negative for negative right hand side of \eq{eq:burchard12},
provided that
\begin{equation}
  \label{eq:burchard14}
  \Delta t > \frac{X^n}{X^nQ^n-P^n}
  \point
\end{equation}

Since it is computationally unreasonable to restrict the time step in
such a way that \eq{eq:burchard14} is avoided, a numerical procedure
first published by \cite{Patankar80} is generally applied
\begin{equation}
  \label{eq:burchard15}
  \frac{X^{n+1}-X^n}{\Delta t}=P^n-Q^nX^{n+1}
  \comma
\end{equation}
which yields an always non-negative solution for $X^{n+1}$,
\begin{equation}
  \label{eq:burchard16}
   X^{n+1}= \frac{X^n+\Delta t P^n}{1+\Delta t Q^n}
   \point
\end{equation}
Thus, the so-called quasi-implicit formulation
\eq{eq:burchard15} by \cite{Patankar80}
is a sufficient condition for positivity applied in almost all
numerical turbulence models.

 
