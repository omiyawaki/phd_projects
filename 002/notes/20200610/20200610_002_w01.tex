% Created 2020-06-08 Mon 16:34
% Intended LaTeX compiler: pdflatex
\documentclass[11pt]{article}
\usepackage[utf8]{inputenc}
\usepackage[T1]{fontenc}
\usepackage{graphicx}
\usepackage{grffile}
\usepackage{longtable}
\usepackage{wrapfig}
\usepackage{rotating}
\usepackage[normalem]{ulem}
\usepackage{amsmath}
\usepackage{textcomp}
\usepackage{amssymb}
\usepackage{capt-of}
\usepackage{hyperref}
\author{Osamu Miyawaki}
\date{Jun 10, 2020}
\title{Research Notes}
\hypersetup{
 pdfauthor={Osamu Miyawaki},
 pdftitle={Research Notes},
 pdfkeywords={},
 pdfsubject={},
 pdfcreator={Emacs 26.3 (Org mode 9.4)}, 
 pdflang={English}}
\begin{document}

\maketitle

\section{Introduction}
\label{sec:orga958d81}

\section{Comparing energy fluxes to Hartmann (2016)}
\label{sec:orgc6a99bb}

Shown in Figure \ref{fig:orgfcea42b}

\begin{figure}[htbp]
\centering
\includegraphics[width=.9\linewidth]{../../../prospectus/figs/fig-6-1-hartmann.png}
\caption{\label{fig:orgfcea42b}Reprint of Fig. 6.1 from \cite{hartmann_important_2002} showing the energy flux terms in the vertically-integrated MSE budget. LE is latent heat, SH is sensible heat, \(\Delta F_a\) is MSE flux divergence, and \(R_a\) is atmospheric radiative cooling.}
\end{figure}

\begin{figure}[htbp]
\centering
\includegraphics[width=.9\linewidth]{../../figures_std/tf/era-fig-6-1-hartmann.png}
\caption{\label{fig:orgd5a751b}Same as Figure \ref{fig:orgfcea42b} but reproduced using ERA-Interim data from 2000--2012. Blue is latent heat, orange is sensible heat, red is MSE flux divergence, gray is atmospheric radiative cooling, and green is the residual, which we infer as the atmospheric MSE storage.}
\end{figure}

\section{Sensitivity of RCE and RAE regimes to various parameters}
\label{sec:org6a7a86b}

\section{Next Steps}
\label{sec:org94035b8}

\bibliographystyle{apalike}
\bibliography{../../../../Papers/references}
\end{document}
