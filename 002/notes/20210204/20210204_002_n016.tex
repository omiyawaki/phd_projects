\documentclass{article}

\usepackage{graphicx}
\usepackage[margin=1in]{geometry}

\title{Research notes}
\date{\today}
\author{Osamu Miyawaki}

\begin{document}
\maketitle

\section{Assessing $R_1$ definitions in the high latitudes using spatio-temporal structure of the near-surface lapse rate}

\begin{figure}
    \includegraphics[width=0.8\textwidth]{/project2/tas1/miyawaki/projects/002/figures_post/test/lr_mon_lat/lr_mon_lat_era5c.pdf}
    \caption{The spatio-temporal structure of the lapse rate deviation from a dry adiabat averaged from 1000 hPa up to (a) 850 hPa (b) 900 hPa (c) 950 hPa. The 100\% contour is highlighted in red, which separates temperature profiles that exhibit an inversion ($>100\%$) and those that do not ($<100\%$). In all cases, an inversion-free temperature profile is found in the NH high latitudes from mid-April through October.}
    \label{fig:dalr-mon-lat-era5c}
\end{figure}

\section{Hemispheric asymmetry in surface energy budget}

\section{Mixed layer depth controls timing of midlatitude regime transition?}

\bibliographystyle{apalike}
\bibliography{../../outline/references.bib}

\end{document}
