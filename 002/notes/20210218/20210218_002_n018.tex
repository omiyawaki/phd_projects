\documentclass{article}

\usepackage{graphicx}
\usepackage[margin=1in]{geometry}
\usepackage{afterpage}

\title{Research notes}
\date{\today}
\author{Osamu Miyawaki}

\begin{document}
\maketitle

\begin{figure}
    \includegraphics[width=\textwidth]{/project2/tas1/miyawaki/projects/002/figures_post/final/temp_echam/temp_echam.pdf}
    \caption{Zonally averaged temperature profiles from the ECHAM6 slab ocean aquaplanets are shown in the midlatitudes and high latitudes for January and June. The temperature profile at 45$^\circ$ in ECHAM6 configured with (a) 15 m mixed layer depth is more stable than a moist adiabat in January and neutrally stable in June and (b) 40 m mixed layer depth is more stable than a moist adiabat yearround. The temperature profile at 85$^\circ$ in ECHAM6 with a 40 m mixed layer depth configured (c) with thermodynamic sea ice exhibits a near surface inversion in January whereas (d) without sea ice remains inversion-free yearround.}
    \label{fig:temp-echam}
\end{figure}

\bibliographystyle{apalike}
\bibliography{../../outline/references.bib}

\end{document}
