\documentclass{article}

\usepackage{graphicx}
\usepackage[margin=1in]{geometry}
\usepackage[parfill]{parskip}
\usepackage{afterpage}
\usepackage{natbib}
\usepackage{xcolor}

\title{Response to reviewers of \\ \textit{When and where do Radiative Convective and Radiative Advective Equilibrium regimes occur on modern Earth?}}
\date{\today}
\author{Osamu Miyawaki, Tiffany A. Shaw, and Malte F. Jansen}

\begin{document}
\maketitle

\textbf{We thank the editor and reviewers for their helpful comments. We revised the manuscript to address many of the concerns and suggestions they raised. Our reponses to their comments are presented in bold. The line numbers referenced in our response correspond to those in our revised manuscript.}

\section*{Editor}
I could imagine a comparison of the vertical structure of the warming response in some climate change simulations with your RCE/RAE index, indicating that this index, defined in a control climate, provides a useful guide to the vertical structure of the warming response geographically and seasonally.  I realize this is a substantial request, and I do not require it, but I think it would make for a more important paper.  You hint at this application, but do not discuss it systematically.  In any case, I would like to see more discussion of what this kind of categorization is useful for in the introduction, to help readers understand your motivation. 

\textbf{We now demonstrate the link between climatological $R_1$ and the vertical temperature response based on CMIP5 RCP8.5 simulations (see Section 5, Fig.~12, 13, 14, and 15). In addition, we expanded our discussion of why quantifying the energy balance regimes is useful in the introduction (lines 71--99). }

Also, the abstract uses the term “lapse rate regimes” that was not clear to me until I read the paper.  Perhaps you could quickly mention the lapse rate features that you are talking about (surface inversion, moist adiabat, mixed) to define this term.

\textbf{We now clarify the term ``lapse rate regimes'' as suggested in the abstract (line 10).}

\section*{Reviewer 1}
A decomposition is presented to diagnose the mechanisms responsible for the seasonal regime transitions (Section 3c, Figure 5-6). Since this is a decomposition of the seasonality of $R_1$, which does not explicitly include surface turbulent fluxes (Eqn. 5), it is not entirely clear to me how to relate all of the terms in the MSE budget back to the decomposition. As I understand the findings, in the northern midlatitudes, the seasonality of $R_1$ is dominated by the dynamic component, which includes both advection and storage. In the other cases (southern midlatitudes, high latitudes of both polar regions), the seasonality of $R_1$ is due to strong compensation between dynamic and radiative components. However, an increase in latent heat flux is also invoked to explain the presence of the regime transition in the high northern latitudes. How are the authors able to say the surface turbulent heat fluxes matter more than the seasonality of advection for the transition from RAE to RCAE?

\textbf{The dynamic and radiative components in equation~(5) are connected to the surface turbulent fluxes via the MSE budget (equation~1). We now explicitly show how the seasonality of $R_1$ is related to that of surface turbulent fluxes (see lines 227--233, equations~6--7). }

I was also confused by the usage of ``residual'' both to refer to the $\Delta R_1$ term reflecting a strong compensation between two other terms and the actual residual of the decomposition (dash-dot line in Figure 6) (L212-214).

\textbf{We apologize for the confusion. We now refer to opposing behavior of the radiative and dynamic components and only refer to the residual in the context of nonlinearities of the decomposition (lines 227--237).}

It would be helpful from a presentation standpoint if the colors didn't have different meanings on the left and right panels of Figures 5 and 6 (e.g., black is the deviation from annual mean $R_1$ on the left but storage on the right).

\textbf{The MSE tendency line in panels b and d of Fig.~5, 6, 8, B5, and B6 is now same as MSE flux divergence to be consistent with the included terms in the dynamic component of $\Delta R_1$ (panels a and c).}

An open question from Section 3b is the inability of storage to explain the discrepancy in timing of energy balance and lapse rate regimes in the Northern Hemisphere high latitudes (L191-193). Does Figure 6 provide any additional insights? Later in the manuscript, it is notable that the aquaplanet simulations with thermodynamic ice in Figure 9 show no such discrepancy. Is there any sensitivity in the relative timing of these regimes to the mixed layer depth?

\textbf{Yes, there is sensitivity in the relative timing of the regimes to the mixed layer depth. We added a figure to demonstrate this (Fig.~10a,c) and discuss it in the text (see lines 333--336).}

The aquaplanet simulations provide a nice demonstration of the role of surface heat capacity on the existence of midlatitude energy balance regime transitions. As noted in the text, the regime transition in AQUA lags that in the reanalysis data. I would find an elaboration on why this might be the case to be helpful.

\textbf{We noted in the original manuscript that the timing of the regime transition depends on mixed layer depth (lines 278--280). In particular, the regime transition occurs earlier for a shallow (3 m) depth compared to a deeper depth (15 m). However, the amplitude of $\Delta R_1$ in the 3 m AQUA simulation is three times larger than those in the reanalyses. We now demonstrate this in the new Fig.~8 and hypothesize that the discrepancy between the amplitude of $\Delta R_1$ between AQUA and reanalyses may reflect the zonally asymmetric surface heat capacity in the Northern Hemisphere midlatitudes (see lines 280--282). We also discuss the importance of studying zonal asymmetries for future work (see lines 376--378).}

I could imagine that idealized simulations with zonal variations in surface heat capacity (e.g., Voigt et al. 2016, JAMES, doi:10.1002/2016MS000748) might enable a better agreement between the aquaplanet simulations and reanalysis, though to be clear I do not think that additional simulations are necessary for this paper to be publishable in Journal of Climate.

\textbf{Thank you for the suggestion. We looked into the TRACMIP simulations. Unfortunately, they only involve zonal variations in surface heat capacity in the tropics ($30^\circ$S--$30^\circ$N). Thus, they cannot be used to understand the signal we see in the midlatitudes. However, we agree that this is an important hypothesis to test as part of future work and added this in the discussion (lines 376--378).}

The discussion of the flattened topography is somewhat brief. I appreciate the authors sharing their null result. It is also interesting that the wintertime $R_1$ is strongly affected by the removal of topography. Does the decomposition of Eqn. 5 applied to the CESM simulation confirm whether this comes purely from an increase in radiative cooling or if there are additional dynamical adjustments?

\textbf{Decomposing the difference in $R_1$ between the simulation with and without Antarctic topography following Equation~(5) shows that there is a large compensation between the radiative and dynamic components (Fig.~\ref{fig:hahn-decomp} below). This reflects the two competing effects of removing topography on $R_1$: 1) radiative cooling increases (a larger negative quantity) due to the optically thicker atmosphere, which acts to decrease $R_1$ (gray line in Fig.~\ref{fig:hahn-decomp}), and 2) MSE flux convergence increases, which acts to increase $R_1$ (red line in Fig.~\ref{fig:hahn-decomp}). The radiative effect dominates over the compensating dynamic effect with the exception of December and January. We agree with reviewer 1 on the value of sharing null results but decided not to include additional results shown in Fig.~\ref{fig:hahn-decomp} below based on the comments we received from reviewer 3.}

\begin{figure}[!h]
  \noindent\includegraphics[width=\textwidth]{/project2/tas1/miyawaki/projects/002/figures/hahn/Control1850/native/dr1/mse_old/lo/0_poleward_of_lat_-80/0_mon_decompr1z_hahn_comp.png}
  \caption{The difference in $R_1$ (black) and its decomposition into radiative (gray) and dynamic (red) components between the CESM simulations performed by \citep{hahn2020} for the flattened Antarctic topography simulation and the control simulation (with Antarctic).}
  \label{fig:hahn-decomp}
\end{figure}

Fig. 2 - I didn't notice the shading to indicate multi-reanalysis spread at first. Consider darkening (or lightening the background colors that indicate energy balance regime).

\textbf{We modified the colors to improve the visibility of the multi-reanalysis spread (see Fig.~2 and B2).}

L172-174 ``... the seasonality of the ... exhibit weak seasonality ...'' $\rightarrow$ ``... the seasonality of the ... are weak ...''

\textbf{Revised text following reviewer's suggestion (see line 182).}

L172-174 The seasonality of the free tropospheric lapse rate deviation does not seem weak in the Southern Hemisphere high latitudes. There is a minimum in September.

\textbf{We now specify that the seasonality of the free tropospheric lapse rate deviation in the midlatitudes is weak (see line 188). While it is interesting that there is a large seasonality in the free tropospheric lapse rate in the high latitudes, we do not consider it here as we focus on the existence of a surface inversion in the high latitudes.}

L242 - The radiative component is negligible in the Northern Hemisphere midlatitudes, but not in the Southern Hemisphere.

\textbf{We now specify that this assumption is satisfied well in the Northern Hemisphere midlatitudes, but less so for the Southern Hemisphere (lines \#REF). Specifically, the error from ignoring the radiative component in the Southern Hemisphere midlatitudes is associated with the phase more so than the magnitude of $\Delta R_1$. Since we use the EBM to predict the existence rather than the timing of the midlatitude regime transition, we proceed by only considering the dynamic component of $\Delta R_1$.}

Eqn. C1 - subscript missing on T* in second term, righthand side

\textbf{We added the missing subscript to Equation C1.}

I suggest stating whether the EBM solution is analytical or numerical.

\textbf{We now specify that the EBM solution is analytical (see line 151).}

L279 - If horizontal black lines in Fig. 6a,c are the annual mean $R_1$, that should be noted in the caption as well.

\textbf{We now specify that the annual mean $R_1$ is represented by the horizontal black line (see caption in Fig.~5).}

Fig. 9 - The right y-axis should be evaluated from 0.9 to 1.0 rather than 0.3 to 0.9 for the boundary layer lapse rate deviation.

\textbf{The y-axis label now shows the correct vertical bounds (see Fig.~9).}

L290-292 - Related to the second major comment above, I don't think the aquaplanet simulation with sea ice entirely captures the observed energy balance and lapse rate regimes, though it is certainly an improvement over the case without ice.

\textbf{We removed this statement and now discuss how the AQUA experiments with ice differ from the reanalyses (lines 334--339).}

\section*{Reviewer 3}

Figure 9 shows that the seasonality of high-latitude $R_1$ depends critically on the presence of sea ice. But, to my mind, sea ice has two important but physically distinct features: high albedo and low heat capacity. Could the authors tease out the relative roles of each in producing seasonal $R_1$ variations? Perhaps simulations in which sea ice has the same albedo as ocean water would help.

\textbf{Thank you for the suggestion. We investigated the relative importance of albedo and surface heat capacity by 
varying the mixed layer depth in AQUA configured without sea ice. This allows us to test whether the seasonality of $R_1$ varies with surface heat capacity while holding the albedo fixed at $\alpha=0.3$. A regime transition occurs for a shallow mixed layer depth (5 m) even in the absence of sea ice (Fig.~10b), suggesting that a high albedo is not necessary for the existence of a regime transition. We added a discussion of the sensitivity of high latitude $R_1$ in AQUA to varying mixed layer depths (lines 344--349).}

Following on to the above, it would be helpful to have a little bit more detail about the sea ice simulation. For instance, how are sea-ice depth and area fraction determined? How do these quantities, averaged over the high-latitudes, evolve seasonally? Etc.

\textbf{The sea ice implementation is the standard thermodynamic sea ice module in ECHAM6, which is represented as a single, motionless slab \citep{giorgetta2013}. We now provide a reference where the sea ice documentation is detailed (lines 153--154). The seasonal evolution of sea ice fraction and depth are shown in Fig.~2 below.}

\begin{figure}[!h]
    \centering
    \noindent\includegraphics[width=\textwidth]{/project2/tas1/miyawaki/projects/002/figures_post/final/r1_decomp_pole/r1_decomp_pole_echamslab_ice_prop.pdf}
    \caption{Seasonality of high latitude ($80^\circ$--$90^\circ$) (a) sea ice fraction and (b) sea ice depth for various mixed layer depths in AQUA with sea ice.}
    \label{fig:echam-sice}
\end{figure}

The discussion of Antarctic topography and lack of seasonality of Antarctic $R_1$ felt a bit unsatisfying, as the former might not be a reader's first guess for explaining the latter. A more obvious candidate would be the land ice boundary condition which imposes a constant high albedo. (The authors acknowledge all this in the discussion.) If the authors could explore this further, perhaps using imposed sea ice fractions and albedo to bridge the gap between simulated land ice and sea ice, that might help. If not, the authors might consider omitting the flattened Antarctica results, and simply leave the understanding of constant Antarctic $R_1$ to future work.

\textbf{We tested the hypothesis that a constant high albedo is responsible for the lack of a regime transition over Antarctica using the 25 m AQUA configured with ice. In this configuration, surface albedo remains constant ($\alpha\approx0.8$, see solid line in Fig.~\ref{fig:echam-alb-ice}a below) but undergoes a high latitude regime transition (solid line in Fig.~\ref{fig:echam-alb-ice}). This suggests that neither topography nor the seasonality of albedo alone can explain the lack of a regime transition over Antarctica. While we understand reviewer 3's thought of omitting these results from the manuscript, we agree with reviewer 1 on the value of sharing null results because it can guide the direction of future work.}

\begin{figure}[!h]
    \centering
    \noindent\includegraphics[width=\textwidth]{/project2/tas1/miyawaki/projects/002/figures_post/test/r1_mld_hl/echam25m_era5/r1_alb_25m.pdf}
    % \noindent\includegraphics[width=0.7\textwidth]{/project2/tas1/miyawaki/projects/002/figures/echam/rp000144/native/alb/albedo_mon_era5c.png}
    % \noindent\includegraphics[width=0.7\textwidth]{/project2/tas1/miyawaki/projects/002/figures/echam/rp000144/native/alb/albedo_mon_icemld.png}\\
    % \noindent\includegraphics[width=\textwidth]{/project2/tas1/miyawaki/projects/002/figures/echam/rp000144/native/alb/icemld_all_legend.png}\\
    \caption{(a) Seasonality of high latitude ($80^\circ$--$90^\circ$) surface albedo for 25 m AQUA with sea ice (solid) and over Antarctica in ERA5 (dashed). (b) Similar, but for $R_1$.}
    \label{fig:echam-alb-ice}
\end{figure}

line 34, `sum of radiative fluxes': Isn't radiative cooling the difference of the net TOA and surface fluxes?

\textbf{Thank you for catching this error. We corrected this in the text (line 35).}

line 92, 'sign definite in the zonal mean': Isn't radiative cooling sign definite column-by-column, not just in the zonal mean?

\textbf{Indeed, radiative cooling is sign definite column-by-column. We modified line 106--107 to specify that $R_a$ is sign definite in the modern climate because the atmosphere can be radiatively heated in a very cold climate such as in Snowball Earth.}

lines 220--225: It is good that the authors took the trouble of analysing CMIP5 data in addition to reanalysis. But this analysis, which requires six figures, also seems only incidental to the main storyline of the paper. The author might, at their discretion, consider moving this content to a supplement.

\textbf{As we now include results showing the connection between the temperature response to climate change and $R_1$ using CMIP5 data, we prefer to keep these figures in the appendix.}

lines 248--253: I assume that this discussion, and the associated Figure~7, concern $R_1$ *averaged over mid-latitudes*? I could not find a specification of this.

\textbf{We now specify that the results presented in Fig.~7 are averaged over the midlatitudes (see Fig.~7 caption).}

lines 294--295: The connection between atmospheric depth and radiative cooling was made explicit in \cite{jeevanjee2018}.

\textbf{Thank you for pointing us to this paper. We now cite \cite{jeevanjee2018} to support our expectation of the influence of Antarctic topography on $R_1$ (line 321).}

Fig.~1: The color of the lines corresponding to $R_1=0.9$ and $R_1=0.1$ are different from those in the other figures. Would be nice if these could be the same. Also, might help to label the RAE and RCE thresholds in the colorbar and/or the figure itself.

\textbf{We changed the colors to be consistent with the other figures and added labels for the RCE and RAE thresholds on the colorbar (see Fig.~1 and B1).}

Fig.~2: Might be useful here, and elsewhere, to directly label the shaded regions as RAE, RCAE, and RCE.

\textbf{We now label the shaded regions as RAE, RCAE, and RCE (see Fig.~2 and B2).}

Fig.~7a: I felt that this is a key figure, in that it supports the key insight that a seasonal transition from RCAE to RCE requires a relatively low heat capacity. This insight is suitably emphasized in the abstract, but I felt the main text in lines 248--253 didn't do it justice.

\textbf{We rewrote this paragraph (lines 289--296) to put more emphasis on the result that a low surface heat capacity is required for the midlatitude regime transition.}


\bibliographystyle{apalike}
\bibliography{references}

\end{document}

