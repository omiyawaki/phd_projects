
\documentclass{article}

\usepackage{graphicx}
\usepackage[margin=1in]{geometry}
\usepackage[parfill]{parskip}
\usepackage{afterpage}
\usepackage{natbib}

\renewcommand{\thefigure}{S\arabic{figure}}

\title{Supplemental Materials for \\ \textit{When and where do Radiative Convective and Radiative Advective Equilibrium regimes occur on modern Earth?}}
\date{\today}
\author{Osamu Miyawaki, Tiffany A. Shaw, and Malte F. Jansen}

\begin{document}
\maketitle

\begin{figure}[t]
  \noindent\includegraphics[width=\textwidth]{/project2/tas1/miyawaki/projects/002/figures/hahn/Control1850/native/dr1/mse_old/lo/0_poleward_of_lat_-80/0_mon_decompr1z_hahn_comp.png}
  \caption{The difference in $R_1$ (black) and its decomposition into radiative (gray) and dynamic (red) components between the CESM simulations performed by \citep{hahn2020} for the flattened Antarctic topography simulation and the control simulation (with Antarctic).}
  \label{fig:hahn-decomp}
\end{figure}

\begin{figure}[t]
  \noindent\includegraphics[width=\textwidth]{/project2/tas1/miyawaki/projects/002/figures/hahn/Control1850/native/dr1/mse_old/lo/0_poleward_of_lat_-80/0_mon_dflux_hahn_comp.png}
  \caption{The difference in energy fluxes between the CESM simulations performed by \citep{hahn2020} for the flattened Antarctic topography simulation and the control simulation (with Antarctic).}
  \label{fig:hahn-eflux}
\end{figure}

\clearpage

\bibliographystyle{apalike}
\bibliography{references}

\end{document}