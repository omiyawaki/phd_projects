\documentclass{article}

\usepackage{graphicx}
\usepackage[margin=1in]{geometry}
\usepackage[parfill]{parskip}
\usepackage{afterpage}
\usepackage{natbib}
\usepackage{xcolor}

\title{\vspace{-2.5cm}Response to Reviewer 1\vspace{-2cm}}
\date{}
\author{}

\begin{document}
\maketitle

\textbf{Thank you for your helpful comments. We revised the manuscript to address many of the concerns and suggestions you raised. Our responses to your comments are presented in bold. The line numbers referenced in our response correspond to those in our revised manuscript.}

A decomposition is presented to diagnose the mechanisms responsible for the seasonal regime transitions (Section 3c, Figure 5-6). Since this is a decomposition of the seasonality of $R_1$, which does not explicitly include surface turbulent fluxes (Eqn. 5), it is not entirely clear to me how to relate all of the terms in the MSE budget back to the decomposition. As I understand the findings, in the northern midlatitudes, the seasonality of $R_1$ is dominated by the dynamic component, which includes both advection and storage. In the other cases (southern midlatitudes, high latitudes of both polar regions), the seasonality of $R_1$ is due to strong compensation between dynamic and radiative components. However, an increase in latent heat flux is also invoked to explain the presence of the regime transition in the high northern latitudes. How are the authors able to say the surface turbulent heat fluxes matter more than the seasonality of advection for the transition from RAE to RCAE?

\textbf{We are able to say that the surface turbulent fluxes matter because the dynamic and radiative components in equation~(5) are connected to the surface turbulent fluxes via the MSE budget [equation~(2)]. We now explicitly show how the seasonality of $R_1$ is related to the seasonality of surface turbulent fluxes in the high latitudes [see lines~230--235, equations~(6)--(7)]. }

I was also confused by the usage of ``residual'' both to refer to the $\Delta R_1$ term reflecting a strong compensation between two other terms and the actual residual of the decomposition (dash-dot line in Figure 6) (L212-214).

\textbf{We apologize for the confusion. We now refer to opposing behavior between the radiative and dynamic components and only refer to the residual in the context of nonlinearities of the decomposition (lines~230--231).}

It would be helpful from a presentation standpoint if the colors didn't have different meanings on the left and right panels of Figures 5 and 6 (e.g., black is the deviation from annual mean $R_1$ on the left but storage on the right).

\textbf{We changed Fig.~5 and 6 so that the black line refers to $R_1$ and changed the atmospheric storage term to dashed red.}

An open question from Section 3b is the inability of storage to explain the discrepancy in timing of energy balance and lapse rate regimes in the Northern Hemisphere high latitudes (L191-193). Does Figure 6 provide any additional insights? Later in the manuscript, it is notable that the aquaplanet simulations with thermodynamic ice in Figure 9 show no such discrepancy. Is there any sensitivity in the relative timing of these regimes to the mixed layer depth?

% \textbf{We don't think that Fig.~6 provides any additional insights into the discrepancy because neither the seasonality of the decomposition of $\Delta R_1$ nor the seasonality of the individual heat flux terms are consistent with the seasonality of the lapse rate regimes.}
%
\textbf{We found that there is sensitivity in the relative timing of the high latitude energy balance and lapse rate regime transitions to the mixed layer depth. The seasonality of the energy balance regime transition as captured in the reanalysis data is more accurately represented by the aquaplanet configured with shallower mixed layer depths (25--40~m). In contrast, the lapse rate regime transition is more accurately represented by deeper mixed layer depths (45 and 50~m, see lines~319--329). The inability for the aquaplanet to capture the differing seasonality of energy balance and lapse rate regimes for a single mixed layer depth suggests that mechanisms missing in the aquaplanet such as zonal asymmetry may be responsible for the discrepancy (see lines~460--468).}

The aquaplanet simulations provide a nice demonstration of the role of surface heat capacity on the existence of midlatitude energy balance regime transitions. As noted in the text, the regime transition in AQUA lags that in the reanalysis data. I would find an elaboration on why this might be the case to be helpful.

\textbf{We noted in the original manuscript that the timing of the regime transition depends on mixed layer depth (lines~278--280 in the original submission). We now show the timing of the midlatitude regime transition is sensitive to the mixed layer depth (Fig.~7b). The regime transition occurs earlier for a shallower mixed layer depth (e.g., see 3 m simulation in Fig.~7b). We hypothesize that the discrepancy between the amplitude of $\Delta R_1$ between AQUA and reanalyses may reflect the zonally asymmetric surface heat capacity in the Northern Hemisphere midlatitudes (see lines~284--286). However, testing this hypothesis would require additional simulations, which is an important subject for future work (see lines~486--488).}

I could imagine that idealized simulations with zonal variations in surface heat capacity (e.g., Voigt et al. 2016, JAMES, doi:10.1002/2016MS000748) might enable a better agreement between the aquaplanet simulations and reanalysis, though to be clear I do not think that additional simulations are necessary for this paper to be publishable in Journal of Climate.

\textbf{Thank you for the suggestion. We looked into the TRACMIP simulations. Unfortunately, they only involve zonal variations in surface heat capacity in the tropics ($30^\circ$S--$30^\circ$N). Thus, they cannot be used to understand the signal we see in the midlatitudes. However, we agree that understanding the role of zonal variations in surface heat capacity for energy balance regime transitions is an important issue for future work and added this in the discussion (lines~486--488).}

The discussion of the flattened topography is somewhat brief. I appreciate the authors sharing their null result. It is also interesting that the wintertime $R_1$ is strongly affected by the removal of topography. Does the decomposition of Eqn. 5 applied to the CESM simulation confirm whether this comes purely from an increase in radiative cooling or if there are additional dynamical adjustments?

\textbf{The decomposition of the change in $R_1$ between the simulation with and without Antarctic topography following equation~(5) shows that there is a large compensation between the radiative and dynamic components (Fig.~\ref{fig:hahn-decomp} below). This reflects the two competing effects of removing topography on $R_1$: 1) radiative cooling increases (a larger negative quantity) due to the optically thicker atmosphere, which acts to decrease $R_1$ (gray line in Fig.~\ref{fig:hahn-decomp}), and 2) MSE flux convergence increases, which acts to increase $R_1$ (red line in Fig.~\ref{fig:hahn-decomp}). The radiative effect dominates over the compensating dynamic effect with the exception of December and January. We agree with you on the value of sharing null results but decided not to include additional results shown in Fig.~\ref{fig:hahn-decomp} below following Reviewer 3's comments.}

\begin{figure}[!h]
  \noindent\includegraphics[width=\textwidth]{/project2/tas1/miyawaki/projects/002/figures/hahn/Control1850/native/dr1/mse_old/lo/0_poleward_of_lat_-80/0_mon_decompr1z_hahn_comp.png}
  \caption{\bf The difference in $R_1$ (black) and its decomposition into radiative (gray) and dynamic (red) components between the CESM simulations performed by \cite{hahn2020} for the flattened Antarctic topography simulation and the control simulation with Antarctic topography.}
  \label{fig:hahn-decomp}
\end{figure}

Fig. 2 - I didn't notice the shading to indicate multi-reanalysis spread at first. Consider darkening (or lightening the background colors that indicate energy balance regime).

\textbf{We modified the colors to improve the visibility of the multi-reanalysis spread (see Fig.~2 and B2).}

L172-174 ``... the seasonality of the ... exhibit weak seasonality ...'' $\rightarrow$ ``... the seasonality of the ... are weak ...''

\textbf{This text was revised (lines~190--193).}

L172-174 The seasonality of the free tropospheric lapse rate deviation does not seem weak in the Southern Hemisphere high latitudes. There is a minimum in September.

\textbf{This text was revised (lines~190--193).}

L242 - The radiative component is negligible in the Northern Hemisphere midlatitudes, but not in the Southern Hemisphere.

\textbf{We now make clear the negligible radiative component assumption is satisfied specifically in the Northern Hemisphere midlatitudes (lines~264--265). The error that arises from ignoring the radiative component in the Southern Hemisphere midlatitudes is associated with the phase more so than the magnitude of $\Delta R_1$. Since we use the EBM to predict the existence rather than the timing of the midlatitude regime transition (see lines~268--270), we proceed by only considering the dynamic component of $\Delta R_1$.}

Eqn. C1 - subscript missing on T* in second term, righthand side

\textbf{We added the missing subscript to Equation~C1.}

I suggest stating whether the EBM solution is analytical or numerical.

\textbf{We now specify that the EBM solution is analytical (see line~267).}

L279 - If horizontal black lines in Fig. 6a,c are the annual mean $R_1$, that should be noted in the caption as well.

\textbf{We now specify that the annual mean $R_1$ is represented by the horizontal black line (see caption in Fig.~5).}

Fig. 9 - The right y-axis should be evaluated from 0.9 to 1.0 rather than 0.3 to 0.9 for the boundary layer lapse rate deviation.

\textbf{The y-axis label now shows the correct vertical bounds (see Fig.~9b,d).}

L290-292 - Related to the second major comment above, I don't think the aquaplanet simulation with sea ice entirely captures the observed energy balance and lapse rate regimes, though it is certainly an improvement over the case without ice.

\textbf{We removed this statement and now discuss how the AQUA experiments with ice differ from the reanalysis data (lines~319--329).}

\bibliographystyle{apalike}
\newsavebox\mytempbib
\savebox\mytempbib{\parbox{\textwidth}{\bibliography{references}}}

\end{document}

