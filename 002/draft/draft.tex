%% Version 5.0, 2 January 2020
%
%%%%%%%%%%%%%%%%%%%%%%%%%%%%%%%%%%%%%%%%%%%%%%%%%%%%%%%%%%%%%%%%%%%%%%
% TemplateV5.tex --  LaTeX-based template for submissions to the 
% American Meteorological Society
%
%%%%%%%%%%%%%%%%%%%%%%%%%%%%%%%%%%%%%%%%%%%%%%%%%%%%%%%%%%%%%%%%%%%%%
% PREAMBLE
%%%%%%%%%%%%%%%%%%%%%%%%%%%%%%%%%%%%%%%%%%%%%%%%%%%%%%%%%%%%%%%%%%%%%

%% Start with one of the following:
% DOUBLE-SPACED VERSION FOR SUBMISSION TO THE AMS
\documentclass{ametsocV5}

% TWO-COLUMN JOURNAL PAGE LAYOUT---FOR AUTHOR USE ONLY
% \documentclass[twocol]{ametsocV5}


% Enter packages here. If too many math alphabets are used,
% remove unnecessary packages or define hmmax and bmmax as necessary.

%\newcommand{\hmmax}{0}
%\newcommand{\bmmax}{0}
\usepackage{amsmath,amsfonts,amssymb,bm}
\usepackage{mathptmx}%{times}
\usepackage{newtxtext}
\usepackage{newtxmath}


%%%%%%%%%%%%%%%%%%%%%%%%%%%%%%%%

%%% To be entered by author:

%% May use \\ to break lines in title:

\title{When and where do Radiative--Convective and Radiative--Advective Equilibrium regimes occur on modern Earth?}

%%% Enter authors' names, as you see in this example:
%%% Use \correspondingauthor{} and \thanks{Current Affiliation:...}
%%% immediately following the appropriate author.
%%%
%%% Note that the \correspondingauthor{} command is NECESSARY.
%%% The \thanks{} commands are OPTIONAL.

    %\authors{Author One\correspondingauthor{Author name, email address}
% and Author Two\thanks{Current affiliation: American Meteorological Society, 
    % Boston, Massachusetts.}}

\authors{Osamu Miyawaki\correspondingauthor{Osamu Miyawaki, miyawaki@uchicago.edu}, Tiffany A. Shaw, and Malte F. Jansen}

%% Follow this form:
    % \affiliation{American Meteorological Society, 
    % Boston, Massachusetts}

\affiliation{The University of Chicago, Chicago, Illinois}

%% If appropriate, add additional authors, different affiliations:
    %\extraauthor{Extra Author}
    %\extraaffil{Affiliation, City, State/Province, Country}

%\extraauthor{}
%\extraaffil{}

%% May repeat for a additional authors/affiliations:

%\extraauthor{}
%\extraaffil{}

%%%%%%%%%%%%%%%%%%%%%%%%%%%%%%%%%%%%%%%%%%%%%%%%%%%%%%%%%%%%%%%%%%%%%
% ABSTRACT
%
% Enter your abstract here
% Abstracts should not exceed 250 words in length!
%
 

\abstract{Conceptual models of an atmospheric column provide a basis to understand the lapse rate structure and its response to climate change. Specifically, Radiative-Convective Equilibrium (RCE) and Radiative-Advective Equilibrium (RAE) have been used for investigating low and high latitude climate change, respectively. Currently we do not have a complete understanding of the spatio-temporal structure of RCE and RAE. Here, we use the vertically-integrated moist static energy budget to define a nondimensional number that quantifies when and where RCE and RAE are approximately satisfied in reanalysis products and models. We find RCE exists yearround in the tropics and in the Northern midlatitudes during summertime. RAE exists yearround poleward of $\approx 70^{\circ}$S over Antarctica but only outside of summer in the Arctic. We show that the lapse rates in RCE and RAE regimes in both reanalyses and GCMs are broadly consistent with lapse rate profiles of a moist adiabat and a surface inversion, respectively. Finally, we vary the mixed layer depth in idealized aquaplanet simulations with or without thermodynamic sea ice to test the following hypotheses: 1) the RCE regime occurs during midlatitude summer for land-like (small heat capacity) surface conditions and 2) sea ice is necessary for the existence of the RAE regime. Consistent with our first hypothesis, we find that an aquaplanet model configured with a 15 m slab ocean (NH-like) transitions to RCE in the summer whereas the 40 m slab ocean (SH-like) does not. Furthermore, we show that sea ice is a necessary and sufficient condition for the existence of RAE during wintertime.}

\begin{document}

%% Necessary!
\maketitle

%%%%%%%%%%%%%%%%%%%%%%%%%%%%%%%%%%%%%%%%%%%%%%%%%%%%%%%%%%%%%%%%%%%%%
% SIGNIFICANCE STATEMENT/CAPSULE SUMMARY
%%%%%%%%%%%%%%%%%%%%%%%%%%%%%%%%%%%%%%%%%%%%%%%%%%%%%%%%%%%%%%%%%%%%%
%
% If you are including an optional significance statement for a journal article or a required capsule summary for BAMS 
% (see www.ametsoc.org/ams/index.cfm/publications/authors/journal-and-bams-authors/formatting-and-manuscript-components for details), 
% please apply the necessary command as shown below:
%
% \statement
% Significance statement here.
%
% \capsule
% Capsule summary here.


%%%%%%%%%%%%%%%%%%%%%%%%%%%%%%%%%%%%%%%%%%%%%%%%%%%%%%%%%%%%%%%%%%%%%
% MAIN BODY OF PAPER
%%%%%%%%%%%%%%%%%%%%%%%%%%%%%%%%%%%%%%%%%%%%%%%%%%%%%%%%%%%%%%%%%%%%%
%

%% In all cases, if there is only one entry of this type within
%% the higher level heading, use the star form: 
%%
% \section{Section title}
% \subsection*{subsection}
% text...
% \section{Section title}

%vs

% \section{Section title}
% \subsection{subsection one}
% text...
% \subsection{subsection two}
% \section{Section title}

%%%
% \section{First primary heading}

% \subsection{First secondary heading}

% \subsubsection{First tertiary heading}

% \paragraph{First quaternary heading}

\section{Introduction}

The Earth's climate is maintained by three types of heat transfer: advection, radiation, and conduction \citep{hartmann2016}. These heat transfer types can be most easily defined using the vertically-integrated annual-mean zonal-mean moist static energy (MSE) budget:
\begin{equation} \label{eq:mse-ann}
    {\underbrace{ \langle\nabla\cdot \overline{[F_{m}]}\rangle}_{\text{advection}}} = {\underbrace{\vphantom{\nabla\cdot \langle [F_{m}]\rangle} \overline{[R_{a}]}}_\text{radiation}} + {\underbrace{\vphantom{\nabla\cdot \langle [F_{m}]\rangle} \mathrm{\overline{[LH]}+\overline{[SH]}}}_\text{conduction}} \, ,
\end{equation}
where $h=c_p T + gz + Lq$ is MSE, $\overline{(\cdot)}$ is the annual mean, $[\cdot]$ is the zonal mean, and $\langle \cdot \rangle$ is the mass-weighted vertical integral \citep{neelin1987}. Advection corresponds to the divergence of MSE flux ($F_m=vh$ where $v$ is the meridional velocity) and represents the heat transferred by the atmospheric circulation, including the Hadley cell and storm tracks. Radiation ($R_a$) corresponds to the sum of the radiative fluxes through the top of the atmosphere and surface. Finally, conduction corresponds to surface latent ($\mathrm{LH}$) and sensible ($\mathrm{SH}$) heat flux.

In the annual mean, the dominant types of heat transfer depend on latitude \citep[e.g., see Fig.~6.1 in][]{hartmann2016}. In the low latitudes, atmospheric radiative cooling is primarily balanced by conductive fluxes \citep{riehl1958}, which destabilize the column to convection by supplying moist, warm air to the boundary layer. The dominant balance between radiative cooling and surface heat fluxes is consistent with the assumptions behind Radiative Convective Equilibrium \citep[RCE,][]{wing2018}. In the high latitudes, atmospheric radiative cooling is primarily balanced by advection \citep{nakamura1988}, consistent with Radiative Advective Equilibrium \citep[RAE,][]{cronin2016}. Finally in the midlatitudes, all three types of heat transfer are important; thus, we introduce the term Radiative Convective Advective Equilibrium (RCAE). In this way, three heat transfer regimes qualitatively characterize the low, mid, and high latitude regions of Earth's modern climate.

The annual mean lapse rate structure also characterizes low, mid, and high latitude regions of Earth's modern climate. Lapse rates in the low latitudes are close to moist adiabatic \citep{stone1979,betts1982,xu1989,williams1993}. The high latitudes typically feature a surface inversion \citep[e.g., see Fig.~1.3 in][]{hartmann2016} because the surface can cool efficiently through the atmospheric window \citep{cronin2016} and the vertical profile of advective heat transport peaks slightly above the surface \citep{oort1974, overland1994, hahn2020, cardinale2021}. Lapse rates in the midlatitudes are typically more stable than a moist adiabat \citep{stone1979,korty2007} due to sensible and latent heating associated with baroclinic eddies, but do not typically show surface inversions.

The lapse rate structure varies through the seasonal cycle in the Northern Hemisphere. For example, the lapse rate in the midlatitudes is within 20\% of a moist adiabat in July, whereas the deviation increases up to 50\% in January \citep{stone1979}. In the Northern Hemisphere high latitudes, the inversion frequency and strength decrease \citep{bradley1992, tjernstrom2009, devasthale2010, zhang2011, cronin2016} and in some cases vanish during summertime \citep{stone1979}.

While heat transfer regimes and lapse rates both characterize low, mid, and high latitude regions of Earth's modern climate, few studies to date have quantified 1) the latitudinal dependence of heat transfer regimes and 2) the link between heat transfer regimes and the lapse rates through the seasonal cycle.

Quantifying the latitudinal distribution of observed heat transfer regimes would allow us to assess where idealized models that assume RCE or RAE hold. This is particularly important for RCE, which has become a standard idealized configuration for tropical theories \citep[e.g.,][]{emanuel1996,nilsson1999,romps2014,singh2015} and simulations \citep[][and the references therein]{wing2018}. Recent work by \cite{jakob2019} partially answered this question for RCE. They define RCE using the dry static energy (DSE) budget and a dimensional threshold (sum of radiative cooling, latent heating, and sensible heating $< \pm 50$ W m$^{-2}$). They find that RCE is approximately satisfied over large spatial ($>5000$ km) and temporal ($>$ daily) scales in the tropics. However, the latitudinal distribution of RCE outside the tropics and the distribution of RAE and RCAE has not been investigated.

Quantifying the link between heat transfer regimes and lapse rates would allow us to better understand the regional response to CO$_2$ forcing. For example, regions of RCE that exhibit a moist adiabatic lapse rate will have amplified warming aloft in response to increased CO$_2$ \citep{held1993, romps2011}. Similarly, regions of RAE that exhibit a surface inversion will have amplified warming at the surface in response to increased CO$_2$ \citep{held1993, cronin2016}.

We therefore seek to answer the questions: when and where do RCE, RAE, and RCAE occur on the modern Earth, and how closely are they linked to moist adiabatic and surface inversion lapse rates? To answer these questions, we develop a quantitative definition for RCE and RAE regimes using the nondimensionalized MSE budget (Section~\ref{sec:methods}\ref{subsec:mse}). We use this definition to examine where and when RCE, RAE, and RCAE occur both in the annual mean and seasonally on modern Earth. We then quantify the connection of heat transfer regimes to the moist adiabatic and surface inversion lapse rates (Section~\ref{sec:diagnostics}). Finally, we use a hierarchy of climate models to test hypotheses that explain the existence of heat transfer regimes in the Northern Hemisphere (Section~\ref{sec:hypo}).

\section{Methods}\label{sec:methods}

    \subsection{Defining heat transfer regimes using the nondimensionalized MSE budget} \label{subsec:mse}

    In order to define heat transfer regimes seasonally, we begin with the vertically-integrated, zonal-mean MSE equation without taking a time mean:
    \begin{equation} \label{eq:mse}
        \left\langle\left[\frac{\partial h}{\partial t}\right]\right\rangle + \langle\nabla\cdot [F_{m}]\rangle = [R_{a}] + \mathrm{[LH]+[SH]} \, ,
    \end{equation}
    where $\langle[\partial_t h]\rangle$ represents atmospheric storage of MSE. We nondimensionalize (\ref{eq:mse}) by dividing by radiative cooling $R_a$:
    \begin{equation}
        {\underbrace{\frac{\frac{\partial h }{\partial t} + \nabla\cdot F_{m}}{R_{a}}}_{R_1}} = 1 + {\underbrace{\frac{\mathrm{LH+SH}}{R_{a}}}_{R_2}} \, ,
    \end{equation}
    where $R_1$ and $R_2$ are nondimensional numbers and the $[\cdot]$ and $\langle\cdot\rangle$ have been dropped for brevity. 

    In the strictest sense, RCE requires a steady-state equilibrium where radiation balances conduction (\(R_{1}=0\)). As this is exactly satisfied only in the global mean, we define RCE as \(R_{1}\le \varepsilon\), where $\varepsilon$ is a small number. This definition includes regions of MSE flux divergence and weak convergence. The temperature profiles of columns with MSE flux divergence in the upper troposphere are also set by convective adjustment \citep{warren2020} and thus are included in the definition of approximate RCE.

    RAE as defined in \cite{cronin2016} requires conduction to be negligibly small (\(R_{2}=0\) or equivalently \(R_{1}=1\)). Although exact RAE further requires $\partial_t h=0$, the framework developed by \cite{cronin2016} could readily be generalized to account for the time tendency term, which would have the same effect as the advective tendency. To be consistent with the approximate definition of RCE, we define RAE as regions where conductive fluxes are small or directed from the atmosphere to the surface (\(R_{2} \ge -\gamma \) or equivalently \(R_{1} \ge 1-\gamma\)).

    In order to choose the values for $\varepsilon$ and $\gamma$, we examine the annual-mean zonal-mean percent deviation of the lapse rate from a moist adiabatic lapse rate binned by the value of $R_1$ using reanalysis data. The lapse rate deviation is plotted in sigma coordinates to ensure that surface inversions are properly represented (see Appendix~A for more details). The tropospheric lapse rate deviation is nearly a monotonic function of $R_1$ (especially above $\sigma=0.7$ and below $\sigma=0.9$), making it a suitable metric not only for quantifying heat transfer regimes, but for categorizing lapse rate regimes as well (Fig.~\ref{fig:rea-binned-r1}). A surface inversion is observed for $R_1 \ge 0.9$ (where lapse rate deviation exceeds 100\% in Fig.~\ref{fig:rea-binned-r1}) and thus we define the RAE regime as $R_1=1-\gamma=0.9$. We define the RCE regime as $R_1=\varepsilon=0.2$, where the free-tropospheric lapse rate deviation (vertically-averaged from $\sigma=0.7$ to 0.3) is 15\% from a moist adiabatic lapse rate. We choose 15\% deviation from a moist adiabat as the threshold for RCE since the lapse rate in the low latitudes ($\pm 30^\circ$) are within 15\% of a moist adiabat yearround (Fig.~\ref{fig:rea-r1-dev}b). Finally, RCAE is defined for the intermediate values of $0.2<R_1<0.9$.

    \subsection{Reanalysis data}\label{subsec:reanalysis}

    We consider three reanalysis data sets from 1980--2005: ERA5 \citep{hersbach2020}, MERRA2 \citep{gelaro2017}, and JRA55 \citep{kobayashi2015}. We focus on the heat transfer regimes and lapse rate structure for the multi-reanalysis mean and show the spread as the range across the three reanalyses. $\partial_t h$ is computed by taking the finite difference of MSE using monthly temperature, specific humidity, and geopotential data following \cite{donohoe2013}. Additionally, we use the monthly radiative ($R_a$) and conductive ($\mathrm{LH}$ and $\mathrm{SH}$) fluxes and infer $\nabla\cdot F_m$ as the residual. We choose to infer $\nabla\cdot F_m$ as the residual because the mass-correction technique for directly computing the MSE flux divergence in reanalysis data is known to produce unphysical results in the high latitudes \citep{porter2010}. 

    \subsection{Climate models}\label{subsec:models}
    We use two idealized climate models to understand the seasonal changes in heat transfer regimes. At intermediate complexity, we examine seasonal changes in the ECHAM6 slab-ocean aquaplanet model \citep{stevens2013}, hereafter referred to as AQUA. The AQUA simulations are configured with a seasonal cycle, no ocean heat transport, modern greenhouse gas concentrations, and with or without thermodynamic sea ice following \cite{shaw2020}. In order to explore the seasonal variation in heat transfer regimes, we vary the mixed layer depth in AQUA from 3 to 50 m following previous work \citep{donohoe2013, barpanda2020}. A monthly climatology is obtained by averaging the last 20 years of the 40 year simulation except for the 3 m configuration, where the last 5 years of a 15 year simulation are averaged due to the faster equilibration time. To be consistent with the reanalysis products, we compute $R_1$ using the monthly tendency of MSE ($\partial_t h$), monthly $R_a$, $\mathrm{LH}$, and $\mathrm{SH}$, and infer $\nabla\cdot F_m $ as the residual.

    At the simple model end, we use the EBM of \cite{rose2017}. The EBM is an equation for the zonal-mean surface temperature:
    \begin{equation}
      \rho c_w d \frac{\partial T_s}{\partial t} = aQ - (A+BT_s)  + D \frac{1}{\cos\phi}\frac{\partial}{\partial \phi}\left( \cos\phi \frac{\partial T_s}{\partial \phi} \right)\, ,
    \end{equation}
    where $\rho$ is the density of water, $c_w$ is the specific heat capacity of liquid water, $d$ is the mixed layer depth, $T_s$ is the surface temperature, $a$ is the co-albedo, $Q$ is insolation, $\mathrm{OLR}=A+BT_s$ is outgoing longwave radiation where $A$ and $B$ are constant coefficients, $\phi$ is latitude, and $D$ is the diffusivity, which is assumed to be a constant. We set $A=-410$ W m$^{-2}$, $B=2.33$ W m$^{-2}$ K$^{-1}$, $D=0.90$ W m$^{-2}$ K$^{-1}$, and $a=0.72$, which are obtained from best fits to AQUA configured with a 25 m mixed layer depth and without sea ice. Best fits of $A$ and $B$ are obtained by taking the least squares linear regression of the zonal-mean $\mathrm{OLR}$ and $T_s$. The best fit of $D$ is obtained similarly by taking the least squares linear regression of $\nabla\cdot F_m$ and $\frac{1}{\cos\phi}\frac{\partial}{\partial \phi} \left( \cos\phi \frac{\partial T_s}{\partial \phi} \right)$ for latitudes poleward of $25^\circ$. Lastly, $a$ is computed as $1-\alpha_p$ where $\alpha_p$ is the globally-averaged diagnosed planetary albedo.

\section{Heat transfer regimes in reanalysis data} \label{sec:diagnostics}

    \subsection{Annual mean heat transfer regimes}
    In the annual mean, the RCE regime, defined by $R_1 \le 0.2$, extends from the deep tropics to $\approx 45^\circ$ (black line overlapping orange region in Fig.~\ref{fig:rea-r1-ann}a). Consistently, the free-tropospheric lapse rate (averaged from $\sigma=0.7$ to 0.3) in the region where $R_1 \le 0.2$ is within $-0.6$\% and 5.6\% of the moist adiabatic lapse rate for the Northern and Southern Hemispheres, respectively (see orange lines in Figs.~\ref{fig:rea-r1-ann}b and c).

    The RAE regime, defined by $R_1 \ge 0.9$, occurs poleward of $\approx 80^\circ$N and $\approx 70^\circ$S in the annual mean (black line overlapping blue region in Fig.~\ref{fig:rea-r1-ann}a). The reanalysis spread in the high latitudes is large in both hemispheres due to high uncertainty in the estimation of conductive fluxes \citep{tastula2013,graham2019}. The largest values of $R_1$ are found over Antarctica whereas $R_1$ is close to the RCAE threshold in the Northern Hemisphere. Consistently, the region where $R_1 \ge 0.9$ exhibits a stronger surface inversion in the Southern high latitudes compared to the North (see blue lines in Figs.~\ref{fig:rea-r1-ann}b and 2c).

    Lastly, the RCAE regime, defined by $0.2 < R_1 < 0.9$, occurs between $45$--$80^\circ$N and $45$--$70^\circ$S in the annual mean (black line overlapping the white region in Fig.~\ref{fig:rea-r1-ann}a). The free-tropospheric lapse rate in the region where $0.2 < R_1 < 0.9$ is 25.9\% and 27.8\% more stable than a moist adiabat in the Northern and Southern Hemisphere, respectively, but does not exhibit a surface inversion (see gray lines in Fig.~\ref{fig:rea-r1-ann}b and 2c).

    \subsection{Seasonality of RCE and RAE} \label{subsec:seasonality}

    Seasonally, the RCE regime, defined by $R_1 \le 0.2$, occurs yearround equatorward of $45^\circ$ and up to $70^\circ$N during Northern Hemisphere summer (region equatorward of the thick orange contour in Fig.~\ref{fig:rea-r1-dev}a). The free-tropospheric lapse rate deviation from a moist adiabat exhibits a similar pattern, where close to moist adiabatic lapse rates are found not only in the low latitudes (from $45^\circ$S to $35^\circ$ N) yearround but also during Northern midlatitude summer (up to $70^\circ$N, see 15\% contour in Fig.~\ref{fig:rea-r1-dev}b). 
    
    Specifically, RCE in the Northern midlatitudes occurs from April through July (solid black line overlapping orange region in Fig.~\ref{fig:rea-r1-ga-temporal}a). In comparson, the lapse rate seasonality in the Northern midlatitudes lags behind $R_1$ by one to two months, and near-moist adiabatic lapse rates occur from May through September (orange line overlapping orange region in Fig.~\ref{fig:rea-r1-ga-temporal}a). This lag is associated with the seasonality of MSE storage, as an alternative definition of RCE based on $R_1=\frac{\nabla\cdot F_m}{R_a}$ leads to an improved agreement with the lapse rate seasonality (compare dashed black and orange lines in Fig.~\ref{fig:rea-r1-ga-temporal}a). %RCE and near-moist adiabatic (within 15\% deviation in the free troposphere) lapse rate regimes are in agreement within $\pm10^\circ$ latitude and 2 months. The Southern midlatitudes are in RCAE yearround and consistently, the free-tropospheric lapse rate deviation exceeds 20\% of a moist adiabat yearround (Fig.~\ref{fig:rea-r1-ga-temporal}a). 

    The RAE regime occurs throughout the seasonal cycle in the high latitudes with the exception of May and June in the Arctic (region poleward of the thick blue contour in Fig.~\ref{fig:rea-r1-dev}a). The boundary layer lapse rate deviation from a moist adiabat is broadly consistent with the seasonality of $R_1$, as the Southern high latitudes remains strongly stable yearround, whereas the stability weakens during summertime in the Northern high latitudes (e.g., deviation decreases to 40\% during June, see Fig.~\ref{fig:rea-r1-dev}c). 

    However, there is a discrepancy in the seasonality of heat transfer and lapse rate regimes in the Northern high latitudes. While RAE occurs during May and June (solid black overlapping white region in Fig.~\ref{fig:rea-r1-ga-temporal}b), boundary layers without a surface inversion occur from May through September (blue line overlapping white region in Fig.~\ref{fig:rea-r1-ga-temporal}b). Unlike in the Northern midlatitudes, the phase shift in the high latitudes is not explained by the seasonality of MSE storage (compare dashed black and blue lines in Fig.~\ref{fig:rea-r1-ga-temporal}b).
    
    % The spatial structure of RAE and inversion lapse rate regimes are in agreement within $\pm 15^\circ$ latitude. For example, in the Southern Hemisphere, RAE occurs poleward of $70^\circ$S in both January and July and boundary layer inversions occur poleward of $80^\circ$S and $70^\circ$S in January and July, respectively. In the Northern Hemisphere, RAE occurs poleward of $75^\circ$N in both January and July, while inversions occur poleward of $60^\circ$N in January and no inversions are present in July. The temporal structure of RAE and inversion regimes exhibit some key discrepancies. In the Southern Hemisphere, RAE occurs yearround and likewise inversions are present nearly yearround except in December (Fig.~\ref{fig:rea-r1-ga-temporal}c). In the Northern Hemisphere, RAE occurs from July through April, while inversions are found from October through April; that is, there is a three month discrepancy between the onset of RAE and an inversion (compare solid black line with blue line in Fig.~\ref{fig:rea-r1-ga-temporal}). Unlike the midlatitudes, these discrepancies are not reconciled by using the alternative definition of $R_1$ where MSE storage is excluded (compare dashed black line to blue line in Fig.~\ref{fig:rea-r1-ga-temporal}).

    \subsection{Decomposition of seasonal heat transfer regime transitions}

    In order to diagnose the physical mechanism responsible for the seasonal regime transitions, we decompose the seasonality of $R_1$ as follows:
    \begin{equation}\label{eq:r1-dev}
      \Delta R_1 = \overline{R_1}\left( \frac{\Delta(\partial_t h + \nabla\cdot F_m)}{\overline{\partial_t h + \nabla\cdot F_m}}  - \frac{\Delta R_a }{\overline{R_a}}\right) + \mathrm{Residual} \, ,
    \end{equation}
    where $\Delta(\cdot)$ is the seasonal deviation and $\overline{(\cdot)}$ is the annual mean. The dynamic component (first term on the right hand side of (\ref{eq:r1-dev})) quantifies the importance of advection and storage, and the radiative component (second term on the right hand side of (\ref{eq:r1-dev})) quantifies the importance of radiative cooling. Lastly, the residual quantifies the importance of nonlinear interactions.

    The RCAE to RCE regime transition in the Northern midlatitudes (where the solid black line intersects the orange region in Fig.~\ref{fig:rea-r1-decomp-mid}a) closely follows the dynamic component (compare black and red lines in Fig.~\ref{fig:rea-r1-decomp-mid}a) whereas the other terms are small (gray and dash-dot line in Fig.~\ref{fig:rea-r1-decomp-mid}a). The dynamic component dominates in the Northern Hemisphere because the seasonality of advection and storage is large ($\Delta(\partial_t h + \nabla\cdot F_m)\approx 40$ Wm$^{-2}$, see red line in Fig.~\ref{fig:rea-r1-decomp-mid}b). In the Southern Hemisphere, the seasonality of advection and storage is weaker ($\Delta(\partial_t h + \nabla\cdot F_m)\approx 15$ Wm$^{-2}$, see red line in Fig.~\ref{fig:rea-r1-decomp-mid}d), the dynamic and radiative components are similar in magnitude (red and gray lines in Fig.~\ref{fig:rea-r1-decomp-mid}c), and the $R_1$ seasonality is small (black line in Fig.~\ref{fig:rea-r1-decomp-mid}). Thus, the large seasonality of advection and storage in the Northern Hemisphere is important for the existence of summertime RCE in the midlatitudes.

    The RAE to RCAE regime transition in the Northern high latitudes is a small residual of the dynamic and radiative components (Fig.~\ref{fig:rea-r1-decomp-pole}a). However, the seasonality in the radiative and dynamic components do not balance completely due to a small but significant increase in latent heat flux during summertime (blue line in Fig.~\ref{fig:rea-r1-decomp-pole}b). In the Southern Hemisphere, there is no regime transition (Fig.~\ref{fig:rea-r1-decomp-pole}c) consistent with the negligibly small latent heat flux yearround (Fig.~\ref{fig:rea-r1-decomp-pole}d). The lack of a regime transition in the Southern Hemisphere is also associated with the hemispheric asymmetry in the sensible heat flux, which is a large negative flux in the Southern Hemisphere (compare orange lines in Fig.~\ref{fig:rea-r1-decomp-pole}b and d). Correspondingly, $\overline{R_1}=1.36$ in the Southern Hemisphere compared to $\overline{R_1}=0.97$ in the Northern Hemisphere, meaning that the Southern high latitudes is farther from the RCAE regime. Thus, the annual mean $R_1$ and the seasonality of latent heat flux are important for the RAE to RCAE regime transition in the Northern high latitudes.

\section{Testing hypotheses to explain seasonal regime transitions using idealized climate models} \label{sec:hypo}
  Next, we use the climate model hierarchy to test hypotheses that explain the seasonality of heat transfer regimes. In the complex end, the CMIP5 historical simulations (Appendix~C) capture the annual mean and seasonal heat transfer regimes that are consistent with the reanalysis (Fig.~C1--C6). The CMIP5 models also suggest that the Northern midlatitude regime transition is associated with the strong seasonality of the advection and storage fluxes (Fig.~C5), and the high latitude regime transition with the annual mean $R_1$ and positive summertime latent heat flux (Fig.~C6). In what follows, we seek a causal understanding of the factors controlling the seasonal regime transitions in the Northern Hemisphere using idealized climate models.

  \subsection{Midlatitude regime transition} \label{subsec:mld}
  Previous studies have found that surface heat capacity plays an important role in the seasonality of various climate phenomena, such as surface temperature \citep{donohoe2014}, ITCZ \citep{bordoni2008}, and storm track intensity \citep{barpanda2020}, due to its effect on the seasonality of surface energy fluxes. Thus, we hypothesize that surface heat capacity controls the existence of midlatitude heat transfer regime transitions. In order to connect the seasonality of $R_1$ to surface heat capacity, we begin by rewriting the MSE budget in terms of fluxes at the top of atmophere (TOA) and the surface (SFC) following \cite{barpanda2020}:
  \begin{equation}\label{eq:mse-toasfc}
    \Delta\left(\partial_t h + \nabla\cdot F_{m} \right) = \Delta F_{\mathrm{TOA}} - \Delta F_{\mathrm{SFC}} \, ,
  \end{equation}
  where \(F_{\mathrm{TOA}}\) and \(F_{\mathrm{SFC}}\) are the net heat fluxes through the top of atmosphere and surface. We can write the seasonality of surface fluxes using the surface energy budget of a mixed layer ocean:
  \begin{equation}
    \Delta F_{\mathrm{SFC}} = \rho c_{w} d \Delta\left(\frac{\partial T_{s}}{\partial t}\right) + \Delta ( \nabla\cdot F_{O}) \approx \rho c_{w} d \Delta\left(\frac{\partial T_{s}}{\partial t}\right) \, ,
  \end{equation}
  where $\rho$ is the density of water, $c_w$ is the specific heat capacity of liquid water, $d$ is the mixed layer depth, and $\Delta(\nabla\cdot F_O)$ is the meridional ocean heat flux convergence. We neglect the seasonality of the meridional ocean heat flux convergence as it is known to be small \citep{roberts2017}. Finally, we divide by $\overline{R_a}$ such that (\ref{eq:mse-toasfc}) becomes
  \begin{equation}\label{eq:mse-toasfc-approx}
    \Delta R_1 \approx \frac{\Delta\left(\partial_t h + \nabla\cdot F_{m} \right)}{\overline{R_a}} = \frac{1}{\overline{R_a}} \left(\Delta F_{\mathrm{TOA}} - \rho c_{w} d \Delta\left(\frac{\partial T_{s}}{\partial t}\right)\right) \, , 
  \end{equation}
  where we assume that the radiative component is negligible in (\ref{eq:r1-dev}) (see Fig.~\ref{fig:rea-r1-decomp-mid}a). In order to close (\ref{eq:mse-toasfc-approx}) and predict the dependence of $R_1$ on mixed layer depth ($d$), we make use of the EBM (see Section~\ref{sec:methods}\ref{subsec:models} and Appendix~B for more details). Following the EBM, we can write (\ref{eq:mse-toasfc-approx}) as
  \begin{equation} \label{eq:r1-linear4}
    \Delta R_{1} = \frac{Q^{*}}{\overline{R_{a}}}\frac{2D}{(B+2D)^{2}+(\rho c_w d \omega)^{2}}\left[(B+2D)\cos(\omega t)+\rho c_w d \omega \sin(\omega t)\right] \, ,
  \end{equation}
  where $Q^*$ is the seasonal amplitude of insolation and $\omega=2\pi/1$ yr. According to (\ref{eq:r1-linear4}), the amplitude of $\Delta R_1$ decreases as the mixed layer depth $d$ increases if all else is equal. 

  The dependence of $R_1$ seasonality on mixed layer depth in AQUA is mostly consistent with the EBM prediction (compare stars to solid black line in Fig.~\ref{fig:amp-r1-echam}a). While the $R_1$ seasonality is not as sensitive to the mixed layer depth in the EBM, it captures well the seasonal amplitude of surface temperature (Fig.~\ref{fig:amp-r1-echam}b). The midlatitude regime transition in AQUA occurs for $d \le 20$ m (intersection of the stars with the orange region in Fig.~\ref{fig:amp-r1-echam}a), whereas the corresponding mixed layer depth in the EBM is $16$ m (compare line to stars in Fig.~\ref{fig:amp-r1-echam}a).

  When AQUA is configured with a mixed layer depth of 15 m, the amplitude of the \(R_{1}\) seasonality closely resembles the Northern midlatitudes (compare Fig.~\ref{fig:echam-rce}a and Fig.~\ref{fig:rea-r1-decomp-mid}a). However, the regime transition in AQUA with a 15 m mixed layer depth occurs later than that in reanalysis data. This phase lag can be partly rectified by choosing a smaller (3 m) mixed layer depth, but this comes at the expense of amplifying the seasonality by a factor of 3. Consistent with the EBM prediction, when AQUA is configured with a large mixed layer depth of 40 m, \(\Delta R_{1}\) closely resembles the Southern midlatitudes; namely, there is no regime transition (compare Fig.~\ref{fig:echam-rce}e and Fig.~\ref{fig:rea-r1-decomp-mid}c). The persistence of the RCAE regime throughout the seasonal cycle in the aquaplanet simulations with 40 m mixed layer depth can be attributed to the weak seasonality of advection and storage, consistent with the results for the Southern Hemisphere midlatitudes (compare Fig.~\ref{fig:echam-rce}f and \ref{fig:rea-r1-decomp-mid}d).

  AQUA simulations also capture the seasonal amplitude of the lapse rate deviation as a function of mixed layer depth. For example, the lapse rate is within 6\% of a moist adiabat during August for AQUA with a 15 m mixed layer compared to 12\% for a 40 m mixed layer (see orange lines in Fig.~\ref{fig:echam-r1-ga-temporal}a,b).

  \subsection{High latitude regime transition} \label{subsec:ice}
  \subsubsection{Existence of wintertime RAE in the Northern Hemisphere}
  In the Northern high latitudes, the regime transition is associated with an increase in latent heat flux during summertime (Fig.~\ref{fig:rea-r1-decomp-pole}b). In other words, the suppression of latent heat flux during the rest of the year is important for the existence of RAE. Sea ice plays an important role on the surface energy balance by increasing the surface albedo and modulating turbulent heat flux exchange between the ocean and atmosphere \citep{andreas1979, maykut1982}. Thus, we hypothesize that sea ice is necessary for the existence of wintertime RAE in the Northern high latitudes. We test this hypothesis by configuring AQUA with and without thermodynamic sea ice (Section~\ref{sec:methods}\ref{subsec:models}). 

  When AQUA is configured with thermodynamic sea ice and a 40 m mixed layer depth, the high latitudes are in RAE from September through April (black line intersects the blue region in Fig.~\ref{fig:echam-rae}a) and in RCAE from May through August. The seasonality of the boundary layer lapse rate deviation is consistent with the seasonality of heat transfer regimes and shows that a surface inversion occurs from September through April (Fig.~\ref{fig:echam-r1-ga-temporal}b). AQUA configured with sea ice thus broadly reproduces the observed heat transfer and lapse rate regime seasonality in the Northern high latitudes (compare Fig.~\ref{fig:echam-rae}a,b with \ref{fig:rea-r1-decomp-pole}a,b).
  
  When AQUA is configured without sea ice and a 40 m mixed layer depth, the high latitudes are in RCAE yearround (black line does not intersect blue region in Fig.~\ref{fig:echam-rae}c). The RCAE regime is associated with positive latent heat flux (blue line in Fig.~\ref{fig:echam-rae}d) and the absence of a surface inversion yearround (Fig.~\ref{fig:echam-r1-ga-temporal}c). This result supports our hypothesis that sea ice is a necessary condition to obtain wintertime RAE in conditions similar to the Northern high latitudes. 
  
  We note that because sea ice persists yearround in both the 40 m AQUA simulation configured with sea ice and the reanalyses, the presence of sea ice cannot explain the existence of the RCAE to RAE regime transition in the Northern high latitude summer. In other words, while sea ice is a necessary and sufficient condition for the existence of wintertime RAE, it does not apply to the existence of RAE more generally. A general criteria for the existence of RAE requires a thorough investigation of the reason behind the increase in latent heat flux over sea ice during summertime. While a detailed analysis of the mechanisms controlling the magnitude of latent heat flux over sea ice is beyond the scope of this study, it represents an interesting avenue for future work.

  \subsubsection{Hemispheric asymmetry in annual mean $R_1$}
  Two key features were associated with the lack of a heat transfer regime transition in the Southern high latitudes: latent heat flux is negligibly small yearround, and the annual mean $R_1$ is higher in the Southern high latitudes, meaning that it is farther from the RCAE threshold. The latter is likely associated with the presence of Antarctic topography, which makes the atmosphere optically thinner and weakens atmospheric radiative cooling, corresponding to larger values of $R_1$. Topography is also known to play an important role in explaining the hemispheric asymmetry in the response of polar amplification to climate change \citep{salzmann2017,hahn2020,singh2020}. Thus, we investigate the role of Antarctic topography on the annual mean $R_1$ and its connection to the existence of a regime transition in the Southern high latitudes using the CESM simulations conducted by \cite{hahn2020}. 
  

  The control CESM simulation (with topography and preindustrial CO$_2$) captures the seasonality of $R_1$ as obtained in the reanalyses and CMIP5 historical runs, where the Northern high latitudes undergo a RAE to RCAE regime transition in June (Fig.~\ref{fig:hahn-aa}a) while the Southern high latitudes remain in RAE yearround (Fig.~\ref{fig:hahn-aa}c). This hemispheric asymmetry is reflected in the large difference in annual mean $R_1$, where $\overline{R_1}=1.04$ in the Northern high latitudes and $\overline{R_1}=1.30$ in the Southern high latitudes (see horizontal lines in Fig.~\ref{fig:hahn-aa}a,c).

  When Antarctic topography is flattened in CESM, the change in $\overline{R_1}$ is negligibly small in the Northern high latitudes whereas $\overline{R_1}$ decreases from 1.30 to 1.12 in the Southern high latitudes (compare horizonal lines in Fig.~\ref{fig:hahn-aa}a,b and c,d, respectively). The change in the Southern high latitudes annual mean $R_1$ comes almost entirely outside of the summer months and hence the Southern high latitudes with flattened topography continue to remain in RAE yearround (Fig.~\ref{fig:hahn-aa}d). This suggests that other factors, such as the asymmetry in the surface type (sea ice and ocean-dominated Arctic vs ice sheets and land-dominated Antarctic), are also important to fully understand the lack of a regime transition in the Southern high latitudes and the hemispheric asymmetry in high latitude regime transitions.

\section{Conclusion and Discussion}

\subsection{Conclusion}
We have quantified when and where heat transfer regimes are observed and their connection to lapse rates in the annual mean and through the seasonal cycle on the modern Earth. We quantified the latitudinal distribution of heat transfer regimes via a nondimensional number that arises in the vertically-integrated MSE equation, $R_1=\frac{\partial_t h + \nabla\cdot F_m}{R_a}$, which quantifies the relative importance of MSE advection and storage to radiative cooling. The annual mean lapse rate as measured by the percent deviation from a moist adiabatic lapse rate is nearly a monotonic function of $R_1$. We define the RCE regime as $R_1 \le 0.2$, the RAE regime as $R_1 \ge 0.9$, and the RCAE regime as $0.2 < R_1 < 0.9$. Within the RCE regime, the vertically-averaged free-tropospheric lapse rate deviation is within 15\% of a moist adiabat, and within the RAE regime, there exists a surface inversion.

In the annual mean, we find that RCE occurs equatorward of $45^\circ$, RAE poleward of $80^\circ$N and $70^\circ$S, and RCAE between $45-70^\circ$N and $45-80^\circ$S. Lapse rates averaged over all regions identified as RCE are within 6\% of a moist adiabat in the free troposphere, RAE exhibit a surface inversion, and RCAE are 27\% more stable than a moist adiabat and do not show a surface inversion.

Heat transfer regimes exhibit negligible seasonality in the Southern Hemisphere. In the Northern Hemisphere, there are seasonal regime transitions from 1) RCAE (September through March) to RCE (April through August) in the midlatitudes, and 2) RAE (July through April) to RCAE (May and June) in the high latitudes. While the lapse rate structure show consistent seasonal regime transitions in the Northern Hemisphere, lapse rate regime transitions lag behind the heat transfer regime transition by 1 to 3 months. In the Northern midlatitudes, this phase lag is associated with the seasonality of MSE storage, such that the discrepancy is eliminated when an alternative definition of $R_1=\frac{\nabla\cdot F_m}{R_a}$ is used.

The Northern midlatitude regime transition is associated with a stronger seasonality of MSE advection and storage relative to the Southern midlatitudes. We confirmed the hypothesis that surface heat capacity controls the seasonal amplitude of MSE advection and storage and correspondingly the existence of summertime RCE, by varying the mixed layer depth in an energy balance model and aquaplanet simulations. Both idealized models show that the seasonality of $R_1$ increases as the mixed layer depth decreases and predict that the RCAE to RCE regime transitions occur for mixed layer depths less than 16 m (EBM) and 20 m (aquaplanets). The seasonality of $R_1$ in the 15 and 40 m aquaplanets closely resemble the observed Northern and Southern midlatitudes, respectively. However, the RCAE to RCE regime transition in the 15 m aquaplanet lags the regime transition in the observed Northern midlatitudes by one month.

The Northern high latitude regime transition is associated with an increase in latent heat flux during summertime and an annual mean $R_1$ that is close to the RCAE threshold. A similar Northern high latitude regime transition is reproduced in an aquaplanet configured with a 40 m mixed layer depth and thermodynamic sea ice. We confirmed the hypothesis that sea ice is a necessary condition for the existence of wintertime RAE by showing that the 40 m aquaplanet configured without sea ice exhibits positive latent heat flux yearround and correspondingly remains in RCAE yearround. In addition, we investigated the role of Antarctic topography on the hemispheric asymmetry of annual mean $R_1$ and hence the asymmetry in the high latitude regime transitions using the CESM experiments conducted by \cite{hahn2020}. While Antarctic topography accounts for much of the hemispheric difference in annual mean $R_1$, the Southern high latitudes remains in RAE yearround even with the absence of Antarctic topography. Thus, mechanisms other than topography must also be considered to fully understand the hemispheric asymmetry in high latitude heat transfer regimes.

\subsection{Discussion}
Our findings are consistent with \cite{jakob2019}, who found that the tropics is near a state of RCE in the annual mean over a sufficiently large spatial average (achieved here through taking the zonal mean). \cite{jakob2019} use the DSE budget to define RCE and primarily focus on the implications of the validity of RCE in the context of CRM configurations and convective aggregation in the tropics. Our work focuses on the nondimensional MSE budget, which has the advantage that it can be used as a more general criterion for defining heat transfer regimes outside of the tropics and in climates different from modern Earth.

The framework we introduced for quantifying heat transfer regimes can be extended in many ways, such as studying the zonal structure of heat transfer regimes and the response of heat transfer regimes to climate change. For example, our aquaplanet results suggest that understanding the zonal structure of $R_1$ may be important to more accurately explain the timing of the Northern Hemisphere regime transitions. In addition, the latitudinal distribution of heat transfer regimes is expected to have changed through Earth's history. There are hints that high latitudes during warm epochs such as the Eocene may have been in a state similar to RCE \citep{abbot2008a} and that RAE was more widespread during a Snowball period \citep{pierrehumbert2005}. Understanding the spatio-temporal changes of heat transfer regimes during and through the transitions across various paleoclimate states are exciting avenues for future work.

%%%%%%%%%%%%%%%%%%%%%%%%%%%%%%%%%%%%%%%%%%%%%%%%%%%%%%%%%%%%%%%%%%%%%
% ACKNOWLEDGMENTS
%%%%%%%%%%%%%%%%%%%%%%%%%%%%%%%%%%%%%%%%%%%%%%%%%%%%%%%%%%%%%%%%%%%%%
\acknowledgments
The authors acknowledge support from the National Science Foundation (AGS-2033467).

%%%%%%%%%%%%%%%%%%%%%%%%%%%%%%%%%%%%%%%%%%%%%%%%%%%%%%%%%%%%%%%%%%%%%
% DATA AVAILABILITY STATEMENT
%%%%%%%%%%%%%%%%%%%%%%%%%%%%%%%%%%%%%%%%%%%%%%%%%%%%%%%%%%%%%%%%%%%%%
% 
%
\datastatement
The data availability statement is where authors should describe how the data underlying 
the findings within the article can be accessed and reused. Authors should attempt to 
provide unrestricted access to all data and materials underlying reported findings. 
If data access is restricted, authors must mention this in the statement.

%%%%%%%%%%%%%%%%%%%%%%%%%%%%%%%%%%%%%%%%%%%%%%%%%%%%%%%%%%%%%%%%%%%%%
% APPENDIXES
%%%%%%%%%%%%%%%%%%%%%%%%%%%%%%%%%%%%%%%%%%%%%%%%%%%%%%%%%%%%%%%%%%%%%
%
% Use \appendix if there is only one appendix.
%\appendix

% Use \appendix[A], \appendix[B], if you have multiple appendixes.
% \appendix[A]

%% Appendix title is necessary! For appendix title:
%\appendixtitle{}

%%% Appendix section numbering (note, skip \section and begin with \subsection)
% \subsection{First primary heading}

% \subsubsection{First secondary heading}

% \paragraph{First tertiary heading}

%% Important!
%\appendcaption{<appendix letter and number>}{<caption>} 
%must be used for figures and tables in appendixes, e.g.,
%
%\begin{figure}
%\noindent\includegraphics[width=19pc,angle=0]{figure01.pdf}\\
%\appendcaption{A1}{Caption here.}
%\end{figure}
%
% All appendix figures/tables should be placed in order AFTER the main figures/tables, i.e., tables, appendix tables, figures, appendix figures.

\appendix[A]
\appendixtitle{Lapse rate deviation from a moist adiabat}
We compute the lapse rate deviation from a moist adiabat to compare whether the lapse rate profiles of RCE, RAE, and RCAE are consistent with the expectations in each region. To avoid the issue of averaging out surface inversions that occur at various surface pressure or height levels in the presence of topography, we compute the lapse rate deviation in sigma coordinates.

We take the central finite difference of monthly pressure level temperature and geopotential data to compute the lapse rate and convert to sigma coordinates by masking out the data below surface pressure and taking a cubic spline interpolation. We perform this conversion for every latitude and longitude grid point. 

Following \cite{stone1979}, we define the deviation of a lapse rate from a moist adiabatic lapse rate as the fractional difference:
  \begin{equation}
    \delta_{c} = \frac{\Gamma_{m}-\Gamma}{\Gamma_{m}}
  \end{equation}
where $\Gamma$ is the actual lapse rate in the reanalysis or GCM and $\Gamma_m$ is the moist adiabatic lapse rate defined as in (3) in \cite{stone1979}.

\appendix[B]
\appendixtitle{Deriving an analytical expression of $\Delta R_1$ as a function of mixed layer depth}
Following the \cite{rose2017} EBM, we write the seasonality of TOA and SFC fluxes as a Fourier-Legendre series. Here, we only consider the first harmonic as it is an order of magnitude larger than the second harmonic in the midlatitudes:
    \begin{itemize}
      \item $\Delta F_{\mathrm{TOA}} \approx a\Delta Q - B\Delta T_{s}$, where $\Delta Q = Q^{*}\cos(\omega t)$. $\omega=\frac{2\pi}{t_{\mathrm{year}}}$, $Q^{*}=as_{11}Q_{g}P_{1}(x)$ is the amplitude of net TOA shortwave radiation, $s_{11}=-2\sin{\beta}$ where $\beta$ is the obliquity, $P_1(x) = \sin\phi$, and $Q_{g}=340$ Wm$^{-2}$. 
      \item $\Delta T_{s} = T_{s}^{*}\cos(\omega t - \Phi)$, where $T_{s}^{*}$ is the amplitude of surface temperature seasonality and $\Phi$ is the phase shift of $\Delta T_{s}$ relative to $\Delta Q$. $T_{s}^{*}=Q^{*}\left[(B+2D)^{2}+(\rho c_w d \omega)^{2}\right]^{-1/2}$ and $\Phi=\arctan\left(\frac{\rho c_w d \omega}{B+2D}\right)$ (see \cite{rose2017} for the derivation of the analytical expression of surface temperature).
    \end{itemize}
  Using the assumptions above, we can write Equation~(\ref{eq:mse-toasfc-approx}) as
  \begin{equation} \label{eq:r1-linear3}
    \Delta R_{1} = \frac{1}{\overline{R_{a}}}\left(Q^{*}\cos(\omega t) -BT^{*}\cos(\omega t - \Phi)+\rho c_{w} d \omega T_{s}^{*}\sin(\omega t - \Phi) \right) \, .
  \end{equation}
  Substituting in $T_{s}^{*}$ and $\Phi$ and simplifying, we obtain
  \begin{equation} \label{eq:r1-linear4-deriv}
    \Delta R_{1} = \frac{Q^{*}}{\overline{R_{a}}}\frac{2D}{(B+2D)^{2}+(\rho c_w d \omega)^{2}}\left[(B+2D)\cos(\omega t)+\rho c_w d \omega \sin(\omega t)\right] \, ,
  \end{equation}

\appendix[C]
\appendixtitle{CMIP5 historical simulations}
We consider the r1i1p1 historical run of the CMIP5 archive from 1980--2005 \citep{taylor2012}. We show the heat transfer regimes and lapse rate structure for the multi-model mean of 41 models (see Table~C1) and show the spread as the interquartile range across the models. To be consistent with the reanalysis products, we compute $R_1$ using the monthly tendency of MSE ($\partial_t h$), monthly $R_a$, $\mathrm{LH}$, and $\mathrm{SH}$, and infer $\nabla\cdot F_m $ as the residual.

%%%%%%%%%%%%%%%%%%%%%%%%%%%%%%%%%%%%%%%%%%%%%%%%%%%%%%%%%%%%%%%%%%%%%
% REFERENCES
%%%%%%%%%%%%%%%%%%%%%%%%%%%%%%%%%%%%%%%%%%%%%%%%%%%%%%%%%%%%%%%%%%%%%
% Make your BibTeX bibliography by using these commands:
\bibliographystyle{ametsoc2014}
\bibliography{references}


%%%%%%%%%%%%%%%%%%%%%%%%%%%%%%%%%%%%%%%%%%%%%%%%%%%%%%%%%%%%%%%%%%%%%
% TABLES
%%%%%%%%%%%%%%%%%%%%%%%%%%%%%%%%%%%%%%%%%%%%%%%%%%%%%%%%%%%%%%%%%%%%%
%% Enter tables at the end of the document, before figures.
%%
%
%\begin{table}[t]
%\caption{This is a sample table caption and table layout.  Enter as many tables as
%  necessary at the end of your manuscript. Table from Lorenz (1963).}\label{t1}
%\begin{center}
%\begin{tabular}{ccccrrcrc}
%\hline\hline
%$N$ & $X$ & $Y$ & $Z$\\
%\hline
% 0000 & 0000 & 0010 & 0000 \\
% 0005 & 0004 & 0012 & 0000 \\
% 0010 & 0009 & 0020 & 0000 \\
% 0015 & 0016 & 0036 & 0002 \\
% 0020 & 0030 & 0066 & 0007 \\
% 0025 & 0054 & 0115 & 0024 \\
%\hline
%\end{tabular}
%\end{center}
%\end{table}

%%%%%%%%%%%%%%%%%%%%%%%%%%%%%%%%%%%%%%%%%%%%%%%%%%%%%%%%%%%%%%%%%%%%%
% FIGURES
%%%%%%%%%%%%%%%%%%%%%%%%%%%%%%%%%%%%%%%%%%%%%%%%%%%%%%%%%%%%%%%%%%%%%
%% Enter figures at the end of the document, after tables.
%%
%
%\begin{figure}[t]
%  \noindent\includegraphics[width=19pc,angle=0]{figure01.pdf}\\
%  \caption{Enter the caption for your figure here.  Repeat as
%  necessary for each of your figures. Figure from \protect\cite{Knutti2008}.}\label{f1}
%\end{figure}

\begin{figure}
  \noindent\includegraphics[width=\textwidth]{/project2/tas1/miyawaki/projects/002/figures/rea/1980_2005/1.00/ga_frac_binned_r1/mse_old/lo/ga_frac_r1_all.png}\\
  \caption{Percent deviation of the reanalysis mean lapse rate from a moist adiabatic lapse rate binned by $R_{1}$. Thick blue and green lines correspond to $R_1=0.9$ and $R_1=0.2$, respectively.}
  \label{fig:rea-binned-r1}
\end{figure}

\begin{figure}[t]
  \noindent\includegraphics[width=\textwidth]{/project2/tas1/miyawaki/projects/002/figures_post/final/r1z_ann/r1z_ann_rea.pdf}\\
  \caption{(a) The annual mean zonal-mean structure of $R_{1}$ for the reanalysis mean. Orange, black, and blue regions indicate RCE, RCAE, and RAE, respectively. The annual-mean zonal-mean vertical profile of the lapse rate deviation from a moist adiabatic lapse rate for RCE, RCAE, and RAE for (b) SH and (c) NH. The shading over the lines indicate the range across ERA5, MERRA2, and JRA55.}
  \label{fig:rea-r1-ann}
\end{figure}

\begin{figure}[t]
  \noindent\includegraphics[width=0.7\textwidth]{/project2/tas1/miyawaki/projects/002/figures_post/final/r1_dev/r1_dev_rea.pdf}\\
  \caption{(a) The seasonality of $R_{1}$ for the reanalysis mean (contour interval is 0.1). The thick orange contour indicates the RCE/RCAE boundary ($R_1=0.2$) and the thick blue contour indicates the RAE/RCAE boundary ($R_1 = 0.9$). (b) The spatio-temporal structure of the vertically averaged free-tropospheric lapse rate deviation from a moist adiabatic lapse rate between $\sigma=0.7$ and 0.3 is shown for the reanalysis mean (contour interval is 5\%). (c) The spatio-temporal structure of the vertically averaged surface lapse rate deviation from a moist adiabatic lapse rate between $\sigma=1$ and 0.9 is shown for the reanalysis mean (contour interval is 10\%).}
  \label{fig:rea-r1-dev}
\end{figure}

\begin{figure}[t]
  \noindent\includegraphics[width=0.9\textwidth]{/project2/tas1/miyawaki/projects/002/figures_post/final/r1_ga_temporal/r1_ga_temporal_rea.pdf}\\
  \caption{The reanalysis mean seasonality of $R_1$ (black lines) is compared to (a) the free tropospheric ($\sigma=0.7$ to 0.3) lapse rate deviation from a moist adiabatic lapse rate in the Northern midlatitudes ($40-60^\circ$N, orange line) and (b) the boundary layer ($\sigma=1$ to 0.9) lapse rate deviation from a moist adiabat in the Northern high latitudes ($80-90^\circ$, blue line). We consider definitions of $R_1$ with (solid black) and without (dashed black) the MSE storage term to investigate its role on the phase discrepancy between heat transfer and lapse rate regimes. The shading over the lines indicate the range across ERA5, MERRA2, and JRA55.}
  % \caption{The seasonality of $R_1$ and the free tropospheric ($\sigma=0.7$ to 0.3) lapse rate deviation from a moist adiabatic lapse rate is compared for the reanalysis mean in the (a) Northern and (b) Southern Hemisphere midlatitudes ($40-60^\circ$). (c,d) Similar, except the seasonality of $R_1$ is compared to the boundary layer ($\sigma=1$ to 0.9) lapse rate deviation from a moist adiabat is compared in the (c) Northern and (d) Southern Hemisphere high latitudes ($80-90^\circ$).}
  \label{fig:rea-r1-ga-temporal}
\end{figure}

\begin{figure}[t]
  \noindent\includegraphics[width=\textwidth]{/project2/tas1/miyawaki/projects/002/figures_post/final/r1_decomp_mid/r1_decomp_mid_rea.pdf}\\
  \caption{The seasonality of $R_{1}$ in midlatitudes ($40$--$60^{\circ}$, black lines, left axis) and its deviation from the annual-mean (right axis) for the (a) Northern and (c) Southern Hemisphere. Note that the x-axis is shifted by 6 months in the Southern Hemisphere to facilitate comparison across the hemispheres. The orange-filled region ($R_1 \le 0.2$) is RCE and the white region ($R_1>0.2$) is RCAE. $\Delta R_1$ is decomposed into the dynamic (red line) and the radiative (gray line) components according to (\ref{eq:r1-dev}). The seasonality of the terms in the MSE budget in the midlatitudes for the (b) Northern and (d) Southern Hemisphere. The shading over the lines indicate the range across ERA5, MERRA2, and JRA55.}
  \label{fig:rea-r1-decomp-mid}
\end{figure}

\begin{figure}[t]
  \noindent\includegraphics[width=\textwidth]{/project2/tas1/miyawaki/projects/002/figures_post/final/r1_decomp_pole/r1_decomp_pole_rea.pdf}\\
  \caption{Same as Fig.~\ref{fig:rea-r1-decomp-mid} but averaged over the high latitudes ($80$--$90^{\circ}$).}
  \label{fig:rea-r1-decomp-pole}
\end{figure}

\begin{figure}
  \noindent\includegraphics[width=0.8\textwidth]{/project2/tas1/miyawaki/projects/002/figures_post/test/amp_r1_echam/amp_echam.pdf}\\
  \caption{(a) $R_1$ seasonality (as measured by $\min(R_1)$) and (b) surface temperature amplitude ($T_s^*=(\max(\Delta T_s)- \min(\Delta T_s))/2$) predicted by the EBM (solid black line) and AQUA (stars). The orange region denotes where an RCE/RCAE regime transition exists through the seasonal cycle.}
  \label{fig:amp-r1-echam}
\end{figure}

\begin{figure}[t]
    \noindent\includegraphics[width=\textwidth]{/project2/tas1/miyawaki/projects/002/figures_post/final/r1_decomp_mid/r1_decomp_mid_echamslab.pdf}\\
    \caption{Same as Fig.~\ref{fig:rea-r1-decomp-mid} but for AQUA with (a,b) 15 m (similar to the Northern midlatitudes) and (c,d) 40 m mixed layer depth (similar to the Southern midlatitudes).}
\label{fig:echam-rce}
\end{figure}

\begin{figure}
  \noindent\includegraphics[width=0.9\textwidth]{/project2/tas1/miyawaki/projects/002/figures_post/final/r1_ga_temporal/r1_ga_temporal_aqua.pdf}\\
    \caption{Similar to Fig.~\ref{fig:rea-r1-ga-temporal}, except AQUA simulations are shown configured with (a) 40 m mixed layer (similar to the Southern midlatitudes), (b) 15 m mixed layer (similar to the Northern midlatitudes), (c) 40 m mixed layer without sea ice, and (d) 40 m mixed layer with sea ice (similar to the Northern high latitudes).}
    \label{fig:echam-r1-ga-temporal}
\end{figure}

\begin{figure}[t]
    \noindent\includegraphics[width=\textwidth]{/project2/tas1/miyawaki/projects/002/figures_post/final/r1_decomp_pole/r1_decomp_pole_echamslab.pdf}\\
    \caption{Same as Fig.~\ref{fig:rea-r1-decomp-pole} but for AQUA with a 40 m mixed layer depth and (a,b) with (similar to the Northern high latitudes) and (c,d) without thermodynamic sea ice.}
    \label{fig:echam-rae}
\end{figure}

\begin{figure}[t]
    \noindent\includegraphics[width=\textwidth]{/project2/tas1/miyawaki/projects/002/figures_post/test/hahn/hahn_aa.pdf}\\
    \caption{Seasonality of $R_1$ in the CESM simulations performed by \cite{hahn2020} in the Northern high latitudes for the (a) control simulation (with Antarctic topography) and (b) flattened Antarctic topography simulation. (c,d) Similar, but for the Southern high latitudes.}
    \label{fig:hahn-aa}
\end{figure}

% APPENDIX FIGURES 

\begin{table}[t]
  \appendcaption{C1}{List of the 41 models that comprise the CMIP5 multi-model mean of the historical run.}
\begin{center}
  \renewcommand{\arraystretch}{1.0}
  \begin{tabular}{ l }
    Models          \\%& Ensemble run \\
    \hline
    ACCESS1-0       \\%& r1i1p1 \\
    ACCESS1-3       \\%& r1i1p1 \\
    bcc-csm1-1      \\%& r1i1p1 \\
    bcc-csm1-1-m    \\%& r1i1p1 \\
    BNU-ESM         \\%& r1i1p1 \\
    CanESM2         \\%& r1i1p1 \\
    CCSM4           \\%& r1i1p1 \\
    CESM1-BGC       \\%& r1i1p1 \\
    CESM1-CAM5      \\%& r1i1p1 \\
    CESM1-WACCM     \\%& r1i1p1 \\
    CMCC-CESM       \\%& r1i1p1 \\
    CMCC-CM         \\%& r1i1p1 \\
    CNRM-CM5        \\%& r1i1p1 \\
    CNRM-CM5-2      \\%& r1i1p1 \\
    CSIRO-Mk3-6-0   \\%& r1i1p1 \\
    FGOALS-g2       \\%& r1i1p1 \\
    FGOALS-s2       \\%& r1i1p1 \\
    GFDL-CM3        \\%& r1i1p1 \\
    GFDL-ESM2G      \\%& r1i1p1 \\
    GFDL-ESM2M      \\%& r1i1p1 \\
    GISS-E2-H       \\%& r1i1p1 \\
    GISS-E2-H-CC    \\%& r1i1p1 \\
    GISS-E2-R       \\%& r1i1p1 \\
    GISS-E2-R-CC    \\%& r1i1p1 \\
    HadCM3          \\%& r1i1p1 \\
    HadGEM2-CC      \\%& r1i1p1 \\
    HadGEM2-ES      \\%& r1i1p1 \\
    inmcm4          \\%& r1i1p1 \\
    IPSL-CM5A-LR    \\%& r1i1p1 \\
    IPSL-CM5A-MR    \\%& r1i1p1 \\
    IPSL-CM5B-LR    \\%& r1i1p1 \\
    MIROC5          \\%& r1i1p1 \\
    MIROC-ESM       \\%& r1i1p1 \\
    MIROC-ESM-CHEM  \\%& r1i1p1 \\
    MPI-ESM-LR      \\%& r1i1p1 \\
    MPI-ESM-MR      \\%& r1i1p1 \\
    MPI-ESM-P       \\%& r1i1p1 \\
    MRI-CGCM3       \\%& r1i1p1 \\
    MRI-ESM1        \\%& r1i1p1 \\
    NorESM1-M       \\%& r1i1p1 \\
    NorESM1-ME      \\%& r1i1p1 

  \end{tabular}
\end{center}
\end{table}

\begin{figure}[t]
  \noindent\includegraphics[width=\textwidth]{/project2/tas1/miyawaki/projects/002/figures/gcm/mmm/historical/198001-200512/1.00/ga_frac_binned_r1/mse_old/lo/ga_frac_r1_all.png}\\
  \appendcaption{C1}{Same as Fig.~\ref{fig:rea-binned-r1} but for the CMIP5 historical multi-model mean.}
  \label{fig:cmip5-binned-r1}
\end{figure}

\begin{figure}[t]
  \noindent\includegraphics[width=\textwidth]{/project2/tas1/miyawaki/projects/002/figures_post/final/r1z_ann/r1z_ann_cmip5hist.pdf}\\
  \appendcaption{C2}{Same as Fig.~\ref{fig:rea-r1-ann} but for the CMIP5 historical multi-model mean. The gray shading indicates one standard deviation from the mean.}
  \label{fig:cmip5hist-r1-ann}
\end{figure}

\begin{figure}[t]
  \noindent\includegraphics[width=\textwidth]{/project2/tas1/miyawaki/projects/002/figures_post/final/r1_dev/r1_dev_cmip5hist.pdf}\\
  \appendcaption{C3}{Same as Fig.~\ref{fig:rea-r1-dev} but for the CMIP5 historical multi-model mean.}
  \label{fig:cmip5hist-r1-dev}
\end{figure}

\begin{figure}[t]
  \noindent\includegraphics[width=0.9\textwidth]{/project2/tas1/miyawaki/projects/002/figures_post/final/r1_ga_temporal/r1_ga_temporal_cmip5hist.pdf}\\
  \appendcaption{C4}{Same as Fig.~\ref{fig:rea-r1-ga-temporal} but for the CMIP5 historical multi-model mean.}
  \label{fig:cmip5hist-r1-ga-temporal}
\end{figure}

\begin{figure}[t]
  \noindent\includegraphics[width=\textwidth]{/project2/tas1/miyawaki/projects/002/figures_post/final/r1_decomp_mid/r1_decomp_mid_cmip5hist.pdf}\\
  \appendcaption{C5}{Same as Fig.~\ref{fig:rea-r1-decomp-mid} but for the CMIP5 historical multi-model mean.}
  \label{fig:cmip5hist-r1-decomp-mid}
\end{figure}

\begin{figure}[t]
  \noindent\includegraphics[width=\textwidth]{/project2/tas1/miyawaki/projects/002/figures_post/final/r1_decomp_pole/r1_decomp_pole_cmip5hist.pdf}\\
  \appendcaption{C6}{Same as Fig.~\ref{fig:rea-r1-decomp-pole} but for the CMIP5 historical multi-model mean.}
  \label{fig:cmip5hist-r1-decomp-pole}
\end{figure}

\end{document}
