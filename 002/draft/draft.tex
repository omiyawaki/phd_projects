%% Version 5.0, 2 January 2020
%
%%%%%%%%%%%%%%%%%%%%%%%%%%%%%%%%%%%%%%%%%%%%%%%%%%%%%%%%%%%%%%%%%%%%%%
% TemplateV5.tex --  LaTeX-based template for submissions to the 
% American Meteorological Society
%
%%%%%%%%%%%%%%%%%%%%%%%%%%%%%%%%%%%%%%%%%%%%%%%%%%%%%%%%%%%%%%%%%%%%%
% PREAMBLE
%%%%%%%%%%%%%%%%%%%%%%%%%%%%%%%%%%%%%%%%%%%%%%%%%%%%%%%%%%%%%%%%%%%%%

%% Start with one of the following:
% DOUBLE-SPACED VERSION FOR SUBMISSION TO THE AMS
\documentclass{ametsocV5}

% TWO-COLUMN JOURNAL PAGE LAYOUT---FOR AUTHOR USE ONLY
% \documentclass[twocol]{ametsocV5}


% Enter packages here. If too many math alphabets are used,
% remove unnecessary packages or define hmmax and bmmax as necessary.

%\newcommand{\hmmax}{0}
%\newcommand{\bmmax}{0}
\usepackage{amsmath,amsfonts,amssymb,bm}
\usepackage{mathptmx}%{times}
\usepackage{newtxtext}
\usepackage{newtxmath}


%%%%%%%%%%%%%%%%%%%%%%%%%%%%%%%%

%%% To be entered by author:

%% May use \\ to break lines in title:

\title{When and where do Radiative--Convective and Radiative--Advective Equilibrium regimes occur on modern Earth?}

%%% Enter authors' names, as you see in this example:
%%% Use \correspondingauthor{} and \thanks{Current Affiliation:...}
%%% immediately following the appropriate author.
%%%
%%% Note that the \correspondingauthor{} command is NECESSARY.
%%% The \thanks{} commands are OPTIONAL.

    %\authors{Author One\correspondingauthor{Author name, email address}
% and Author Two\thanks{Current affiliation: American Meteorological Society, 
    % Boston, Massachusetts.}}

\authors{Osamu Miyawaki\correspondingauthor{Osamu Miyawaki, miyawaki@uchicago.edu}, Tiffany A. Shaw, and Malte F. Jansen}

%% Follow this form:
    % \affiliation{American Meteorological Society, 
    % Boston, Massachusetts}

\affiliation{The University of Chicago, Chicago, Illinois}

%% If appropriate, add additional authors, different affiliations:
    %\extraauthor{Extra Author}
    %\extraaffil{Affiliation, City, State/Province, Country}

%\extraauthor{}
%\extraaffil{}

%% May repeat for a additional authors/affiliations:

%\extraauthor{}
%\extraaffil{}

%%%%%%%%%%%%%%%%%%%%%%%%%%%%%%%%%%%%%%%%%%%%%%%%%%%%%%%%%%%%%%%%%%%%%
% ABSTRACT
%
% Enter your abstract here
% Abstracts should not exceed 250 words in length!
%
 

\abstract{Conceptual models of an atmospheric column provide a basis to understand the vertical temperature profile and its response to climate change. Specifically, Radiative-Convective Equilibrium (RCE) and Radiative-Advective Equilibrium (RAE) are the standard idealized models for investigating tropical and polar climate change, respectively. Currently we do not have a complete understanding of the spatio-temporal structure of RCE and RAE. Here we use the vertically-integrated Moist Static Energy budget to define a non-dimensional number that quantifies when and where RCE and RAE are approximately satisfied in observations and models. We find RCE exists year-round in the deep tropics and in the northern midlatitudes during summertime. RAE exists year-round poleward of $\approx 70^{\circ}$ latitude. We show the stratification in RCE and RAE regimes in both reanalyses and GCMs are consistent with a moist adiabatic and stable near-surface temperature profile, respectively. Finally, we vary the mixed layer depth in idealized aquaplanet simulations with thermodynamic sea ice to test the following hypotheses: 1) the RCE regime occurs during midlatitude summer for land-like (small heat capacity) surface conditions and 2) the equatorward edge of the RAE regime is determined by the sea ice edge. We find that an aquaplanet model configured with a 20 m slab ocean (NH-like) transitions to RCE in the summer whereas the 40 m slab ocean (SH-like) does not.}

\begin{document}

%% Necessary!
\maketitle

% Key points
\section*{Key points}
\begin{itemize}
  \item We introduce a non-dimensional metric $R_{1}$ for identifying heat transfer regimes.
\item Vertical temperature profiles are a monotonic function of $R_{1}$, making $R_{1}$ a suitable metric to define heat transfer regimes and their associated temperature profiles (Fig. 1(a,b)).
  \item RCE exists yearround in the deep tropics and in the NH midlatitudes during summer. RAE exists yearround poleward of 70 S in the Antarctic but only during winter in the Arctic (Figs. 3(a) and 4(a)).
  \item In the NH winter, the observed stratification is more stable than a moist adiabat in the midlatitudes and exhibits a near surface inversion in the high latitudes, consistent with the expected temperature profiles for RCAE and RAE regimes, respectively (Fig. 3(b) and 4(b)).
  \item In NH summer, the observed stratification is moist adiabatic in the midlatitudes and does not exhibit a near surface inversion in the high latitudes, consistent with RCE and RCAE regimes, respectively (Figs. 3(c) and 4(c)).
  \item The observed stratification is more stable than a moist adiabat in the SH midlatitudes and exhibits a near surface inversion in the SH high latitudes yearround, consistent with RCAE and RAE regimes, respectively (Figs. 3(d,e) and 4(d,e)).
  \item The seasonality of $R_{1}$ in the midlatitudes mostly follows the seasonality of MSE flux divergence. Thus, the asymmetry in the seasonality of MSE flux divergence explains the hemispheric asymmetry in the midlatitude regime transitions (Fig. 6(a,b) and 7(a,b)).
  \item Modifying the surface heat capacity in an aquaplanet captures the hemispheric asymmetry in the seasonality of MSE flux divergence and $R_{1}$ (Figs. 10(a,b)). The strong seasonality of MSE flux divergence in the NH midlatitudes is consistent with a lower heat capacity (20 m mixed layer depth) and the weak seasonality of MSE flux divergence in the SH midlatitudes is consistent with a higher heat capacity (40 m mixed layer depth). The control of mixed layer depth on MSE flux divergence seasonality is consistent with Barpanda and Shaw (2020).
  \item On modern Earth, the melting of sea ice appears to be a necessary condition for RCAE in the high latitudes:
  \item In other words, our current hypothesis is that the presence of sea ice that is not melting is a necessary and sufficient condition for high latitude RAE.
\end{itemize}

\newpage

%%%%%%%%%%%%%%%%%%%%%%%%%%%%%%%%%%%%%%%%%%%%%%%%%%%%%%%%%%%%%%%%%%%%%
% SIGNIFICANCE STATEMENT/CAPSULE SUMMARY
%%%%%%%%%%%%%%%%%%%%%%%%%%%%%%%%%%%%%%%%%%%%%%%%%%%%%%%%%%%%%%%%%%%%%
%
% If you are including an optional significance statement for a journal article or a required capsule summary for BAMS 
% (see www.ametsoc.org/ams/index.cfm/publications/authors/journal-and-bams-authors/formatting-and-manuscript-components for details), 
% please apply the necessary command as shown below:
%
% \statement
% Significance statement here.
%
% \capsule
% Capsule summary here.


%%%%%%%%%%%%%%%%%%%%%%%%%%%%%%%%%%%%%%%%%%%%%%%%%%%%%%%%%%%%%%%%%%%%%
% MAIN BODY OF PAPER
%%%%%%%%%%%%%%%%%%%%%%%%%%%%%%%%%%%%%%%%%%%%%%%%%%%%%%%%%%%%%%%%%%%%%
%

%% In all cases, if there is only one entry of this type within
%% the higher level heading, use the star form: 
%%
% \section{Section title}
% \subsection*{subsection}
% text...
% \section{Section title}

%vs

% \section{Section title}
% \subsection{subsection one}
% text...
% \subsection{subsection two}
% \section{Section title}

%%%
% \section{First primary heading}

% \subsection{First secondary heading}

% \subsubsection{First tertiary heading}

% \paragraph{First quaternary heading}

\section{Introduction}
    Three types of heat transfers dominate in the atmosphere of modern Earth: radiation, convection, and advection (horizontal heat transport by large-scale eddies). On Earth, longwave cooling dominates over shortwave heating due to the strong greenhouse effect and leads to a nearly meridionally uniform net radiative cooling of approximately $-100$ W m$^{-2}$ \citep{lin2008}. In steady state, the net radiative cooling is balanced by either convective heating, advective heating, or a combination of both, corresponding to three heat transfer regimes: radiative-convective equilibrium (RCE), radiative advective equilibrium (RAE), and radiative-convective-advective equilibrium (RCAE).
    
    In the modern tropics, strong insolation yearround heats the surface, destabilizing the lower troposphere through the supply of heat and moisture from the surface to the atmosphere. Moist convection redistributes heat throughout the troposphere, which is largely balanced by longwave cooling. Thus, the tropics is commonly idealized to be in RCE \citep{wing2018}. The dual assumption of RCE and the weak temperature gradient \citep{bretherton2002} forms the foundation of many theoretical models of the tropics. However, to date only one study has investigated where and when RCE is observed on modern Earth \citep{jakob2019}.
    
    In the modern polar regions, high albedo near the surface keeps the surface cool, leading to surface heat fluxes that are negligibly small relative to advection \citep{nakamura1988}. Indeed, \cite{cronin2016} use the assumption that the high latitudes is in a state of RAE to develop a simple model for the high latitude temperature profile. Temperature profiles in RAE are characterized by a near-surface inversion because advective heating tends to have a maximum above the surface. The presence of a near-surface inversion plays an important role in the contribution of the lapse rate feedback to polar amplification \cite{pithan2014,payne2015}.
    
    
    
    
    \cite{jakob2019} use the vertically integrated dry static energy (DSE) budget to identify the temporal and spatial scales where RCE is approximately satisfied in the tropics. They define RCE as regions where the DSE flux convergence is less than 50 W m$^{-2}$. A disadvantage of defining the criteria for RCE using a dimensional value is that it is difficult to generalize the framework to other climates.

    Radiative convective equilibrium (RCE) and radiative advective equilibrium (RAE) are the standard conceptual models used to model the vertical temperature structure of the low and high latitudes of the modern climate \citep{wing2018, cronin2016}. For example in the tropics, the temperature profile associated with RCE is used to quantify the strength of tropical storms through convective available potential energy \citep{singh2013, seeley2015}. At the poles, RAE is used to predict the strength of the high latitude near-surface inversion \citep{payne2015,cronin2016}. Thus, identifying the spatio-temporal structure of heat transfer regimes is important to understand where and when the predictions provided by these conceptual models are valid.
    
     \cite{jakob2019} use the vertically integrated dry static energy (DSE) budget to identify the temporal and spatial scales where RCE is approximately satisfied in the tropics. They define RCE as regions where the DSE flux convergence is less than 50 W m$^{-2}$. A disadvantage of defining the criteria for RCE using a dimensional value is that it is difficult to generalize the framework to other climates.
    
    While RCE and RAE regimes are the standard idealized heat transfer regimes for the modern low and high latitudes, respectively, the spatial structure of heat transfer regimes is thought to have varied significantly over the history of Earth's climate. For example, \cite{pierrehumbert2005} shows that a strong near surface inversion exists in the snowball midlatitudes, suggesting the possibility of RAE expanding farther equatorward during the snowball climate. \cite{abbot2008} show that the high latitudes of the past equable climates (e.g. Eocene, PETM) may have been convective, suggesting the possibility of RCE expanding to the poles in warm climates.
    
    In order to develop a general framework for quantifying heat transfer regimes, we use a nondimensional number \(R_{1}\) that appears in the vertically integrated moist static energy (MSE) equation to quantify an approximate state of RCE and RAE. In this paper, we focus on understanding the regime transitions that occur on the annual time scale in the modern Earth climate. We show that the hemispheric asymmetry in the midlatitude regime transition is due to the asymmetry in surface heat capacity by varying the mixed layer depth in a slab ocean aquaplanet model. We show that the hemispheric asymmetry in the high latitude regime transition is due to the asymmetry in sensible heat flux.

\section{Methods}

\subsection{Defining heat transfer regimes using the MSE budget}

    Here, we use the MSE budget to define heat transfer regimes. We choose to use the MSE budget over the DSE budget because precipitation is generally not a good proxy for convective activity in the extratropics. We write the zonal mean, vertically integrated MSE equation as: 
    \begin{equation} \label{eq:mse}
        \frac{\partial \langle h \rangle}{\partial t} + \nabla\cdot \langle F_{m} \rangle = R_{a} + \mathrm{LH+SH} \, ,
    \end{equation}
    where $h=c_p T + gz + Lq$ is MSE, $F_m$ is the MSE flux, $R_a$ is radiative cooling, $\mathrm{LH}$ is surface latent flux, $\mathrm{SH}$ is surface sensible heat flux, and $\langle \cdot \rangle$ is the mass-weighted vertical integral. We nondimensionalize Equation (\ref{eq:mse}) by dividing by \(R_{a}\). We choose to divide by $R_a$ because it is a negative definite quantity (other terms changes signs over the annual cycle) on modern Earth. The nondimensionalized equation is
    \begin{align}
        \frac{\frac{\partial \langle h \rangle}{\partial t} + \nabla\cdot \langle F_{m} \rangle }{R_{a}} &= 1 + \frac{\mathrm{LH+SH}}{R_{a}} \\
        R_{1} &= 1 + R_{2}
    \end{align}
    As the MSE tendency term is one order of magnitude smaller than the MSE flux divergence term, \(R_{1}\) quantifies the relative importance of horizontal fluxes through an atmospheric column and \(R_{2}\) quantifies the relative importance of vertical fluxes through an atmospheric column.
    
    Strict radiative convective equilibrium requires that all fluxes in the column be oriented in the vertical direction (i.e., \(R_{1}=0\)). This is only satisfied in the global mean and where the MSE flux divergence changes sign in the midlatitudes, so we define approximate radiative convective equilibrium as regions where the MSE flux convergence is negligibly small (\(0 \le R_{1} \le \epsilon\)) or the MSE flux is divergent \(R_{1}\le 0\). The reason we allow all magnitudes of a positive MSE flux divergence is because MSE flux divergence cools the atmosphere aloft and thus is a destabilizing flux that is balanced by stronger surface turbulent fluxes (convection).
    
    Strict radiative advective equilibrium requires that all fluxes be in the horizontal direction (i.e., \(R_{2}=0\) or equivalently \(R_{1}=1\)). To be consistent with the approximate definition of RCE, we define approximate radiative advective equilibrium as regions where positive surface turbulent fluxes are small (\(-\epsilon \le R_{2} \le 0 \) or equivalently \(1-\epsilon \le R_{1} \le 1\)) or the surface turbulent fluxes are negative (\(R_{2} \ge 0 \) or equivalently \(R_{1} \ge 1\)).
    
    Tropospheric temperature profiles are a monotonic function of $R_{1}$ (Fig.~\ref{fig:temp-binned-r1}(a,b)). Thus, $R_{1}$ is a useful metric for identifying heat transfer regimes and its associated vertical temperature profiles. We choose $\epsilon=0.1$ on the basis that the near-surface inversion exists for $R_{1} \ge 0.9$ (bold lines in Fig.~\ref{fig:temp-binned-r1}(a,b)).
    
    In summary, we define RCE as regions that satisfy \(R_{1}\le 0.1\) and RAE as regions that satisfy \(R_{1}\ge 0.9\). We identify intermediate values of \(R_{1}\) (i.e. \(0.1 < R_{1} < 0.9\)) as RCAE, where radiation, convection, and advection all contribute to its vertical temperature profile.

\subsection{Data}
    We use the ERA5 reanalysis data to evaluate heat transfer regimes in the observed modern Earth and the historical run of the Coupled Model Intercomparison Project Phase 5 (CMIP5) archive for state-of-the-art climate model simulations of modern Earth. We perform our analysis on a monthly climatology averaged between 1979 and 2005, which corresponds to the time period where both ERA5 and CMIP5 historical data are both available. 
  
    \subsubsection{ERA5 Reanalysis}
    ERA5 is the latest generation reanalysis product provided by the European Centre for Medium-Range Weather Forecasts \citep{hersbach2020}. $R_a$, $\mathrm{LH}$, and $\mathrm{SH}$ are available as part of the standard single level output in ERA5. We infer $\partial_t \langle h \rangle + \nabla\cdot\langle F_m \rangle$ as the sum $R_a + \mathrm{LH} + \mathrm{SH}$ to compute $R_1$.
    
    \subsubsection{CMIP5 Historical Runs}
    The historical run of the CMIP5 archive are atmosphere-ocean general circulation models (AOGCMs) forced with the historically observed atmospheric composition evolution. We obtain a multimodel mean of the CMIP5 historical run that is comprised of 41 models. As with ERA5, we use the output of $R_a$, $\mathrm{LH}$, and $\mathrm{SH}$ and infer $\partial_t \langle h \rangle + \nabla\cdot\langle F_m \rangle$ as the sum $R_a + \mathrm{LH} + \mathrm{SH}$.

\subsection{ECHAM6 slab ocean aquaplanet experiments}
\begin{itemize}
  \item ECHAM6 is the atmospheric component of the MPI-ESM-LR GCM.
  \item We configure ECHAM6 with a slab ocean to test the hypothesis that the mixed layer depth controls the seasonal amplitude of the poleward boundary of RCE.
\end{itemize}

% \subsection{ECHAM6 AGCM experiments}
% \begin{itemize}
%   \item ECHAM6 is the atmospheric component of the MPI-ESM-LR GCM.
%   \item We configure ECHAM6 with a slab ocean to test the hypothesis that the mixed layer depth controls the seasonal amplitude of the poleward boundary of RCE.
% \end{itemize}

\section{Results} \label{sec:results}

\subsection{RCE and RAE regimes}
    \subsubsection{Annual mean $R_1$}
    In the annual mean, RCE is satisfied in the low latitudes and RAE in the high latitudes, as expected (Fig.~\ref{fig:era5-r1-ann}(a)). The temporally and spatially averaged temperature profile over the RCE regime is close to moist adiabatic in both the NH (orange line in Fig.~\ref{fig:era5-r1-ann}(b)) and SH (orange line in Fig.~\ref{fig:era5-r1-ann}(c)). Over the RAE regime, the temporally and spatially averaged temperature profile exhibits a near surface inversion (blue lines in Figs.~\ref{fig:era5-r1-ann}(b) and (c)). The inversion is stronger in the SH, consistent with the higher values of $R_{1}$ indicating that it is in a stronger state of RAE. Over the RCAE regime, the temporally and spatially averaged temperature profile is more stable than a moist adiabat and does not exhibit a near surface inversion, distinct from both RCE and RAE regimes (gray lines in Figs.~\ref{fig:era5-r1-ann}(b) and (c)).
    
    \subsubsection{Seasonality of $R_1$}
    The seasonality of \(R_{1}\) in the mid and high latitudes is larger in the NH (Fig.~\ref{fig:era5-r1-dev}). The strong increase in \(R_{1}\) during NH summer drives a regime transition from RCAE to RCE in the NH midlatitudes and from RAE to RCAE in the NH high latitudes.
  
    The seasonality of vertical temperature profiles are consistent with the diagnosed heat transfer regimes in both the NH and SH. For example, the temperature profile at 45$^{\circ}$ N in January is significantly more stable than a moist adiabat (compare solid to dashed gray lines in Fig.~\ref{fig:era5-r1-dev}(b)) while the temperature profile in July is close to a moist adiabat (compare solid to dashed orange lines in Fig.~\ref{fig:era5-r1-dev}(c)). Similarly, the temperature profile at 85 $^{\circ}$ N in January exhibits a strong near surface inversion (blue line in Fig.~\ref{fig:era5-r1-dev}(b)) while no surface inversion is present in July (gray line in Fig.~\ref{fig:era5-r1-dev}(c)). In the SH, the temperature profile at 45$^{\circ}$S remains more stable than a moist adiabat yearround (gray lines in Fig.~\ref{fig:era5-r1-dev}(d,e)), consistent with $R_{1}$ indicating that the SH midlatitudes is in a state of RCAE yearround. The temperature profile at 85$^{\circ}$S exhibits a near surface inversion yearround (blue lines in Fig.~\ref{fig:era5-r1-dev}(d,e)), consistent with $R_{1}$ indicating that the SH high latitudes is in a state of RAE yearround.
    
    More generally, the spatio-temporal structure of the critical \(R_{1}\) values for RCE (\(R_{1}\le 0.1\)) and RAE (\(R_{1}\ge 0.9\)) closely follow the structure of convective (\(\delta_{c}\le 10\%\)) and inversion (\(\delta_{i}\ge 90\%\)) lapse rate regimes, respectively (compare orange and cyan contours in Fig.~\ref{fig:era5-r1-dev}(a) with Figs.~A1 and A2).
    
    These results suggest that RCE and RAE as diagnosed by \(R_{1}\) is a good proxy for identifying convective and inversion lapse rate regimes.
    
    Next, in order to understand what physical mechanism is responsible for the hemispheric asymmetry in heat transfer regimes in the mid and high latitudes, we diagnose which terms in the MSE budget dominates the hemispheric asymmetry.


\subsection{Hemispheric asymmetry} \label{subsec:asym}
\begin{itemize}
  \item We now focus on latitude bands in the midlatitudes (40--50$^{\circ}$ N/S) and the high latitudes (80--90$^{\circ}$ N/S) to understand what sets the difference in the seasonality of the RCE and RAE boundaries across the northern and southern hemisphere.
  \item We find that the transition from RCAE to RCE in the NH midlatitude summer (solid black line in Fig.~\ref{fig:era5-r1-decomp-mid}(a)) arises primarily due to a weakening of MSE flux convergence (red line in Fig.~\ref{fig:era5-r1-decomp-mid}(a)). The significant weakening of MSE flux convergence (red line in Fig.~\ref{fig:era5-r1-decomp-mid}(b)) comes from a combination of weaker radiative cooling (gray line in Fig.~\ref{fig:era5-r1-decomp-mid}(b)) and stronger surface turbulent fluxes in the summer (blue line in Fig.~\ref{fig:era5-r1-decomp-mid}(b)).
  \item In contrast, the SH midlatitudes remains in RCAE yearround (solid black line in Fig.~\ref{fig:era5-r1-decomp-mid}(c)). As in the NH, the MSE flux convergence weakens during summer (red line in Fig.~\ref{fig:era5-r1-decomp-mid}(d)) but the amplitude is smaller than in the NH. This can be attributed to the opposite phase of the seasonality of surface turbulent fluxes (blue line in Fig.~\ref{fig:era5-r1-decomp-mid}(d)).
  \item The transition from RAE to RCAE in the NH high latitude summer (Fig.~\ref{fig:era5-r1-decomp-pole}(a)) arises due to strong weakening of the MSE flux convergence that is supported by a strengthening of the surface turbulent fluxes during NH summer (Fig.~\ref{fig:era5-r1-decomp-pole}(b)).
  \item In contrast, the SH high latitudes remains in RAE yearround (Fig.~\ref{fig:era5-r1-decomp-pole}(c)). The amplitude of the weakening MSE flux convergence and strengthening surface turbulent fluxes in SH summer are similar to those in the NH (Fig.~\ref{fig:era5-r1-decomp-pole}(d)). However, the SH remains in RAE yearround because its annual mean state is farther away from the RAE boundary (see horizontal line in Fig.~\ref{fig:era5-r1-decomp-pole}(c)).
  \item In summary, we find that 1) the hemispheric asymmetry in midlatitude regime transitions is due to the asymmetry in the seasonal amplitude of MSE flux divergence and 2) the asymmetry in high latitude regime transitions is due to the difference in the annual mean $R_{1}$.
\end{itemize}


\subsection{Varying mixed layer depth explains hemispheric asymmetry of RCE}
\begin{itemize}
  \item When ECHAM AQUA is configured with a mixed layer depth of 20 m (shallower than the critical depth of 25 m derived earlier), the seasonality of \(R_{1}\) (Fig.~\ref{fig:echam-rce}(a)) closely resembles that of the NH midlatitudes (cf. Fig.~\ref{fig:era5-r1-decomp-mid}(a)).
  \item The decrease of \(R_{1}\) in the summer months is associated with a weakening of MSE flux divergence and a strengthening of turbulent heat fluxes (Fig.~\ref{fig:echam-rce}(b)).
  \item The seasonality of turbulent heat fluxes in ECHAM appears to lag behind insolation by 2 months. Thus, the months of RCE is also shifted toward late summer/early fall.
  \item When ECHAM is configured with a mixed layer depth of 40 m (deeper than the critical depth of 25 m), the seasonal amplitude of \(R_{1}\) (Fig.~\ref{fig:echam-rce}(c)) is comparable to that of the SH midlatitudes (cf. Fig.~\ref{fig:era5-r1-decomp-mid}(c)). Interestingly, \(R_{1}\) in the 40 m ECHAM run is nearly perfectly out of phase with \(R_{1}\) in the SH midlatitudes.
  \item The weaker amplitude of \(R_{1}\) is associated with a seasonality of turbulent heat fluxes that is out of phase with insolation (Fig.~\ref{fig:echam-rce}(d)).
  \item In summary, we find that the difference in the seasonal amplitude of \(R_{1}\) between the NH and SH can be explained by varying the mixed layer depth in an aquaplanet. However, the phase of \(R_{1}\) in the SH is opposite of that found in the SH midlatitudes.
\end{itemize}

\subsection{Winter RAE requires sea ice}
\begin{itemize}
  \item When ECHAM is configured without sea ice, the annual mean state in the high latitudes is in RCAE (Fig.~\ref{fig:echam-rae}(a)). This can be attributed to positive turbulent fluxes that persist yearround in the absence of sea ice (Fig.~\ref{fig:echam-rae}(b)).
  \item This holds for all mixed layer depths tested here (10--50 m).
  \item When ECHAM is configured with thermodynamic sea ice (40 m chosen here as it fits the NH high latitudes well), the annual mean state in the high latitudes is in RAE (Fig.~\ref{fig:echam-rae}(c)). In the presence of sea ice, turbulent fluxes are negative most of the year (Fig.~\ref{fig:echam-rae}(d)). Only when the sea ice melts during the summer do positive turbulent fluxes arise, which drives the transition to RCAE.
  \item This holds for mixed layer depths between 25--50 m. For mixed layer depths below 20 m, a runaway ice-albedo feedback leads the model to equilibrate to a snowball climate.
\end{itemize}

\section{Summary and Discussion}
% \begin{itemize}
%   \item We find that the northern boundary of the convective lapse rate regime expands poleward to the NH midlatitudes during NH summer in both ERA5 and the CMIP5 multi-model mean. The southern boundary does not expand poleward during SH summer. The amplitude of the seasonality of the southern boundary is different in ERA-Interim (contracts equatorward in SH winter) compared to CMIP5HIST-ESM-LR (negligible contraction).
%   \item The equatorward boundary of the NH inversion lapse rate regime extends to \(60^{\circ}\) N during NH winter and vanishes during summer in both ERA-Interim and CMIP5HIST-ESM-LR. The boundary of the SH inversion lapse rate regime also vanishes during SH summer in ERA-Interim, whereas there is negligible migration of the boundary in CMIP5HIST-ESM-LR.

% \end{itemize}


%%%%%%%%%%%%%%%%%%%%%%%%%%%%%%%%%%%%%%%%%%%%%%%%%%%%%%%%%%%%%%%%%%%%%
% ACKNOWLEDGMENTS
%%%%%%%%%%%%%%%%%%%%%%%%%%%%%%%%%%%%%%%%%%%%%%%%%%%%%%%%%%%%%%%%%%%%%
\acknowledgments
Keep acknowledgments (note correct spelling: no ``e'' between the ``g'' and
``m'') as brief as possible. In general, acknowledge only direct help in
writing or research. Financial support (e.g., grant numbers) for the work
done, for an author, or for the laboratory where the work was performed is
best acknowledged here rather than as footnotes to the title or to an
author's name. Contribution numbers (if the work has been published by the
author's institution or organization) should be included as footnotes on the title page,
not in the acknowledgments.

%%%%%%%%%%%%%%%%%%%%%%%%%%%%%%%%%%%%%%%%%%%%%%%%%%%%%%%%%%%%%%%%%%%%%
% DATA AVAILABILITY STATEMENT
%%%%%%%%%%%%%%%%%%%%%%%%%%%%%%%%%%%%%%%%%%%%%%%%%%%%%%%%%%%%%%%%%%%%%
% 
%
\datastatement
The data availability statement is where authors should describe how the data underlying 
the findings within the article can be accessed and reused. Authors should attempt to 
provide unrestricted access to all data and materials underlying reported findings. 
If data access is restricted, authors must mention this in the statement.

%%%%%%%%%%%%%%%%%%%%%%%%%%%%%%%%%%%%%%%%%%%%%%%%%%%%%%%%%%%%%%%%%%%%%
% APPENDIXES
%%%%%%%%%%%%%%%%%%%%%%%%%%%%%%%%%%%%%%%%%%%%%%%%%%%%%%%%%%%%%%%%%%%%%
%
% Use \appendix if there is only one appendix.
%\appendix

% Use \appendix[A], \appendix[B], if you have multiple appendixes.
% \appendix[A]

%% Appendix title is necessary! For appendix title:
%\appendixtitle{}

%%% Appendix section numbering (note, skip \section and begin with \subsection)
% \subsection{First primary heading}

% \subsubsection{First secondary heading}

% \paragraph{First tertiary heading}

%% Important!
%\appendcaption{<appendix letter and number>}{<caption>} 
%must be used for figures and tables in appendixes, e.g.,
%
%\begin{figure}
%\noindent\includegraphics[width=19pc,angle=0]{figure01.pdf}\\
%\appendcaption{A1}{Caption here.}
%\end{figure}
%
% All appendix figures/tables should be placed in order AFTER the main figures/tables, i.e., tables, appendix tables, figures, appendix figures.

\appendix[A]
\appendixtitle{Convective and inversion lapse rate regimes}
\subsection{Methods}
\begin{itemize}
  \item We use a cubic spline interpolation to convert the temperature profile to sigma coordinates. We do this to avoid the issue of averaging out inversions that occur at various pressure or height levels in the presence of topography.
  \item Following \cite{stone1979}, we define the deviation of a lapse rate from a convective lapse rate as the percent difference from a moist adiabatic lapse rate:
        \begin{equation}
          \delta_{c} = \frac{\Gamma_{m}-\Gamma}{\Gamma_{m}}
        \end{equation}
  \item We vertically average \(\delta_{c}\) from 0.85--0.4 in linear sigma coordinates to obtain the free tropospheric deviation \(\langle \delta_{c} \rangle\).
  \item To quantify the presence of a near surface inversion, we define the deviation of a lapse rate from a dry adiabatic lapse rate in a similar manner:
        \begin{equation}
          \delta_{i} = \frac{\Gamma_{d}-\Gamma}{\Gamma_{d}}
        \end{equation}
  \item Note that \(\delta_{i}=1\) corresponds to an isothermal stratification and thus \(\delta_{i}>1\) indicates the presence of an inversion.
  \item We use the dry adiabat as the reference lapse rate here because in general the boundary layer (where the near surface inversion forms) is not saturated.
  \item We vertically average \(\delta_{i}\) from 1--0.85 in linear sigma coordinates to obtain the near surface deviation \(\langle \delta_{i} \rangle\).
\end{itemize}

\subsection{Results}
\begin{itemize}
  \item The free tropospheric stratification is either conditionally unstable (orange filled contours in Fig.~A1) or close to neutrally stable (white filled contours in Fig.~A1) to a saturated moist adiabat equatorward of 30$^{\circ}$ N/S yearround.
  \item Thus the tropical free tropospheric stratification is set by convection yearround.
  \item The northern boundary of the convective lapse rate regime migrates poleward out to 60$^{\circ}$ N in July (Fig.~A1).
  \item The seasonal expansion of the northern boundary of the convective regime is consistent with the results obtained by \cite{stone1979} (cf. their Fig. 7).
  \item The southern boundary of the convective lapse rate regime varies less throughout the annual cycle, expanding out to only 40$^{\circ}$ in January (Fig.~A1). The southern boundary contracts equatorward to 20$^{\circ}$ S during the SH winter in ERA-Interim (Fig.~A1(a)) whereas there is negligible contraction in MPI-ESM-LR.
  \item The near surface stratification exhibits an inversion (100\% contour in Fig.~A2) poleward of 60 $^{\circ}$ N and 70$^{\circ}$ S in the respective winter hemispheres.
  \item The southern boundary of the NH inversion lapse rate regime migrates poleward following the seasonality of insolation until the inversion vanishes in the summer.
  \item The northern boundary of the SH inversion lapse rate regime also migrates poleward and vanishes in the summer in ERA-Interim (Fig.~A2(a)) whereas there is negligible seasonality in MPI-ESM-LR (Fig.~A2(b)).
\end{itemize}

\appendix[B]
\appendixtitle{Analytical relationship between $R_{1}$ and mixed layer depth}
\begin{itemize}
  \item In order to understand what physical mechanisms are responsible for the hemispheric asymmetry, we work toward expressing $R_{1}$ in terms of external parameters.
  \item Section~\label{subsec:asym} showed that the existence of a midlatitude regime transition depends on the seasonality of $R_{1}$. Thus, we start by focusing on the seasonality of $R_{1}$ into independent contributions from the seasonality of MSE flux divergence and radiative cooling:
        \begin{equation} \label{eq:r1-linear}
          \Delta R_{1} = \Delta\left(\frac{\nabla\cdot F_{m}}{R_{a}}\right) \approx \frac{1}{\overline{R_{a}}}\Delta(\nabla\cdot F_{m}) - \frac{\overline{\nabla\cdot F_{m}}}{\overline{R_{a}}^{2}}\Delta R_{a} \, ,
        \end{equation}
  \item where $\overline{(\cdot)}$ denotes the annual mean and $\Delta(\cdot)$ denotes the deviation from the annual mean. Fig.~\ref{fig:era5-r1-decomp-mid}(a,c) show that the second term on the RHS is smaller than the first term in the midlatitudes. Thus, we approximate
        \begin{equation} \label{eq:r1-term1}
          \Delta R_{1} \approx \frac{1}{\overline{R_{a}}}\Delta(\nabla\cdot F_{m}) \, .
        \end{equation}
  \item To connect the seasonality of MSE flux divergence to external parameters in the climate system, we rewrite the MSE equation in terms of fluxes at the top of atmophere (TOA) and surface (SFC):
        \begin{equation}\label{eq:mse-toasfc}
          \frac{\partial \langle h \rangle}{\partial t} + \nabla\cdot \langle F_{m} \rangle = F_{\mathrm{TOA}} - F_{\mathrm{SFC}}
        \end{equation}
  \item The seasonality of MSE tendency is small, so we approximate the seasonality of each term in the MSE equation as
        \begin{equation}\label{eq:mse-toasfc-approx}
          \Delta (\nabla\cdot \langle F_{m} \rangle) \approx \Delta F_{\mathrm{TOA}} - \Delta F_{\mathrm{SFC}}
        \end{equation}
  \item Note that \(F_{\mathrm{SFC}}\) is defined to be positive from the atmosphere to the surface here, hence the negative sign.
  \item We can write the seasonality of surface fluxes using the surface energy budget of a mixed layer ocean:
        \begin{equation}
          \Delta F_{\mathrm{SFC}} = \Delta\left(\rho c_{w} d \frac{\partial T_{s}}{\partial t}\right) + \Delta ( \nabla\cdot F_{O})
        \end{equation}
  \item The seasonality of ocean flux divergence is negligible on the seasonal time scale \citep{roberts2017}, so we can approximately relate the seasonality of surface fluxes to the mixed layer depth (d):
        \begin{equation} \label{eq:sfc-simple}
          \Delta F_{\mathrm{SFC}} = \Delta\left(\rho c_{w} d \frac{\partial T_{s}}{\partial t}\right)
        \end{equation}
  \item We substitute Equation~(\ref{eq:mse-toasfc-approx}) and (\ref{eq:sfc-simple}) into (\ref{eq:r1-term1}):
        \begin{equation} \label{eq:r1-linear2}
          \Delta R_{1} = \frac{1}{\overline{R_{a}}}\left(\Delta F_{\mathrm{TOA}}-\rho c_{w} d \Delta\frac{\partial T_{s}}{\partial t}\right) \, .
        \end{equation}
  \item Following \cite{rose2017}, we make the following assumptions to rewrite the RHS:
        \begin{enumerate}
          \item $\Delta F_{\mathrm{TOA}} \approx a\Delta Q - B\Delta T_{s}$, where $a$ is the co-albedo, $Q$ is insolation, and $B$ is the sensitivity of OLR to surface temperature $T_{s}$.
          \item $\Delta Q = Q^{*}\cos(\omega t)$, where $Q^{*}=as_{11}Q_{g}P_{1}(x)$ is the amplitude of net TOA shortwave, $s_{11}=-2\sin{\beta}$ where $\beta$ is the obliquity, and $Q_{g}=340$ Wm$^{-2}$. For simplicity we only consider the annual mode.
          \item $\Delta T_{s} = T_{s}^{*}\cos(\omega t - \Phi)$, where $T_{s}^{*}$ is the amplitude of surface temperature and $\Phi$ is the phase shift of $\Delta T_{s}$ relative to $\Delta Q$.
          \item As with insolation, we only consider the annual mode, so $T_{s}^{*}=Q^{*}B^{-1}\left((1+2\delta)^{2}+\gamma^{2}\right)^{-1/2}$ and $\Phi=\arctan\left(\frac{\gamma}{1+2\delta}\right)$.
        \end{enumerate}
  \item Using the above assumptions, we can write Equation~(\ref{eq:r1-linear2}) as
        \begin{equation} \label{eq:r1-linear3}
          \Delta R_{1} = \frac{1}{\overline{R_{a}}}\left(Q^{*}\cos(\omega t) -BT^{*}\cos(\omega t - \Phi)+\rho c_{w} d \omega T_{s}^{*}\sin(\omega t - \Phi) \right) \, .
        \end{equation}
  \item Substituting in $T_{s}^{*}$ and $\Phi$ and simplifying, we obtain
        \begin{equation} \label{eq:r1-linear4}
          \Delta R_{1} = \frac{Q^{*}}{\overline{R_{a}}}\frac{2\delta}{(1+2\delta)^{2}+\gamma^{2}}\left((1+2\delta)\cos(\omega t)+\gamma\sin(\omega t)\right) \, .
        \end{equation}
        \item While the predicted relationship between $T_{s}^{*}$ and the mixed layer depth agrees well with the ECHAM simulations (Fig.~B1(a)), the predicted amplitude of $R_{1}$ underestimates those in ECHAM (Fig.~B1(b)).
\end{itemize}

\subsection{Critical mixed layer depth}
\begin{itemize}
  \item We can define a critical mixed layer depth ($d_{c}$) as the depth that corresponds to the threshold of the RCE/RCAE regime transition. The threshold of the RCE/RCAE regime transition is expressed as
        \begin{equation}
          \min(R_{1}) = \overline{R_{1}} + \min(\Delta R_{1}) = \epsilon
        \end{equation}
  \item From Equation~(\ref{eq:r1-linear4}), we can compute the expression for $\min(\Delta R_{1})$
        \begin{equation}
          \min(\Delta R_{1}) = \frac{Q^{*}}{\overline{R_{a}}\sqrt{1+\tilde{C}^{2}}}
        \end{equation}
  \item Thus the critical mixed layer depth $d_{c}$ is
        \begin{equation}
          d_{c} = \frac{B}{\rho c_{w} \omega}\sqrt{\left(\frac{Q^{*}}{\overline{R_{a}}(\epsilon-\overline{R_{1}})}\right)^{2}-1} \, .
        \end{equation}
\item For $B=2$ Wm$^{-2}$K$^{-1}$, $\rho=1000$ kg m$^{-3}$, $c_{w}=4000$ J kg$^{-1} $K$^{-1}$, $Q^{*}=200$ W m$^{-2}$, $\overline{R_{a}}=-100$ W m$^{-2}$, $\epsilon=0.1$, and $\overline{R_{1}}=0.3$, we obtain a critical mixed layer depth of 25 m.

\end{itemize}


%%%%%%%%%%%%%%%%%%%%%%%%%%%%%%%%%%%%%%%%%%%%%%%%%%%%%%%%%%%%%%%%%%%%%
% REFERENCES
%%%%%%%%%%%%%%%%%%%%%%%%%%%%%%%%%%%%%%%%%%%%%%%%%%%%%%%%%%%%%%%%%%%%%
% Make your BibTeX bibliography by using these commands:
\bibliographystyle{ametsoc2014}
\bibliography{/home/omiyawaki/Sync/papers/references}


%%%%%%%%%%%%%%%%%%%%%%%%%%%%%%%%%%%%%%%%%%%%%%%%%%%%%%%%%%%%%%%%%%%%%
% TABLES
%%%%%%%%%%%%%%%%%%%%%%%%%%%%%%%%%%%%%%%%%%%%%%%%%%%%%%%%%%%%%%%%%%%%%
%% Enter tables at the end of the document, before figures.
%%
%
%\begin{table}[t]
%\caption{This is a sample table caption and table layout.  Enter as many tables as
%  necessary at the end of your manuscript. Table from Lorenz (1963).}\label{t1}
%\begin{center}
%\begin{tabular}{ccccrrcrc}
%\hline\hline
%$N$ & $X$ & $Y$ & $Z$\\
%\hline
% 0000 & 0000 & 0010 & 0000 \\
% 0005 & 0004 & 0012 & 0000 \\
% 0010 & 0009 & 0020 & 0000 \\
% 0015 & 0016 & 0036 & 0002 \\
% 0020 & 0030 & 0066 & 0007 \\
% 0025 & 0054 & 0115 & 0024 \\
%\hline
%\end{tabular}
%\end{center}
%\end{table}

%%%%%%%%%%%%%%%%%%%%%%%%%%%%%%%%%%%%%%%%%%%%%%%%%%%%%%%%%%%%%%%%%%%%%
% FIGURES
%%%%%%%%%%%%%%%%%%%%%%%%%%%%%%%%%%%%%%%%%%%%%%%%%%%%%%%%%%%%%%%%%%%%%
%% Enter figures at the end of the document, after tables.
%%
%
%\begin{figure}[t]
%  \noindent\includegraphics[width=19pc,angle=0]{figure01.pdf}\\
%  \caption{Enter the caption for your figure here.  Repeat as
%  necessary for each of your figures. Figure from \protect\cite{Knutti2008}.}\label{f1}
%\end{figure}

\begin{figure}
  \noindent\includegraphics[width=\textwidth]{/project2/tas1/miyawaki/projects/002/figures_post/final/temp_binned_r1/temp_binned_r1.pdf}\\
  \caption{Temperature profiles binned according to $R_{1}$ for (a) ERA5 reanalysis and (b) CMIP5 historical multi-model mean. Temperature profiles corresponding to the threshold of RCE ($R_1=0.1$) and RAE ($R_1=0.9$) are presented as thicker lines.}
  \label{fig:temp-binned-r1}
\end{figure}

\begin{figure}[t]
  \noindent\includegraphics[width=\textwidth]{/project2/tas1/miyawaki/projects/002/figures_post/final/r1z_ann/r1z_ann_era5.pdf}\\
  \caption{(a) The annual mean zonal-mean structure of $R_{1}$ for the ERA5 reanalysis. Orange, black, and blue regions indicate RCE, RCAE, and RAE, respectively. The annual-mean zonal-mean vertical temperature structure for RCE, RCAE, and RAE for (b) SH and (c) NH. The dotted lines indicate a moist adiabat.}
  \label{fig:era5-r1-ann}
\end{figure}

\begin{figure}[t]
  \noindent\includegraphics[width=\textwidth]{/project2/tas1/miyawaki/projects/002/figures_post/final/r1z_ann/r1z_ann_cmip5hist.pdf}\\
  \caption{Same as Fig.~\ref{fig:era5-r1-ann} but for the CMIP5 historical multi-model mean.}
  \label{fig:cmip5hist-r1-ann}
\end{figure}

\begin{figure}[t]
  \noindent\includegraphics[width=0.7\textwidth]{/project2/tas1/miyawaki/projects/002/figures_post/final/r1_dev/r1_dev_era5.pdf}\\
  \caption{(a) The seasonality of $R_{1}$ for the ERA5 reanalysis. The vertical temperature structure during (b,d) January and (c,e) June at 45$^{\circ}$ and 85$^{\circ}$ in the NH and SH.}
  \label{fig:era5-r1-dev}
\end{figure}

\begin{figure}[t]
  \noindent\includegraphics[width=0.7\textwidth]{/project2/tas1/miyawaki/projects/002/figures_post/final/r1_dev/r1_dev_cmip5hist.pdf}\\
  \caption{Same as Fig.~\ref{fig:era5-r1-dev} but for the CMIP5 historical multi-model mean.}
  \label{fig:cmip5hist-r1-dev}
\end{figure}

\begin{figure}[t]
  \noindent\includegraphics[width=\textwidth]{/project2/tas1/miyawaki/projects/002/figures_post/final/r1_decomp_mid/r1_decomp_mid_era5.pdf}\\
  \caption{The seasonality of $R_{1}$ in midlatitudes ($30$--$60^{\circ}$) and its deviation from the annual-mean for the (a) NH and (b) SH. The seasonality of the terms in the MSE budget in midlatitudes for the (c) NH and (d) SH.}
  \label{fig:era5-r1-decomp-mid}
\end{figure}

\begin{figure}[t]
  \noindent\includegraphics[width=\textwidth]{/project2/tas1/miyawaki/projects/002/figures_post/final/r1_decomp_mid/r1_decomp_mid_cmip5hist.pdf}\\
  \caption{Same as Fig.~\ref{fig:era5-r1-decomp-mid} but for the CMIP5 historical multi-model mean.}
  \label{fig:cmip5hist-r1-decomp-mid}
\end{figure}

\begin{figure}[t]
  \noindent\includegraphics[width=\textwidth]{/project2/tas1/miyawaki/projects/002/figures_post/final/r1_decomp_pole/r1_decomp_pole_era5.pdf}\\
  \caption{Same as Fig.~\ref{fig:era5-r1-decomp-mid} but averaged over the polar region ($80$--$90^{\circ}$).}
  \label{fig:era5-r1-decomp-pole}
\end{figure}

\begin{figure}[t]
  \noindent\includegraphics[width=\textwidth]{/project2/tas1/miyawaki/projects/002/figures_post/final/r1_decomp_pole/r1_decomp_pole_cmip5hist.pdf}\\
  \caption{Same as Fig.~\ref{fig:era5-r1-decomp-pole} but for the CMIP5 historical multi-model mean.}
  \label{fig:cmip5hist-r1-decomp-pole}
\end{figure}

\begin{figure}[t]
    \noindent\includegraphics[width=\textwidth]{/project2/tas1/miyawaki/projects/002/figures_post/final/r1_decomp_mid/r1_decomp_mid_echamslab.pdf}\\
    \caption{Same as Fig.~\ref{fig:era5-r1-decomp-mid} but for the ECHAM6 aquaplanet model with (a), (b) 15 m and (c),(d) 40 m mixed layer depth}
\label{fig:echam-rce}
\end{figure}

\begin{figure}[t]
    \noindent\includegraphics[width=\textwidth]{/project2/tas1/miyawaki/projects/002/figures_post/final/r1_decomp_pole/r1_decomp_pole_echamslab.pdf}\\
    \caption{Same as Fig.~\ref{fig:era5-r1-decomp-pole} but for the ECHAM6 aquaplanet with a 40 m mixed layer depth and (a),(b) with and (c),(d) without thermodynamic sea ice.}
    \label{fig:echam-rae}
\end{figure}

\begin{figure}[t]
  \noindent\includegraphics[width=0.8\textwidth]{malr-mon-lat.png}\\
  \appendcaption{A1}{The spatio-temporal structure of the free tropospheric lapse rate deviation from a moist adiabatic lapse rate is shown for a) ERA-Interim reanalysis and b) MPI-ESM-LR. We idenfity the region where \(\delta_{c}\le 10\%\) (thick orange contour) as the convective lapse rate regime.}
  \label{fig:malr-mon-lat}
\end{figure}

\begin{figure}[t]
  \noindent\includegraphics[width=0.8\textwidth]{dalr-mon-lat.png}\\
  \appendcaption{A2}{The spatio-temporal structure of the near surface lapse rate deviation from a dry adiabatic lapse rate is shown for a) ERA-Interim reanalysis and b) MPI-ESM-LR. We idenfity the regions where \(\delta_{i}\ge 90\%\) (thick blue contour) as the inversion lapse rate regime.}
  \label{fig:dalr-mon-lat}
\end{figure}

\begin{figure}
  \noindent\includegraphics[width=0.8\textwidth]{/project2/tas1/miyawaki/projects/002/figures_post/test/amp_r1_echam/amp_echam_del0-5.pdf}\\
  \appendcaption{B1}{(a) The seasonal amplitude of surface temperature between 30--60$^{\circ}$ latitude as diagnosed from ECHAM with varied mixed layer depths (asterisks) and that predicted from the \cite{rose2017} energy balance model (line) for $B=1.45$ Wm$^{-2}$K$^{-1}$, $\rho=1000$ kg m$^{-3}$, $c_{w}=4000$ J kg$^{-1} $K$^{-1}$, $a=0.68$, and $\delta=0.5$. (b) Minimum of the seasonal deviation of $R_{1}$ as diagnosed from ECHAM (asterisks) and the EBM (line).}
  \label{fig:amp-r1-echam}
\end{figure}

\end{document}
