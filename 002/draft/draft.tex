%% Version 5.0, 2 January 2020
%
%%%%%%%%%%%%%%%%%%%%%%%%%%%%%%%%%%%%%%%%%%%%%%%%%%%%%%%%%%%%%%%%%%%%%%
% TemplateV5.tex --  LaTeX-based template for submissions to the 
% American Meteorological Society
%
%%%%%%%%%%%%%%%%%%%%%%%%%%%%%%%%%%%%%%%%%%%%%%%%%%%%%%%%%%%%%%%%%%%%%
% PREAMBLE
%%%%%%%%%%%%%%%%%%%%%%%%%%%%%%%%%%%%%%%%%%%%%%%%%%%%%%%%%%%%%%%%%%%%%

%% Start with one of the following:
% DOUBLE-SPACED VERSION FOR SUBMISSION TO THE AMS
\documentclass{ametsocV5}

% TWO-COLUMN JOURNAL PAGE LAYOUT---FOR AUTHOR USE ONLY
% \documentclass[twocol]{ametsocV5}


% Enter packages here. If too many math alphabets are used,
% remove unnecessary packages or define hmmax and bmmax as necessary.

%\newcommand{\hmmax}{0}
%\newcommand{\bmmax}{0}
\usepackage{amsmath,amsfonts,amssymb,bm}
\usepackage{mathptmx}%{times}
\usepackage{newtxtext}
\usepackage{newtxmath}


%%%%%%%%%%%%%%%%%%%%%%%%%%%%%%%%

%%% To be entered by author:

%% May use \\ to break lines in title:

\title{When and where do Radiative--Convective and Radiative--Advective Equilibrium regimes occur on modern Earth?}

%%% Enter authors' names, as you see in this example:
%%% Use \correspondingauthor{} and \thanks{Current Affiliation:...}
%%% immediately following the appropriate author.
%%%
%%% Note that the \correspondingauthor{} command is NECESSARY.
%%% The \thanks{} commands are OPTIONAL.

    %\authors{Author One\correspondingauthor{Author name, email address}
% and Author Two\thanks{Current affiliation: American Meteorological Society, 
    % Boston, Massachusetts.}}

\authors{Osamu Miyawaki\correspondingauthor{Osamu Miyawaki, miyawaki@uchicago.edu}, Tiffany A. Shaw, and Malte F. Jansen}

%% Follow this form:
    % \affiliation{American Meteorological Society, 
    % Boston, Massachusetts}

\affiliation{The University of Chicago, Chicago, Illinois}

%% If appropriate, add additional authors, different affiliations:
    %\extraauthor{Extra Author}
    %\extraaffil{Affiliation, City, State/Province, Country}

%\extraauthor{}
%\extraaffil{}

%% May repeat for a additional authors/affiliations:

%\extraauthor{}
%\extraaffil{}

%%%%%%%%%%%%%%%%%%%%%%%%%%%%%%%%%%%%%%%%%%%%%%%%%%%%%%%%%%%%%%%%%%%%%
% ABSTRACT
%
% Enter your abstract here
% Abstracts should not exceed 250 words in length!
%
 

\abstract{Conceptual models of an atmospheric column provide a basis to understand the lapse rate structure and its response to climate change. Specifically, Radiative-Convective Equilibrium (RCE) and Radiative-Advective Equilibrium (RAE) have been used for investigating low and high latitude climate change, respectively. Currently we do not have a complete understanding of the spatio-temporal structure of RCE and RAE. Here, we use the vertically-integrated moist static energy budget to define a nondimensional number that quantifies when and where RCE and RAE are approximately satisfied in reanalysis products and models. We find RCE exists yearround in the tropics and in the Northern midlatitudes during summertime. RAE exists yearround poleward of $\approx 70^{\circ}$S over Antarctica but only outside of summer in the Arctic. We show that the lapse rates in RCE and RAE regimes in both reanalyses and GCMs are broadly consistent with lapse rate profiles of a moist adiabat and a surface inversion, respectively. Finally, we vary the mixed layer depth in idealized aquaplanet simulations with or without thermodynamic sea ice to test the following hypotheses: 1) the RCE regime occurs during midlatitude summer for land-like (small heat capacity) surface conditions and 2) sea ice is necessary for the existence of the RAE regime. Consistent with our first hypothesis, we find that an aquaplanet model configured with a 15 m slab ocean (NH-like) transitions to RCE in the summer whereas the 40 m slab ocean (SH-like) does not. Furthermore, we show that sea ice is a necessary and sufficient condition for the existence of RAE during wintertime.}

\begin{document}

%% Necessary!
\maketitle

%%%%%%%%%%%%%%%%%%%%%%%%%%%%%%%%%%%%%%%%%%%%%%%%%%%%%%%%%%%%%%%%%%%%%
% SIGNIFICANCE STATEMENT/CAPSULE SUMMARY
%%%%%%%%%%%%%%%%%%%%%%%%%%%%%%%%%%%%%%%%%%%%%%%%%%%%%%%%%%%%%%%%%%%%%
%
% If you are including an optional significance statement for a journal article or a required capsule summary for BAMS 
% (see www.ametsoc.org/ams/index.cfm/publications/authors/journal-and-bams-authors/formatting-and-manuscript-components for details), 
% please apply the necessary command as shown below:
%
% \statement
% Significance statement here.
%
% \capsule
% Capsule summary here.


%%%%%%%%%%%%%%%%%%%%%%%%%%%%%%%%%%%%%%%%%%%%%%%%%%%%%%%%%%%%%%%%%%%%%
% MAIN BODY OF PAPER
%%%%%%%%%%%%%%%%%%%%%%%%%%%%%%%%%%%%%%%%%%%%%%%%%%%%%%%%%%%%%%%%%%%%%
%

%% In all cases, if there is only one entry of this type within
%% the higher level heading, use the star form: 
%%
% \section{Section title}
% \subsection*{subsection}
% text...
% \section{Section title}

%vs

% \section{Section title}
% \subsection{subsection one}
% text...
% \subsection{subsection two}
% \section{Section title}

%%%
% \section{First primary heading}

% \subsection{First secondary heading}

% \subsubsection{First tertiary heading}

% \paragraph{First quaternary heading}

\section{Introduction}

The Earth's climate is maintained by three types of heat transfer: advection, radiation, and conduction \citep{hartmann2016}. These heat transfer types can be most easily defined using the vertically-integrated annual-mean zonal-mean moist static energy (MSE) budget:
\begin{equation} \label{eq:mse-ann}
    {\underbrace{\nabla\cdot \langle \overline{[F_{m}]}\rangle}_{\text{advection}}} = {\underbrace{\vphantom{\nabla\cdot \langle [F_{m}]\rangle} \overline{[R_{a}]}}_\text{radiation}} + {\underbrace{\vphantom{\nabla\cdot \langle [F_{m}]\rangle} \mathrm{\overline{[LH]}+\overline{[SH]}}}_\text{conduction}} \, ,
\end{equation}
where $h=c_p T + gz + Lq$ is MSE, $\overline{(\cdot)}$ is the annual mean, $[\cdot]$ is the zonal mean, and $\langle \cdot \rangle$ is the mass-weighted vertical integral \citep{neelin1987}. Advection corresponds to the divergence of MSE flux ($F_m=vh$ where $v$ is the meridional velocity) and represents the heat transferred by the atmospheric circulation, including the Hadley cell and storm tracks. Radiation ($R_a$) corresponds to the sum of the radiative fluxes through the top of the atmosphere and surface. Finally, conduction corresponds to surface latent ($\mathrm{LH}$) and sensible ($\mathrm{SH}$) heat flux.

The dominant types of heat transfer depend on latitude \citep[e.g., see Fig.~6.1 in][]{hartmann2016}. In the low latitudes, atmospheric radiative cooling is primarily balanced by conductive fluxes \citep{riehl1958}, which destabilize the column to convection by supplying moist, warm air to the boundary layer. The dominant balance between radiative cooling and surface heat fluxes is consistent with the assumptions behind Radiative Convective Equilibrium \citep[RCE,][]{wing2018}. In the high latitudes, atmospheric radiative cooling is primarily balanced by advection \citep{nakamura1988}, consistent with Radiative-Advective Equilibrium \citep[RAE,][]{cronin2016}. Finally in the midlatitudes, all three types of heat transfer are important; thus, we introduce the term Radiative-Convective-Advective Equilibrium (RCAE). In this way, three heat transfer regimes qualitatively characterize the low, mid, and high latitude regions of Earth's modern climate.

The lapse rate structure can also characterize low, mid, and high latitude regions of Earth's modern climate. Lapse rates in the low latitudes are close to moist adiabatic \citep{stone1979,betts1982,xu1989,williams1993}. The high latitudes typically feature a surface inversion \citep[e.g., see Fig.~1.3 in][]{hartmann2016} because the surface can cool efficiently through the atmospheric window \citep{cronin2016} and the vertical profile of advective heat transport peaks slightly above the surface \citep{oort1974, overland1994, hahn2020, cardinale2021}. The inversions exhibit seasonality in the Northern Hemisphere, where the inversion frequency and strength decrease \citep{bradley1992, tjernstrom2009, devasthale2010, zhang2011, cronin2016} and in some cases vanish during summertime \citep{stone1979}. Lapse rates in the midlatitudes are typically more stable than a moist adiabat \citep{stone1979,korty2007} due to sensible and latent heating associated with baroclinic eddies, but do not typically show surface inversions. An exception occurs during Northern Hemisphere summer where the midlatitude lapse rate is within 20\% of a moist adiabat \citep{stone1979}.

Heat transfer regimes and lapse rates clearly share a similar latitudinal dependence in the annual mean. However, few studies to date have quantified 1) where heat transfer regimes are observed and 2) the link between heat transfer regimes and the lapse rate structure through the seasonal cycle.

% Clearly, heat transfer regimes and vertical temperature profiles share a qualitatively similar latitudinal structure. Here, we seek to answer two questions: when and where are idealized heat transfer regimes satisfied on the modern Earth, and how closely are heat transfer regimes and the vertical temperature profiles linked? The answer to these questions would have several important consequences.

Quantifying where heat transfer regimes are observed would allow us to assess where idealized models that assume RCE or RAE hold. This is particularly important for RCE, which has become a standard idealized configuration for tropical theories \citep[e.g.,][]{emanuel1996,nilsson1999,romps2014,singh2015} and simulations \citep[][and the references therein]{wing2018}. Recent work by \cite{jakob2019} partially answered this question for RCE. They define RCE using the dry static energy (DSE) budget and a dimensional threshold (sum of radiative cooling, latent heating, and sensible heating $< \pm 50$ W m$^{-2}$). They find that RCE is approximately satisfied over large spatial ($>5000$ km) and temporal ($>$ daily) scales in the tropics. However, the occurence of the remaining two heat transfer regimes and RCE outside the tropics has not been investigated.

Quantifying the link between heat transfer regimes and lapse rate structure would allow us to better understand the regional response to CO$_2$ forcing. For example, regions of RCE that exhibit a moist adiabatic lapse rate will have amplified warming aloft in response to increased CO$_2$ \citep{romps2011}. Similarly, regions of RAE that exhibit a surface inversion will have amplified warming at the surface in response to increased CO$_2$ \citep{cronin2016}.% Then, simple energy balance models (EBMs) can be used to understand and predict the vertical temperature response, even if the latter is not explicitly solved by the model.

We therefore seek to answer the questions: when and where do RCE, RAE, and RCAE occur on the modern Earth, and how closely are they linked to the moist adiabatic and surface inversion lapse rate structure? To answer these questions, we develop a quantitative definition for RCE and RAE regimes using the nondimensionalized MSE budget (Section~\ref{sec:methods}\ref{subsec:mse}). We use this definition to examine where and when RCE and RAE occur both in the annual mean and seasonally on modern Earth and quantify the connection to the moist adiabatic and surface inversion lapse rates (Section~\ref{sec:diagnostics}). Finally, we use a hierarchy of climate models to test hypotheses that explain the seasonality of heat transfer regimes (Section~\ref{sec:hypo}).

\section{Methods}\label{sec:methods}

    \subsection{Defining heat transfer regimes using the nondimensionalized MSE budget} \label{subsec:mse}

    In order to define heat transfer regimes seasonally, we begin with the vertically-integrated, zonal-mean MSE equation including the storage term:
    \begin{equation} \label{eq:mse}
        \frac{\partial \langle [h] \rangle}{\partial t} + \nabla\cdot \langle [F_{m}]\rangle = [R_{a}] + \mathrm{[LH]+[SH]} \, ,
    \end{equation}
    where $\partial_t h$ represents atmospheric storage of MSE. We nondimensionalize (\ref{eq:mse}) by dividing by radiative cooling $R_a$:
    \begin{equation}
        {\underbrace{\frac{\frac{\partial h }{\partial t} + \nabla\cdot F_{m}}{R_{a}}}_{R_1}} = 1 + {\underbrace{\frac{\mathrm{LH+SH}}{R_{a}}}_{R_2}} \, ,
    \end{equation}
    where $R_1$ and $R_2$ are nondimensional numbers and the $[\cdot]$ and $\langle\cdot\rangle$ have been dropped for brevity. 
    
    In the strictest sense, RCE requires a steady-state equilibrium where radiation balances conduction (\(R_{1}=0\)). As this is exactly satisfied only in the global mean, we define RCE as \(R_{1}\le \varepsilon\), where $\varepsilon$ is a threshold parameter. This definition includes regions of MSE flux divergence and weak convergence. MSE flux divergence typically occurs in the upper troposphere and destabilizes the column. The temperature profiles of such columns are also set by convective adjustment \citep{warren2020} and thus are included in the definition of approximate RCE.
    
    RAE as defined in \cite{cronin2016} requires conduction to be negligibly small (\(R_{2}=0\) or equivalently \(R_{1}=1\)). Although exact RAE further requires $\partial_t h=0$, the framework developed by \cite{cronin2016} could readily be generalized to account for the time tendency term, which would have the same effect as the advective tendency. To be consistent with the approximate definition of RCE, we define RAE as regions where conductive fluxes are small or directed from the atmosphere to the surface (\(R_{2} \ge -\varepsilon \) or equivalently \(R_{1} \ge 1-\varepsilon\)).
      
    In order to choose the value for $\varepsilon$, we examine the annual-mean zonal-mean percent deviation of the lapse rate from a moist adiabatic lapse rate binned by the value of $R_1$ using reanalysis data. The lapse rate deviation is plotted in sigma coordinates to ensure that surface inversions are properly represented (see Appendix~A for more details). The tropospheric lapse rate deviation is nearly a monotonic function of $R_1$, making it a suitable metric not only for quantifying heat transfer regimes, but for categorizing lapse rate structures as well. A surface inversion is observed for $R_1 \ge 0.9$ (where lapse rate deviation exceeds 100\% in Fig.~\ref{fig:rea-binned-r1}) and thus we choose $R_1=1-\varepsilon=0.9$ as the threshold for the RAE regime. We use $R_1=\varepsilon=0.3$ to define the boundary of the RCE regime, where the lapse rate deviation vertically-averaged from $\sigma=0.8$ to 0.3 is 20\% from a moist adiabatic lapse rate. We choose 20\% deviation from a moist adiabat as the threshold for RCE since the lapse rate in the low latitudes are known to be within 20\% of a moist adiabat \citep{stone1979}. Finally, RCAE is defined for the intermediate values of $0.3<R_1<0.9$.
    
    \subsection{Reanalysis data}\label{subsec:reanalysis}
    
    We consider three reanalysis data sets from 1980--2005: ERA5 \citep{hersbach2020}, MERRA2 \citep{gelaro2017}, and JRA55 \citep{kobayashi2015}. We focus on the heat transfer regimes and lapse rate structure for the multi-reanalysis mean and show the spread as the range across the three reanalyses. $\partial_t h$ is computed by taking the finite difference of MSE using monthly temperature, specific humidity, and geopotential data. Additionally, we use the monthly radiative ($R_a$) and conductive ($\mathrm{LH}$ and $\mathrm{SH}$) fluxes and infer $\nabla\cdot F_m$ as the residual. We choose to infer $\nabla\cdot F_m$ as the residual because the mass-correction technique for directly computing the MSE flux divergence in reanalysis data is known to produce unphysical results in the high latitudes \citep{porter2010}. 
    
    %This is important for ensuring that a well-posed comparison between the heat transfer regimes and the vertical temperature profile. 

    %Monthly radiation ($R_a$) and conduction ($\mathrm{LH}$ and $\mathrm{SH}$) are available as standard outputs for all reanalysis datasets. There are two ways to compute the $\partial_t h + \nabla\cdot F_m$ term for reanalysis data: 1) directly computing it using sub-daily frequency data and applying a mass correction technique to address the issue of mass conservation in reanalyses \citep{trenberth1997} and 2) inferring it as the residual of the remaining terms in the MSE budget, i.e. $R_a + \mathrm{LH + SH}$. Here, we take the second approach and infer $\partial_t h +\nabla\cdot F_m$ as the residual. We choose to infer $\partial_t h +\nabla\cdot F_m$ as the residual here because the mass-correction technique for directly computing the MSE flux divergence in reanalysis data is known to produce unphysical results in the high latitudes \citep{porter2010}. The second approach is also important for ensuring a well-posed comparison between the heat transfer regimes and the vertical temperature profile.
    
    % As the high latitudes are characterized by surface turbulent fluxes that are close to zero in the annual mean, small changes in the energy balance (of the order of $\epsilon R_a \approx 10$ W m$^{-2}$) are enough to result in regime transitions through the annual cycle; thus, it is vital that the magnitude and sign of surface turbulent fluxes used for this analysis are as physically consistent as possible. This approach is particularly important for ensuring that a proper comparison between the heat transfer regimes and the vertical temperature structure can be made. 
    
    \subsection{Climate model hierarchy}\label{subsec:models}
    We use a hierarchy of climate models to understand the seasonal changes in heat transfer regimes. To be consistent with the reanalysis products, we compute $R_1$ for all climate models using the monthly tendency of MSE ($\partial_t h$), monthly $R_a$, $\mathrm{LH}$, and $\mathrm{SH}$, and infer $\nabla\cdot F_m $ as the residual.
    
    At the complex end, we consider the historical run of the CMIP5 archive from 1980--2005 \citep{taylor2012}. The historical runs are atmosphere-ocean general circulation models forced with the observed atmospheric composition. We focus on the heat transfer regimes and vertical temperature profile for the multi-model mean of 41 models (see Table~D1) and show the spread as the interquartile range across the models.
    
    At intermediate complexity, we examine seasonal changes in the ECHAM6 slab-ocean aquaplanet model \citep{stevens2013}, hereafter referred to as AQUA. The AQUA simulations are configured with a seasonal cycle, no ocean heat transport, modern greenhouse gas concentrations, and with or without thermodynamic sea ice following \cite{shaw2020}. In order to explore the seasonal variation in heat transfer regimes, we vary the mixed layer depth in AQUA from 3 to 50 m following previous work \citep{donohoe2013, barpanda2020}. A monthly climatology is obtained by averaging the last 20 years of the 40 year simulation except for the 3 m configuration, where the last 5 years of a 15 year simulation are averaged due to the faster equilibration time.
    
    % We use the aquaplanet to test our hypotheses that mixed layer depth controls the seasonality of heat transfer regimes in the midlatitudes and sea ice controls the seasonality in polar regions.

    At the simple model end, we use the EBM of \cite{rose2017}. The EBM is an equation for the zonal-mean surface temperature:
    \begin{equation}
      \rho c_w d \frac{\partial T_s}{\partial t} = aQ - (A+BT_s)  + D \frac{1}{\cos\phi}\frac{\partial}{\partial \phi}\left( \cos\phi \frac{\partial T_s}{\partial \phi} \right)\, ,
    \end{equation}
    where $\rho$ is the density of water, $c_w$ is the specific heat capacity of liquid water, $d$ is the mixed layer depth, $T_s$ is the surface temperature, $a$ is the co-albedo, $Q$ is insolation, $\mathrm{OLR}=A+BT_s$ is outgoing longwave radiation where $A$ and $B$ are constant coefficients, $\phi$ is latitude, and $D$ is the diffusivity, which is assumed to be a constant. We set $A=-410$ W m$^{-2}$, $B=2.33$ W m$^{-2}$ K$^{-1}$, $D=0.90$ W m$^{-2}$ K$^{-1}$, and $a=0.72$, which are obtained from best fits to AQUA configured with a 25 m mixed layer depth and without sea ice. Best fits of $A$ and $B$ are obtained by taking the least squares linear regression of the zonal-mean $\mathrm{OLR}$ and $T_s$. The best fit of $D$ is obtained similarly by taking the least squares linear regression of $\nabla\cdot F_m$ and $\frac{1}{\cos\phi}\frac{\partial}{\partial \phi} \left( \cos\phi \frac{\partial T_s}{\partial \phi} \right)$ for latitudes poleward of $25^\circ$. Lastly, $a$ is computed as $1-\alpha_p$ where $\alpha_p$ is the globally-averaged diagnosed planetary albedo.

\section{Heat transfer regimes in reanalysis data} \label{sec:diagnostics}

    \subsection{Annual mean heat transfer regimes}

    In the annual mean, the RCE regime, defined by $R_1 \le 0.3$, extends from the deep tropics to $\approx 45^\circ$ (black line overlapping orange region in Fig.~\ref{fig:rea-r1-ann}a). Consistently, the lapse rate in the region where $R_1 \le 0.3$ is close to the moist adiabatic lapse rate ($0.6$\% and 6.0\% deviation averaged from $\sigma=0.8$ to 0.3 for the Northern and Southern Hemispheres, respectively; see orange lines in Figs.~\ref{fig:rea-r1-ann}b and \ref{fig:rea-r1-ann}c).

    The RAE regime defined by $R_1 \ge 0.9$ occurs poleward of $\approx 80^\circ$N and $\approx 70^\circ$S in the annual mean (black line overlapping blue region in Fig.~\ref{fig:rea-r1-ann}a). The reanalysis spread in the high latitudes is large in both hemispheres, consistent with the uncertainty in the estimation of conductive fluxes \citep{tastula2013,graham2019}. The largest values of $R_1$ are found over Antarctica whereas $R_1$ is close to the RCAE threshold in the Northern Hemisphere. Consistently, the region where $R_1 \ge 0.9$ exhibits a stronger surface inversion in the Southern high latitudes compared to the North (see blue lines in Figs.~\ref{fig:rea-r1-ann}b and 2c).

    Lastly, the RCAE regime defined by $0.3 < R_1 < 0.9$ occurs between $45$--$80^\circ$N and $45$--$70^\circ$S in the annual mean (black line overlapping the white region in Fig.~\ref{fig:rea-r1-ann}a). The lapse rate profile in the region where $0.3 < R_1 < 0.9$ is clearly more stable than a moist adiabat (27.8\% and 28.1\% more stable averaged from $\sigma=0.8$ to 0.3 in the Northern and Southern Hemisphere, respectively) but does not exhibit a surface inversion (see gray lines in Fig.~\ref{fig:rea-r1-ann}b and 2c).

    \subsection{Seasonality of RCE and RAE} \label{subsec:seasonality}

    Seasonally, the RCE regime defined by $R_1 \le 0.3$ occurs yearround equatorward of $45^\circ$ and up to $70^\circ$N during Northern Hemisphere summer (region equatorward of the thick orange contour in Fig.~\ref{fig:rea-r1-dev}a). The free-tropospheric (averaged from $\sigma=0.8$ to 0.3) lapse rate deviation from a moist adiabat exhibits a similar pattern, where close to moist adiabatic lapse rates are found not only in the low latitudes yearround but during Northern midlatitude summer (Fig.~\ref{fig:rea-r1-dev}b). More specifically, the spatial structure of RCE and near-moist adiabatic lapse rate regimes are in agreement within $\pm5^\circ$ latitude. For example, during January, RCE occurs from $45^\circ$S to $45^\circ$N and the free tropospheric lapse rate is within 20\% from $50^\circ$S to $40^\circ$N. Similarly, during July, RCE occurs from $45^\circ$S to $70^\circ$N and the free tropospheric lapse rate is within 20\% from $50^\circ$S to $75^\circ$N. The temporal structure of RCE and near-moist adiabatic lapse rate regimes are in agreement within 2 months. The Southern midlatitudes are in RCAE yearround and consistently, the free-tropospheric lapse rate deviation exceeds 20\% of a moist adiabat yearround (Fig.~\ref{fig:rea-r1-ga-temporal}a). In the Northern midlatitudes, RCE occurs from April through August. In comparson, the lapse rate seasonality is delayed by one to two months, where lapse rates within 20\% of a moist adiabat are found from May through October (compare solid black and red lines in Fig.~\ref{fig:rea-r1-ga-temporal}b). Interestingly, when $R_1$ is defined excluding the MSE storage term, the seasonality of RCE and near moist adiabatic lapse rate are in good agreement (compare dashed black and red lines in Fig.~\ref{fig:rea-r1-ga-temporal}b).
    
    % closely follows the poleward migration of the northern boundary of RCE during summer, where near moist adiabatic stratification extends out to $\approx 60^\circ$N in the summer (thick red contour in Fig.~D1a).

    The RAE regime occurs throughout the seasonal cycle in the high latitudes with the exception of May and June in the Arctic (region poleward of the thick blue contour in Fig.~\ref{fig:rea-r1-dev}a). The boundary layer lapse rate deviation from a moist adiabat is broadly consistent with the seasonality of $R_1$ (compare Fig.~\ref{fig:rea-r1-dev}c and \ref{fig:rea-r1-dev}a), as the Southern high latitudes remains strongly stable yearround, whereas the stability weakens during summertime in the Northern high latitudes. The spatial structure of RAE and inversion lapse rate regimes are in agreement within $\pm 15^\circ$ latitude. For example, in the Southern Hemisphere, RAE occurs poleward of $70^\circ$S in both January and July and boundary layer inversions occur poleward of $80^\circ$S and $70^\circ$S in January and July, respectively. In the Northern Hemisphere, RAE occurs poleward of $75^\circ$N in both January and July, while inversions occur poleward of $60^\circ$N in January and no inversions are present in July. The temporal structure of RAE and inversion regimes exhibit some key discrepancies. In the Southern Hemisphere, RAE occurs yearround and likewise inversions are present nearly yearround except in December (Fig.~\ref{fig:rea-r1-ga-temporal}c). In the Northern Hemisphere, RAE occurs from July through April, while inversions are found from October through April; that is, there is a three month discrepancy between the onset of RAE and an inversion (compare solid black line with blue line in Fig.~\ref{fig:rea-r1-ga-temporal}). Unlike the midlatitudes, these discrepancies are not reconciled by using the alternative definition of $R_1$ where MSE storage is excluded (compare dashed black line to blue line in Fig.~\ref{fig:rea-r1-ga-temporal}).

    %Consistently, the temperature profiles at $85^\circ$S in both January and June (blue lines in Fig.~\ref{fig:rea-temp-sel}c,d) show a surface inversion, while at $85^\circ$N a surface inversion is seen in January but not in June (Fig.~\ref{fig:rea-temp-sel}a and b).
    
    %More generally, the surface lapse rate deviation from a moist adiabat shows that the Southern high latitudes remains strongly stable yearround, whereas the stability weakens during summertime in the Northern high latitudes (compare blue contours in Fig.~\ref{fig:rea-r1-dev}c and \ref{fig:rea-r1-dev}a). %The timing of the reduced stability in the Northern high latitude lapse rate as measured by the surface lapse rate deviation from a dry adiabat (defined in Appendix C) corresponds well with that of the heat transfer regime transition.
     
    Lastly, RCAE occurs predominantly in the midlatitudes, where it is found yearround in the Southern Hemisphere and from August to April in the Northern Hemisphere. In addition, RCAE occurs in the Northern high latitudes during May and June (Fig.~\ref{fig:rea-r1-dev}a). It follows from the analysis above that the equatorward boundary of RCAE and the $>20$\% lapse rate deviation regime are in agreement within $5^\circ$ latitude and 2 months, while the poleward boundary, particularly in the Northern Hemisphere, exhibits larger discrepancies. %The temperature profile at $85^\circ$N in June is also consistent with a state of RCAE, as the vertical temperature profile is more stable than a moist adiabat but does not exhibit a surface inversion (gray line in Fig.~\ref{fig:rea-ga-fr-sel}b).

    % Overall, the Northern Hemisphere mid and high latitudes exhibit heat transfer regime transitions whereas the Southern Hemisphere does not. These regime transitions also correspond to changes in the vertical temperature profile.

    \subsection{Decomposition of seasonality in heat transfer regimes}

    In order to diagnose the physical mechanism responsible for the seasonal regime transitions, we decompose the seasonality of $R_1$ as follows:
    \begin{equation}\label{eq:r1-dev}
      \Delta R_1 = \overline{R_1}\left( \frac{\Delta\partial_t h + \nabla\cdot F_m}{\overline{\partial_t h + \nabla\cdot F_m}}  - \frac{\Delta R_a }{\overline{R_a}}\right) + \mathrm{Residual} \, ,
      % \Delta R_1 = \frac{1}{\overline{R_a}}\left( \Delta(\partial_t h + \nabla\cdot F_m)  - \overline{R_1} \Delta R_a \right) + \mathrm{Residual} \, ,
      % \Delta R_1 = R_1 - \overline{R_1} = \underbrace{\frac{\Delta(\partial_t h + \nabla\cdot F_m)}{\overline{R_a}} \vphantom{\frac{\overline{R_1}}{\overline{R_a}}} }_{\text{dynamic component}} - \underbrace{\frac{\overline{R_1}}{\overline{R_a}}\Delta R_a}_{\text{radiative component}} + \mathrm{Residual} \, ,
      % \Delta R_1 = R_1 - \overline{R_1} = \underbrace{\frac{\Delta(\partial_t h + \nabla\cdot F_m)}{\overline{R_a}} \vphantom{\frac{\overline{\partial_t h + \nabla\cdot F_m}}{\overline{R_a}^2}} }_{\text{dynamic component}} - \underbrace{\frac{\overline{\partial_t h + \nabla\cdot F_m}}{\overline{R_a}^2}\Delta R_a}_{\text{radiative component}} + \mathrm{Residual} \, ,
    \end{equation}
    % \begin{equation}
    %   \Delta R_1 = R_1 - \overline{R_1} = \frac{1}{\overline{R_a}^2}\left(\overline{R_a} \Delta(\partial_t h + \nabla\cdot F_m) - \overline{(\partial_t h + \nabla\cdot F_m)}\Delta R_a\right) + \mathcal{O}(\Delta(\cdot)^2) \, ,
    % \end{equation}
    where $\Delta(\cdot)$ is the seasonal deviation and $\overline{(\cdot)}$ is the annual mean. The dynamic component (first term) quantifies the importance of the fractional seasonality (meaning seasonality relative to the annual mean) of advection and storage. The radiative component (second term) quantifies the importance of the fractional seasonality of radiative cooling. Lastly, the residual quantifies the importance of nonlinear interactions.

    The regime transition in the Northern midlatitudes (where the solid black line intersects the orange region in Fig.~\ref{fig:rea-r1-decomp-mid}a) closely follows the dynamic component (compare black and red lines in Fig.~\ref{fig:rea-r1-decomp-mid}a) and the other terms are small (gray and dash-dot line in Fig.~\ref{fig:rea-r1-decomp-mid}a). The dynamic component dominates in the Northern Hemisphere because the fractional seasonality of advection and storage is stronger than the fractional seasonality of radiative cooling (red and gray lines in Fig.~\ref{fig:rea-r1-decomp-mid}b). In the Southern Hemisphere, the seasonality of advection and storage is weaker (red line in Fig.~\ref{fig:rea-r1-decomp-mid}d), the dynamic and radiative components are similar in magnitude (red and gray lines in Fig.~\ref{fig:rea-r1-decomp-mid}c), and the $R_1$ seasonality is small (black line in Fig.~\ref{fig:rea-r1-decomp-mid}). Thus, the hemispheric asymmetry in the seasonality of advection and storage plays an important role in the asymmetry of midlatitude regime transitions. %The seasonality of the conductive fluxes is weak (blue and orange lines in Fig.~\ref{fig:rea-r1-decomp-mid}b) and thus .

    % In contrast, there is no regime transition in the Southern midlatitudes because the dynamic component exhibits smaller seasonality (Fig.~\ref{fig:rea-r1-decomp-mid}c). While the seasonality of radiative cooling is similar in both hemispheres (compare gray lines in Fig.~\ref{fig:rea-r1-decomp-mid}b and \ref{fig:rea-r1-decomp-mid}d), the seasonality of advection and storage is damped in the Southern Hemisphere as the seasonality of conductive fluxes are larger and compensate radiative cooling. %Unlike in the Northern Hemisphere, the seasonality in radiative cooling is largely balanced by surface latent and sensible heat fluxes (Fig.~\ref{fig:rea-r1-decomp-mid}d).

    The regime transition in the Northern high latitudes is a small residual of the dynamic and radiative components (Fig.~\ref{fig:rea-r1-decomp-pole}a). As a result, the seasonality of $R_1$ closely follows the seasonality of conductive fluxes. In the Northern Hemisphere, the regime transition can be associated with a small but significant increase in latent heat flux during summertime (blue line in Fig.~\ref{fig:rea-r1-decomp-pole}b). In the Southern Hemisphere, there is no regime transition (Fig.~\ref{fig:rea-r1-decomp-pole}c) despite having a similar amplitude of $R_1$, which is associated with the seasonality in the sensible heat flux (orange line in Fig.~\ref{fig:rea-r1-decomp-pole}d). The lack of a regime transition in the Southern Hemisphere is associated with the hemispheric asymmetry in annual mean $R_1$, where $\overline{R_1}=1.36$ in the Southern Hemisphere compared to $R_1=0.97$ in the Northern Hemisphere, meaning that the Southern high latitudes is farther from the regime transition threshold. Thus, the hemispheric asymmetry in the annual mean $R_1$ plays an important role in the asymmetry of high latitude regime transitions.

    %$\Delta R_1$ in the Northern high latitudes is a small residual of the compensation between the large dynamic and radiative components. While there is a regime transition during May and June in the multi-reanalysis mean (black line intersecting white region in Fig.~\ref{fig:rea-r1-decomp-pole}a), the spread is large. The Northern high latitude regime transition seems to be associated with the small but significant increase of the latent heat flux from the ocean to the atmosphere (blue line in Fig.~\ref{fig:rea-r1-decomp-pole}b).


    % Similarly, the dynamic and radiative components in the Southern high latitudes strongly oppose each other and $R_1$ seasonality is weak (Fig.~\ref{fig:rea-r1-decomp-pole}c). In this case the there is no regime transition because the annual mean value of $R_1=1.36$ in the Southern high latitudes is well over the threshold of the regime transition. Consistently, radiative cooling is weaker and conductive fluxes transfer heat from the atmosphere to the surface yearround (Fig.~\ref{fig:rea-r1-decomp-pole}d).

\section{Testing hypotheses to explain seasonal regime transitions using a climate model hierarchy} \label{sec:hypo}

  % The diagnostic analysis of the seasonal regime transitions in the Northern Hemisphere suggests that the seasonality of MSE storage and advection is important for the midlatitudes, whereas the seasonality of conduction is important for the high latitudes. 
  
  The seasonality of $R_1$ and the regime transitions are also captured by the CMIP5 multi-model mean (see Figs.~D1--D6). The diagnostic analysis of the seasonal regime transitions in the CMIP5 multi-model mean also suggests that the Northern midlatitude regime transition is associated with a strong seasonality of MSE storage and advection, and the Northern high latitude regime transition is associated with a positive latent heat flux during summertime. Next, we seek a causal understanding of the factors controlling the seasonal regime transitions in the Northern mid and high latitudes using the EBM and aquaplanet simulations (AQUA).
  
  %Here, we use AQUA to test our hypotheses that mixed layer depth and sea ice control the seasonality of heat transfer regimes.
  
  \subsection{Midlatitude regime transition} \label{subsec:mld}

  Previous studies have found that surface heat capacity plays an important role in the seasonality of various climate phenomena, such as surface temperature \citep{donohoe2014}, ITCZ \citep{bordoni2008}, and storm track intensity \citep{barpanda2020}, due to its effect on the seasonality of surface energy fluxes. Thus, we hypothesize that surface heat capacity controls the heat transfer regime transition in the Northern midlatitudes. In order to connect the seasonality of $R_1$ to surface heat capacity, we begin by rewriting the MSE budget in terms of fluxes at the top of atmophere (TOA) and the surface (SFC) following \cite{barpanda2020}:
  \begin{equation}\label{eq:mse-toasfc}
    \Delta\left(\partial_t h + \nabla\cdot F_{m} \right) = \Delta F_{\mathrm{TOA}} - \Delta F_{\mathrm{SFC}} \, ,
  \end{equation}
  where \(F_{\mathrm{TOA}}\) and \(F_{\mathrm{SFC}}\) are the net heat fluxes through the top of atmosphere and surface. We can write the seasonality of surface fluxes using the surface energy budget of a mixed layer ocean:
  \begin{equation}
    \Delta F_{\mathrm{SFC}} = \rho c_{w} d \Delta\left(\frac{\partial T_{s}}{\partial t}\right) + \Delta ( \nabla\cdot F_{O}) \approx \rho c_{w} d \Delta\left(\frac{\partial T_{s}}{\partial t}\right) \, ,
  \end{equation}
  where $\rho$ is the density of water, $c_w$ is the specific heat capacity of liquid water, $d$ is the mixed layer depth, and $\Delta(\nabla\cdot F_O)$ is the meridional ocean heat flux convergence. We neglect the seasonality of the meridional ocean heat flux convergence as it is known to be small \citep{roberts2017}. Finally, we divide by $\overline{R_a}$ such that (\ref{eq:mse-toasfc}) becomes
  \begin{equation}\label{eq:mse-toasfc-approx}
    \Delta R_1 \approx \frac{\Delta\left(\partial_t h + \nabla\cdot F_{m} \right)}{\overline{R_a}} = \frac{1}{\overline{R_a}} \left(\Delta F_{\mathrm{TOA}} - \rho c_{w} d \Delta\left(\frac{\partial T_{s}}{\partial t}\right)\right) \, , 
  \end{equation}
  where we assume that the radiative component is negligible to simplify the expression of $\Delta R_1$. This assumption generally holds well for the Northern midlatitudes because the dynamic component is $\approx 3$ times larger than the radiative component (compare red and gray lines in Fig.~\ref{fig:rea-r1-decomp-mid}a). In order to close (\ref{eq:mse-toasfc-approx}) and predict the dependence of $R_1$ on mixed layer depth ($d$), we make use of the EBM (see Section~\ref{sec:methods}\ref{subsec:models} and Appendix~B for more details). Following the EBM, we can write (\ref{eq:mse-toasfc-approx}) as
  \begin{equation} \label{eq:r1-linear4}
    \Delta R_{1} = \frac{Q^{*}}{\overline{R_{a}}}\frac{2D}{(B+2D)^{2}+(\rho c_w d \omega)^{2}}\left[(B+2D)\cos(\omega t)+\rho c_w d \omega \sin(\omega t)\right] \, ,
  \end{equation}
  where $Q^*$ is the seasonal amplitude of insolation, $B$ is the sensitivity of OLR to surface temperature, $D$ is diffusivity, and $\omega=2\pi/1$ yr. According to (\ref{eq:r1-linear4}), we expect that the amplitude of $\Delta R_1$ decreases as the mixed layer depth $d$ increases if all else is equal. 
  
  %Since insolation is here assumed to be proportional to $\cos(\omega t)$, Eq.~(\ref{eq:r1-linear4}) further predicts that $\Delta R_1$ is in phase with insolation for shallow mixed layers ($d \approx 0$), whereas $\Delta R_1$ is in quadrature with insolation ($\propto\sin(\omega t)$) for deep mixed layers ($d \gg 1 $).
    
  % We can quantitatively define shallow and deep mixed layers in a way that is relevant for the existence of a midlatitude regime transition. Recalling that RCE is defined where $R_1 \le \epsilon$, a RCE/RCAE regime transition will exist if
  %   \begin{equation} \label{eq:regime-trans}
  %     \min(R_{1}) \le \varepsilon \, .
  %   \end{equation}
  % $\min(R_1)$ is obtained by evaluating $R_1(\omega t_{\mathrm{min}})$ where $t_{\mathrm{min}}$ is found by setting $\frac{\mathrm{d}\Delta R_1}{\mathrm{d}t}(\omega t_{\mathrm{min}})=0$. Then,
  %   \begin{equation} \label{eq:min-dr1}
  %     \min(R_{1}) = \overline{R_1} + \frac{Q^{*}}{\overline{R_{a}}}\frac{2D}{\sqrt{(B+2D)^2+(\rho c_w d \omega)^{2}}} \, .
  %   \end{equation}
  % Thus, we obtain the critical mixed layer depth $d_{c}$ by combining equations~(\ref{eq:regime-trans}) and (\ref{eq:min-dr1}) and solving for $d$:
  %   \begin{equation} \label{eq:crit-d}
  %     d_{c} = \frac{1}{\rho c_{w} \omega}\sqrt{\left(\frac{2 Q^{*} D}{\overline{R_{a}}(\varepsilon-\overline{R_{1}})}\right)^{2}-(B+2D)^2} \, .
  %   \end{equation}
  % Using the EBM values in Section~\ref{sec:methods}\ref{subsec:models} and $\overline{R_{a}}=-100$ W m$^{-2}$, $\varepsilon=0.1$, and $\overline{R_{1}}=0.3$, we obtain a critical mixed layer depth of 16 m (where the solid black line intersects orange region in Fig.~\ref{fig:amp-r1-echam}a). Thus, we hypothesize that a midlatitude regime transition will occur for $d \le 16$ m. We test this hypothesis by varying the mixed layer depth from 3--50 m in the standard configuration of AQUA.
  
  The dependence of $R_1$ seasonality on mixed layer depth in AQUA is mostly consistent with the EBM prediction (compare stars to solid black line in Fig.~\ref{fig:amp-r1-echam}a). While the $R_1$ seasonality is not as sensitive to the mixed layer depth in the EBM, it captures well the seasonal amplitude of surface temperature (Fig.~\ref{fig:amp-r1-echam}b). The midlatitude regime transition in AQUA occurs for $d \le 20$ m and the mixed layer depth where the regime transition occurs is between 20--25 m (intersection of the stars with the orange region in Fig.~\ref{fig:amp-r1-echam}a), whereas the corresponding mixed layer depth in the EBM is $16$ m (compare line to stars in Fig.~\ref{fig:amp-r1-echam}a).% in part because it overpredicts $\Delta T_s$ (Fig.~\ref{fig:amp-r1-echam}b), which acts to dampen $\Delta R_1$ (\ref{eq:mse-toasfc-approx}).

  When AQUA is configured with a mixed layer depth of 15 m, the amplitude of the \(R_{1}\) seasonality closely resembles the Northern midlatitudes (compare Fig.~\ref{fig:echam-rce}a and Fig.~\ref{fig:rea-r1-decomp-mid}a). However, the regime transition in AQUA with a 15 m mixed layer depth occurs later than that in reanalysis data. This phase lag can be partly rectified by choosing a smaller (3 m) mixed layer depth, but this comes at the expense of amplifying the seasonality (Fig.~\ref{fig:echam-rce}c,d). Consistent with the EBM prediction, when AQUA is configured with a mixed layer depth of 40 m, \(\Delta R_{1}\) closely resembles the Southern midlatitudes; namely, there is no regime transition (compare Fig.~\ref{fig:echam-rce}e and Fig.~\ref{fig:rea-r1-decomp-mid}c). The persistence of the RCAE regime throughout the seasonal cycle in the aquaplanet simulations with 40 m mixed layer depth can be attributed to the weak seasonality of advection and storage, consistent with the results for the Southern Hemisphere midlatitudes.

  AQUA simulations also capture the seasonal amplitude of the lapse rate deviation as a function of mixed layer depth. For example, the lapse rate is within 6\% of a moist adiabat during August for AQUA with a 15 m mixed layer compared to 12\% for a 40 m mixed layer (see red lines in Fig.~\ref{fig:echam-r1-ga-temporal}a,b).
  
  %AQUA with a large mixed layer depth captures the weak seasonality of the MSE flux divergence and tendency seen in the Southern midlatitudes (compare Fig.~\ref{fig:echam-rce}f and \ref{fig:rea-r1-decomp-mid}d).
  
  \subsection{High latitude regime transition in the Northern Hemisphere} \label{subsec:ice}

  In the Northern high latitudes, the regime transition is associated with an increase in latent heat flux during summertime (Fig.~\ref{fig:rea-r1-decomp-pole}b). In the annual mean, the spatial structure of latent heat flux is well correlated with the presence of sea ice in the Northern high latitudes (see Fig.~\ref{fig:rea-sice}). Thus, we hypothesize that the presence of sea ice modulates latent heat flux over the high latitude oceans and thus the existence of RAE. We test this hypothesis by configuring AQUA with and without thermodynamic sea ice (Section~\ref{sec:methods}\ref{subsec:models}). 

  % As the presence of sea ice strongly influences the surface albedo and is known to insulate turbulent heat flux exchange between the ocean and atmosphere \citep{andreas1979, maykut1982}, 
  
  When AQUA is configured without sea ice and a 40 m mixed layer depth, the high latitudes are in RCAE yearround (Fig.~\ref{fig:echam-rae}a). The RCAE regime is associated with positive latent heat flux (blue line in Fig.~\ref{fig:echam-rae}b) and the absence of a surface inversion yearround (Fig.~\ref{fig:echam-r1-ga-temporal}c). In contrast, when AQUA is configured with thermodynamic sea ice and a 40 m mixed layer depth, the high latitudes are in RAE from September through April (black line intersects the blue region in Fig.~\ref{fig:echam-rae}c) and in RCAE from May through August. The seasonality of the boundary layer lapse rate deviation is consistent with the seasonality of heat transfer regimes and shows that a surface inversion form from September through April (Fig.~\ref{fig:echam-r1-ga-temporal}d). Thus, sea ice is a necessary and sufficient condition to obtain winter RAE in AQUA. %Finally, latent heat flux is suppressed and sensible heat flux is negative outside of the summer months (Fig.~\ref{fig:echam-rae}(d)). 

  %However, the increase in latent heat flux during summertime is not controlled by the presence of sea ice since high latitude sea ice persists yearround in AQUA. Thus, while the presence of sea ice is clearly a necessary and sufficient condition for the existence of wintertime RAE, the loss of sea ice is not necessary for the increase in summertime latent heat flux.

\section{Conclusion and Discussion}

\subsection{Conclusion}

We have quantified when and where heat transfer regimes are observed and their connection to the lapse rate structure in the annual mean and through the seasonal cycle on the modern Earth. We identified heat transfer regimes via a nondimensional number that arises in the vertically-integrated MSE equation, $R_1=\frac{\partial_t h + \nabla\cdot F_m}{R_a}$, which quantifies the relative importance of MSE advection and storage relative to radiative cooling. The lapse rate structure as measured by the percent deviation from a moist adiabatic lapse rate is nearly a monotonic function of $R_1$. We define the RCE regime as $R_1 \le 0.3$ and the RAE regime as $R_1 \ge 0.9$, on the basis that the free tropospheric lapse rate deviation in the RCE regime is within 20\% of a moist adiabat and that the RAE regime exhibits a surface inversion.

In the annual mean, we find that RCE occurs equatorward of $45^\circ$, RAE poleward of $70^\circ$S and $80^\circ$N, and RCAE between $45-70^\circ$N and $45-80^\circ$S. Lapse rate profiles averaged over all regions identified as RCE are within 6\% of a moist adiabat in the free troposphere, RAE exhibit a surface inversion, and RCAE are 28\% more stable than a moist adiabat and do not show a surface inversion.

Heat transfer regimes exhibit negligible seasonality in the Southern Hemisphere whereas the seasonality is large in the North where it leads to two regime transitions during summertime: 1) the Northern midlatitudes transition from RCAE (September through March) to RCE (April through August), and 2) the Northern high latitudes transition from RAE (July through April) to RCAE (May and June). While the lapse rate structure show consistent seasonal regime transitions in the Northern Hemisphere, the timing and duration of heat transfer and lapse rate regime transitions exhibit discrepancies from 1 to 3 months.

The Northern midlatitude regime transition is associated with a stronger seasonality of MSE advection and storage relative to the South. By varying the mixed layer depth in an energy balance model and aquaplanet simulations, we confirmed the hypothesis that the surface heat capacity controls the seasonality of MSE advection and storage, and thus the existence of a midlatitude regime transition.

The Northern high latitude regime transition is associated with an increase in latent heat flux during summertime. A similar Northern high latitude regime transition is reproduced in an aquaplanet configured with sea ice, whereas an aquaplanet configured without sea ice remains in RCAE yearround. Thus, we find that sea ice is a necessary and sufficient condition for the existence of winter RAE.

\subsection{Discussion}
Our findings are consistent with those of \cite{jakob2019}, who found that the tropics is near a state of RCE in the annual mean over a sufficiently large spatial average (achieved here through taking the zonal mean). \cite{jakob2019} use the DSE budget to define RCE and primarily focus on the implications of the validity of RCE in the context of CRM configurations and convective aggregation in the tropics. Our work proposes an alternative definition of RCE using the nondimensional MSE budget, which has the advantage that it can be used as a more general criterion for defining heat transfer regimes outside of the tropics and in climates different from modern Earth.

Although we found that the annual mean difference in $R_1$ plays an important role in the hemispheric asymmetry of high latitude regime transitions, we have not yet identified the physical mechanism associated with the asymmetry in $R_1$. One plausible hypothesis is that the presence of Antarctic topography makes the atmosphere optically thinner and weakens atmospheric radiative cooling, which corresponds to larger values of $R_1$. Topography is also known to play an important role in explaining the hemispheric asymmetry in the response of polar amplification to climate change \citep{salzmann2017,hahn2020,singh2020}. However, we find that the Southern high latitudes still remain in RAE yearround in the CESM preindustrial control simulation conducted by \cite{hahn2020} where Antarctic topography is removed (Fig.~E1). This suggests that other factors, such as the asymmetry in the surface type (sea ice and ocean-dominated Arctic vs ice sheets and land-dominated Antarctic), are also important for explaining the hemispheric asymmetry in high latitude regime transitions.

The framework we introduced in this study for quantifying heat transfer regimes can be extended in many ways, such as studying the zonal structure of heat transfer regimes and the response of heat transfer regimes to climate change. For example, our result that capturing the amplitude and the phase of the observed midlatitude regime transition in the aquaplanet requires two separate mixed layer depths suggests that understanding the zonal structure of $R_1$ may be important to more accurately explain the timing of the midlatitude regime transition. In addition, the spatio-temporal structure of heat transfer regimes is expected to have significantly changed through Earth's history. There are hints that high latitudes during warm epochs such as the Eocene may have been in a state similar to RCE \citep{abbot2008a} and that RAE was more widespread during a Snowball period \citep{pierrehumbert2005}. Understanding the spatio-temporal structure of heat transfer regimes during and through the transitions across various paleoclimate states are exciting avenues for future work.

% A moist adiabatic temperature profile being observed not only in the tropics but also in the northern midlatitude summer is consistent with previous studies \citep{stone1979,korty2007}. We find that the transition from a convectively-stable to a moist adiabatic profile in the midlatitude summer is associated with a regime transition from RCAE to RCE. This allows us to use the heat transfer regime transitions as a proxy for understanding lapse rate regimes, which is a framework better suited for relating phenomena to physical mechanisms. In particular, the control of surface heat capacity on the magnitude of MSE flux divergence seasonality and thus the existence of a midlatitude regime transition is consistent with \cite{barpanda2020}, where similar results were found in the context of the hemispheric asymmetry in the seasonality of storm track intensity. 

%%%%%%%%%%%%%%%%%%%%%%%%%%%%%%%%%%%%%%%%%%%%%%%%%%%%%%%%%%%%%%%%%%%%%
% ACKNOWLEDGMENTS
%%%%%%%%%%%%%%%%%%%%%%%%%%%%%%%%%%%%%%%%%%%%%%%%%%%%%%%%%%%%%%%%%%%%%
\acknowledgments
Keep acknowledgments (note correct spelling: no ``e'' between the ``g'' and
``m'') as brief as possible. In general, acknowledge only direct help in
writing or research. Financial support (e.g., grant numbers) for the work
done, for an author, or for the laboratory where the work was performed is
best acknowledged here rather than as footnotes to the title or to an
author's name. Contribution numbers (if the work has been published by the
author's institution or organization) should be included as footnotes on the title page,
not in the acknowledgments.

%%%%%%%%%%%%%%%%%%%%%%%%%%%%%%%%%%%%%%%%%%%%%%%%%%%%%%%%%%%%%%%%%%%%%
% DATA AVAILABILITY STATEMENT
%%%%%%%%%%%%%%%%%%%%%%%%%%%%%%%%%%%%%%%%%%%%%%%%%%%%%%%%%%%%%%%%%%%%%
% 
%
\datastatement
The data availability statement is where authors should describe how the data underlying 
the findings within the article can be accessed and reused. Authors should attempt to 
provide unrestricted access to all data and materials underlying reported findings. 
If data access is restricted, authors must mention this in the statement.

%%%%%%%%%%%%%%%%%%%%%%%%%%%%%%%%%%%%%%%%%%%%%%%%%%%%%%%%%%%%%%%%%%%%%
% APPENDIXES
%%%%%%%%%%%%%%%%%%%%%%%%%%%%%%%%%%%%%%%%%%%%%%%%%%%%%%%%%%%%%%%%%%%%%
%
% Use \appendix if there is only one appendix.
%\appendix

% Use \appendix[A], \appendix[B], if you have multiple appendixes.
% \appendix[A]

%% Appendix title is necessary! For appendix title:
%\appendixtitle{}

%%% Appendix section numbering (note, skip \section and begin with \subsection)
% \subsection{First primary heading}

% \subsubsection{First secondary heading}

% \paragraph{First tertiary heading}

%% Important!
%\appendcaption{<appendix letter and number>}{<caption>} 
%must be used for figures and tables in appendixes, e.g.,
%
%\begin{figure}
%\noindent\includegraphics[width=19pc,angle=0]{figure01.pdf}\\
%\appendcaption{A1}{Caption here.}
%\end{figure}
%
% All appendix figures/tables should be placed in order AFTER the main figures/tables, i.e., tables, appendix tables, figures, appendix figures.

% \appendix[A]
% \appendixtitle{Vertical temperature profiles}

% We compute the vertical temperature profiles to compare whether the temperature profiles of RCE, RAE, and RCAE are consistent with the expectations in each region. To avoid the issue of averaging out surface inversions that occur at various surface pressure or height levels in the presence of topography, we compute temperature profiles in sigma coordinates.

% We use monthly pressure level temperature data and convert to sigma coordinates by masking out the data below surface pressure, inserting the 2 m temperature data into the $\sigma=1$ level, and taking a cubic spline interpolation. We perform this conversion for every latitude and longitude grid point. 

% To compare the temperature profiles in RCE and RCAE to moist adiabats, we compute the reversible moist adiabatic temperature profile in height coordinates and convert to sigma coordinates at each latitude and longitude. The reversible moist adiabat considers condensed water loading in the definition of buoyancy, which is known to be important when evaluating the neutrality of the tropical temperature profile to a moist adiabat \citep{xu1989}. We do not consider the latent heat of fusion associated with the ice phase here, as the tropical temperature profile is known to be less stable than a moist adiabat that includes this effect \citep{williams1993}. We assume a lifted condensation level of $\sigma=0.95$ and integrate upward using the reversible moist adiabatic lapse rate $\Gamma_{rm}$ following the American Meteorological Society (AMS) glossary \citep{ams2021}:
% \begin{equation} \label{eq:malr}
%   \Gamma_{rm} = \Gamma_d \frac{(1+r_t)\left(1+\frac{L r_v}{R_d T}\right)}{1+r_v\frac{c_{p_v}}{c_{p_d}}+r_l\frac{c_{p_l}}{c_{p_d}}+\frac{L^2 r^*(\epsilon+r_v)}{R_d T^2 c_{p_d}}},
% \end{equation}
% where $\Gamma_d=\frac{g}{c_{p_d}}$ is the dry adiabatic lapse rate, $g$ is gravitational acceleration, $c_{p_d}$, $c_{p_v}$, and $c_{p_l}$ are the specific heat capacities of dry air, water vapor, and liquid water, respectively, $r_v$, $r_l$, and $r_t$ are the mixing ratios of water vapor, liquid water, and total water, respectively, $R_d$ is the gas constant of dry air, $\epsilon=0.622$ is the ratio of the gas constants of dry air to water vapor, $L$ is the latent heat of vaporization, and $T$ is temperature.

% The moist adiabat can be computed alternatively by initiating the parcel at the surface and following the dry adiabatic lapse rate up to the lifted condensation level (LCL) as in \cite{miyawaki2020}. The LCL is computed as where the air reaches saturation assuming that the vapor mixing ratio is conserved from its 2 m value. Above the LCL, we compute the moist adiabat following the reversible moist adiabat as in (\ref{eq:malr}).

% The moist adiabat in the Northern Hemisphere midlatitudes is sensitive to the initial condition of the parcel. The boundary layer here is more stable than a moist adiabat and thus the predicted LCL does not align well with the actual LCL in ERA5 (Fig.~A1a,b). While the moist adiabatic temperature is sensitive to the initial conditions of the parcel, the free tropospheric lapse rate still closely follows the moist adiabatic lapse rate (compare slope of solid to dashed line Fig.~A1b).

% In contrast, the Southern Hemisphere midlatitude (Fig.~A1c,d) and the equatorial boundary layers (Fig.~A2c,d) are both close to dry adiabatic. Thus, in the southern midlatitudes and the tropics, the moist adiabat is robust to the choice of the initial condition of the rising parcel.

\appendix[A]
\appendixtitle{Lapse rate deviation from a moist adiabat}

We compute the lapse rate deviation from a moist adiabat to compare whether the lapse rate profiles of RCE, RAE, and RCAE are consistent with the expectations in each region. To avoid the issue of averaging out surface inversions that occur at various surface pressure or height levels in the presence of topography, we compute the lapse rate deviation in sigma coordinates.

We take the central finite difference of monthly pressure level temperature and geopotential data to compute the lapse rate and convert to sigma coordinates by masking out the data below surface pressure and taking a cubic spline interpolation. We perform this conversion for every latitude and longitude grid point. 

Following \cite{stone1979}, we define the deviation of a lapse rate from a moist adiabatic lapse rate as the fractional difference:
  \begin{equation}
    \delta_{c} = \frac{\Gamma_{m}-\Gamma}{\Gamma_{m}}
  \end{equation}
where $\Gamma$ is the actual lapse rate in the reanalysis or GCM and $\Gamma_m$ is the moist adiabatic lapse rate defined as in (3) in \cite{stone1979}.


% We vertically average \(\delta_{c}\) from $\sigma=0.9$--0.3 in linear sigma coordinates to obtain the vertically-integrated deviation \(\langle \delta_{c} \rangle\).

% To quantify the presence of a near surface inversion, we define the deviation of a lapse rate from a dry adiabatic lapse rate in a similar manner. We use the dry adiabat as the reference lapse rate here because the boundary layer, where the near surface inversion forms, is typically not saturated. Thus, the existence of an inversion is quantified as:
%   \begin{equation}
%     \delta_{i} = \frac{\Gamma_{d}-\Gamma}{\Gamma_{d}}\, .
%   \end{equation}
% We vertically average \(\delta_{i}\) from $\sigma=$1--0.9 in linear sigma coordinates to obtain the average near surface deviation from a dry adiabat \(\langle \delta_{i} \rangle\).

  % The stratification is either conditionally unstable (orange filled contours in Fig.~C1) or close to neutrally stable (white filled contours) to a moist adiabat equatorward of 30$^{\circ}$N/S yearround. The northern boundary of the convective lapse rate regime migrates poleward out to 60$^{\circ}$N in July (thick red contour in Fig.~C1). In the SH, the boundary of the convective lapse rate regime varies less throughout the annual cycle, expanding out to only 40$^{\circ}$S in January.

  % To quantify the presence of a near surface inversion, we define the deviation of a lapse rate from a dry adiabatic lapse rate in a similar manner. We use the dry adiabat as the reference lapse rate here because the boundary layer, where the near surface inversion forms, is typically not saturated. Thus, the existence of an inversion is quantified as:
  %   \begin{equation}
  %     \delta_{i} = \frac{\Gamma_{d}-\Gamma}{\Gamma_{d}}
  %   \end{equation}
  % where \(\delta_{i}=1\) corresponds to an isothermal stratification and thus \(\delta_{i}>1\) indicates the presence of an inversion. To be consistent with the approximate thresholds used to define heat transfer regimes and the convective lapse rate regime, we choose $\delta_i>0.9$ as the threshold for defining the inversion lapse rate regime. We vertically average \(\delta_{i}\) from 1--0.9 in linear sigma coordinates to obtain the average near surface deviation from a dry adiabat \(\langle \delta_{i} \rangle\). The inversion lapse rate regime (90\% contour in Fig.~A1(b)) is found poleward of 60$^{\circ}$N and 70$^{\circ}$S during wintertime. In the NH, the inversion lapse rate regime migrates poleward and vanishes in the summer (Fig.~A1(b)) whereas the SH remains in the inversion regime yearround.

\appendix[B]
\appendixtitle{Deriving an analytical expression of $\Delta R_1$ as a function of mixed layer depth}
%  Then, equation~(\ref{eq:r1-dev}) simplifies to
% \begin{equation} \label{eq:r1-term1}
%   \Delta R_{1} \approx \frac{1}{\overline{R_{a}}}\Delta(\partial_t h + \nabla\cdot F_{m}) \, .
% \end{equation}
% Substituting equation~(\ref{eq:mse-toasfc-approx}), we obtain
% \begin{equation} \label{eq:r1-linear2}
%   \Delta R_{1} = \frac{1}{\overline{R_{a}}}\left(\Delta F_{\mathrm{TOA}}-\rho c_{w} d \Delta\frac{\partial T_{s}}{\partial t}\right) \, .
% \end{equation}
Following the \cite{rose2017} EBM, we write the seasonality of TOA and SFC fluxes as a Fourier-Legendre series. Here, we only consider the first harmonic as it is an order of magnitude larger than the second harmonic in the midlatitudes:
    \begin{itemize}
      \item $\Delta F_{\mathrm{TOA}} \approx a\Delta Q - B\Delta T_{s}$, where $\Delta Q = Q^{*}\cos(\omega t)$. $\omega=\frac{2\pi}{t_{\mathrm{year}}}$, $Q^{*}=as_{11}Q_{g}P_{1}(x)$ is the amplitude of net TOA shortwave radiation, $s_{11}=-2\sin{\beta}$ where $\beta$ is the obliquity, $P_1(x) = \sin\phi$, and $Q_{g}=340$ Wm$^{-2}$. 
      \item $\Delta T_{s} = T_{s}^{*}\cos(\omega t - \Phi)$, where $T_{s}^{*}$ is the amplitude of surface temperature seasonality and $\Phi$ is the phase shift of $\Delta T_{s}$ relative to $\Delta Q$. $T_{s}^{*}=Q^{*}\left[(B+2D)^{2}+(\rho c_w d \omega)^{2}\right]^{-1/2}$ and $\Phi=\arctan\left(\frac{\rho c_w d \omega}{B+2D}\right)$ (see \cite{rose2017} for the derivation of the analytical expression of surface temperature).
    \end{itemize}
  Using the assumptions above, we can write Equation~(\ref{eq:mse-toasfc-approx}) as
  \begin{equation} \label{eq:r1-linear3}
    \Delta R_{1} = \frac{1}{\overline{R_{a}}}\left(Q^{*}\cos(\omega t) -BT^{*}\cos(\omega t - \Phi)+\rho c_{w} d \omega T_{s}^{*}\sin(\omega t - \Phi) \right) \, .
  \end{equation}
  Substituting in $T_{s}^{*}$ and $\Phi$ and simplifying, we obtain
  \begin{equation} \label{eq:r1-linear4-deriv}
    \Delta R_{1} = \frac{Q^{*}}{\overline{R_{a}}}\frac{2D}{(B+2D)^{2}+(\rho c_w d \omega)^{2}}\left[(B+2D)\cos(\omega t)+\rho c_w d \omega \sin(\omega t)\right] \, ,
  \end{equation}

%%%%%%%%%%%%%%%%%%%%%%%%%%%%%%%%%%%%%%%%%%%%%%%%%%%%%%%%%%%%%%%%%%%%%
% REFERENCES
%%%%%%%%%%%%%%%%%%%%%%%%%%%%%%%%%%%%%%%%%%%%%%%%%%%%%%%%%%%%%%%%%%%%%
% Make your BibTeX bibliography by using these commands:
\bibliographystyle{ametsoc2014}
\bibliography{references}


%%%%%%%%%%%%%%%%%%%%%%%%%%%%%%%%%%%%%%%%%%%%%%%%%%%%%%%%%%%%%%%%%%%%%
% TABLES
%%%%%%%%%%%%%%%%%%%%%%%%%%%%%%%%%%%%%%%%%%%%%%%%%%%%%%%%%%%%%%%%%%%%%
%% Enter tables at the end of the document, before figures.
%%
%
%\begin{table}[t]
%\caption{This is a sample table caption and table layout.  Enter as many tables as
%  necessary at the end of your manuscript. Table from Lorenz (1963).}\label{t1}
%\begin{center}
%\begin{tabular}{ccccrrcrc}
%\hline\hline
%$N$ & $X$ & $Y$ & $Z$\\
%\hline
% 0000 & 0000 & 0010 & 0000 \\
% 0005 & 0004 & 0012 & 0000 \\
% 0010 & 0009 & 0020 & 0000 \\
% 0015 & 0016 & 0036 & 0002 \\
% 0020 & 0030 & 0066 & 0007 \\
% 0025 & 0054 & 0115 & 0024 \\
%\hline
%\end{tabular}
%\end{center}
%\end{table}

%%%%%%%%%%%%%%%%%%%%%%%%%%%%%%%%%%%%%%%%%%%%%%%%%%%%%%%%%%%%%%%%%%%%%
% FIGURES
%%%%%%%%%%%%%%%%%%%%%%%%%%%%%%%%%%%%%%%%%%%%%%%%%%%%%%%%%%%%%%%%%%%%%
%% Enter figures at the end of the document, after tables.
%%
%
%\begin{figure}[t]
%  \noindent\includegraphics[width=19pc,angle=0]{figure01.pdf}\\
%  \caption{Enter the caption for your figure here.  Repeat as
%  necessary for each of your figures. Figure from \protect\cite{Knutti2008}.}\label{f1}
%\end{figure}

\begin{figure}
  \noindent\includegraphics[width=\textwidth]{/project2/tas1/miyawaki/projects/002/figures/rea/1980_2005/1.00/ga_frac_binned_r1/mse_old/lo/ga_frac_r1_all.png}\\
  \caption{Percent deviation of the reanalysis mean lapse rate deviation from a moist adiabatic lapse rate binned by $R_{1}$. Thick blue and green lines correspond to $R_1=0.9$ and $R_1=0.3$, respectively.}
  \label{fig:rea-binned-r1}
\end{figure}

\begin{figure}[t]
  \noindent\includegraphics[width=\textwidth]{/project2/tas1/miyawaki/projects/002/figures_post/final/r1z_ann/r1z_ann_rea.pdf}\\
  \caption{(a) The annual mean zonal-mean structure of $R_{1}$ for the reanalysis mean. Orange, black, and blue regions indicate RCE, RCAE, and RAE, respectively. The annual-mean zonal-mean vertical temperature profile for RCE, RCAE, and RAE for (b) SH and (c) NH. The dotted lines indicate a moist adiabat. The shading over the lines indicate the range across the three reanalysis products.}
  \label{fig:rea-r1-ann}
\end{figure}

\begin{figure}[t]
  \noindent\includegraphics[width=0.7\textwidth]{/project2/tas1/miyawaki/projects/002/figures_post/final/r1_dev/r1_dev_rea.pdf}\\
  \caption{(a) The seasonality of $R_{1}$ for the reanalysis mean (contour interval is 0.1). The thick orange contour indicates the RCE/RCAE boundary ($R_1=0.3$) and the thick blue contour indicates the RAE/RCAE boundary ($R_1 = 0.9$). (b) The spatio-temporal structure of the vertically averaged free tropospheric lapse rate deviation from a moist adiabatic lapse rate between $\sigma=0.8$ and 0.3 is shown for the reanalysis mean (contour interval is 5\%). (c) The spatio-temporal structure of the vertically averaged surface lapse rate deviation from a moist adiabatic lapse rate between $\sigma=1$ and 0.9 is shown for the reanalysis mean (contour interval is 10\%).}
  \label{fig:rea-r1-dev}
\end{figure}

\begin{figure}[t]
  \noindent\includegraphics[width=0.9\textwidth]{/project2/tas1/miyawaki/projects/002/figures_post/test/r1_ga_temporal/r1_ga_temporal_rea.pdf}\\
  \caption{The seasonality of $R_1$ and the free tropospheric ($\sigma=0.8$ to 0.3) lapse rate deviation from a moist adiabatic lapse rate is compared for the reanalysis mean in the (a) Northern and (b) Southern Hemisphere midlatitudes ($40-60^\circ$). (c,d) Similar, except the seasonality of $R_1$ is compared to the boundary layer ($\sigma=1$ to 0.9) lapse rate deviation from a moist adiabat is compared in the (c) Northern and (d) Southern Hemisphere high latitudes ($80-90^\circ$).}
  \label{fig:rea-r1-ga-temporal}
\end{figure}

% \begin{figure}[t]
%   \noindent\includegraphics[width=0.9\textwidth]{/project2/tas1/miyawaki/projects/002/figures_post/final/ga_fr_sea/ga_fr_sea_rea.pdf}\\
%   \caption{The seasonality of the reanalysis mean percentage lapse rate deviation from a moist adiabatic lapse rate at (a,b) 85$^{\circ}$N, (c,d) $45^{\circ}$N (e,f) $45^{\circ}$S, and (g,h) 85$^{\circ}$S. The lapse rate deviation profiles are colored according to values of $R_1$ during the corresponding month.}
%   \label{fig:rea-ga-fr-sel}
% \end{figure}

\begin{figure}[t]
  \noindent\includegraphics[width=\textwidth]{/project2/tas1/miyawaki/projects/002/figures_post/final/r1_decomp_mid/r1_decomp_mid_rea.pdf}\\
  \caption{The seasonality of $R_{1}$ in midlatitudes ($40$--$60^{\circ}$, black lines, left axis) and its deviation from the annual-mean (right axis) for the (a) Northern and (c) Southern Hemisphere. Note that the x-axis is shifted by 6 months in the Southern Hemisphere to facilitate comparison across the hemispheres. The orange-filled region ($R_1 \le 0.3$) is RCE and the white region ($R_1>0.3$) is RCAE. $\Delta R_1$ is decomposed into the dynamic (red line) and the radiative (gray line) components according to (\ref{eq:r1-dev}). The seasonality of the terms in the MSE budget in the midlatitudes for the (b) Northern and (d) Southern Hemisphere. The shading over the lines indicate the range across the three reanalysis products.}
  \label{fig:rea-r1-decomp-mid}
\end{figure}

\begin{figure}[t]
  \noindent\includegraphics[width=\textwidth]{/project2/tas1/miyawaki/projects/002/figures_post/final/r1_decomp_pole/r1_decomp_pole_rea.pdf}\\
  \caption{Same as Fig.~\ref{fig:rea-r1-decomp-mid} but averaged over the high latitudes ($80$--$90^{\circ}$).}
  \label{fig:rea-r1-decomp-pole}
\end{figure}

\begin{figure}
  \noindent\includegraphics[width=0.8\textwidth]{/project2/tas1/miyawaki/projects/002/figures_post/test/amp_r1_echam/amp_echam.pdf}\\
  \caption{(a) $R_1$ seasonality (as measured by $\min(R_1)$) and (b) surface temperature amplitude ($T_s^*=(\max(\Delta T_s)- \min(\Delta T_s))/2$) predicted by the EBM (solid black line) and AQUA (stars). The orange region denotes where an RCE/RCAE regime transition exists through the seasonal cycle.}
  % \caption{(a) Minimum of the seasonal deviation of $R_{1}$ as diagnosed from ECHAM (asterisks) and the EBM prediction (line) for $B=2.32$ W m$^{-2}$ K$^{-1}$, $D=0.89$ W m$^{-2}$ K$^{-1}$, $\rho=1000$ kg m$^{-3}$, $c_{w}=4000$ J kg$^{-1} $K$^{-1}$, and $a=0.80$. The red line denotes the threshold where an RCE/RCAE regime transition occurs based on the criteria that $\min(\Delta R_1)=\varepsilon - \overline{R_1} = -0.2$. (b) The seasonal amplitude of surface temperature between 40--60$^{\circ}$ latitude as diagnosed from ECHAM6 across varied mixed layer depths (asterisks) and that predicted from the EBM (line).}
  % \appendcaption{C1}{(a) The seasonal amplitude of surface temperature between 40--60$^{\circ}$ latitude as diagnosed from ECHAM with varied mixed layer depths (asterisks) and that predicted from the \cite{rose2017} energy balance model (line) for $B=2.32$ W m$^{-2}$ K$^{-1}$, $D=0.89$ W m$^{-2}$ K$^{-1}$, $\rho=1000$ kg m$^{-3}$, $c_{w}=4000$ J kg$^{-1} $K$^{-1}$, and $a=0.80$. (b) Minimum of the seasonal deviation of $R_{1}$ as diagnosed from ECHAM (asterisks) and the EBM (line).}
  \label{fig:amp-r1-echam}
\end{figure}

\begin{figure}[t]
    \noindent\includegraphics[width=\textwidth]{/project2/tas1/miyawaki/projects/002/figures_post/final/r1_decomp_mid/r1_decomp_mid_echamslab.pdf}\\
    \caption{Same as Fig.~\ref{fig:rea-r1-decomp-mid} but for AQUA with (a,b) 15 m, (c,d) 3 m, and (e,f) 40 m mixed layer depth}
\label{fig:echam-rce}
\end{figure}

% \begin{figure}
%     \includegraphics[width=\textwidth]{/project2/tas1/miyawaki/projects/002/figures_post/final/temp_echam/temp_echam.pdf}
%     \caption{The vertical temperature profile at $45^\circ$ for (a) 15 m and (b) 40 m mixed layer depth AQUA without ice, and temperature profile at $85^\circ$ for 40 m mixed layer depth AQUA (c) with and (d) without sea ice.}
%     % \caption{Zonally averaged temperature profiles from the ECHAM6 slab ocean aquaplanets are shown in the midlatitudes and high latitudes for January and June. The temperature profile at 45$^\circ$ in ECHAM6 configured with (a) 15 m mixed layer depth is more stable than a moist adiabat in January and neutrally stable in June and (b) 40 m mixed layer depth is more stable than a moist adiabat yearround. The temperature profile at 85$^\circ$ in ECHAM6 with a 40 m mixed layer depth configured (c) with thermodynamic sea ice exhibits a near surface inversion in January whereas (d) without sea ice remains inversion-free yearround.}
%     \label{fig:temp-echam}
% \end{figure}

\begin{figure}
  \noindent\includegraphics[width=0.9\textwidth]{/project2/tas1/miyawaki/projects/002/figures_post/test/r1_ga_temporal/r1_ga_temporal_aqua.pdf}\\
    \caption{Similar to Fig.~\ref{fig:rea-r1-ga-temporal}, except AQUA simulations are shown configured with (a) 40 m mixed layer (Southern midlatitude like), (b) 15 m mixed layer (Northern midlatitude like), (c) 40 m mixed layer without sea ice, and (d) 40 m mixed layer with sea ice (Northern high latitudes like).}
    \label{fig:echam-r1-ga-temporal}
\end{figure}

% \begin{figure}[t]
%   \noindent\includegraphics[width=\textwidth]{/project2/tas1/miyawaki/projects/002/figures_post/final/r1_decomp_pole/r1_decomp_pole_era5.pdf}\\
%   \caption{Same as Fig.~\ref{fig:era5-r1-decomp-mid} but averaged over the polar region ($80$--$90^{\circ}$).}
%   \label{fig:era5-r1-decomp-pole}
% \end{figure}

\begin{figure}[t]
    \noindent\includegraphics[width=\textwidth]{/project2/tas1/miyawaki/projects/002/figures_post/test/sice_ann/sice_ann_rea.pdf}\\
    \caption{Annual mean (a) sea ice fraction and (b) latent heat flux are shown for the reanalysis mean over the Northern high latitudes.}
    \label{fig:rea-sice}
\end{figure}

\begin{figure}[t]
    \noindent\includegraphics[width=\textwidth]{/project2/tas1/miyawaki/projects/002/figures_post/final/r1_decomp_pole/r1_decomp_pole_echamslab.pdf}\\
    \caption{Same as Fig.~\ref{fig:rea-r1-decomp-pole} but for AQUA with a 40 m mixed layer depth and (a,b) without and (c,d) with thermodynamic sea ice.}
    \label{fig:echam-rae}
\end{figure}

% APPENDIX FIGURES 

% \begin{figure}[t]
%   \noindent\includegraphics[width=\textwidth]{/project2/tas1/miyawaki/projects/002/figures_post/final/temp_surf/temp_surf_era5.pdf}\\
%   \appendcaption{A1}{Same as Fig.~\ref{fig:rea-temp-sel}a--d in the midlatitudes except the parcel is initiated at the surface following a dry adiabat up to the LCL.}
%   \label{fig:temp-surf-era5}
% \end{figure}

% \begin{figure}[t]
%   \noindent\includegraphics[width=\textwidth]{/project2/tas1/miyawaki/projects/002/figures_post/final/temp_surf/temp_surf_eq_era5.pdf}\\
%   \appendcaption{A2}{Equatorial temperature profiles (solid) and the moist adiabats (dotted) initiated at (a,b) 950 hPa assuming the parcel is saturated and (c,d) the surface following a dry adiabat up to the LCL.}
%   \label{fig:temp-surf-eq-era5}
% \end{figure}

\begin{table}[t]
  \appendcaption{D1}{List of the 41 models that comprise the CMIP5 multi-model mean of the historical run.The r1i1p1 ensemble run is used to compute the multi-model mean.}
\begin{center}
  \renewcommand{\arraystretch}{1.0}
  \begin{tabular}{ l }
    Models          \\%& Ensemble run \\
    \hline
    ACCESS1-0       \\%& r1i1p1 \\
    ACCESS1-3       \\%& r1i1p1 \\
    bcc-csm1-1      \\%& r1i1p1 \\
    bcc-csm1-1-m    \\%& r1i1p1 \\
    BNU-ESM         \\%& r1i1p1 \\
    CanESM2         \\%& r1i1p1 \\
    CCSM4           \\%& r1i1p1 \\
    CESM1-BGC       \\%& r1i1p1 \\
    CESM1-CAM5      \\%& r1i1p1 \\
    CESM1-WACCM     \\%& r1i1p1 \\
    CMCC-CESM       \\%& r1i1p1 \\
    CMCC-CM         \\%& r1i1p1 \\
    CNRM-CM5        \\%& r1i1p1 \\
    CNRM-CM5-2      \\%& r1i1p1 \\
    CSIRO-Mk3-6-0   \\%& r1i1p1 \\
    FGOALS-g2       \\%& r1i1p1 \\
    FGOALS-s2       \\%& r1i1p1 \\
    GFDL-CM3        \\%& r1i1p1 \\
    GFDL-ESM2G      \\%& r1i1p1 \\
    GFDL-ESM2M      \\%& r1i1p1 \\
    GISS-E2-H       \\%& r1i1p1 \\
    GISS-E2-H-CC    \\%& r1i1p1 \\
    GISS-E2-R       \\%& r1i1p1 \\
    GISS-E2-R-CC    \\%& r1i1p1 \\
    HadCM3          \\%& r1i1p1 \\
    HadGEM2-CC      \\%& r1i1p1 \\
    HadGEM2-ES      \\%& r1i1p1 \\
    inmcm4          \\%& r1i1p1 \\
    IPSL-CM5A-LR    \\%& r1i1p1 \\
    IPSL-CM5A-MR    \\%& r1i1p1 \\
    IPSL-CM5B-LR    \\%& r1i1p1 \\
    MIROC5          \\%& r1i1p1 \\
    MIROC-ESM       \\%& r1i1p1 \\
    MIROC-ESM-CHEM  \\%& r1i1p1 \\
    MPI-ESM-LR      \\%& r1i1p1 \\
    MPI-ESM-MR      \\%& r1i1p1 \\
    MPI-ESM-P       \\%& r1i1p1 \\
    MRI-CGCM3       \\%& r1i1p1 \\
    MRI-ESM1        \\%& r1i1p1 \\
    NorESM1-M       \\%& r1i1p1 \\
    NorESM1-ME      \\%& r1i1p1 

  \end{tabular}
\end{center}
\end{table}

\begin{figure}[t]
  \noindent\includegraphics[width=\textwidth]{/project2/tas1/miyawaki/projects/002/figures/gcm/mmm/historical/198001-200512/1.00/ga_frac_binned_r1/mse_old/lo/ga_frac_r1_all.png}\\
  \appendcaption{D1}{Same as Fig.~\ref{fig:rea-binned-r1} but for the CMIP5 historical multi-model mean.}
  \label{fig:cmip5-binned-r1}
\end{figure}

\begin{figure}[t]
  \noindent\includegraphics[width=\textwidth]{/project2/tas1/miyawaki/projects/002/figures_post/final/r1z_ann/r1z_ann_cmip5hist.pdf}\\
  \appendcaption{D2}{Same as Fig.~\ref{fig:rea-r1-ann} but for the CMIP5 historical multi-model mean. The gray shading indicates one standard deviation from the mean.}
  \label{fig:cmip5hist-r1-ann}
\end{figure}

\begin{figure}[t]
  \noindent\includegraphics[width=\textwidth]{/project2/tas1/miyawaki/projects/002/figures_post/final/r1_dev/r1_dev_cmip5hist.pdf}\\
  \appendcaption{D3}{Same as Fig.~\ref{fig:rea-r1-dev} but for the CMIP5 historical multi-model mean.}
  \label{fig:cmip5hist-r1-dev}
\end{figure}

\begin{figure}[t]
  \noindent\includegraphics[width=0.9\textwidth]{/project2/tas1/miyawaki/projects/002/figures_post/test/r1_ga_temporal/r1_ga_temporal_cmip5hist.pdf}\\
  \appendcaption{D4}{Same as Fig.~\ref{fig:rea-r1-ga-temporal} but for the CMIP5 historical multi-model mean.}
  \label{fig:cmip5hist-r1-ga-temporal}
\end{figure}

\begin{figure}[t]
  \noindent\includegraphics[width=\textwidth]{/project2/tas1/miyawaki/projects/002/figures_post/final/r1_decomp_mid/r1_decomp_mid_cmip5hist.pdf}\\
  \appendcaption{D5}{Same as Fig.~\ref{fig:rea-r1-decomp-mid} but for the CMIP5 historical multi-model mean.}
  \label{fig:cmip5hist-r1-decomp-mid}
\end{figure}

\begin{figure}[t]
  \noindent\includegraphics[width=\textwidth]{/project2/tas1/miyawaki/projects/002/figures_post/final/r1_decomp_pole/r1_decomp_pole_cmip5hist.pdf}\\
  \appendcaption{D6}{Same as Fig.~\ref{fig:rea-r1-decomp-pole} but for the CMIP5 historical multi-model mean.}
  \label{fig:cmip5hist-r1-decomp-pole}
\end{figure}

\begin{figure}[t]
  \noindent\includegraphics[width=\textwidth]{/project2/tas1/miyawaki/projects/002/figures/hahn/Flat1850/native/flux/mse_old/lo/0_r1z_mon_lat.png}\\
  \appendcaption{E1}{Same as Fig.~\ref{fig:rea-r1-dev}a but for the CESM preindustrial simulation conducted by \cite{hahn2020} with flattened Antarctic topography. Note that there is no high latitude regime transition in the Southern high latitudes even when Antarctica is flattened.}
  \label{fig:hahn-r1-dev}
\end{figure}

% \begin{figure}[t]
%   \noindent\includegraphics[width=\textwidth]{/project2/tas1/miyawaki/projects/002/figures_post/test/amp_r1_echam/dr1_sub.pdf}\\
%   \appendcaption{E1}{Same as Fig.~\ref{fig:echam-rce}(a) but for other mixed layer depths.}
%   \label{fig:dr1-sub-echam}
% \end{figure}

\end{document}
