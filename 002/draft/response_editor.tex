\documentclass{article}

\usepackage{graphicx}
\usepackage[margin=1in]{geometry}
\usepackage[parfill]{parskip}
\usepackage{afterpage}
\usepackage{natbib}
\usepackage{xcolor}

\title{\vspace{-2.5cm}Response to Editor\vspace{-2cm}}
\date{}
\author{}

\begin{document}
\maketitle

\textbf{We thank the editor for their helpful comments. We revised the manuscript to address many of the concerns and suggestions they raised. Our reponses to their comments are presented in bold. The line numbers referenced in our response correspond to those in our revised manuscript.}

I could imagine a comparison of the vertical structure of the warming response in some climate change simulations with your RCE/RAE index, indicating that this index, defined in a control climate, provides a useful guide to the vertical structure of the warming response geographically and seasonally.  I realize this is a substantial request, and I do not require it, but I think it would make for a more important paper.  You hint at this application, but do not discuss it systematically.  In any case, I would like to see more discussion of what this kind of categorization is useful for in the introduction, to help readers understand your motivation. 

\textbf{We now demonstrate the link between climatological $R_1$ and the vertical temperature response based on CMIP5 RCP8.5 simulations (see Section 5, Fig.~12,13). In addition, we expanded our discussion of why quantifying the energy balance regimes is useful in the introduction (lines 64--80). }

Also, the abstract uses the term “lapse rate regimes” that was not clear to me until I read the paper.  Perhaps you could quickly mention the lapse rate features that you are talking about (surface inversion, moist adiabat, mixed) to define this term.

\textbf{Thank you for the suggestion. We now define what we mean by lapse rate regimes in the abstract (lines 9--10). We also revised the paper to use consistent lapse rate regime terminology throughout the manuscript.}

% \bibliographystyle{apalike}
% \bibliography{references}

\end{document}

