\documentclass{article}

\usepackage{graphicx}
\usepackage[margin=1in]{geometry}
\usepackage[parfill]{parskip}
\usepackage{afterpage}
\usepackage{natbib}
\usepackage{xcolor}

\title{\vspace{-2.5cm}Response to Editor\vspace{-2cm}}
\date{}
\author{}

\begin{document}
\maketitle

\textbf{Thank you for your helpful comments. We revised the manuscript to address the concerns and suggestions you raised. Our responses to your comments are presented in bold. The line numbers referenced in our response correspond to those in our revised manuscript.}

I could imagine a comparison of the vertical structure of the warming response in some climate change simulations with your RCE/RAE index, indicating that this index, defined in a control climate, provides a useful guide to the vertical structure of the warming response geographically and seasonally.  I realize this is a substantial request, and I do not require it, but I think it would make for a more important paper.  You hint at this application, but do not discuss it systematically.  In any case, I would like to see more discussion of what this kind of categorization is useful for in the introduction, to help readers understand your motivation. 

\textbf{Thank you for the suggestion. We added results to the manuscript quantifying the link between energy balance regimes in the modern climate and the vertical structure of the warming response latitudinally and seasonally using the CMIP5 RCP8.5 scenario (see Section 5, Fig.~12, 13). We find that energy balance regimes in the modern climate defined using $R_1$ provide a useful guide to the vertical structure of the warming response in the annual mean, and seasonally across the tropics and the Southern high latitudes (lines~369--384). In the Northern Hemisphere midlatitudes, there is a discrepancy between the seasonality of energy balance regimes in the modern climate and the warming response that is similar to that seen for the modern climate itself (lines~385--392). The response in the Arctic is complicated by large changes in $R_1$ with warming (lines~393--397). Thus, we leave understanding the changes in $R_1$ and their implication for the vertical structure of the warming response in the Northern Hemisphere extratropics for future work.} 

\textbf{In addition, we expanded our discussion of why quantifying the energy balance regimes is useful in the introduction (lines~64--82).}

Also, the abstract uses the term ``lapse rate regimes'' that was not clear to me until I read the paper.  Perhaps you could quickly mention the lapse rate features that you are talking about (surface inversion, moist adiabat, mixed) to define this term.

\textbf{Thank you for the suggestion. We now define lapse rate regimes in the abstract (lines~9--10). We also revised the manuscript to use consistent lapse rate regime terminology.}

% \bibliographystyle{apalike}
% \bibliography{references}

\end{document}

